%----------------------------------------------------------------------------------------
%	PACKAGES AND OTHER DOCUMENT CONFIGURATIONS
%----------------------------------------------------------------------------------------

\documentclass[12pt]{article}
\usepackage[spanish]{babel}
\usepackage[extreme]{savetrees}
\usepackage[utf8]{inputenc}
\usepackage{helvet}
\usepackage{multicol}
\usepackage{lipsum}
\usepackage[breakable,skins]{tcolorbox}
\usepackage{fancyhdr} % Required for custom headers
\usepackage{lastpage} % Required to determine the last page for the footer
\usepackage{amsmath,amsfonts,amssymb,amsthm}
\usepackage{mathabx}

%%%%%%%%% === Document Configuration === %%%%%%%%%%%%%%

\pagestyle{fancy}
\setlength{\headheight}{14.49998pt} %preceding warning said make it at least this
\lhead{Ignacio Rojas} % Top left header
\chead{\textbf{Notas Análisis}} % Top center header
\rhead{}%\firstxmark} % Top right header
\lfoot{}%\lastxmark} % Bottom left footer
\cfoot{} % Bottom center footer
\rfoot{P\'ag.\ \thepage\ de\ \pageref{LastPage}} % Bottom right footer

%%%%%%%%% === My T Color Box === %%%%%%%%%%%%%%

\newtcolorbox{ptcb}{
colframe = black,
colback = white,
breakable,
enhanced
}

\newtcolorbox{ptcbp}{
colframe = black,
colback = white,
coltitle = black,
colbacktitle = black!40,
title = Prueba,
breakable,
enhanced
}

%%%%%%%%% === Theorems and suchlike === %%%%%%%%%%%%%%

\theoremstyle{plain}
\newtheorem{Th}{Teorema}[subsection]   %%% Theorem 1.1
\newtheorem*{nonum-Th}{Teorema}        %%% No-numbered Theorem
\newtheorem{Prop}[Th]{Proposición}     %%% Proposition 1.2
\newtheorem{Lem}[Th]{Lema}             %%% Lemma 1.3
\newtheorem{Cor}[Th]{Corolario}        %%% Corollary 1.4
\newtheorem*{nonum-Cor}{Corolario}     %%% No-numbered Corollary

\theoremstyle{definition}
\newtheorem{Def}[Th]{Definición}       %%%Definition 1.5
\newtheorem*{nonum-Ex}{Ejemplo}        %%% No number Example
\newtheorem{Ex}[Th]{Ejemplo}               %%% Example
\newtheorem{Ej}[Th]{Ejercicio}
\newtheorem*{nEj}{Ejercicio}

\theoremstyle{remark}
\newtheorem{Rmk}[Th]{Observación}      %%%Remark 1.6

\numberwithin{equation}{section}

\setlength{\parindent}{3ex}

%%====== Useful macros: =======%%%

\DeclareMathOperator{\End}{End}     %%%space of endomorphisms
\DeclareMathOperator{\Hom}{Hom}     %%%space of homomorphisms
\DeclareMathOperator{\id}{id}       %%%identity map
\DeclareMathOperator{\gen}{gen}     %%%set generated by...

\newcommand{\la}{\lambda}           %%%short for \lambda
\newcommand{\Om}{\varOmega}         %%%short for \varOmega
\newcommand{\sg}{\sigma}            %%%short for \sigma

\newcommand{\bC}{\mathbb{C}}        %%%complex numbers
\newcommand{\bN}{\mathbb{N}}        %%%natural numbers
\newcommand{\bQ}{\mathbb{Q}}        %%%rational numbers
\newcommand{\bR}{\mathbb{R}}        %%%real numbers
\newcommand{\bS}{\mathbb{S}}        %%%sphere
\newcommand{\bZ}{\mathbb{Z}}        %%%integer numbers
\newcommand{\cF}{\mathcal{F}}       %%%set family
\newcommand{\cB}{\mathcal{B}}       %%%basis
\newcommand{\bFp}{\mathbb{F}_p}     %%%integer numbers
\newcommand{\mm}{\mathfrak{m}}      %%%measure
\newcommand{\cU}{\mathcal{U}}       %%%open set family

\renewcommand{\geq}{\geqslant}      %%%(to save typing)
\renewcommand{\leq}{\leqslant}      %%%(to save typing)
\newcommand{\ox}{\otimes}           %%%tensor product
\renewcommand{\:}{\colon}           %%%colon in  f: A -> B
\let\oldvec=\vec
\renewcommand{\vec}[1]{\mathbf{#1}}
\newcommand{\vx}{\vec{x}}           %%%vectors
\newcommand{\vy}{\vec{y}}
\newcommand{\vz}{\vec{z}}

\newcommand*\quot[2]{{^{\textstyle #1}\big/_{\textstyle #2}}}

\newcommand{\conj}[1]{\left\lbrace#1\right\rbrace}
\newcommand{\bonj}[1]{\left\lbrack#1\right\rbrack}
\newcommand{\obonj}[1]{\left\rbrack#1\right\lbrack}

\def\Circlearrowright{\ensuremath{%
  \rotatebox[origin=c]{180}{$\circlearrowright$}}}


%----------------------------------------------------------------------------------------
%	ARTICLE CONTENTS
%----------------------------------------------------------------------------------------
\begin{document}
\begin{multicols}{2}

\section{Parcial 1}
\subsection{Día 1| 13-3-18}
Recordar definición de norma.
\begin{Def}
Si $\vec{x}\in\bR^d$, defina $||\vec{x}||=\left(\sum_{i}x_i^2\right)^{\frac{1}{2}}$
\end{Def}

\begin{Th}
La norma cumple las siguientes propiedades:
\begin{enumerate}
  \item $||\vec{x}||\geq 0$ y $||\vec{x}|| = 0\iff \vec{x}= 0$
  \item $\forall a\in\bR\forall \vx\in\bR^d: ||a\vec{x}|| = |a|||\vec{x}||$
  \item $\forall \vx,\vy \in\bR^d : ||\vec{x}+\vec{y}||\leq ||\vec{x}||+||\vec{y}||$
\end{enumerate}
\end{Th}

\begin{Def}
  Dado $x_0\in\bR^d$ definimos $$B(\vx_0,r)\colon = \conj{\vx\in\bR^d: ||\vx-\vx_0||< r}$$
\end{Def}

\begin{Def}
 Decimos que $C\subseteq\bR^d$ es abierto si para todo $\vx_0\in C$ existe $r>0$ tal que $B(\vx_0,r)\subseteq C$.
\end{Def}
Cúales son los abiertos sobre $\bR$? Pensaríamos en intervalos abiertos, pero la respuesta correcta es uniones disjuntas contables de intervalos abiertos.

\begin{Lem}\label{pabiertos}
Se cumple que:
  \begin{enumerate}
    \item $\emptyset, \bR^d$ son abiertos
    \item Toda bola es abierta.
    \item Dados $C_1,C_2\subseteq \bR^d$, entonces $C_1\cap C_2$ es abierto.
    \item La unión de abiertos es abierta y la intersección finita de abiertos es abierta.
  \end{enumerate}
\end{Lem}

\begin{ptcbp}
Para $\mathit{2}.$, si $\vy\in B(\vx_0,r)$ tenemos $||\vx_0-\vy||\leq r$. Tome $\vz\in B(\vy,r_1)$ y considere $||\vx_0-r||$. Vea que
\begin{align*}
  ||\vx_0-\vz|| &\leq ||\vx_0-\vy||+||\vy-\vz|| \\
  &<||\vx_0-\vy||+r-||\vx_0-\vy|| = r
\end{align*}



Para $\mathit{3}.$ tome $\vx\in C_1\cap C_2$. Existen $r_1,r_2$ tales que $B(\vx,r_1)\subseteq C_1$ y $B(\vx,r_2)\subseteq C_2$. Sin perdida de generalidad, asuma que $r_1<r_2$, entonces $B(\vx,r_1)\subseteq B(\vx,r_2)\subseteq C_2$. Por lo tanto $B(\vx,r_1)\subseteq C_1\cap C_2$.\par
En $\mathit{4}.$ tome $\vx\in\cup_{i\in I} C_i$, entonces existe $j\in\bonj{d}: \vx\in C_j$. Tenemos que $C_j$ es abierto y por tanto hay una bola adentro de $C_j$. Esta bola está dentro de la unión. Por lo tanto la unión es abierta.
\end{ptcbp}

\begin{Ej}
  Terminar de probar el lema.
\end{Ej}

\begin{ptcb}
Observe que $\emptyset$ es abierto por vacuidad. Para todo punto en $\emptyset$, la bola de cualquier radio está contenida en $\emptyset$. Esto es cierto por vacuidad. Ahora $\bR^d$ es abierto pues toda bola está dentro del espacio.\par
Ahora suponga que tenemos $(A_i)_{i\in\bonj{n}}$ una familia finita de abiertos. Considere $A\colon=\cap_{i\in\bonj{n}}A_i$ y $\vec{x}\in A$, luego $\forall i\in\bonj{n}: \vx\in A_i$. Entonces como todos los $A_i$ son abiertos, $\exists r_i>0$ tal que $B(\vx,r_i)\subseteq A_i$ para todo $i\in\bonj{n}$. Tome $\tilde{r}=\min_i\conj{r_i}$. Entonces $B(\vx,\tilde{r})\subseteq A_i$ para todo $i$, por lo que $B(\vx,\tilde{r})\subseteq A$ y por lo tanto $A$ es abierto.

\end{ptcb}

\begin{Def}
  $F\subseteq\bR^d$ es cerrado si $F^C\colon= \bR^d\setminus F$ es abierto.
\end{Def}

\begin{Lem}\label{propCerrados} Las siguientes son propiedades de cerrados.
  \begin{enumerate}
    \item $\emptyset, \bR^d$ son cerrados.
    \item La unión finita de cerrados es cerrada.
    \item La intersección infinita, inclusive no numerable, de cerrados es cerrada.
  \end{enumerate}
\end{Lem}

\begin{ptcbp}
$\mathit{1}.$ es inmediato de la definición de conjunto cerrado. Como $\emptyset,\bR^d$ son abiertos, sus complementos son cerrados. Estos son $\bR^d$ y $\emptyset$ respectivamente.
Suponga que $(F_i)_{i\in I}$ es una familia arbitraria de cerrados. Note que $\left(\cap_{i=1}^\infty F_i\right)^C=\cup_{i=1}^\infty (F_i)^C$ es abierto pues $(F_i)^C$ es abierto.
\end{ptcbp}

\begin{Ej}
  Terminar de probar el lema.
\end{Ej}

\begin{ptcb}
Sean $(F_i)_{i\in\bonj{n}}$ conjuntos cerrados. Considere $F=\cup_{i\in\bonj{n}}F_i$ y vea que $F^C=\cap_{i\in\bonj{n}}F_i^C$ es una intersección finita de abiertos. Por lo tanto $F^C$ es abierto e inmediatamente $F$ es cerrado.
\end{ptcb}

\begin{Def}
Decimos que $\vx_0$ es un punto de acumulación de $H\subseteq \bR^d$ si existe $(\vy_n)_{n\in\bN}\subseteq H$ tal que $\vy_n\to \vx_0$.
\end{Def}

\begin{Ex}
  El conjunto de los números de la forma $\left(\frac{1}{n}\right)_{n\in\bN}$ tiene como punto de acumulación $0$. Sin embargo $0$ no es de la forma $\frac{1}{n}$.
\end{Ex}


\begin{Lem}
  $F$ es cerrado si y sólo si contiene todos sus puntos de acumulación.
\end{Lem}
\begin{ptcbp}
\begin{enumerate}
  \item[($\Rightarrow$)] Suponga que tenemos un conjunto cerrado $F$ y sea $x$ un punto de acumulación de $F$. Asuma por contradicción que $x\in F^C$. Esto significa que existe $r>0$ tal que $B(x,r)\subseteq F^C$. Esto nos lleva a una contradicción, pues existe $(y_n)_{n\in\bN}\subseteq F$ tal que $y_n\to x$. O sea existe $n_0\geq 0$ tal que para todo $n>n_0: ||y_n-x||< r$.\par
  \item[($\Leftarrow$)] Ahora suponga que $F$ contiene todos sus puntos de acumulación. Suponga que $F$ no es cerrado lo que nos dice que $F^C$ no es abierto. Así, existe un punto $x_0\in F^C$ tal que no existe $r>0: B(x_0,r)\subseteq F^C$. Decir esto es lo mismo que decir que existe un punto de $F$ que está en $B(x_0,r)$. Luego $\exists y_r\in B(x_0,r)\cap F$. En particular $y_n\in B(x_0,\frac{1}{n})\cap F$ o sea $||x_0-y_n||<\frac{1}{n}\Rightarrow y_n\to x_0$ y así $x_0$ es un punto de acumulación.
\end{enumerate}



\end{ptcbp}

\begin{Rmk}
  La siguiente es una prueba, en prosa, distinta del hecho anterior.
\end{Rmk}

\begin{ptcb}
\begin{enumerate}
  \item[($\Rightarrow$)] Suponga que $F$ es cerrado, entonces $F^C$ es abierto. Entonces existe un vecindario alrededor de cada punto de $F^C$ contenido dentro de $F^C$. En otras palabras, cualquier punto fuera de $F$ no es aproximable (no hay un vecindario) por puntos dentro de él. Por contraposición, cualquier punto aproximable de $F$ está en $F$.
  \item[($\Leftarrow$)] Ahora suponga que $F$ contiene todos sus puntos de acumulación. Esto nos dice que todo punto fuera de $F$ no es punto de acumulación de $F$. Viéndolo de otra manera, cualquier punto fuera de $F$ tiene un vecindario enteramente contenido en $F^C$. Esto nos dice que todo punto de $F^C$ tiene un vecindario dentro de él y por lo tanto $F^C$ es abierto.
\end{enumerate}
\end{ptcb}
Hasta ahora, sólo hemos usado propiedades de la norma y el hecho de que las bolas son abiertas. Las cosas que cumplen la desigualdad triangular no son necesariamente normas, sino distancias. No es necesario estar en un espacio vectorial.

\subsubsection*{Espacios Métricos}

\begin{Def}
  Dado un conjunto $E$, una métrica es una función $$\mm\colon E\times E\to \lbrack 0,\infty\lbrack$$ tal que
  \begin{enumerate}
    \item $\forall x,y\in E\colon\,\, \mm(x,y)\geq 0$ y $\mm(x,y) =0\iff x=y$.
    \item $\forall x,y\in E\colon\,\, \mm(x,y) = \mm(y,x)$.
    \item $\forall x,y,z\in E\colon\,\, \mm(x,y)\leq \mm(x,z)+\mm(z,y)$.
  \end{enumerate}
  Decimos que $(E,\mm)$ es un espacio métrico.
\end{Def}

\begin{Ex}
Sea $\vec{x},\vy\in\bR^d$, defina
  \begin{enumerate}
    \item $||\vx||_\infty = \sup\conj{|x_i|\colon i\in\bonj{d}}$ y $\mm_\infty(\vx,\vy)=||\vx-\vy||_\infty$
    \item $||\vx||_1 = \sum_{i\in\bonj{d}}|x_i|$ y $\mm_1(\vx,\vy)=||\vx-\vy||_1$
  \end{enumerate}
\end{Ex}

Veamos que estas funciones en efecto son normas y que las distancias también cumplen la definición de serlo.
\begin{ptcbp}
Sean $\vx,\vy\in\bR^d$. Ahora:
\begin{itemize}
  \item $||\vx||_1=0\iff \sum_{i\in\bonj{d}}|x_i|=0 \iff \forall i\in\bonj{d}\colon x_i=0 \iff \vx=0$
  \item $||a\vx||_1=\sum_{i\in\bonj{d}}|ax_i|=|a|\sum_{i\in\bonj{d}}|x_i|=|a|||\vx||_1$
  \item $||\vx+\vy||_1=\sum_{i\in\bonj{d}}|x_i+y_i|\leq\sum_{i\in\bonj{d}}|x_i|+|y_i|=||\vx||_1+||\vy||_1$
\end{itemize}


\end{ptcbp}

\begin{Ej}
  Muestre que la función $||\cdot||_\infty$ también define una norma. Muestre que las métricas inducidas por estas normas en efecto son métricas.
\end{Ej}

\begin{ptcb}
Nuevamente sean $\vx,\vy\in\bR^d$.
\begin{itemize}
  \item $||\vx||_\infty = 0\iff s\colon=\sup\conj{|x_i|\colon i\in\bonj{d}}=0\Rightarrow 0\leq|x_i|\leq s =0\Rightarrow \forall i\in\bonj{d}\colon x_i=0 \iff \vx=0 $. La otra dirección es inmediata. Todas las coordenadas son 0, entonces $s=0$.
  \item $||a\vx||_\infty=\sup\conj{|ax_i|\colon i\in\bonj{d}}=|a|\sup\conj{|x_i|\colon i\in\bonj{d}}=|a|||\vx||_\infty$.
  \item $||\vx+\vy||_\infty=\sup\conj{|x_i+y_i|\colon i\in\bonj{d}}\leq\sup\conj{|x_i|+|y_i|\colon i\in\bonj{d}}\leq\sup\conj{|x_i|\colon i\in\bonj{d}}+\sup\conj{|y_i|\colon i\in\bonj{d}}=||\vx||_\infty+||\vy||_\infty$.
\end{itemize}
La segunda propiedad viene de $\sup(aX)=a\sup(X)$ y la tercera ocurre pues $|\cdot|$, el valor absoluto en $\bR$, cumple la desigualdad triangular y $\sup(A+B)\leq\sup(A)+\sup(B)$.
\end{ptcb}

%item 2 https://proofwiki.org/wiki/Multiple_of_Supremum
%item 3 https://proofwiki.org/wiki/Supremum_of_Sum_equals_Sum_of_Suprema y http://mathonline.wikidot.com/the-supremum-and-infimum-of-the-sum-of-nonempty-subsets-of-r

\begin{Rmk}
  Del ejercicio anterior, basta mostrar que cualquier norma sobre un e.v. finito-dimensional induce una métrica.
\end{Rmk}

\begin{ptcbp}
Sea $(V,N)$ un e.v. dim. finita con norma $N$.  Sean $x,y,z\in V$, defina $\mm(x,y)=N(x-y)$:
\begin{itemize}
  \item $\mm(x,y)=0\iff N(x-y)=0\iff x-y=0_V\iff x=y$.

  \item $\mm(x,y)=N(x-y)=N((-1)(y-x))=|-1|N(y-x)=N(y-x)=\mm(y,x)$
   \item $\mm(x,z)=N(x-z)=N(x+(-y+y)-z)\leq N(x-y)+N(y-z)=\mm(x,y)+\mm(y,x)$.
\end{itemize}
Por lo tanto $(V,\mm)$ es un espacio métrico.
\end{ptcbp}

\begin{Th}
  $||\cdot||$ y $||\cdot||_1$ son equivalentes.
\end{Th}
\begin{ptcbp}

Note que $||x||_1\leq d\sup\conj{|x_i|\colon i\in\bonj{d}}\leq d||x||$ y elevando a ambos lados al cuadrado la definición de norma Euclídea inmediatamente tenemos $||x||\leq ||x||_1$.

\end{ptcbp}

\begin{Ej}
  Pruebe que la norma $||\cdot||_\infty$ y $||\cdot||$ son equivalentes. Más aún muestre que todas las normas sobre $\bR^d$ son equivalentes.
\end{Ej}

\begin{ptcb}
En efecto sea $\vx\in\bR^d$, inmediatamente tenemos que $||\vx||_\infty\leq ||\vx||$ pues si $|x_j|=\sup\conj{|x_i|\colon i\in\bonj{d}}$ para $j\in\bonj{d}$, entonces $|x_j|^2\leq\sum_{i\in\bonj{d}}|x_i|^2\iff 0\leq \sum_{i\neq j}|x_i|^2$. Ahora note que por definición de $|x_j|$ tenemos que $\sum_{i\in\bonj{d}}|x_i|^2\leq d|x_j|^2$. Tomando raices en ambos lados obtenemos $||\vx||\leq \sqrt{d}||\vx||_\infty$. Por lo tanto:
$$||\vx||_\infty\leq ||\vx||\leq \sqrt{d}||\vx||_\infty$$
Para el segundo apartado note que la relación $N$ es equivalente a $N'$ es transitiva. Suponga que $N,N'\widetilde{N}$ son normas sobre $\bR^d$ y que
\begin{gather*}
  c_1N'(\vx)\leq N(\vx)\leq c_2N'(\vx)\\
  d_1\widetilde{N}(\vx)\leq N'(\vx)\leq d_2\widetilde{N}(\vx)
\end{gather*}
Inmediatamente $c_1d_1\widetilde{N}\leq N(\vx)\leq c_2d_2\widetilde{N}$ por lo que la relación es transitiva.\par
Así, ver que todas las normas son equivalentes es lo mismo que ver que cualquier norma es equivalente a la norma usual sobre $\bR^d$.\textbf{FINISH}
\end{ptcb}

\subsection{Día 2| 16-3-18}

Recuerde que con el concepto de norma podemos definir una distacia. Probar que las normas son equivalentes es lo mismo que lo siguiente.

\begin{Ej}
  Para cualquier norma $N$ sobre $\bR^d$, muestre que existen $c_1=c_1(d), c_2=c_2(d)$ tales que $$c_1||\vx||\leq N(\vx)\leq c_2||\vx||$$
\end{Ej}

\begin{ptcb}
\bf{TO DO}
\end{ptcb}
Continuamos con otro ejemplo de espacio métrico.

\begin{Ex}
Sea $A$ un conjunto, defina $E\colon=\conj{f\colon A\to\bR\colon f\hspace{1mm}\text{es acotada}}$ con $\mm(f,g)=\sup_{x\in A}|f(x)-g(x)|$. Entonces $(E,\mm)$ es un espacio métrico.
\end{Ex}

\begin{ptcbp}
Note que $\mm(f,g)\geq 0$ y $\mm(f,g)=0\iff 0=\sup_{x\in A}|f(x)-g(x)|\iff \forall x\in A 0=|f(x)-g(x)|$. Además si $f,g,h\in E$, se tiene que
\begin{align*}
  |f(x)-g(x)| &\leq |f(x)-h(x)|+|h(x)-g(x)| \\
&\leq \mm(f,h)+\mm(h,g)
\end{align*}
Así $\mm(f,g)\leq \mm(f,h)+\mm(h,g)$
\end{ptcbp}
En el ejercicio anterior, es necesario que las funciones sean acotadas. De lo contrario dicho supremo puede no existir.
A esta métrica se le conoce como la métrica de convergencia uniforme por lo siguiente.

\begin{Ej}
  Si $(f_k)_{k\in\bN}\subseteq E$, entonces $\mm(f_k,f)\to 0$ si $k\to\infty$ si y sólo si $f_k\to f$ uniformemente.
 \end{Ej}
\begin{ptcb}
 Por definición de convergencia uniforme, si $f_k\to f$ uniformemente entonces $\forall\varepsilon >0\exists k_0\in\bN$ tal que $\forall x\in A, k\geq k_0\colon |f_k(x)-f(x)|<\varepsilon$. Por definición de $\sup$ tenemos que $\sup_{x\in A}|f_k(x)-f(x)|<\varepsilon$. O sea que visto en $E$ como sucesión tenemos que $\mm(f_k,f)<\varepsilon$ para todo $\varepsilon>0$ cuando $k\geq k_0$. Esto corrobora la equivalencia.

\end{ptcb}
\subsubsection*{Conceptos Topológicos}

\begin{Def}
  Sea $(E,\mm)$ un espacio métrico. Defina la bola abierta como
  $$B(x_0,r)=\conj{y\in E\colon\mm(x_0,y)<r}$$
  Diremos que $G\subseteq E$ es abierto si $$\forall x\in G\exists r>0\colon B(x,r)\subseteq G$$
\end{Def}

De igual forma que en $\bR^d$, tenemos las siguientes propiedades.

\begin{Lem}
Sea $E$ un espacio métrico.
  \begin{enumerate}
    \item $\emptyset, E$ son abiertos.
    \item Si $G_1, G_2$ son abiertos, entonces $G_1\cap G_2$ es abierto.
    \item Si $A$ es un conjunto y $G_\lambda$ es abierto para todo $\lambda\in A$ entonces $\cup_{\lambda\in A}G_\lambda$ es abierto.
  \end{enumerate}
\end{Lem}

\begin{Ej}
  Adapte la prueba del lema \ref{pabiertos} a este lema.
\end{Ej}

\begin{Def}
  Se dice que $F\subseteq E$ es cerrado si $E\setminus F$ es abierto.
\end{Def}

Las propiedades de los cerrados son las mismas que en \ref{propCerrados} intercambiando $\bR^d$ por $E$.

\begin{Def}
  Dado $A\subseteq E$, decimos que $x_0\in A$ es un punto interior de $A$ si existe $r>0$ tal que $B(x_0,r)\subseteq A$. Análogamente definimos el interior de $A$ como el conjunto de los puntos interiores de $A$. Denotamos $A^o$ al interior.
\end{Def}

  Es importante notar que $A^o$ puede ser vacío.


\begin{Lem}
  Un conjunto $G\subseteq E$ es abierto si y sólo si $G=G^o$.
\end{Lem}

\begin{Def}\label{dvecindario1}
  Decimos que $V\subseteq E$ es un vecindario de $x_0\in E$ si $x_0\in V^o$. Es decir, existe $r>0$ tal que $B(x_0,r)\subseteq V$.
\end{Def}

Veamos un hecho interesante sobre abiertos. Sea $G\subseteq A$ con $G$ abierto. Si $x_0\in G$ existe $r>0$ tal que $B(x_0,r)\subseteq G\subseteq A$. Luego $x_0\in A^o$ y $G\subseteq A^o$. Es decir, $A^o$ es el abierto más grande contenido en $A$. Esto prueba el apartado $\mathit{2}.$ del siguiente lema.

\begin{Lem}
  Sea $A,B\subseteq E$, entonces
  \begin{enumerate}
    \item $A$ es abierto si y sólo si $A=A^o$.
    \item $A^o$ es el abierto más grande contenido en $A$.
    \item Si $A\subseteq B$, entonces $A^o\subseteq B^o$.
    \item $(A\cap B)^o=A^o\cap B^o$.
  \end{enumerate}
\end{Lem}

\begin{ptcbp}
%Si puede meter una bola en A la puede meter en B. Todos los puntos en los que usted puede meter una bola en A, puede meterlo en B.\par
Para $\mathit{4}.$ note que $A^o\cap B^o\subseteq A\cap B$. Luego $A^o\cap B^o\subseteq (A\cap B)^o$. Además $(A\cap B)^o\subseteq A\cap B\subseteq A$. Como $(A\cap B)^o$ es abierto entonces $(A\cap B)^o\subseteq A^o$. De igual forma $(A\cap B)^o\subseteq B^o$.
\end{ptcbp}

\begin{ptcb}
\begin{enumerate}
  \item[$\mathit 1$.] Suponga que $A$ es abierto, por el apartado $\mathit{2}$ tenemos que si $G$ es un subconjunto abierto de $A$ entonces $G\subseteq A^o$. En particular como $A$ es abierto, $A\subseteq A^o$. La otra dirección es inmediata pues $A^o$ es abierto por definición.
  \item[$\mathit 3$.] Observe que por el apartado $\mathit{2}$ tenemos que $A^o\subseteq A\subseteq B$ es un subconjunto abierto de $B$, o sea está contenido en su interior. Así $A^o\subseteq B^o$.
\end{enumerate}
\end{ptcb}

\begin{Def}
  Se dice que $(x_n)_{n\in\bN}\subseteq E$ converge a $x$ si $\lim_{n\to\infty}\mm(x_n,x)=0$. Esto es lo mismo que decir que dado $\varepsilon >0$ existe $n_0$ tal que $\mm(x_n,x)<\varepsilon$ cuando $n\geq n_0$. Denotamos $x_n\to x$.
\end{Def}

Sea $x$ un punto y $(x_n)_{n\in\bN}\subseteq B^C$ tal que $x_n\to x$. Si $x\in B^o$, entonces existe $r>0$ tal que $B(x,r)\subseteq B$. Pero existe $n_0$ tal que $x_n\in B(x,r)$ para todo $n\geq n_0$.

\begin{Def}
  Dado $x_0\in E$, decimos que $x_0$ está en la frontera de $A$ si para todo $r>0$ se cumple $B(x_0,r)\cap A \neq \emptyset \neq B(x_0,r)\cap A^C$. En palabras de sucesiones, existen $(x_n)_{n\in\bN}\subseteq A$ y $(y_n)_{n\in\bN}\subseteq A^C$ tales que $x_n\to x_0$ y $y_n\to x_0$. Denotamos la frontera de $A$ como $\partial A$.
\end{Def}

Inmediatamente se sigue que $\partial(A) = \partial(A^C)$.

\begin{Def}
  Decimos que $x_0\in E$ está en la adherencia de $A$ si para todo $r>0$ se cumple $B(x_0,r)\cap A \neq \emptyset$. Definimos la clausura de $A$ como el conjunto de puntos de adherencia de $A$, denotado por $\overline{A}$. En términos de sucesiones, $x_0\in\overline{A}$ si y sólo si $(x_n)_{n\in\bN}\subseteq A$ tal que $x_n\to x_0$.
\end{Def}

Note que $\mm(x_n,x_0)\to 0$ entonces $\inf\conj{\mm(x_0,a)\colon a\in A}=0$ pues no existe $r>0$ tal que $\mm(x_0,a)> r$.

\begin{Def}
  Dados $A,B\subseteq E$ definimos $\mm(A,B) = \inf\conj{\mm(x,y)\colon (x,y)\in A\times B}$.
\end{Def}

Luego, si $x_0\in\overline{A}$, se tiene $\mm(\conj{x_0},A)=0$.

\begin{Rmk}\label{distCeroClausura}
  El detalle anterior no es obvio. Más aún es una equivalencia.
\end{Rmk}

\begin{ptcb}
\begin{enumerate}
  \item[$(\Rightarrow)$] Suponga que $x_0$ es un punto de adherencia. Entonces $\forall r>0\colon B(x_0,r)\cap A\neq \emptyset$. Ahora sea $y\in B(x_0,r)\cap A$, luego $\mm(\conj{x_0},A)\leq\mm(x_0,y)< r$ y como $r>0$ es arbitrario se sigue que  $\mm(\conj{x_0},A)=0$.
  \item[$(\Leftarrow)$]Ahora suponga que $x_0$ es un punto tal que $\mm(\conj{x_0},A)=0$. Entonces $\forall r>0\exists y_r\in A\colon \mm(x_0,y_r)<r$. Así $B(x_0,r)\cap A\neq \emptyset$. Pero entonces cualquier bola abierta centrada en $x_0$ tendrá intersección no vacía con $A$ por lo que $x_0$ está en su adherencia.
\end{enumerate}
\end{ptcb}
\begin{Ex}
  Considere $A=\conj{n\colon n\geq 1}$ y $B=\conj{n+\frac{1}{n}\colon n\geq 1}$. Como $\mm(n,n+\frac{1}{n})=|n-(n+\frac{1}{n})|=\frac{1}{n}$ entonces $\mm(A,B)=0$.
\end{Ex}
Usualmente la mente lo traiciona a uno pues uno piensa en situaciones con conjuntos acotados. El ejemplo anterior muestra que estos conjuntos se tocan en infinito.\par
Ahora un punto de acumulación es un punto de adherencia sólo que la intersección con $A$ no puede ser trivial. O sea la intersección no puede ser el mismo punto.
\begin{Def}
  Un punto $x_0\in E$ es un punto de acumulación de $A$ si $\left(B(x_0,r)\setminus\conj{x_0}\right)\cap A \neq \emptyset$ para todo $r>0$. El conjunto de los puntos de acumulación se denota $A'$
  Un punto $x_0\in A$ es un punto aislado de $A$ si existe $r>0$ tal que $B(x_0,r)\cap A =\conj{x_0}$.
\end{Def}

\begin{Ex}
  Considere $A=\conj{(x,y)\colon y\geq 0}\subseteq \bR^2$ con la métrica usual. Encuentre el interior, frontera y adherencia de $A$.
   \end{Ex}

   \begin{ptcb}
   Vea que $A^o=A\setminus\conj{(x,y)\colon y=0}$.\par
    Sea $(x,y)\in A^o$, tome $r=\frac{y}{2}$ y $(z,w)\in B\left((x,y),r\right)$. Entonces $||(z,w)-(x,y)||<r=\frac{y}{2}$, así $|w-y|\leq\sqrt{(z-x)^2+(w-y)^2}<\frac{y}{2}$ .
  Por lo que
  \begin{align*}
    &|w-y| <\frac{y}{2} \\
    \iff & -\frac{y}{2}<w-y<\frac{y}{2} \\
    \iff & 0<\frac{y}{2}<w<y
  \end{align*}
  Además $\partial(A)=\conj{(x,0)\colon x\in\bR}$ pues $B\left((x,0),r\right)$ contiene $(x,\frac{r}{2}), (x,-\frac{r}{2})$.\par
  Finalmente $\overline{A}=A$.
\end{ptcb}

\begin{Ej}
  Caracterice los puntos de acumulación en términos de sucesiones.
\end{Ej}

\begin{ptcb}
TO DO
\end{ptcb}
\begin{Lem}
  Sea $A\subseteq E$. Entonces $A$ es cerrado si y sólo si $A=\overline{A}$.
\end{Lem}
\begin{ptcbp}
\begin{enumerate}
  \item[$(\Rightarrow)$] Tenemos $A\subseteq\overline{A}$. Ahora si $A$ es cerrado, $A^C$ es abierto. Entonces dado $x_0\in A^C$, existe $r>0$ tal que $B(x_0,r)\subseteq A^C$. Tome $x_0\in\overline{A}\setminus A$, entonces $B(x_0,r)\cap A\neq\emptyset$. Esto es contradictorio por lo que $\overline{A}\setminus A=\emptyset$ i.e. $A=\overline{A}$.
   \item[$(\Leftarrow)$] Sea $x_0\not\in\overline{A}$, entonces existe $r>0$ tal que $B(x_0,r)\cap A=\emptyset$. Luego $B(x_0,r)\subseteq A^C=(\overline{A})^C$.
\end{enumerate}
\end{ptcbp}

En general tenemos el siguiente hecho.

\begin{Lem}
  Dado $A\subseteq E$, $\overline{A}$ es cerrado.
\end{Lem}

\begin{ptcbp}
Suponga que $\overline{A}$  no es cerrado. Entonces $(\overline{A})^C$ no es abierto. Entonces existe $x_0\in(\overline{A})^C$ tal que para todo $r>0$ tenemos $B(x_0,r)\subsetneq (\overline{A})^C$. Es decir que existe $y_0\in \overline{A}$ tal que $y_0\in B(x_0,r)$. Como $B(x_0,r)$ es abierto, existe $r_1 > 0$ tal que $B(y_0,r_1)\subseteq B(x_0,r)$. Además $y_0\in \overline{A}$ entonces $B(y_0,r_1)\cap A\neq \emptyset\Rightarrow B(x_0,r)\cap A\neq \emptyset\Rightarrow x_0\in \overline{A}$, pues $r$ es arbitrario.

\end{ptcbp}

\begin{Ej}
  Sea $F$ cerrado. Si $A\subseteq F$, entonces $\overline{A}\subseteq F$.
\end{Ej}

\begin{ptcb}
En efecto, sea $x\in\overline{A}$ un punto de adeherencia de $A$. Podemos encontrar una sucesión $(x_n)_{n\in\bN}$ de puntos en $A$ que convergen a $x$. En particular esta sucesión está contenida en $F$ y dado que $F$ es cerrado, contiene todos sus puntos de adherencia. Como $x$ era arbitrario, $F$ contiene todos los puntos de adherencia de $A$, $\overline{A}\subseteq F$.
\end{ptcb}

\begin{Rmk}
  Podemos interpretar el hecho anterior de la siguiente manera. El conjunto de puntos de adherencia de $A$ es el cerrado más pequeño que contiene a $A$.
\end{Rmk}
Realizar los ejercicios de las notas Santiago Cambronero. Sección 2.2.11(1,2,6,7,11,12,16,17,19,20).

\subsection{Día 3| 20-3-18}

\begin{Def}\label{dvecindario2}
  Dado $A\subseteq E$ definimos
  $$V_r(A)=\conj{x\in E\colon \mm(x,A)=\mm(\conj{x},A)<r)}$$
\end{Def}
Será que este conjunto es un vecindario de $A$ según nuestra definición \ref{dvecindario1} anterior?
Tenemos que agarrar una bola y meterla dentro de este conjunto.\par
Tome $a\in A$, entonces $B(a,r)\subseteq V_r(A)$ pues si $y\in B(a,r)\Rightarrow |y-a|<r$ y por definición de distancia al conjunto $\mm(y,A)<|y-a|<r$ y por lo tanto $V_r(A)$ es un vecindario de $A$.\par
Ahora si $A\subseteq E$ es cerrado, $y\not\in A$ entonces $\mm(y,A)>0$. O sea existe $r_0>0$ tal que $\mm(y,A)>r_0$ pues de lo contrario $\mm(y,A)=0$. Entonces $y\not\in V_r(A)$ para cualquier $r<r_0$.\par Esto también se puede corroborar con la observación \ref{distCeroClausura}, ya que si $A$ es cerrado, $y\in A\iff\mm(y,A)=0$. Tomando la contrapositiva de la implicación hacia la izquierda tenemos el resultado.
\begin{Ej}
  Si $A$ es cerrado entonces $V_r(A)$ es abierto.
\end{Ej}

\begin{ptcb}
%Hay que agarrar un señor ahí que no está en $A$ y mostrar que su bola está en el conjunto.
Tome $x\in V_r(A)$ y considere el radio $\tilde{r}=\min\conj{\mm(x,A),r-\mm(x,A)}$. La hipótesis de que $A$ sea cerrado es necesaria para garantizar que $\mm(x,A)>0$, pues $x\not\in A$. De esta manera $B(x,\tilde{r})\subseteq V_r(A)$. \par
Sea $y\in B(x,\tilde{r})$, tenemos que verificar que $\mm(y,A)<r$. Por desigualdad triangular tenemos que
\begin{align*}
 \mm(y,A)&\leq\mm(y,x)+\mm(x,A)\\
 &<\tilde{r}+\mm(x,A) \\
 &=2\mm(x,A)\quad\text{ó}\quad r
\end{align*}
 \textcolor{red}{Cuando se cumple que $\mm(y,A)<2\mm(x,A)$ es porque $\tilde{r}=\mm(x,A)$. Entonces $\mm(x,A)\leq\frac{r}{2}$}. En ambos casos se corrobora que $y\in V_r(A)$.
\end{ptcb}

Además $A\subseteq\cap_{r>0}V_r(A)$. Se cumple la igualdad?
\begin{Ej}
  Si $A$ es cerrado entonces $A=\cap_{r>0}V_r(A)$. En caso de que $A$ no sea cerrado, la igualdad es con $\overline{A}$.
\end{Ej}

\begin{ptcb}
\textbf{TO DO}\par
La intersección es vecindario de tamaño cero. Los puntos a distancia cero de un conjunto es la adherencia.\par
Será que podemos ver $V_r(A)$ como la unión de todas las bolas de radio $r$ en puntos de $A$?
\end{ptcb}

\begin{Def}
  Para un conjunto $F\subseteq E$, decimos que $F$ es denso en $E$ si $\overline{F}=E$.
\end{Def}

\begin{Def}
  Un espacio $(E,\mm)$ es separable si contiene un conjunto denso y numerable.
\end{Def}

\subsubsection*{Métricas Inducidas}

Vamos a cambiar nuestra notación de espacio de $E$ a $X$.

\begin{Def}
  Sea $(X,\mm)$ un espacio métrico y $H\subseteq X$. Defina $\mm_H\colon H\times H\to\bR;\, (h_1,h_2)\mapsto \mm(h_1,h_2)$ es una métrica.
\end{Def}

Entonces qué es una bola en $H$? Vea que $B_H(a,r)=\conj{y\in H\colon \mm(a,y)<r}$ pero esto es $B(a,r)\cap H=\conj{y\in X\colon \mm(y,a)<r}$.
\par
Ahora los abiertos quienes son?\par
Note que si $G$ es abierto y $g\in G$ entonces existe $r_g>0$ tal que $B(g,r_g)\subseteq G\Rightarrow \cup_{g\in G}B(g,r_g)\subseteq G$. Luego si $G$ es abierto respecto a $(H,\mm_H)$ entonces
\begin{align*}
  G&=\cup_{\lambda\in A}B_H(x_\lambda,r_\lambda) \\
   &=\cup_{\lambda\in A}(B(x_\lambda,r_\lambda)\cap H)\\
   &=H\cap \left(\cup_{\lambda\in A}(B(x_\lambda,r_\lambda)\right)\\
   &= H\cap G_X
\end{align*}
Donde $G_X$ es un abierto pero en el espacio grande.

\begin{Lem}
  Todo abierto es unión de bolas.
\end{Lem}

\begin{Rmk}\label{abiertoIffBolas}
  Más aún, un conjunto dentro de un espacio métrico es abierto si y sólo si es unión de bolas abiertas.
\end{Rmk}

\begin{ptcb}
\begin{enumerate}
  \item[$(\Rightarrow)$] Sea $A\subseteq X$ abierto y $a\in A$. Como $A$ es abierto, $\exists r_a>0\colon B(a,r_a)\subseteq A$. Para cualquier $a\in A$ se cumple, por lo que $\cup_{a\in A}B(a,r_a)\subseteq A$. La otra inclusión es inmediata pues por como fue definida, la unión tiene todos los puntos de $A$. Entonces $A$ es unión de bolas.
  \item[$(\Leftarrow)$] Tenemos que la unión de abiertos es un abierto. En este caso, las bolas son abiertas.
\end{enumerate}
\end{ptcb}

\begin{Lem}
  Todo abierto en $(H,\mm_H)$ es de la forma $G\cap H$ con $G$ abierto en $(X,\mm)$.
\end{Lem}

Queremos responder la misma pregunta para cerrados. Cómo son los cerrados en $(H,\mm_H)$?\par
Sea $F$ cerrado en $(H,\mm_H)$ entonces $H\setminus F$ es abierto en $(H,\mm_H)$. Luego, existe $G$ abierto en $(X,\mm)$ tal que $H\setminus F = G\cap H$. Tomando complementos tenemos que $F=H\setminus(G\cap H)= H\cap G^C$. O sea que un cerrado en $H$ es la intersección de un cerrado original con el espacio $H$.

\subsubsection*{Continuidad}

Para poder hablar de la noción de continuidad necesitamos dos espacios métricos.

\begin{Def}
  Sean $(X,\mm),(Y,\mm')$ dos espacios métricos y $f\colon X\to Y$. Se dice que $\lim_{x\to a}f(x)=\ell\in Y$ si para todo $\varepsilon>0$ existe $\delta>0$ tal que
  $0<\mm(a,x)<\delta\Rightarrow \mm'(\ell,f(x))<\varepsilon$. Además $f$ es continua en $a$ si $\lim_{x\to a}f(x)=f(a)$.
\end{Def}
 Observe que es necesario que $f$ esté definida en $a$ para hablar de continuidad. Ahora, una caracterización por bolas es la siguiente. Si $f$ es continua en $a$ entonces para todo $\varepsilon>0$ existe $\delta >0$ tal que $f\left(B(a,\delta)\right)\subseteq B(f(a),\varepsilon)$. Además $\lim_{x\to a} f(x)=\ell$ si y sólo si $f\left(B(a,\delta)\setminus\conj{a}\right)\subseteq B(\ell,\varepsilon)$.

 \begin{Ex}
   Dado $x_0\in X$ defina $f(x)=\mm(x,x_0)$, entonces $\mm(f(x),f(a))=|\mm(x,x_0)-\mm(a,x_0)|\leq \mm(x,a)$. De hecho $f$ es Lipschitz y por tanto es continua.
 \end{Ex}
%% A LEER: https://math.stackexchange.com/questions/370224/proving-something-is-1-lipschitz?utm_medium=organic&utm_source=google_rich_qa&utm_campaign=google_rich_qa

\begin{Ej}
  De igual forma si $A\subseteq X$ es un conjunto, $f(x)=\mm(x,A)$ es continua. Pruebe y utilice que \textcolor{red}{$\mm(x,A)+\mm(y,A)\leq \mm(x,y)$(?)} $\mm(x,A)-\mm(y,A)\leq \mm(x,y)$ .
\end{Ej}

%%A LEER https://math.stackexchange.com/questions/370224/proving-something-is-1-lipschitz?utm_medium=organic&utm_source=google_rich_qa&utm_campaign=google_rich_qa
\begin{ptcb}
Hay que mostrar
$$|\mm(x,A)-\mm(y,A)|\leq \mm(x,y)$$
Sean $z_x,z_y\in A$ tales que $\mm(x,A)=\mm(x,z_x)$ y análogamente para $y$. Así $\mm(y,z_x)\leq\mm(x,z_x)+\mm(x,y)$ y $\mm(x,z_y)\leq\mm(y,z_y)+\mm(x,y)$. Restando desigualdades tenemos que
$$\mm(y,z_x)-\mm(x,z_y)\leq\mm(x,z_x)-\mm(y,z_y)$$
\textbf{FINISH}
\end{ptcb}
\begin{ptcb}
\textcolor{red}{Estoy usando algo más, que $x,y$ y el punto en $A$ se ve como un triángulo.}
Por desigualdad triangular tenemos que $\mm(x,A)\leq\mm(x,y)+\mm(y,A)$. Ahora aplicada a $y$, tenemos que $\mm(y,A)\leq\mm(x,y)+\mm(x,A)$. Así tenemos que
\begin{gather*}
  \mm(x,A)-\mm(y,A)\leq\mm(x,y)\\
  -(\mm(x,A)-\mm(y,A))=\mm(y,A)-\mm(x,A)\leq\mm(x,y)
\end{gather*}
Así $|\mm(x,A)-\mm(y,A)|\leq\mm(x,y)$, luego $f$ es 1-Lipschitz y por tanto continua.
\end{ptcb}
\begin{Def}
  Una función $f\colon X\to Y$ se dice Lipschitz si existe $\lambda >0$ tal que $\mm(f(x),f(y))\leq \lambda\mm(x,y)$. Toda función Lipschitz es continua.
\end{Def}

El siguiente teorema es la caracterización topológica de continuidad. \par
Sea $G\subseteq Y$ abierto y considere $f^{-1}(G)=\conj{x\in X\colon f(x)\in G}$. Tome $x_0\in f^{-1}(G)$, existe $r>0$ tal que $B(x_0,r)\subseteq f^{-1}(G)$? Es decir existe $r>0$ tal que $x\in B(x_0,r)$ entonces $f(x)\in G$?\par
Como $G$ es abierto, existe $\varepsilon>0$ tal que $B(f(x_0),\varepsilon)\subseteq G$. Luego si $f$ es continua, existe $\delta>0$ tal que $f(B(x_0,\delta))\subseteq B(f(x_0),\varepsilon)\subseteq G$.\par Las funciones continuas mandan abiertos de vuelta en abiertos, por imagen inversa.

\begin{Th}
  Si $f\colon X\to Y$ es continua y $G\subseteq Y$ es abierto entonces $f^{-1}(G)$ es abierto en $X$.
\end{Th}

Observe que las imagenes directas no preservan esta propiedad. Es decir, si $G_1\subseteq X$ es abierto, no es siempre cierto que $f(G_1)$ es abierto.\par
De hecho el teorema anterior es una equivalencia.
\begin{Th}
  Sea $f\colon X\to Y$, son equivalentes:
  \begin{enumerate}
    \item $f$ es continua
    \item $f^{-1}(G)$ para todo $G\subseteq Y$ abierto.
    \item $f^{-1}(F)$ es cerrado para todo $F\subseteq Y$ cerrado.
    \item $f(\overline{B})\subseteq \overline{f(B)}$ para todo $B\subseteq X$. Donde la primera cerradura es respecto a $X$ y la segunda respecto a $Y$.
  \end{enumerate}
\end{Th}

\begin{ptcbp}
$(\mathit{2}\Rightarrow\mathit{1})$ Dado $\varepsilon>0$ y $x_0\in X$, $B(f(x_0),\varepsilon))\Rightarrow f^{-1}\lbrack B(f(x_0),\varepsilon)\rbrack$ es abierto. Así $\exists\delta>0$ tal que $B(x_0,\delta)\subseteq f^{-1}\lbrack B(f(x_0),\varepsilon)\rbrack\Rightarrow f(B(x_0,\delta))\subseteq B(f(x_0),\varepsilon)$

\end{ptcbp}

\begin{Ej}
  Mostar que $\mathit{4}$ es equivalente a alguna característica anterior.
\end{Ej}
%%https://math.stackexchange.com/questions/114462/a-map-is-continuous-if-and-only-if-for-every-set-the-image-of-closure-is-contai?utm_medium=organic&utm_source=google_rich_qa&utm_campaign=google_rich_qa
\begin{ptcb}
Vamos a mostrar que $\mathit{2}.\Rightarrow\mathit{4}.$ y $\mathit{4}.\Rightarrow\mathit{3}.$
\begin{enumerate}
  \item[$(\Rightarrow)$] Como $f$ es continua manda abiertos de vuelta en abiertos,  $f^{-1}(Y\setminus \overline{f(B)})$ es abierto de $X$. Además $X\setminus f^{-1}(\overline{f(B)})=f^{-1}(Y\setminus \overline{f(B)})$. De esta manera $f^{-1}(\overline{f(B)})$ es cerrado. Ahora considere la siguiente cadena de inclusiones
      $$A\subseteq f^{-1}(f(B))\subseteq f^{-1}(\overline{f(B)})$$
      Lo que nos dice que $\overline{B}\subseteq f^{-1}(\overline{f(B)})$ y por lo tanto
      $$f(\overline{B})\subseteq f(f^{-1}(\overline{f(B)}))\subseteq\overline{f(B)} $$
  \item[$(\Leftarrow)$] Sea $F\subseteq Y$ cerrado. Tenemos que
  $$f(\overline{f^{-1}(F)}) \subseteq \overline{f(f^{-1}(F))} \subseteq \overline{F} = F$$
  $$\Rightarrow f(\overline{f^{-1}(F)}) \subseteq F$$
  Esto significa que $\overline{f^{-1}(F)}\subseteq f^{-1}(F)$ y por lo tanto $f$ manda cerrados de vuelta en cerrados.
  %Ahora sea $V\subseteq Y$ un abierto y considere $X\setminus f^{-1}(V)$. Queremos ver que tal conjunto es cerrado, o sea $X\setminus f^{-1}(V)=\overline{X\setminus f^{-1}(V)}$. Por hipótesis $f(\overline{X\setminus f^{-1}(V)})\subseteq \overline{f(X\setminus f^{-1}(V))}$, pero $X\setminus f^{-1}(V)=f^{-1}(Y\setminus V)\Rightarrow f(X\setminus f^{-1}(V))\subseteq Y\setminus V$ que es conjunto cerrado.
\end{enumerate}
\end{ptcb}
\begin{Th}
  Sea $f\colon X\to Y$, $f$ es continua en $a$ si y sólo si para cualquier sucesión $(x_n)_{n\in\bN}\subseteq X$ tal que $x_n\to a$ se tiene que $f(x_n)\to f(a)$.
\end{Th}

\begin{ptcbp}
Considere $(f(x_n))_{n\in\bN}$, dado $\varepsilon>0$ hay que mostrar que existe $n_0$ tal que $\mm(f(x_0),f(a))<\varepsilon$ para todo $n\geq n_0$. Sabemos que para todo $\eta>0$ existe $m_0$ tal que $\mm(x_n,a)<\eta$ si $n\geq m_0$. Además existe $\eta_0$ tal que $\mm(x_n,a)<\eta_0 \Rightarrow \mm(f(x_n),f(a))<\varepsilon$. Luego si $n\geq m_0$, entonces $\mm(x_n,a)<\eta_0 \Rightarrow \mm(f(x_n),f(a))<\varepsilon$.\par
Por otro lado asuma que $f$ no es continua, o sea existe $\varepsilon>0$ tal que para todo $\delta$ existe un $x_\delta$ que cumple $\mm(x_\delta,a)<\delta$ y $\mm(f(x_0),f(a))\geq\varepsilon$. En particular si $\delta =\frac{1}{n}$ existe $x_n$ tal que $\mm(x_n,a)<\frac{1}{n}$ y $\mm(f(x_n),f(a))\geq \varepsilon$. Esto nos da una contradicción porque tenemos $f(x_n)$ no converge a $f(a)$.
\end{ptcbp}

Verificar continuidad es lo mismo que verificarlo para sucesiones. Recuerde además que la sucesión no puede ser escogida, debe ser arbitraria.\par
La siguiente definición nos va a servir en algún momento.
\begin{Def}
  Una función $f\colon X\to Y$ es un homeomorfismo si $f$ es biyectiva y $f$ y su inversa son continuas.
\end{Def}

Estas funciones preservan abiertos. Sin embargo, estas funciones no preservan bolas.

\begin{Def}
   Una función $f\colon X\to Y$ es una isometría si es sobreyectiva y $\mm'(f(x),f(y))=\mm(x,y)$.
\end{Def}

Será cierto que las isometrías son inyectivas? En efecto, por un argumento de kernel se obtiene el resultado.
%%https://math.stackexchange.com/questions/120539/every-isometry-is-a-homeomorphism?utm_medium=organic&utm_source=google_rich_qa&utm_campaign=google_rich_qa
\begin{Ej}
  Una isometría es un homeomorfismo.
\end{Ej}

\begin{ptcb}
Primero corroboramos que una isometría es inyectiva. Sea $f$ una isometría con $f(a)=f(b)$ entonces $\mm'(f(a),f(b))=0$. Tenemos que $\mm(a,b)=0$, pues $f$ es una isometría, y por lo tanto $a=b$.\par
Ahora vemos que $f$ es continuas. Pero esto es inmediato, suponga que $\mm(a,b)<\varepsilon$ entonces $\mm'(f(a),f(b))<\varepsilon$ para cualquier $\varepsilon>0$.\par
Entonces nos falta ver que la inversa de $f$ preserva distancias. Como $f$ es una isometría $\mm(a,b)=\mm(f(a),f(b))$ para $a,b\in X$. Como $f$ es biyectiva, existen $c,d\in Y$ tales que $a=f^{-1}(c), b=f^{-1}(d)$. De la igualdad anterior tenemos $\mm(f^{-1}(c),f^{-1}(d))=\mm(f(f^{-1}(c)),f(f^{-1}(d)))=\mm(c,d)$. Así $f^{-1}$ también es una isometría que por lo tanto es continua. \par
Concluímos que una isometría es un homeomorfismo.
\end{ptcb}

\subsubsection*{Compacidad}

Sea $f\colon\bonj{a,b}\to \bR$ continua. Qué propiedades tiene esta función? Es uniformemente continua, alcanza puntos extremos, manda intervalos en intervalos. Estos intervalos son los compactos de $\bR$.

\begin{Def}\label{defCompacto}
  Un conjunto $C\subseteq X$ es compacto si toda sucesión $(x_n)_{n\in\bN}\subseteq C$ tiene una subsucesión $(x_{n_k})_{k\in\bN}$ tal que $x_{n_k}\to c\in C$.
\end{Def}

Bajo esta definición todo compacto es cerrado pues todos los puntos de adherencia son límites de sucesiones y no hay otro lugar que caer para los puntos de adherencia que en $C$.

Note que si $x_{n_k}\to c$ entonces $c$ es un punto de adherencia de $(x_n)_{n\in\bN}$.
Ahora tome $c\in\overline{(x_n)_{n\in\bN}}$, tome $n_2> n_1$ y $x_{n_1}\in B(c,1)\cap(x_n)_{n\in\bN}$. En general $x_{n_k}\in B(c,\frac{1}{k})\cap(x_n)_{n\in\bN}$, esta sucesión converge a $c$. Ahora yo quiero que mi sucesión conste de puntos distintos. Si queremos que $x_{n_1}\neq x_{n_2}$, tome $x_{n_2}\in B(c,r)$ con $r<\min\conj{\frac{1}{2},\mm(x_{n_1},c)}$ entonces $x_{n_1}\neq x_{n_2}$.\par
En general, cuando $n_k>n_{k-1}$, $x_{n_k}\in B(c,r)\cap (x_n)_{n\in\bN}$ con $r <\min\conj{\frac{1}{k},\mm(c,x_{n_1}),\cdots,\mm(c,x_{n_{k-1}})}$.

\begin{Def}
  Decimos que una colección $\mathcal{U}=\conj{U_\alpha\colon \alpha\in A}$ de abiertos de $X$ cubre $C$ si $C\subseteq\cup_{\alpha\in A}U_\alpha$.
\end{Def}

\begin{Lem}
  Sea $C\subseteq X$ compacto y $\cU$ un cubrimiento de $C$. Entonces existe $\varepsilon >0$ tal que para todo $x\in C$ existe $\alpha\in A$ que satisface $B(x,\varepsilon)\subseteq U_\alpha$.
\end{Lem}
Aquí el $\varepsilon$ nos sirve para todos $U_\alpha$'s.
\begin{ptcbp}
Asuma que el resultado es falso, o sea que para todo epsilon puedo encontrar un x tal que $B(x,\varepsilon)$ no está contenido en todos los $U_\alpha$. Van a haber pedazos en $U_\alpha$ pero no completamente. \par
Mejor dicho, para todo $n$ existe $x_n\in C$ tal que $B(x_n,\frac{1}{n})\subsetneq U_\alpha$ para todo $\alpha\in A$. Sabemos que existe $x_{n_k}\to c_0\in C$. Existe un $\alpha$ tal que $c_0\in U_\alpha$.\par
Todavía no hemos usado la hipótesis acerca de $\cU$ como cubrimiento. Vea que como $U_\alpha$ es abierto podemos tomar $\varepsilon > 0$ tal que $B(c_0,\varepsilon)\subseteq U_\alpha$. Basta ver que las bolas anteriores las podemos meter en esta nueva bola para obtener una contradicción, pues según nuestra hipótesis por contradicción no podemos meter las bolas anteriores en un $U_\alpha$.\par
Sea $n_0$ tal que $\mm(x_n,c_0)<\frac{\varepsilon}{2}$ y $\frac{1}{n}<\frac{\varepsilon}{2}$ para todo $n\geq n_0$. Entonces $B(x_n,\frac{1}{n})\subseteq B(c_0,\varepsilon)$ Esto pues $y\in B(x_n,\frac{1}{n})\Rightarrow \mm(y,c_0)\leq \mm(y,x_n)+\mm(x_n,c_0)<\frac{1}{n}+\frac{\varepsilon}{2}<\varepsilon$.
\end{ptcbp}

\subsection{Día 4| 22-3-18}

\subsubsection*{Primera sesión de ejercicios}

\begin{Ej}[2.2.11.1.b Santiago Cambronero]
  Sea $A=\conj{(x,e^x)\colon x\in\bQ}$. Es $A$ acotado? Cerrado? Abierto? Encuentre $A^o, \overline{A}$ y $\partial(A)$.
\end{Ej}
\begin{ptcb}
Tome $(x_n)_{n\in\bN}\subseteq\bR\setminus\bQ$ y $x_n\to a$ entonces como la exponencial es continua $e^{x_n}\to e^a$. Así $(x_n,e^{x_n})\to (a,e^a)$.
\end{ptcb}

\begin{Ej}[2.2.11.7. Santiago Cambronero]
  Mostrar que $\overline{A}$ es el menor cerrado que contiene a $A$
\end{Ej}
\begin{ptcb}
Tome $B$ un cerrado tal que $A\subseteq B\subseteq \overline{A}$. Así $B^C\subseteq A^C$, existe $x\in B^C$ tal que $x\in\overline{A}$. Como $B^C$ es abierto entonces existe $r>0$ tal que $B(x,r)\subseteq B^C$. Esto implica que esta bola está contenida en $A^C$, sin embargo $x\in\overline{A}$. Eso implica que para todo $r_1$ si tomamos la bola $B(x,r_1)$ pescamos a alguien en $A$. Esto es contradictorio pues la bola está completamente contenida en el complemento.
\end{ptcb}

\begin{ptcb}
Estar en $\overline{A}$ significa que existe una sucesión de $A$ que converge al punto. En otras palabras $\exists(x_n)_{n\in\bN}\subseteq A\colon x_n\to x$. Esa misma sucesión está contenida en $B$ y como $B$ es cerrado, es igual a su clausura. Esto nos dice que $x\in B$ y por tanto $\overline{A}\subseteq B$.
\end{ptcb}

\begin{Ej}[2.2.11.7.a Santiago Cambronero]
  Si $A\subseteq B$ entonces $\overline{A}\subseteq\overline{B}$
\end{Ej}

\begin{ptcb}
Tenemos que $A\subseteq \overline{B}$. Como $\overline{B}$ es cerrado, luego $\overline{A}$ es el menor cerrado bajo inclusión que contiene a $A$. Entonces $\overline{A}\subseteq \overline{B}$.
\end{ptcb}

\begin{Ej}[2.2.11.7.b Santiago Cambronero]
 Mostrar $\overline{A\cup B}=\overline{A}\cup\overline{B}$
\end{Ej}

\begin{ptcb}
\begin{enumerate}
  \item[$``\supseteq''$] Como $\overline{A}\cup\overline{B}$ es cerrado entonces $A\cup B\subseteq \overline{A}\cup\overline{B} \Rightarrow \overline{A\cup B}\subseteq \overline{A}\cup\overline{B}$.
  \item[$``\subseteq''$] Si $x\in\overline{A\cup B}$ existe $(x_n)_{n\in\bN}\subseteq A\cup B$ tal que $x_n\to x$. Entonces la sucesión está en uno de los conjuntos. Esto significa que el límite está en uno de los dos conjuntos.
\end{enumerate}
\end{ptcb}

\begin{ptcb}
Por la definición de cerradura, como $\overline{A\cup B}$ es cerrado y $A,B\subseteq A\cup B\subseteq\overline{A\cup B}$. Entonces sus cerraduras están contenidas en la cerradura de la unión.
\end{ptcb}

\begin{Ej}[2.2.11.7.c Santiago Cambronero]
 Mostrar $\overline{A\cap B}\subseteq\overline{A}\cap\overline{B}$
\end{Ej}

\begin{ptcb}
Como $\overline{A}\cap\overline{B}$ es cerrado tenemos que $A\cap B\subseteq \overline{A}\cap\overline{B}$. Por definición de cerradura $\overline{A\cap B}\subseteq\overline{A}\cap\overline{B}$.
\end{ptcb}

\begin{Ej}[2.2.11.12 Santiago Cambronero]
 Mostrar $A\subseteq\bR$ es abierto $\iff A=\cup_i I_i$ con $I_i$ intervalos abiertos. La unión es disjunta.
\end{Ej}

\begin{ptcb}
\begin{enumerate}
  \item[$(\Leftarrow)$] $A$ sería unión de abiertos, entonces $A$ es abierto.
  \item[$(\Rightarrow)$]
 % Si $A$ es un abierto arbitrario, para $x\in\bR$ defina $I_x=\cup_{x\in I\subseteq U}I$ con $I$ intervalos abiertos. Para $x\in A$ existe $r>0$ tal que $\obonj{x-r,x+r}\subseteq A$. Además si $y$ es un racional que está en $\obonj{x-r,x+r}$ entonces $\obonj{x-r,x+r}\subseteq I_y$. Así $x\in I_y$, y como $x$ es arbitrario, cualquier $x\in A$ está en $I_q$ para $q\in A\cap\bQ$. Así $A\subseteq \cup_{q\in A\cap\bQ} I_q$. Pero $I_q\subseteq A$ para $q\in A\cap\bQ$. Entonces $A=\cup_{q\in A\cap\bQ} I_q$. Los intervalos $I_q$ son disjuntos por nuestra definición, ya que si $x\in I_p\cap I_q$ entonces la unión está metida en ambos. Así si $I_q$ es distinto a $I_p$ entonces la unión es disjunta.
\end{enumerate}
\end{ptcb}


\begin{Ej}[2.2.11.16.a Santiago Cambronero]
 Si $A$ es abierto, entonces $A\subseteq(\overline{A})^o$. Encuentre un ejemplo donde la inclusión es estricta.
\end{Ej}

\begin{ptcb}
Tome $a\in A$ entonces existe $r>0$ tal que $B(x,r)\subseteq A\subseteq\overline{A}$. Entonces $a\in (\overline{A})^o$. \par
A manera de ejemplo considere $A=\obonj{a,b}\cup\obonj{b,c}$. Entonces $\overline{A}=\bonj{a,c}$ y $(\overline{A})^o=\obonj{a,c}$.
\end{ptcb}

Otra forma usando la definición de interior
\begin{ptcb}
Si A es abierto, al clausurarlo A barra interior es el mayor abierto contenido en A barra y como A está contenido en A barra debe estar contendio en el interior.
\end{ptcb}


\begin{Ej}[2.2.11.16.b Santiago Cambronero]
 Si $A$ es cerrado, entonces $\overline{(A^o)}\subseteq A$.
\end{Ej}

\begin{ptcb}
Sea $(x_n)_{n\in\bN}\subseteq A^o$ tal que $x_n\to x$. Esto significa que $x\in \overline{(A^o)}$ y así $x\in A$
\end{ptcb}

Parte c
\begin{ptcb}
beta A barra subset A barra, o sea barra alpha A = beta A barra subset A barra
\end{ptcb}

\subsection{Día 5| 3-4-18}

Retomando un ejemplo anterior, para $c\in(x_n)_{n\in\bN}$ nos interesa lo que pasa en la cola. El comportamiento de esta sucesión se determina no por una cantidad finita de puntos sino por los últimos infinitos puntos. Tome $x_{n_1}\in B(c,1)$ y $\varepsilon_2=\min\conj{\frac{1}{2},\mm(c,x_{n_1})}$. Sea $x_{n_2}\in B(c,\varepsilon_2)$. \par
Lo primero que hay que hacer es asumir que el conjunto $\conj{x_n\colon n\in\bN}$ sea infinito. Si fuera finito, la sucesión sólo toma 10 puntos. Es agarrar sucesiones constantes y cada una de estas nos da un punto de acumulación. No aproximándolo, sino repitiéndolo. Llega un momento donde puedo hacer el segundo paso finitas veces, el de escoger el $\varepsilon_2$.\par
Bajo esta nueva hipótesis con $x_{n_1}\neq c$, sea $n_2>n_1$ tal que $x_{n_2}\in B(c,\varepsilon_2)$. Iterando el proceso existe $n_k>n_{k-1}>\cdots>n_1$ tal que $x_{n_k}\in B(c,\varepsilon_k)$ con $\varepsilon_k=\min\conj{\frac{1}{k},\mm(c,x_{n_1}),\cdots,\mm(c,x_{n_k})}$. Esto me asegura el proceso de escoger una sucesión donde todos los elementos son distintos.\par
Recordamos la siguiente definición \ref{defCompacto} de conjuntos compactos.\par
\textcolor{blue}{Cómo hago para que esta definición coincida con la anterior? Se me ocurre usar nonum-Def ó en vez de escribir la definición nuevamnete hacer un hipervínculo desde el ref hasta la ubicación. Sin embargo eso no es bueno para la versión a papel pues cuando toco el papel no cambia automáticamente de página.}
\begin{Def}
  Un conjunto $C$ se dice compacto si dada $(x_n)_{n\in\bN}\subseteq C$ existe $(x_{n_k})_{k\in\bN}\subseteq C$ que converge a un punto de $C$.
\end{Def}
Por la construcción tenemos la siguiente equivalencia.

\begin{Lem}
  Un conjunto $C$ es compacto si y sólo si para cualquier $(x_n)_{n\in\bN}\subseteq C$ se cumple $\cap_{k\in\bN}\overline{\conj{x_n\colon n>k}}\neq\emptyset$.
\end{Lem}

\begin{Lem}
  Sea $C$ un conjunto compacto y $\cU=\conj{U_\alpha\colon\alpha\in A}$ un cubrimiento por abiertos de $C$. Entonces existe $\varepsilon>0$ tal que para todo $x\in C$ existe $\alpha_0\in A$ tal que $B(x,\varepsilon)\subseteq U_{\alpha_0}$.
\end{Lem}

%Uno de los probabilistas más importantes de la historia se llamaba Donald Von Holder. Sólo publicaba artículos muy importantes. Durante una conferencia con gente joven un estudiante le pregunta "profesor, cómo hace para mantenerse activo?". Donald responde "muy sencillo, después de almuerzo me voy a caminar por una hora." L amoraleja es, hacer algo una hora para llegar enteros hasta la vejez.

\begin{Th}\label{equivBWCubrimientos}
  Sea $C\subseteq X$, entonces $C$ es compacto si y sólo si para todo cubrimiento por abiertos $\cU=\conj{U_\alpha\colon \alpha\in A}$ de $C$ existen $(\alpha_i)_{i\in\bonj{m}}\subseteq A$ tal que $C\subseteq\cup_{i\in\bonj{m}}U_{\alpha_i}$.
\end{Th}

\begin{ptcbp}
\begin{enumerate}
  \item[$(\Rightarrow)$] Sea $\cU=\conj{U_\alpha\colon \alpha\in A}$ tal que $C\subseteq\cup_{\alpha\in A}U_{\alpha}$. Sea $\varepsilon>0$ según el lema anterior. Tome $x_1\in C$, así existe $\alpha_1$ tal que $B(x,\varepsilon)\subseteq U_{\alpha_1}$. \par
      Si $C\subseteq U_{\alpha_1}$, estamos listos. De lo contrario tome $x_2\in C\setminus U_{\alpha_1}\subseteq C\setminus B(x_1,\varepsilon)$. Al escogerlo fuera de $U_{\alpha_1}$ ya sé que la distancia $\mm(x_1,x_2)$ es mayor a $\varepsilon$. Tome $\alpha_2$ tal que $B(x_2,\varepsilon)\subseteq U_{\alpha_2}$. En el caso de que $C\subseteq U_{\alpha_1}\cup U_{\alpha_2}$, tenemos el resultado.\par
      De lo contrario existe $x_3\in C\setminus(U_{\alpha_1}\cup U_{\alpha_2})$. En otras palabras $\mm(x_3,x_1)\geq\varepsilon, \mm(x_3,x_2)\geq\varepsilon, \mm(x_2,x_1)\geq\varepsilon$. Esto contradice que esta sucesión sea convergente, contradiciendo la condición de Cauchy.\par
      Si iteramos este proceso, podemos encontrar $x_{k+1}\in C\setminus(\cup_{i\in\bonj{k}}U_{\alpha_i})$ y $\mm(x_i,x_j)\geq\varepsilon$ para $i,j\in\bonj{k+1}, i\neq j$. Si este proceso acaba, significa que existe $m$ tal que $C\subseteq\cup_{i\in\bonj{m}}U_{\alpha_i}$. En el caso contrario construimos una sucesión $(x_n)_{n\in\bN}\subseteq C$ tal que $\mm(x_i,x_j)\geq \varepsilon$ si $i\neq j$ y esto es una contradicción.\par
      Esto pues $\varepsilon\leq\mm(x_i,x_j)\leq \mm(x_i,c)+\mm(x_j,c)$ lo que nos dice que no pueden haber subsucesiones convergentes, contradiciendo la definición de compacidad.
  \item[$(\Leftarrow)$] Asuma que $(x_n)_{n\in\bN}\subseteq C$ no tiene una subsucesión convergente. Defina $U_k=X\setminus (\overline{\conj{x_n\colon n\geq k}})$. Note que $C\subseteq\cup_{k\in\bN}U_k=X\setminus (\cap_{k\in\bN}\overline{\conj{x_n\colon n\geq k}})$.
      %Para este punto ya está la prueba
      y este conjunto es $X$. Luego, por compacidad, existen $k_1<\cdots<k_m$ tales que $C\subseteq\cup_{i\in\bonj{m}}U_{k_i}$. Al unir todos, como van creciendo, es lo mismo que poner el más grande. O sea $C\subseteq U_{k_m}=X\setminus (\overline{\conj{x_n\colon n\geq k_m}})$ y esto es una contradicción pues $x_n\in C$.
\end{enumerate}
\end{ptcbp}

En los textos se toma la siguiente deinición de compacidad.
\begin{Def}
  Dado $\cU=\conj{\widetilde{U}_\alpha\colon \alpha\in A}$ con $\widetilde{U}_\alpha$ abierto respecto a $C$. Diremos que $C$ es compacto si existen $\alpha_1,\cdots,\alpha_n$ tal que $C=\cup_{i\in\bonj{m}}\widetilde{U}_{\alpha_i}$.
\end{Def}

Recuerde que como $\widetilde{U}_\alpha$ es abierto respecto a $C$, existe $U_\alpha$ abierto en $X$ tal que $\widetilde{U}_\alpha=U_\alpha\cap C$.
\begin{Ej}
  La definición anterior coincide con la definición \ref{defCompacto}.
\end{Ej}

\begin{ptcb}

\end{ptcb}
Veamos en $\bR$, cuáles son los conjuntos compactos? En $\bonj{0,1}$ una sucesión dentro de este conjunto tiene una subsucesión convergente por Bolzano-Weierstra{\ss}.\par
Primero veremos un resultado, las funciones continuas mandan compactos en compactos.

\begin{ptcbp}
Sea $f\colon X\to Y$ continua y $K\subseteq X$ compacto. Defina $K_1=f(K)$. Sea $\cU=\conj{U_\alpha\colon\alpha\in A}$ un cubrimiento de $K$. Así $K_1\subseteq\cup_{\alpha\in A}\Rightarrow f^{-1}(K_1)\subseteq\cup_{\alpha\in A}f^{-1}(U_\alpha)$. \par
\begin{ptcb}
Recuerde que $c\in f^{-1}(f(K))\iff f(c)\in f(K)\Rightarrow k\in K$ entonces $f(k)\in f(K)\Rightarrow K\subseteq f^{-1}(f(K))$.
\end{ptcb}
Esto nos dice que $K\subseteq f^{-1}(f(K))\subseteq\cup_{\alpha\in A}f^{-1}(U_\alpha)$. Como $K$ es compacto, existen $\alpha_1,\cdots,\alpha_{m}$ tales que $K\subseteq\cup_{i\in\bonj{m}}f^{-1}(U_{\alpha_i})\Rightarrow f(K)\subseteq\cup_{i\in\bonj{m}}f(f^{-1}(U_{\alpha_i}))$. Entonces $f(K)=K_1$ y $\cup_{i\in\bonj{m}}f(f^{-1}(U_{\alpha_i}))\subseteq\cup_{i\in\bonj{m}}U_{\alpha_i}$ nos da el resultado.
\end{ptcbp}
\begin{Ej}
  Sea $f\colon X\to Y$ y $\tilde{K}\subseteq X, K\subseteq Y$. Pruebe que $\tilde{K}\subseteq f^{-1}(f(\tilde{K}))$ y $f(f^{-1}(K))\subseteq K$. Se cumple la igualdad cuando $f$ es inyectiva y sobreyectiva respectivamente.
\end{Ej}
%No todos los elementos de $K$ tienen que ser imagenes de alguien.
\begin{Th}\label{intervalosEncajados}
  Dado $(X,\mm)$ un espacio métrico. Las siguientes aseveraciones son equivalentes.
  \begin{enumerate}
    \item $X$ es compacto.
    \item Dados $\conj{F_{\alpha}\colon\alpha\in A}$ con $F_\alpha$ cerrados tal que para todos $\alpha_1,\cdots,\alpha_m$ se tiene $\cap_{i\in\bonj{m}}F_{\alpha_i}\neq\emptyset\Rightarrow\cap_{\alpha\in A}F_{\alpha}\neq\emptyset$.
  \end{enumerate}
\end{Th}

\begin{ptcbp}
Probamos $\mathit{1}.\Rightarrow\mathit{2.}$ por contradicción. Suponga que la conclusión es falso o sea existe $\conj{F_\alpha\colon\alpha\in A}$ tal que $\cap_{\alpha\in A}F_\alpha=\emptyset\Rightarrow\cup_{\alpha\in A}X\setminus F_\alpha= X$. Luego $\conj{X\setminus F_\alpha\colon\alpha\in A}$ es un cubrimiento de $X$. Así existen $\alpha_1,\cdots,\alpha_m$ tal que $X=\cup_{i\in\bonj{m}}X\setminus F_{\alpha_i}\Rightarrow\cap_{i\in\bonj{m}}F_{\alpha_i}=\emptyset$. Esto es una contradicción. \textcolor{red}{Por qué?}
\end{ptcbp}

\begin{Def}
  Un espacio $(X,\mm)$ es acotado si para todo $x_0\in X$ existe $r>0$ tal que $B(x_0,r)\supseteq X$. O equivalentemente, existe $M$ tal que $\mm(x,y)\leq M$ para todos los $x,y\in X$.\par
  De manera análoga $B\subseteq X$ es acotado si $(B,\mm_B)$ es acotado. O sea para cualquier $x_0\in X$ existe $r>0$ tal que $B\subseteq B(x_0,r)$.
\end{Def}
Será que un conjunto compacto es acotado? Si $C$ es compacto $C\subseteq\cup_{n\in\bN}B(x_0,n)$. Por compacidad existen $n_1<n_2<\cdots<n_m$ tales que $C\subseteq\cup_{n\in\bonj{m}}B(x_0,n)=B(x_0,n_m)$.

\begin{Lem}
  Si $C\subseteq X$ es compacto, entonces $C$ es cerrado y acotado.
\end{Lem}

%bigtimes adelante
Considere $C=\bigtimes_{i\in\bonj{d}}\bonj{a_i,b_i}\subseteq\bR^d$ con la métrica euclídea. Si $C$ no es compacto, existe $\cU=\conj{U_\alpha\colon\alpha\in A}$ tal que $C\subsetneq\cup_{i\in\bonj{m}}U_{\alpha_i}$ para todo $\alpha_1,\cdots,\alpha_m$.\par
INSERTAR FIG5.1
\par
Divida el rectángulo en $2^d$ rectángulos de la forma $\bigtimes_{j\in\bonj{d}}\bonj{c^i_j,d^i_j}$, $i\in\bonj{2^d}$. Donde $\bonj{c^i_j,d^i_j}$ es de la forma $\bonj{a_j,\frac{a_i+b_i}{2}}$ ó $\bonj{\frac{a_i+b_i}{2},b_j}$. Por hipótesis existe un $i_0\in\bonj{2^d}$ tal que $C_1=\bigtimes_{j\in\bonj{d}}\bonj{c_j^{i_0},d_j^{i_0}}$ tal que no puede ser cubierto por una cantidad finita de $U_\alpha$'s.
Iterando el proceso $C_k=\bigtimes_{j\in\bonj{d}}\bonj{c_j^k,d_j^k}$ con $\bonj{c_j^k,d_j^k}\supseteq\bonj{c_j^{k+1},d_j^{k+1}}$ y $|d_j^k-c_j^k|=\frac{b_j-a_j}{2^k}$. Por el teorema \ref{intervalosEncajados} de los intervalos encajados existe $z_j=\cap_{k\in\bN}\bonj{c_j^k,d_j^k}\Rightarrow \cap_{k\in\bN}C_k=\conj{(z_1,\cdots,z_{algo})}$. Como $z\in C$, existe $\beta\in A\colon z\in U_\beta$. Como $U_\beta$ es abierto, existe $\varepsilon>0$ tal que $B(z,\varepsilon)\subseteq U_\beta$\par
INSERTAR FIG5.2
\par
Si $\frac{\sqrt{d}\sup_{i\in\bonj{d}}|b_i-a_i|}{2^k}<\varepsilon$ entonces $C_k\subseteq B(z,\varepsilon)\subseteq U_\beta$.

\begin{Lem}
  Si $C_1\subseteq C_2$ con $C_2$ compacto y $C_1$ cerrado entonces $C_1$ es compacto.
\end{Lem}

\begin{Ej}
  Muestre el lema anterior.
\end{Ej}

\begin{ptcb}

\end{ptcb}

Como $\bigtimes_{i\in\bonj{d}}\conj{a_i,b_i}$ es compacto,  si $C$ es cerrado y acotado existe $n$ tal que $C\subseteq\bigtimes_{i\in\bonj{d}}\bonj{-n,n}$ tenemos que $C$ es compacto. Con esto probamos el teorema de Heine-Borel.
\begin{Th}[Heine-Borel]\label{Heine-Borel}
  Sea $C\subseteq\bR^d$ con la métrica euclídea. Entonces $C$ es compacto si y sólo si $C$ es cerrado y acotado.
\end{Th}

\subsection{Día 6| 5-4-18}

\subsubsection*{Completitud}

\begin{Def}
  Sea $(X,\mm)$ un espacio métrico y $(x_n)_{n\in\bN}\subseteq X$. Decimos que la sucesión es de Cauchy si para todo $\varepsilon>0$ existe $n_0$ tal que $\mm{x_n,x_m}<\varepsilon$ cuando $n,m\geq n_0$.
\end{Def}

Ahora al igual que en $\bR$ se cumple que:

\begin{Lem}
  Toda sucesión convergente es de Cauchy y toda sucesión de Cauchy es acotada.
\end{Lem}

\begin{Def}
  Decimos que $(X,\mm)$ es completo si toda sucesión de Cauchy en $X$ es convergente.
\end{Def}

\begin{Ex}
  Sea $X =\conj{f\colon\bonj{a,b}\to\bR\colon f\text{ es continua}}$ con la métrica $\mm_{\infty}=\sup_{x\in\bonj{a,b}}\conj{|f(x)-g(x)|}=||f-g||_{\infty}$. Sea $(f_n)_{n\in\bN}$ de Cauchy. \par
  Dado $\varepsilon>0$ existe $n_0$ tal que $||f_n-f_m||_{\infty}\leq\varepsilon$, ahora con $x\in\bonj{a,b}$, $|f_n(x)-f_m(x)|\leq ||f_n-f_m||<\varepsilon$ si $m,n\geq n_0$. Defina $f=\lim_{n\to\infty} f_n$, si $x\in\bonj{a,b}$ existe $n_1$ tal que $|f(x)-f_m(x)|<\varepsilon$ para $m\geq n$. Note que
  $$|f(x)-f_m(x)|\leq |f(x)-f_n(x)|+|f_m(x)-f_n(x)|$$
  El primer término es pequeño pues $f_n\to f$ puntualmente y el segundo término pues $(f_n)_{n\in\bN}$ es Cauchy, esto nos permite empequeñecer independiente del $x$. \par
  De esta manera, si $m,n\geq n_0$ y $m,n\geq n_1$ entonces $|f(x)-f_n(x)|\leq 2\varepsilon$ para $n\geq\max\conj{n_0,n_1}$. Esto prueba coonvergencia uniforme, lo que implica que $f$ además es continua.
\end{Ex}

Considere un subconjunto cerrado $C$ de un espacio completo $X$. Será cierto que toda sucesión convergente en $C$ converge en $C$?\par
Sea $(x_n)_{n\in\bN}\subseteq C\subseteq X$, como $X$ es completo tenemos que $x_n\to y\in X$. Como $C$ es cerrado, se tiene que $y$ está en $C$, toda sucesión convergente en $C$ converge a un límite dentro de $C$.\par
Esto prueba el lema siguiente:

\begin{Lem}
  Sea $(X,\mm)$ un espacio completo, $C\subseteq X$ con $C$ cerrado. Entonces $C$ con métrica inducida es un espacio completo. Además si $(C,\mm_C)$ es completo, entonces es cerrado en $X$.
\end{Lem}

\begin{ptcbp}
Suponga que $C$ es completo, entonces toda sucesión de Cauchy es convergente dentro $C$. De esta manera toda sucesión convergente es $C$ converge dentro de $C$ que es la definición de ser cerrado.
\end{ptcbp}

Habrá alguna relación entre completitud y compacidad?\par
Vea que $\bR$ es completo pero no es compacto. Por otra parte, si $X$ es compacto y $(x_n)_{n\in\bN}\subseteq X$ es de Cauchy existe $(x_{n_k})_{k\in\bN}\subseteq(x_n)_{n\in\bN}$ que converge. O sea $x_{n_k}\to y\in X$, entonces $\mm(x_n,y)\leq \mm(x_n,x_{n_k})+\mm(x_{n_k},y)$. Esto nos dice que $\mm(x_n,y)\to 0$ y por tanto $(x_n)_{n\in\bN}$ es convergente.

\begin{Lem}
  Sea $(X,\mm)$ un espacio compacto, entonces $(X,\mm)$ es completo.
\end{Lem}

\begin{Ej}
  Completar los detalles de la prueba anterior.
\end{Ej}

Aún si completitud no implica directamente compacidad, podemos encontrar una propiedad que junto a completitud nos de compacidad. A esto se le conoce como estar totalmente acotado.

\begin{Def}
  Un espacio métrico $(X,\mm)$ se dice ser totalmente acotado o paracompacto si $\forall\varepsilon>0$ existen $y_1,\cdots,y_n\in X\colon X\subseteq\cup_{i\in\bonj{m}}B(y_i,\varepsilon)$.
\end{Def}

\begin{Ex}
  Tome $X=\bN$ y $\mm(m,n)= m\neq n? 1\colon 0$, la métrica discreta. Entonces $X$ es acotado pero no totalmente acotado. Esto pues el $\varepsilon$ de la definición puede ser menor a 1 y la cantidad de bolas no es contable sino finita.
\end{Ex}

\begin{Th}
  En un espacio totalmente acotado, toda sucesión poseé una subsucesión de Cauchy.
\end{Th}

\begin{ptcbp}
Sea $(x_n)_{n\in\bN}\subseteq X$, entonces como $X$ es compacto existen $y_1,\cdots, y_m$ tal que
$$(x_n)_{n\in\bN}\subseteq X\subseteq\cup_{i\in\bonj{m}}B(y_i,1)$$
Si la sucesión no tiene infinitos puntos, el resultado es inmediato pues tenemos subsucesiones constantes. Sin perdida de generalidad podemos asumir que $\conj{x_n\colon n\in\bN}$ tiene cardinalidad infinita. Entonces existe $z_1$ tal que $B(z_1,1)$ contiene infinitos de la sucesión.\par
Sea $(x_{n,1})_{n\in\bN}\subseteq B(z_1,1)$ y $\conj{x_{n,1}\colon n\geq 1}$ tiene cardinalidad infinita. Si iteramos este proceso, tenemos que $(x_{n,k})_{n\in\bN}\subseteq B(z_k,\frac{1}{k})$ es una sucesión de cardinalidad infinita. De esta manera extraemos la subsucesión $(x_{n,k+1})_{n\in\bN}\subseteq B(z_{k+1},\frac{1}{k+1})$. Note que $\mm(x_{n,k},x_{m,k})\leq\frac{2}{k}$.\par
Tome la sucesión $(x_{k,k})_{k\in\bN}$ y note que $\mm(x_{k,k},x_{\ell,\ell})\leq\frac{2}{k}$.
\end{ptcbp}

Aplicamos un argumento diagonal, creamos muchas sucesiones y así construimos una matriz de sucesiones. De aquí lléndonos por la diagonal podemos agarrar esta.

\iffalse
Totalmente acotado nos da la cantidad finita de bolas. Tenemos infitos señores en una cantidad finita de bolas. Hay una bola con infinitos de estos. Seguimos cubriendo el espacio, pero con bolas de radio menor $(1\to \frac{1}{2})$. Nuevamente una de estas bolas tiene una cantidad infinita de elementos. Otra vez existe una de estas bolas de medio radio que contiene infinitos señores. Seguimos sacando infinitos en bolas de tamaño un tercio, un cuarto, un quinto,... La subsucesión en el quinto paso es subsucesión de la cuarto paso que a su vez es del tercer paso y así. Pero estas subsucesiones son de los señores que ya había filtrado. \par
En algún momento llegue al paso $k$ y agarro una subsucesión infinita en una bola aún más pequeña. En el paso $k$ todos los puntos están apelotados. En el paso $\ell\geq k$ todos los del paso $\ell$ eran del paso $k$. Por la manera escogida no se repiten puntos.
\fi

La otra dirección de la implicación también es cierta.

\begin{Lem}
  Sea $(X,\mm)$ un espacio métrico. Entonces $X$ es totalmente acotado si y sólo si toda sucesión $(x_n)_{n\in\bN}\subseteq X$ poseé una subsucesión de Cauchy.
\end{Lem}

\begin{ptcbp}
Suponga que $X$ no es totalmente acotado, entonecs existe $\varepsilon >0$ tal que $X$ no se puede cubrir con una cantidad finita de bolas con radio $\varepsilon$. Tome $y_1\in X$, entonces $X\subsetneq B(y_1,\varepsilon)$. Sea $y_2\in X\setminus B(y_1,\varepsilon)$. Como $X\subsetneq B(y_1,\varepsilon)\cup B(y_2,\varepsilon)$. En general tome
$$y_k\in X\setminus\left(\cup_{j\in\bonj{k-1}}B(y_j,\varepsilon)\right)$$
Entonces $\mm(y_k,y_\ell)\geq\varepsilon$, o sea la sucesión $(y_n)_{n\in\bN}$ no puede tener una subsucesión de Cauchy.
\end{ptcbp}

\begin{Th}
  Dado un espacio métrico $(X,\mm)$. Se tiene que $C$ es compacto si y sólo si completo y totalmente acotado.
\end{Th}

\subsubsection*{Compleción de un espacio}

Sea $(X,\mm)$ un espacio. Tome $(x_n)_{n\in\bN},(y_n)_{n\in\bN}\subseteq X$ sucesiones de Cauchy. Vea que
$$\mm(x_m,y_m)\leq \mm(x_m,x_n)+\mm(x_n,y_n)+\mm(y_m,y_n)$$
Y a su vez, podemos hacer lo mismo con $\mm(x_n,y_n)$. Entonces
$$|\mm(x_m,y_m)-\mm(x_n,y_n)|\leq \mm(x_n,x_m)+\mm(y_n,y_m)$$
Sabemos que dado $\varepsilon\geq 0$, existe $n_0$ tal que cuando $m,n\geq n_0$:
\begin{gather*}
  \mm(x_n,x_m)<\frac{\varepsilon}{2} \\
  \mm(y_n,y_m)<\frac{\varepsilon}{2}
\end{gather*}
Así $(\mm(x_m,y_m))_{m\in\bN}$ es una sucesión de Cauchy.\par
Definimos
$$\mm^\#((x_n)_{n\in\bN},(y_n)_{n\in\bN})=\lim_{m\to\infty}(\text{\textcolor{red}{no lo escribí}})$$
Sea $(z_n)_{n\in\bN}$ de Cauchy, como $\mm(x_m,y_m)\leq \mm(x_m,z_m)+\mm(z_m,y_m)$ al tomar limites se tiene que
\begin{align*}
  \mm^\#((x_n)_{n\in\bN},(y_n)_{n\in\bN})\leq  & \mm^\#((x_n)_{n\in\bN},(z_n)_{n\in\bN}) \\
   +&\mm^\#((z_n)_{n\in\bN},(y_n)_{n\in\bN})
\end{align*}

Esto nos dice que $\mm^\#$ es una semimétrica ya que no cumple definición positiva.
\end{multicols}
\end{document} 