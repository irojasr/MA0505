%----------------------------------------------------------------------------------------
%	PACKAGES AND OTHER DOCUMENT CONFIGURATIONS
%----------------------------------------------------------------------------------------

\documentclass[12pt]{article}
\usepackage[spanish]{babel}
\usepackage[extreme]{savetrees}
\usepackage[utf8]{inputenc}
\usepackage{helvet}
\usepackage{multicol}
\usepackage{lipsum}
\usepackage[breakable,skins]{tcolorbox}
\usepackage{fancyhdr} % Required for custom headers
\usepackage{lastpage} % Required to determine the last page for the footer
\usepackage{amsmath,amsfonts,amssymb,amsthm}
\usepackage{mathabx}
\usepackage{physics}
\usepackage{mathrsfs}
%\usepackage{bm}

%%%%%%%%% === Document Configuration === %%%%%%%%%%%%%%

\pagestyle{fancy}
\setlength{\headheight}{14.49998pt} %preceding warning said make it at least this
\lhead{Ignacio Rojas} % Top left header
\chead{\textbf{Notas Análisis}} % Top center header
\rhead{}%\firstxmark} % Top right header
\lfoot{}%\lastxmark} % Bottom left footer
\cfoot{} % Bottom center footer
\rfoot{P\'ag.\ \thepage\ de\ \pageref{LastPage}} % Bottom right footer

%%%%%%%%% === My T Color Box === %%%%%%%%%%%%%%

\newtcolorbox{ptcb}{
colframe = black,
colback = white,
breakable,
enhanced
}

\newtcolorbox{ptcbp}{
colframe = black,
colback = white,
coltitle = black,
colbacktitle = black!40,
title = Prueba,
breakable,
enhanced
}

%%%%%%%%% === Theorems and suchlike === %%%%%%%%%%%%%%

\theoremstyle{plain}
\newtheorem{Th}{Teorema}[subsection]   %%% Theorem 1.1
\newtheorem*{nonum-Th}{Teorema}        %%% No-numbered Theorem
\newtheorem{Prop}[Th]{Proposición}     %%% Proposition 1.2
\newtheorem{Lem}[Th]{Lema}             %%% Lemma 1.3
\newtheorem{Cor}[Th]{Corolario}        %%% Corollary 1.4
\newtheorem*{nonum-Cor}{Corolario}     %%% No-numbered Corollary

\theoremstyle{definition}
\newtheorem{Def}[Th]{Definición}       %%%Definition 1.5
\newtheorem*{nonum-Def}{Definición}    %%% No number Definition
\newtheorem*{nonum-Ex}{Ejemplo}        %%% No number Example
\newtheorem{Ex}[Th]{Ejemplo}               %%% Example
\newtheorem{Ej}[Th]{Ejercicio}
\newtheorem*{nEj}{Ejercicio}

\theoremstyle{remark}
\newtheorem{Rmk}[Th]{Observación}      %%%Remark 1.6

\numberwithin{equation}{section}

\setlength{\parindent}{3ex}

%%====== Useful macros: =======%%%

\DeclareMathOperator{\End}{End}     %%%space of endomorphisms
\DeclareMathOperator{\Hom}{Hom}     %%%space of homomorphisms
\DeclareMathOperator{\id}{id}       %%%identity map
\DeclareMathOperator{\gen}{gen}     %%%set generated by...
\DeclareMathOperator{\Var}{Var}     %%%variation
\DeclareMathOperator{\GL}{GL}       %%%general linear group
\DeclareMathOperator{\diam}{diam}   %%%diameter
\DeclareMathOperator*{\esssup}{ess\hspace{0.5mm}sup}
\DeclareMathOperator*{\essinf}{ess\hspace{0.5mm}inf}
\DeclareMathOperator{\sgn}{sgn}     %%%sign of function
\DeclareMathOperator{\ebal}{ev}     %%%evalutation

\newcommand{\la}{\lambda}           %%%short for \lambda
\newcommand{\Om}{\varOmega}         %%%short for \varOmega
\newcommand{\sg}{\sigma}            %%%short for \sigma

\newcommand{\bC}{\mathbb{C}}        %%%complex numbers
\newcommand{\bN}{\mathbb{N}}        %%%natural numbers
\newcommand{\bQ}{\mathbb{Q}}        %%%rational numbers
\newcommand{\bR}{\mathbb{R}}        %%%real numbers
\newcommand{\obR}{\overline{\bR}}   %%%overline real numbers
\newcommand{\bS}{\mathbb{S}}        %%%sphere
\newcommand{\bZ}{\mathbb{Z}}        %%%integer numbers
\newcommand{\cF}{\mathcal{F}}       %%%set family
\newcommand{\cA}{\mathcal{A}}       %%%no me puse creativo con esta letra
\newcommand{\cB}{\mathcal{B}}       %%%basis
\newcommand{\cC}{\mathcal{C}}       %%%class
\newcommand{\cL}{\mathcal{L}}       %%%laplace
\newcommand{\bFp}{\mathbb{F}_p}     %%%integer numbers
\newcommand{\mm}{\mathfrak{m}}      %%%measure
\newcommand{\cP}{\mathcal{P}}       %%%power set
\newcommand{\cR}{\mathcal{R}}       %%%relations
\newcommand{\cU}{\mathcal{U}}       %%%open set family
\newcommand{\cM}{\mathcal{M}}       %%%measureable set family
\newcommand{\cN}{\mathcal{N}}       %%%norm
\newcommand{\ind}{\mathbf{1}}       %%%indicator function
\newcommand{\sF}{\mathscr{F}}       %%%calig f
\newcommand{\sB}{\mathscr{B}}       %%%calig b

\renewcommand{\geq}{\geqslant}      %%%(to save typing)
\renewcommand{\leq}{\leqslant}      %%%(to save typing)
\newcommand{\ox}{\otimes}           %%%tensor product
\renewcommand{\:}{\colon}           %%%colon in  f: A -> B
\let\oldvec=\vec
\renewcommand{\vec}[1]{\mathbf{#1}}
\newcommand{\vx}{\vec{x}}           %%%vectors
\newcommand{\vy}{\vec{y}}
\newcommand{\vz}{\vec{z}}

\newcommand*\quot[2]{{^{\textstyle #1}\big/_{\textstyle #2}}}

\newcommand{\conj}[1]{\left\lbrace#1\right\rbrace}
\newcommand{\bonj}[1]{\left\lbrack#1\right\rbrack}
\newcommand{\obonj}[1]{\left\rbrack#1\right\lbrack}
\newcommand{\rbonj}[1]{\left\rbrack#1\right\rbrack}
\newcommand{\lbonj}[1]{\left\lbrack#1\right\lbrack}
\newcommand{\ov}{\overline}
\newcommand{\nm}[1]{\left\|#1\right\|} %norma pegadita
\newcommand{\pnm}[1]{\biggl|\biggl|#1\biggr|\biggr|}
\newcommand{\Lloc}{L_{\text{loc}}}
\newcommand{\ipd}[1]{\left\langle #1\right\rangle}


\def\Circlearrowright{\ensuremath{%
  \rotatebox[origin=c]{180}{$\circlearrowright$}}}


%----------------------------------------------------------------------------------------
%	ARTICLE CONTENTS
%----------------------------------------------------------------------------------------
\begin{document}
\begin{multicols}{2}

\section{Parcial 1}
\subsection{Día 1| 13-3-18}
Recordar definición de norma.
\begin{Def}
Si $\vec{x}\in\bR^d$, defina $\nm{\vec{x}}=\left(\sum_{i}x_i^2\right)^{\frac{1}{2}}$
\end{Def}

\begin{Th}
La norma cumple las siguientes propiedades:
\begin{enumerate}
  \item $\nm{\vec{x}}\geq 0$ y $\nm{\vec{x}} = 0\iff \vec{x}= 0$
  \item $\forall a\in\bR\forall \vx\in\bR^d: \nm{a\vec{x}} = |a|\nm{\vec{x}}$
  \item $\forall \vx,\vy \in\bR^d : \nm{\vec{x}+\vec{y}}\leq \nm{\vec{x}}+\nm{\vec{y}}$
\end{enumerate}
\end{Th}

\begin{Def}
  Dado $x_0\in\bR^d$ definimos $$B(\vx_0,r)\colon = \conj{\vx\in\bR^d: \nm{\vx-\vx_0}< r}$$
\end{Def}

\begin{Def}
 Decimos que $C\subseteq\bR^d$ es abierto si para todo $\vx_0\in C$ existe $r>0$ tal que $B(\vx_0,r)\subseteq C$.
\end{Def}
Cúales son los abiertos sobre $\bR$? Pensaríamos en intervalos abiertos, pero la respuesta correcta es uniones disjuntas contables de intervalos abiertos.

\begin{Lem}\label{lem:propiedadesConjsAbiertos}
Se cumple que:
  \begin{enumerate}
    \item $\emptyset, \bR^d$ son abiertos
    \item Toda bola es abierta.
    \item Dados $C_1,C_2\subseteq \bR^d$, entonces $C_1\cap C_2$ es abierto.
    \item La unión de abiertos es abierta y la intersección finita de abiertos es abierta.
  \end{enumerate}
\end{Lem}

\begin{ptcbp}
Para $\mathit{2}.$, si $\vy\in B(\vx_0,r)$ tenemos $\nm{\vx_0-\vy}\leq r$. Tome $\vz\in B(\vy,r_1)$ y considere $\nm{\vx_0-r}$. Vea que
\begin{align*}
  \nm{\vx_0-\vz} &\leq \nm{\vx_0-\vy}+\nm{\vy-\vz} \\
  &<\nm{\vx_0-\vy}+r-\nm{\vx_0-\vy} = r
\end{align*}



Para $\mathit{3}.$ tome $\vx\in C_1\cap C_2$. Existen $r_1,r_2$ tales que $B(\vx,r_1)\subseteq C_1$ y $B(\vx,r_2)\subseteq C_2$. Sin perdida de generalidad, asuma que $r_1<r_2$, entonces $B(\vx,r_1)\subseteq B(\vx,r_2)\subseteq C_2$. Por lo tanto $B(\vx,r_1)\subseteq C_1\cap C_2$.\par
En $\mathit{4}.$ tome $\vx\in\cup_{i\in I} C_i$, entonces existe $j\in\bonj{d}: \vx\in C_j$. Tenemos que $C_j$ es abierto y por tanto hay una bola adentro de $C_j$. Esta bola está dentro de la unión. Por lo tanto la unión es abierta.
\end{ptcbp}

\begin{Ej}
  Terminar de probar el lema.
\end{Ej}

\begin{ptcb}
Observe que $\emptyset$ es abierto por vacuidad. Para todo punto en $\emptyset$, la bola de cualquier radio está contenida en $\emptyset$. Esto es cierto por vacuidad. Ahora $\bR^d$ es abierto pues toda bola está dentro del espacio.\par
Ahora suponga que tenemos $(A_i)_{i\in\bonj{n}}$ una familia finita de abiertos. Considere $A\colon=\cap_{i\in\bonj{n}}A_i$ y $\vec{x}\in A$, luego $\forall i\in\bonj{n}: \vx\in A_i$. Entonces como todos los $A_i$ son abiertos, $\exists r_i>0$ tal que $B(\vx,r_i)\subseteq A_i$ para todo $i\in\bonj{n}$. Tome $\tilde{r}=\min_i\conj{r_i}$. Entonces $B(\vx,\tilde{r})\subseteq A_i$ para todo $i$, por lo que $B(\vx,\tilde{r})\subseteq A$ y por lo tanto $A$ es abierto.

\end{ptcb}

\begin{Def}
  $F\subseteq\bR^d$ es cerrado si $F^C\colon= \bR^d\setminus F$ es abierto.
\end{Def}

\begin{Lem}\label{lem:propiedadesConjsCerrados} Las siguientes son propiedades de cerrados.
  \begin{enumerate}
    \item $\emptyset, \bR^d$ son cerrados.
    \item La unión finita de cerrados es cerrada.
    \item La intersección infinita, inclusive no numerable, de cerrados es cerrada.
  \end{enumerate}
\end{Lem}

\begin{ptcbp}
$\mathit{1}.$ es inmediato de la definición de conjunto cerrado. Como $\emptyset,\bR^d$ son abiertos, sus complementos son cerrados. Estos son $\bR^d$ y $\emptyset$ respectivamente.
Suponga que $(F_i)_{i\in I}$ es una familia arbitraria de cerrados. Note que $\left(\cap_{i=1}^\infty F_i\right)^C=\displaystyle\cup_{i=1}^\infty (F_i)^C$ es abierto pues $(F_i)^C$ es abierto.
\end{ptcbp}

\begin{Ej}
  Terminar de probar el lema.
\end{Ej}

\begin{ptcb}
Sean $(F_i)_{i\in\bonj{n}}$ conjuntos cerrados. Considere $F=\cup_{i\in\bonj{n}}F_i$ y vea que $F^C=\cap_{i\in\bonj{n}}F_i^C$ es una intersección finita de abiertos. Por lo tanto $F^C$ es abierto e inmediatamente $F$ es cerrado.
\end{ptcb}

\begin{Def}
Decimos que $\vx_0$ es un punto de acumulación de $H\subseteq \bR^d$ si existe $(\vy_n)_{n\in\bN}\subseteq H$ tal que $\vy_n\to \vx_0$.
\end{Def}

\begin{Ex}
  El conjunto de los números de la forma $\left(\frac{1}{n}\right)_{n\in\bN}$ tiene como punto de acumulación $0$. Sin embargo $0$ no es de la forma $\frac{1}{n}$.
\end{Ex}


\begin{Lem}
  $F$ es cerrado si y sólo si contiene todos sus puntos de acumulación.
\end{Lem}
\begin{ptcbp}
\begin{enumerate}
  \item[($\Rightarrow$)] Suponga que tenemos un conjunto cerrado $F$ y sea $x$ un punto de acumulación de $F$. Asuma por contradicción que $x\in F^C$. Esto significa que existe $r>0$ tal que $B(x,r)\subseteq F^C$. Esto nos lleva a una contradicción, pues existe $(y_n)_{n\in\bN}\subseteq F$ tal que $y_n\to x$. O sea existe $n_0\geq 0$ tal que para todo $n>n_0: \nm{y_n-x}< r$.\par
  \item[($\Leftarrow$)] Ahora suponga que $F$ contiene todos sus puntos de acumulación. Suponga que $F$ no es cerrado lo que nos dice que $F^C$ no es abierto. Así, existe un punto $x_0\in F^C$ tal que no existe $r>0: B(x_0,r)\subseteq F^C$. Decir esto es lo mismo que decir que existe un punto de $F$ que está en $B(x_0,r)$. Luego $\exists y_r\in B(x_0,r)\cap F$. En particular $y_n\in B(x_0,\frac{1}{n})\cap F$ o sea $\nm{x_0-y_n}<\frac{1}{n}\Rightarrow y_n\to x_0$ y así $x_0$ es un punto de acumulación.
\end{enumerate}



\end{ptcbp}

\begin{Rmk}
  La siguiente es una prueba, en prosa, distinta del hecho anterior.
\end{Rmk}

\begin{ptcb}
\begin{enumerate}
  \item[($\Rightarrow$)] Suponga que $F$ es cerrado, entonces $F^C$ es abierto. Entonces existe un vecindario alrededor de cada punto de $F^C$ contenido dentro de $F^C$. En otras palabras, cualquier punto fuera de $F$ no es aproximable (no hay un vecindario) por puntos dentro de él. Por contraposición, cualquier punto aproximable de $F$ está en $F$.
  \item[($\Leftarrow$)] Ahora suponga que $F$ contiene todos sus puntos de acumulación. Esto nos dice que todo punto fuera de $F$ no es punto de acumulación de $F$. Viéndolo de otra manera, cualquier punto fuera de $F$ tiene un vecindario enteramente contenido en $F^C$. Esto nos dice que todo punto de $F^C$ tiene un vecindario dentro de él y por lo tanto $F^C$ es abierto.
\end{enumerate}
\end{ptcb}
Hasta ahora, sólo hemos usado propiedades de la norma y el hecho de que las bolas son abiertas. Las cosas que cumplen la desigualdad triangular no son necesariamente normas, sino distancias. No es necesario estar en un espacio vectorial.

\subsubsection*{Espacios Métricos}

\begin{Def}
  Dado un conjunto $E$, una métrica es una función $$\mm\colon E\times E\to \lbrack 0,\infty\lbrack$$ tal que
  \begin{enumerate}
    \item $\forall x,y\in E\colon\,\, \mm(x,y)\geq 0$ y $\mm(x,y) =0\iff x=y$.
    \item $\forall x,y\in E\colon\,\, \mm(x,y) = \mm(y,x)$.
    \item $\forall x,y,z\in E\colon\,\, \mm(x,y)\leq \mm(x,z)+\mm(z,y)$.
  \end{enumerate}
  Decimos que $(E,\mm)$ es un espacio métrico.
\end{Def}

\begin{Ex}
Sea $\vec{x},\vy\in\bR^d$, defina
  \begin{enumerate}
    \item $\nm{\vx}_\infty = \sup\conj{|x_i|\colon i\in\bonj{d}}$ y $\mm_\infty(\vx,\vy)=\nm{\vx-\vy}|_\infty$
    \item $\nm{\vx}_1 = \sum_{i\in\bonj{d}}|x_i|$ y $\mm_1(\vx,\vy)=\nm{\vx-\vy}|_1$
  \end{enumerate}
\end{Ex}

Veamos que estas funciones en efecto son normas y que las distancias también cumplen la definición de serlo.
\begin{ptcbp}
Sean $\vx,\vy\in\bR^d$. Ahora:
\begin{itemize}
  \item $\nm{\vx}_1=0\iff \sum_{i\in\bonj{d}}|x_i|=0 \iff \forall i\in\bonj{d}\colon x_i=0 \iff \vx=0$
  \item $\nm{a\vx}_1=\sum_{i\in\bonj{d}}|ax_i|=|a|\sum_{i\in\bonj{d}}|x_i|=|a|\nm{\vx}_1$
  \item $\nm{\vx+\vy}_1=\sum_{i\in\bonj{d}}|x_i+y_i|\leq\sum_{i\in\bonj{d}}|x_i|+|y_i|=\nm{\vx}_1+\nm{\vy}_1$
\end{itemize}


\end{ptcbp}

\begin{Ej}
  Muestre que la función $\nm{\cdot}_\infty$ también define una norma. Muestre que las métricas inducidas por estas normas en efecto son métricas.
\end{Ej}

\begin{ptcb}
Nuevamente sean $\vx,\vy\in\bR^d$.
\begin{itemize}
  \item $\nm{\vx}_\infty = 0\iff s\colon=\sup\conj{|x_i|\colon i\in\bonj{d}}=0\Rightarrow 0\leq|x_i|\leq s =0\Rightarrow \forall i\in\bonj{d}\colon x_i=0 \iff \vx=0 $. La otra dirección es inmediata. Todas las coordenadas son 0, entonces $s=0$.
  \item $\nm{a\vx}_\infty=\sup\conj{|ax_i|\colon i\in\bonj{d}}=|a|\sup\conj{|x_i|\colon i\in\bonj{d}}=|a|\nm{\vx}_\infty$.
  \item $\nm{\vx+\vy}_\infty=\sup\conj{|x_i+y_i|\colon i\in\bonj{d}}\leq\sup\conj{|x_i|+|y_i|\colon i\in\bonj{d}}\leq\sup\conj{|x_i|\colon i\in\bonj{d}}+\sup\conj{|y_i|\colon i\in\bonj{d}}=\nm{\vx}_\infty+\nm{\vy}_\infty$.
\end{itemize}
La segunda propiedad viene de $\sup(aX)=a\sup(X)$ y la tercera ocurre pues $|\cdot|$, el valor absoluto en $\bR$, cumple la desigualdad triangular y $\sup(A+B)\leq\sup(A)+\sup(B)$.
\end{ptcb}

%item 2 https://proofwiki.org/wiki/Multiple_of_Supremum
%item 3 https://proofwiki.org/wiki/Supremum_of_Sum_equals_Sum_of_Suprema y http://mathonline.wikidot.com/the-supremum-and-infimum-of-the-sum-of-nonempty-subsets-of-r

\begin{Rmk}
  Del ejercicio anterior, basta mostrar que cualquier norma sobre un e.v. finito-dimensional induce una métrica.
\end{Rmk}

\begin{ptcbp}
Sea $(V,N)$ un e.v. dim. finita con norma $N$.  Sean $x,y,z\in V$, defina $\mm(x,y)=N(x-y)$:
\begin{itemize}
  \item $\mm(x,y)=0\iff N(x-y)=0\iff x-y=0_V\iff x=y$.

  \item $\mm(x,y)=N(x-y)=N((-1)(y-x))=|-1|N(y-x)=N(y-x)=\mm(y,x)$
   \item $\mm(x,z)=N(x-z)=N(x+(-y+y)-z)\leq N(x-y)+N(y-z)=\mm(x,y)+\mm(y,x)$.
\end{itemize}
Por lo tanto $(V,\mm)$ es un espacio métrico.
\end{ptcbp}

\begin{Th}
  $\nm{\cdot}$ y $\nm{\cdot}_1$ son equivalentes.
\end{Th}
\begin{ptcbp}

Note que $\nm{x}_1\leq d\sup\conj{|x_i|\colon i\in\bonj{d}}\leq d\nm{x}$ y elevando a ambos lados al cuadrado la definición de norma Euclídea inmediatamente tenemos $\nm{x}\leq \nm{x}_1$.

\end{ptcbp}

\begin{Ej}
  Pruebe que la norma $\nm{\cdot}_\infty$ y $\nm{\cdot}$ son equivalentes. Más aún muestre que todas las normas sobre $\bR^d$ son equivalentes.
\end{Ej}

\begin{ptcb}
En efecto sea $\vx\in\bR^d$, inmediatamente tenemos que $\nm{\vx}_\infty\leq \nm{\vx}$ pues si $|x_j|=\sup\conj{|x_i|\colon i\in\bonj{d}}$ para $j\in\bonj{d}$, entonces $|x_j|^2\leq\sum_{i\in\bonj{d}}|x_i|^2\iff 0\leq \sum_{i\neq j}|x_i|^2$. Ahora note que por definición de $|x_j|$ tenemos que $\sum_{i\in\bonj{d}}|x_i|^2\leq d|x_j|^2$. Tomando raices en ambos lados obtenemos $\nm{\vx}\leq \sqrt{d}\nm{\vx}_\infty$. Por lo tanto:
$$\nm{\vx}_\infty\leq \nm{\vx}\leq \sqrt{d}\nm{\vx}_\infty$$
Para el segundo apartado note que la relación $N$ es equivalente a $N'$ es transitiva. Suponga que $N,N'\widetilde{N}$ son normas sobre $\bR^d$ y que
\begin{gather*}
  c_1N'(\vx)\leq N(\vx)\leq c_2N'(\vx)\\
  d_1\widetilde{N}(\vx)\leq N'(\vx)\leq d_2\widetilde{N}(\vx)
\end{gather*}
Inmediatamente $c_1d_1\widetilde{N}\leq N(\vx)\leq c_2d_2\widetilde{N}$ por lo que la relación es transitiva.\par
Así, ver que todas las normas son equivalentes es lo mismo que ver que cualquier norma es equivalente a la norma usual sobre $\bR^d$.\textbf{FINISH}
\end{ptcb}

\subsection{Día 2| 16-3-18}

Recuerde que con el concepto de norma podemos definir una distacia. Probar que las normas son equivalentes es lo mismo que lo siguiente.

%%https://sites.math.washington.edu/~morrow/334_16/norms2.pdf
\begin{Ej}
  Para cualquier norma $N$ sobre $\bR^d$, muestre que existen $c_1=c_1(d), c_2=c_2(d)$ tales que $$c_1N(\vx)\leq\nm{\vx}\leq c_2N(\vx)$$
\end{Ej}


\begin{ptcb}
En efecto, sea $\vx\in\bR^d$. Tenemos que si $\vx=(x_1,x_2,\cdots,x_d)$, por desigualdad triangular:
$$N(\vx)\leq\sum_{j\in\bonj{d}}|x_j|N(e_j)$$
Por Cauchy-Schwarz tenemos que
$$\sum_{j\in\bonj{d}}|x_j|N(e_j)\leq\left(\sum_{j\in\bonj{d}}x_j^2\right)^{\frac{1}{2}}\left(\sum_{j\in\bonj{d}}N(e_j)^2\right)^{\frac{1}{2}}$$
Ahora, tome $c_1=\left(\sum_{j\in\bonj{d}}N(e_j)^2\right)^{-\frac{1}{2}}$, entonces tenemos que $c_1N(\vx)\leq \nm{\vx}$.\par
Por otro lado considere la función $N$ sobre el conjunto $K=\conj{\vx\in\bR^d\: \nm{\vec{v}}=1}$. Note además que $K$ es compacto, sea $m=\min_{\vx\in K}\conj{N(\vx)}$ y $\vy=\frac{\vx}{\nm{\vx}}$. Así $\nm{\vy}=1$ y por tanto $N(\vy)\geq m$. Luego tenemos que $$N(\vx)\geq m\nm{\vx}$$
Que a la vez es lo mismo que $\nm{\vx}\leq c_2N(\vx)$ con $c_2=\frac{1}{m}$.
\end{ptcb}

Continuamos con otro ejemplo de espacio métrico.

\begin{Ex}
Sea $A$ un conjunto, defina $E\colon=\conj{f\colon A\to\bR\colon f\hspace{1mm}\text{es acotada}}$ con $\mm(f,g)=\sup_{x\in A}|f(x)-g(x)|$. Entonces $(E,\mm)$ es un espacio métrico.
\end{Ex}

\begin{ptcbp}
Note que $\mm(f,g)\geq 0$ y $\mm(f,g)=0\iff 0=\sup_{x\in A}|f(x)-g(x)|\iff \forall x\in A 0=|f(x)-g(x)|$. Además si $f,g,h\in E$, se tiene que
\begin{align*}
  |f(x)-g(x)| &\leq |f(x)-h(x)|+|h(x)-g(x)| \\
&\leq \mm(f,h)+\mm(h,g)
\end{align*}
Así $\mm(f,g)\leq \mm(f,h)+\mm(h,g)$
\end{ptcbp}
En el ejercicio anterior, es necesario que las funciones sean acotadas. De lo contrario dicho supremo puede no existir.
A esta métrica se le conoce como la métrica de convergencia uniforme por lo siguiente.

\begin{Ej}
  Si $(f_k)_{k\in\bN}\subseteq E$, entonces $\mm(f_k,f)\to 0$ si $k\to\infty$ si y sólo si $f_k\to f$ uniformemente.
 \end{Ej}
\begin{ptcb}
 Por definición de convergencia uniforme, si $f_k\to f$ uniformemente entonces $\forall\varepsilon >0\exists k_0\in\bN$ tal que $\forall x\in A, k\geq k_0\colon |f_k(x)-f(x)|<\varepsilon$. Por definición de $\sup$ tenemos que $\sup_{x\in A}|f_k(x)-f(x)|<\varepsilon$. O sea que visto en $E$ como sucesión tenemos que $\mm(f_k,f)<\varepsilon$ para todo $\varepsilon>0$ cuando $k\geq k_0$. Esto corrobora la equivalencia.

\end{ptcb}
\subsubsection*{Conceptos Topológicos}

\begin{Def}
  Sea $(E,\mm)$ un espacio métrico. Defina la bola abierta como
  $$B(x_0,r)=\conj{y\in E\colon\mm(x_0,y)<r}$$
  Diremos que $G\subseteq E$ es abierto si $$\forall x\in G\exists r>0\colon B(x,r)\subseteq G$$
\end{Def}

De igual forma que en $\bR^d$, tenemos las siguientes propiedades.

\begin{Lem}
Sea $E$ un espacio métrico.
  \begin{enumerate}
    \item $\emptyset, E$ son abiertos.
    \item Si $G_1, G_2$ son abiertos, entonces $G_1\cap G_2$ es abierto.
    \item Si $A$ es un conjunto y $G_\lambda$ es abierto para todo $\lambda\in A$ entonces $\cup_{\lambda\in A}G_\lambda$ es abierto.
  \end{enumerate}
\end{Lem}

\begin{Ej}
  Adapte la prueba del lema \ref{lem:propiedadesConjsAbiertos} a este lema.
\end{Ej}

\begin{Def}
  Se dice que $F\subseteq E$ es cerrado si $E\setminus F$ es abierto.
\end{Def}

Las propiedades de los cerrados son las mismas que en \ref{lem:propiedadesConjsCerrados} intercambiando $\bR^d$ por $E$.

\begin{Def}
  Dado $A\subseteq E$, decimos que $x_0\in A$ es un punto interior de $A$ si existe $r>0$ tal que $B(x_0,r)\subseteq A$. Análogamente definimos el interior de $A$ como el conjunto de los puntos interiores de $A$. Denotamos $A^o$ al interior.
\end{Def}

  Es importante notar que $A^o$ puede ser vacío.


\begin{Lem}
  Un conjunto $G\subseteq E$ es abierto si y sólo si $G=G^o$.
\end{Lem}

\begin{Def}\label{def:vecindarioTopologico}
  Decimos que $V\subseteq E$ es un vecindario de $x_0\in E$ si $x_0\in V^o$. Es decir, existe $r>0$ tal que $B(x_0,r)\subseteq V$.
\end{Def}

Veamos un hecho interesante sobre abiertos. Sea $G\subseteq A$ con $G$ abierto. Si $x_0\in G$ existe $r>0$ tal que $B(x_0,r)\subseteq G\subseteq A$. Luego $x_0\in A^o$ y $G\subseteq A^o$. Es decir, $A^o$ es el abierto más grande contenido en $A$. Esto prueba el apartado $\mathit{2}.$ del siguiente lema.

\begin{Lem}
  Sea $A,B\subseteq E$, entonces
  \begin{enumerate}
    \item $A$ es abierto si y sólo si $A=A^o$.
    \item $A^o$ es el abierto más grande contenido en $A$.
    \item Si $A\subseteq B$, entonces $A^o\subseteq B^o$.
    \item $(A\cap B)^o=A^o\cap B^o$.
  \end{enumerate}
\end{Lem}

\begin{ptcbp}
%Si puede meter una bola en A la puede meter en B. Todos los puntos en los que usted puede meter una bola en A, puede meterlo en B.\par
Para $\mathit{4}.$ note que $A^o\cap B^o\subseteq A\cap B$. Luego $A^o\cap B^o\subseteq (A\cap B)^o$. Además $(A\cap B)^o\subseteq A\cap B\subseteq A$. Como $(A\cap B)^o$ es abierto entonces $(A\cap B)^o\subseteq A^o$. De igual forma $(A\cap B)^o\subseteq B^o$.
\end{ptcbp}

\begin{ptcb}
\begin{enumerate}
  \item[$\mathit 1$.] Suponga que $A$ es abierto, por el apartado $\mathit{2}$ tenemos que si $G$ es un subconjunto abierto de $A$ entonces $G\subseteq A^o$. En particular como $A$ es abierto, $A\subseteq A^o$. La otra dirección es inmediata pues $A^o$ es abierto por definición.
  \item[$\mathit 3$.] Observe que por el apartado $\mathit{2}$ tenemos que $A^o\subseteq A\subseteq B$ es un subconjunto abierto de $B$, o sea está contenido en su interior. Así $A^o\subseteq B^o$.
\end{enumerate}
\end{ptcb}

\begin{Def}
  Se dice que $(x_n)_{n\in\bN}\subseteq E$ converge a $x$ si $\lim_{n\to\infty}\mm(x_n,x)=0$. Esto es lo mismo que decir que dado $\varepsilon >0$ existe $n_0$ tal que $\mm(x_n,x)<\varepsilon$ cuando $n\geq n_0$. Denotamos $x_n\to x$.
\end{Def}

Sea $x$ un punto y $(x_n)_{n\in\bN}\subseteq B^C$ tal que $x_n\to x$. Si $x\in B^o$, entonces existe $r>0$ tal que $B(x,r)\subseteq B$. Pero existe $n_0$ tal que $x_n\in B(x,r)$ para todo $n\geq n_0$.

\begin{Def}
  Dado $x_0\in E$, decimos que $x_0$ está en la frontera de $A$ si para todo $r>0$ se cumple $B(x_0,r)\cap A \neq \emptyset \neq B(x_0,r)\cap A^C$. En palabras de sucesiones, existen $(x_n)_{n\in\bN}\subseteq A$ y $(y_n)_{n\in\bN}\subseteq A^C$ tales que $x_n\to x_0$ y $y_n\to x_0$. Denotamos la frontera de $A$ como $\partial A$.
\end{Def}

Inmediatamente se sigue que $\partial(A) = \partial(A^C)$.

\begin{Def}
  Decimos que $x_0\in E$ está en la adherencia de $A$ si para todo $r>0$ se cumple $B(x_0,r)\cap A \neq \emptyset$. Definimos la clausura de $A$ como el conjunto de puntos de adherencia de $A$, denotado por $\overline{A}$. En términos de sucesiones, $x_0\in\overline{A}$ si y sólo si $(x_n)_{n\in\bN}\subseteq A$ tal que $x_n\to x_0$.
\end{Def}

Note que $\mm(x_n,x_0)\to 0$ entonces $\inf\conj{\mm(x_0,a)\colon a\in A}=0$ pues no existe $r>0$ tal que $\mm(x_0,a)> r$.

\begin{Def}
  Dados $A,B\subseteq E$ definimos $\mm(A,B) = \inf\conj{\mm(x,y)\colon (x,y)\in A\times B}$.
\end{Def}

Luego, si $x_0\in\overline{A}$, se tiene $\mm(\conj{x_0},A)=0$.

\begin{Ej}\label{ej:clausuraMetricaIffDistanciaCero}
  El detalle anterior no es obvio. Más aún es una equivalencia, pruébelo.
\end{Ej}

\begin{ptcb}
\begin{enumerate}
  \item[$(\Rightarrow)$] Suponga que $x_0$ es un punto de adherencia. Entonces $\forall r>0\colon B(x_0,r)\cap A\neq \emptyset$. Ahora sea $y\in B(x_0,r)\cap A$, luego $\mm(\conj{x_0},A)\leq\mm(x_0,y)< r$ y como $r>0$ es arbitrario se sigue que  $\mm(\conj{x_0},A)=0$.
  \item[$(\Leftarrow)$]Ahora suponga que $x_0$ es un punto tal que $\mm(\conj{x_0},A)=0$. Entonces $\forall r>0\exists y_r\in A\colon \mm(x_0,y_r)<r$. Así $B(x_0,r)\cap A\neq \emptyset$. Pero entonces cualquier bola abierta centrada en $x_0$ tendrá intersección no vacía con $A$ por lo que $x_0$ está en su adherencia.
\end{enumerate}
\end{ptcb}
\begin{Ex}
  Considere $A=\conj{n\colon n\geq 1}$ y $B=\conj{n+\frac{1}{n}\colon n\geq 1}$. Como $\mm(n,n+\frac{1}{n})=|n-(n+\frac{1}{n})|=\frac{1}{n}$ entonces $\mm(A,B)=0$.
\end{Ex}
Usualmente la mente lo traiciona a uno pues uno piensa en situaciones con conjuntos acotados. El ejemplo anterior muestra que estos conjuntos se tocan en infinito.\par
Ahora un punto de acumulación es un punto de adherencia sólo que la intersección con $A$ no puede ser trivial. O sea la intersección no puede ser el mismo punto.
\begin{Def}
  Un punto $x_0\in E$ es un punto de acumulación de $A$ si $\left(B(x_0,r)\setminus\conj{x_0}\right)\cap A \neq \emptyset$ para todo $r>0$. El conjunto de los puntos de acumulación se denota $A'$
  Un punto $x_0\in A$ es un punto aislado de $A$ si existe $r>0$ tal que $B(x_0,r)\cap A =\conj{x_0}$.
\end{Def}

\begin{Ex}
  Considere $A=\conj{(x,y)\colon y\geq 0}\subseteq \bR^2$ con la métrica usual. Encuentre el interior, frontera y adherencia de $A$.
   \end{Ex}

   \begin{ptcb}
   Vea que $A^o=A\setminus\conj{(x,y)\colon y=0}$.\par
    Sea $(x,y)\in A^o$, tome $r=\frac{y}{2}$ y $(z,w)\in B\left((x,y),r\right)$. Entonces $\nm{(z,w)-(x,y)}<r=\frac{y}{2}$, así $|w-y|\leq\sqrt{(z-x)^2+(w-y)^2}<\frac{y}{2}$ .
  Por lo que
  \begin{align*}
    &|w-y| <\frac{y}{2} \\
    \iff & -\frac{y}{2}<w-y<\frac{y}{2} \\
    \iff & 0<\frac{y}{2}<w<y
  \end{align*}
  Además $\partial(A)=\conj{(x,0)\colon x\in\bR}$ pues $B\left((x,0),r\right)$ contiene $(x,\frac{r}{2}), (x,-\frac{r}{2})$.\par
  Finalmente $\overline{A}=A$.
\end{ptcb}

\begin{Ej}
  Caracterice los puntos de acumulación en términos de sucesiones.
\end{Ej}

\begin{ptcb}
TO DO
\end{ptcb}
\begin{Lem}
  Sea $A\subseteq E$. Entonces $A$ es cerrado si y sólo si $A=\overline{A}$.
\end{Lem}
\begin{ptcbp}
\begin{enumerate}
  \item[$(\Rightarrow)$] Tenemos $A\subseteq\overline{A}$. Ahora si $A$ es cerrado, $A^C$ es abierto. Entonces dado $x_0\in A^C$, existe $r>0$ tal que $B(x_0,r)\subseteq A^C$. Tome $x_0\in\overline{A}\setminus A$, entonces $B(x_0,r)\cap A\neq\emptyset$. Esto es contradictorio por lo que $\overline{A}\setminus A=\emptyset$ i.e. $A=\overline{A}$.
   \item[$(\Leftarrow)$] Sea $x_0\not\in\overline{A}$, entonces existe $r>0$ tal que $B(x_0,r)\cap A=\emptyset$. Luego $B(x_0,r)\subseteq A^C=(\overline{A})^C$.
\end{enumerate}
\end{ptcbp}

En general tenemos el siguiente hecho.

\begin{Lem}
  Dado $A\subseteq E$, $\overline{A}$ es cerrado.
\end{Lem}

\begin{ptcbp}
Suponga que $\overline{A}$  no es cerrado. Entonces $(\overline{A})^C$ no es abierto. Entonces existe $x_0\in(\overline{A})^C$ tal que para todo $r>0$ tenemos $B(x_0,r)\subsetneq (\overline{A})^C$. Es decir que existe $y_0\in \overline{A}$ tal que $y_0\in B(x_0,r)$. Como $B(x_0,r)$ es abierto, existe $r_1 > 0$ tal que $B(y_0,r_1)\subseteq B(x_0,r)$. Además $y_0\in \overline{A}$ entonces $B(y_0,r_1)\cap A\neq \emptyset\Rightarrow B(x_0,r)\cap A\neq \emptyset\Rightarrow x_0\in \overline{A}$, pues $r$ es arbitrario.

\end{ptcbp}

\begin{Ej}
  Sea $F$ cerrado. Si $A\subseteq F$, entonces $\overline{A}\subseteq F$.
\end{Ej}

\begin{ptcb}
En efecto, sea $x\in\overline{A}$ un punto de adeherencia de $A$. Podemos encontrar una sucesión $(x_n)_{n\in\bN}$ de puntos en $A$ que convergen a $x$. En particular esta sucesión está contenida en $F$ y dado que $F$ es cerrado, contiene todos sus puntos de adherencia. Como $x$ era arbitrario, $F$ contiene todos los puntos de adherencia de $A$, $\overline{A}\subseteq F$.
\end{ptcb}

\begin{Rmk}
  Podemos interpretar el hecho anterior de la siguiente manera. El conjunto de puntos de adherencia de $A$ es el cerrado más pequeño que contiene a $A$.
\end{Rmk}
Realizar los ejercicios de las notas Santiago Cambronero. Sección 2.2.11(1,2,6,7,11,12,16,17,19,20).

\subsection{Día 3| 20-3-18}

\begin{Def}\label{def:bolasEspacioMetrico}
  Dado $A\subseteq E$ definimos
  $$V_r(A)=\conj{x\in E\colon \mm(x,A)=\mm(\conj{x},A)<r)}$$
\end{Def}
Será que este conjunto es un vecindario de $A$ según nuestra definición \ref{def:vecindarioTopologico} anterior?
Tenemos que agarrar una bola y meterla dentro de este conjunto.\par
Tome $a\in A$, entonces $B(a,r)\subseteq V_r(A)$ pues si $y\in B(a,r)\Rightarrow |y-a|<r$ y por definición de distancia al conjunto $\mm(y,A)<|y-a|<r$ y por lo tanto $V_r(A)$ es un vecindario de $A$.\par
Ahora si $A\subseteq E$ es cerrado, $y\not\in A$ entonces $\mm(y,A)>0$. O sea existe $r_0>0$ tal que $\mm(y,A)>r_0$ pues de lo contrario $\mm(y,A)=0$. Entonces $y\not\in V_r(A)$ para cualquier $r<r_0$.\par Esto también se puede corroborar con el ejercicio \ref{ej:clausuraMetricaIffDistanciaCero}, ya que si $A$ es cerrado, $y\in A\iff\mm(y,A)=0$. Tomando la contrapositiva de la implicación hacia la izquierda tenemos el resultado.
\begin{Ej}
  Si $A$ es cerrado entonces $V_r(A)$ es abierto.
\end{Ej}

\begin{ptcb}
%Hay que agarrar un señor ahí que no está en $A$ y mostrar que su bola está en el conjunto.
Tome $x\in V_r(A)$ y considere el radio $\tilde{r}=\min\conj{\mm(x,A),r-\mm(x,A)}$. La hipótesis de que $A$ sea cerrado es necesaria para garantizar que $\mm(x,A)>0$, pues $x\not\in A$. De esta manera $B(x,\tilde{r})\subseteq V_r(A)$. \par
Sea $y\in B(x,\tilde{r})$, tenemos que verificar que $\mm(y,A)<r$. Por desigualdad triangular tenemos que
\begin{align*}
 \mm(y,A)&\leq\mm(y,x)+\mm(x,A)\\
 &<\tilde{r}+\mm(x,A) \\
 &=2\mm(x,A)\quad\text{ó}\quad r
\end{align*}
 \textcolor{red}{Cuando se cumple que $\mm(y,A)<2\mm(x,A)$ es porque $\tilde{r}=\mm(x,A)$. Entonces $\mm(x,A)\leq\frac{r}{2}$}. En ambos casos se corrobora que $y\in V_r(A)$.
\end{ptcb}

Además $A\subseteq\cap_{r>0}V_r(A)$. Se cumple la igualdad?
\begin{Ej}
  Si $A$ es cerrado entonces $A=\cap_{r>0}V_r(A)$. En caso de que $A$ no sea cerrado, la igualdad es con $\overline{A}$.
\end{Ej}

\begin{ptcb}
\textbf{TO DO}\par
La intersección es vecindario de tamaño cero. Los puntos a distancia cero de un conjunto es la adherencia.\par
Será que podemos ver $V_r(A)$ como la unión de todas las bolas de radio $r$ en puntos de $A$?
\end{ptcb}

\begin{Def}
  Para un conjunto $F\subseteq E$, decimos que $F$ es denso en $E$ si $\overline{F}=E$.
\end{Def}

\begin{Def}
  Un espacio $(E,\mm)$ es separable si contiene un conjunto denso y numerable.
\end{Def}

\subsubsection*{Métricas Inducidas}

Vamos a cambiar nuestra notación de espacio de $E$ a $X$.

\begin{Def}
  Sea $(X,\mm)$ un espacio métrico y $H\subseteq X$. Defina $\mm_H\colon H\times H\to\bR;\, (h_1,h_2)\mapsto \mm(h_1,h_2)$ es una métrica.
\end{Def}

Entonces qué es una bola en $H$? Vea que $B_H(a,r)=\conj{y\in H\colon \mm(a,y)<r}$ pero esto es $B(a,r)\cap H=\conj{y\in X\colon \mm(y,a)<r}$.
\par
Ahora los abiertos quienes son?\par
Note que si $G$ es abierto y $g\in G$ entonces existe $r_g>0$ tal que $B(g,r_g)\subseteq G\Rightarrow \cup_{g\in G}B(g,r_g)\subseteq G$. Luego si $G$ es abierto respecto a $(H,\mm_H)$ entonces
\begin{align*}
  G&=\cup_{\lambda\in A}B_H(x_\lambda,r_\lambda) \\
   &=\cup_{\lambda\in A}(B(x_\lambda,r_\lambda)\cap H)\\
   &=H\cap \left(\cup_{\lambda\in A}(B(x_\lambda,r_\lambda)\right)\\
   &= H\cap G_X
\end{align*}
Donde $G_X$ es un abierto pero en el espacio grande.

\begin{Lem}
  Todo abierto es unión de bolas.
\end{Lem}

\begin{Rmk}\label{rmk:topoEspMetrico}
  Más aún, un conjunto dentro de un espacio métrico es abierto si y sólo si es unión de bolas abiertas.
\end{Rmk}

\begin{ptcb}
\begin{enumerate}
  \item[$(\Rightarrow)$] Sea $A\subseteq X$ abierto y $a\in A$. Como $A$ es abierto, $\exists r_a>0\colon B(a,r_a)\subseteq A$. Para cualquier $a\in A$ se cumple, por lo que $\cup_{a\in A}B(a,r_a)\subseteq A$. La otra inclusión es inmediata pues por como fue definida, la unión tiene todos los puntos de $A$. Entonces $A$ es unión de bolas.
  \item[$(\Leftarrow)$] Tenemos que la unión de abiertos es un abierto. En este caso, las bolas son abiertas.
\end{enumerate}
\end{ptcb}

\begin{Lem}
  Todo abierto en $(H,\mm_H)$ es de la forma $G\cap H$ con $G$ abierto en $(X,\mm)$.
\end{Lem}

Queremos responder la misma pregunta para cerrados. Cómo son los cerrados en $(H,\mm_H)$?\par
Sea $F$ cerrado en $(H,\mm_H)$ entonces $H\setminus F$ es abierto en $(H,\mm_H)$. Luego, existe $G$ abierto en $(X,\mm)$ tal que $H\setminus F = G\cap H$. Tomando complementos tenemos que $F=H\setminus(G\cap H)= H\cap G^C$. O sea que un cerrado en $H$ es la intersección de un cerrado original con el espacio $H$.

\subsubsection*{Continuidad}

Para poder hablar de la noción de continuidad necesitamos dos espacios métricos.

\begin{Def}
  Sean $(X,\mm),(Y,\mm')$ dos espacios métricos y $f\colon X\to Y$. Se dice que $\lim_{x\to a}f(x)=\ell\in Y$ si para todo $\varepsilon>0$ existe $\delta>0$ tal que
  $0<\mm(a,x)<\delta\Rightarrow \mm'(\ell,f(x))<\varepsilon$. Además $f$ es continua en $a$ si $\lim_{x\to a}f(x)=f(a)$.
\end{Def}
 Observe que es necesario que $f$ esté definida en $a$ para hablar de continuidad. Ahora, una caracterización por bolas es la siguiente. Si $f$ es continua en $a$ entonces para todo $\varepsilon>0$ existe $\delta >0$ tal que $f\left(B(a,\delta)\right)\subseteq B(f(a),\varepsilon)$. Además $\lim_{x\to a} f(x)=\ell$ si y sólo si $f\left(B(a,\delta)\setminus\conj{a}\right)\subseteq B(\ell,\varepsilon)$.

 \begin{Ex}
   Dado $x_0\in X$ defina $f(x)=\mm(x,x_0)$, entonces $\mm(f(x),f(a))=|\mm(x,x_0)-\mm(a,x_0)|\leq \mm(x,a)$. De hecho $f$ es Lipschitz y por tanto es continua.
 \end{Ex}
%% A LEER: https://math.stackexchange.com/questions/370224/proving-something-is-1-lipschitz?utm_medium=organic&utm_source=google_rich_qa&utm_campaign=google_rich_qa

\begin{Ej}
  De igual forma si $A\subseteq X$ es un conjunto, $f(x)=\mm(x,A)$ es continua. Pruebe y utilice el hecho: $\mm(x,A)-\mm(y,A)\leq \mm(x,y)$.
\end{Ej}

%%A LEER https://math.stackexchange.com/questions/370224/proving-something-is-1-lipschitz?utm_medium=organic&utm_source=google_rich_qa&utm_campaign=google_rich_qa

\begin{ptcb}
Para ver que $\mm(x,A)\leq \mm(x,y)+\mm(y,A)$ vea que para cualquier $a\in A$ tenemos que
$$\mm(x,a)\leq \mm(x,y)+\mm(y,a)$$
Por definición de $\inf$ tenemos $\mm(x,A)\leq \mm(x,a)\leq\mm(x,y)+\mm(y,a) $. Pero entonces $\mm(x,A)-\mm(x,y)$ es cota inferior de $\mm(y,A)$. Nuevamente por definición de $\inf$ tenemos que $\mm(x,A)-\mm(x,y)\leq \mm(y,A)$. Esto prueba la desigualdad y por tanto procedemos por simetría.\par
Como tenemos que $\mm(x,A)\leq\mm(x,y)+\mm(y,A)$, podemos aplicar a $y$. Esto es $\mm(y,A)\leq\mm(x,y)+\mm(x,A)$. Así tenemos que
\begin{gather*}
  \mm(x,A)-\mm(y,A)\leq\mm(x,y)\\
  -(\mm(x,A)-\mm(y,A))=\mm(y,A)-\mm(x,A)\leq\mm(x,y)
\end{gather*}
Así $|\mm(x,A)-\mm(y,A)|\leq\mm(x,y)$, luego $f$ es 1-Lipschitz y por tanto continua.
\end{ptcb}
\begin{Def}
  Una función $f\colon X\to Y$ se dice Lipschitz si existe $\lambda >0$ tal que $\mm(f(x),f(y))\leq \lambda\mm(x,y)$. Toda función Lipschitz es continua.
\end{Def}

El siguiente teorema es la caracterización topológica de continuidad. \par
Sea $G\subseteq Y$ abierto y considere $f^{-1}(G)=\conj{x\in X\colon f(x)\in G}$. Tome $x_0\in f^{-1}(G)$, existe $r>0$ tal que $B(x_0,r)\subseteq f^{-1}(G)$? Es decir existe $r>0$ tal que $x\in B(x_0,r)$ entonces $f(x)\in G$?\par
Como $G$ es abierto, existe $\varepsilon>0$ tal que $B(f(x_0),\varepsilon)\subseteq G$. Luego si $f$ es continua, existe $\delta>0$ tal que $f(B(x_0,\delta))\subseteq B(f(x_0),\varepsilon)\subseteq G$.\par Las funciones continuas mandan abiertos de vuelta en abiertos, por imagen inversa.

\begin{Th}\label{thm:ContinuidadConTopo}
  Si $f\colon X\to Y$ es continua y $G\subseteq Y$ es abierto entonces $f^{-1}(G)$ es abierto en $X$.
\end{Th}

Observe que las imagenes directas no preservan esta propiedad. Es decir, si $G_1\subseteq X$ es abierto, no es siempre cierto que $f(G_1)$ es abierto.\par
De hecho el teorema anterior es una equivalencia.
\begin{Th}
  Sea $f\colon X\to Y$, son equivalentes:
  \begin{enumerate}
    \item $f$ es continua
    \item $f^{-1}(G)$ para todo $G\subseteq Y$ abierto.
    \item $f^{-1}(F)$ es cerrado para todo $F\subseteq Y$ cerrado.
    \item $f(\overline{B})\subseteq \overline{f(B)}$ para todo $B\subseteq X$. Donde la primera cerradura es respecto a $X$ y la segunda respecto a $Y$.
  \end{enumerate}
\end{Th}

\begin{ptcbp}
$(\mathit{2}\Rightarrow\mathit{1})$ Dado $\varepsilon>0$ y $x_0\in X$, $B(f(x_0),\varepsilon))\Rightarrow f^{-1}\lbrack B(f(x_0),\varepsilon)\rbrack$ es abierto. Así $\exists\delta>0$ tal que $B(x_0,\delta)\subseteq f^{-1}\lbrack B(f(x_0),\varepsilon)\rbrack\Rightarrow f(B(x_0,\delta))\subseteq B(f(x_0),\varepsilon)$

\end{ptcbp}

\begin{Ej}
  Mostar que $\mathit{4}$ es equivalente a alguna característica anterior.
\end{Ej}
%%https://math.stackexchange.com/questions/114462/a-map-is-continuous-if-and-only-if-for-every-set-the-image-of-closure-is-contai?utm_medium=organic&utm_source=google_rich_qa&utm_campaign=google_rich_qa
\begin{ptcb}
Vamos a mostrar que $\mathit{2}.\Rightarrow\mathit{4}.$ y $\mathit{4}.\Rightarrow\mathit{3}.$
\begin{enumerate}
  \item[$(\Rightarrow)$] Como $f$ es continua manda abiertos de vuelta en abiertos,  $f^{-1}(Y\setminus \overline{f(B)})$ es abierto de $X$. Además $X\setminus f^{-1}(\overline{f(B)})=f^{-1}(Y\setminus \overline{f(B)})$. De esta manera $f^{-1}(\overline{f(B)})$ es cerrado. Ahora considere la siguiente cadena de inclusiones
      $$A\subseteq f^{-1}(f(B))\subseteq f^{-1}(\overline{f(B)})$$
      Lo que nos dice que $\overline{B}\subseteq f^{-1}(\overline{f(B)})$ y por lo tanto
      $$f(\overline{B})\subseteq f(f^{-1}(\overline{f(B)}))\subseteq\overline{f(B)} $$
  \item[$(\Leftarrow)$] Sea $F\subseteq Y$ cerrado. Tenemos que
  $$f(\overline{f^{-1}(F)}) \subseteq \overline{f(f^{-1}(F))} \subseteq \overline{F} = F$$
  $$\Rightarrow f(\overline{f^{-1}(F)}) \subseteq F$$
  Esto significa que $\overline{f^{-1}(F)}\subseteq f^{-1}(F)$ y por lo tanto $f$ manda cerrados de vuelta en cerrados.
  %Ahora sea $V\subseteq Y$ un abierto y considere $X\setminus f^{-1}(V)$. Queremos ver que tal conjunto es cerrado, o sea $X\setminus f^{-1}(V)=\overline{X\setminus f^{-1}(V)}$. Por hipótesis $f(\overline{X\setminus f^{-1}(V)})\subseteq \overline{f(X\setminus f^{-1}(V))}$, pero $X\setminus f^{-1}(V)=f^{-1}(Y\setminus V)\Rightarrow f(X\setminus f^{-1}(V))\subseteq Y\setminus V$ que es conjunto cerrado.
\end{enumerate}
\end{ptcb}
\begin{Th}
  Sea $f\colon X\to Y$, $f$ es continua en $a$ si y sólo si para cualquier sucesión $(x_n)_{n\in\bN}\subseteq X$ tal que $x_n\to a$ se tiene que $f(x_n)\to f(a)$.
\end{Th}

\begin{ptcbp}
Considere $(f(x_n))_{n\in\bN}$, dado $\varepsilon>0$ hay que mostrar que existe $n_0$ tal que $\mm(f(x_0),f(a))<\varepsilon$ para todo $n\geq n_0$. Sabemos que para todo $\eta>0$ existe $m_0$ tal que $\mm(x_n,a)<\eta$ si $n\geq m_0$. Además existe $\eta_0$ tal que $\mm(x_n,a)<\eta_0 \Rightarrow \mm(f(x_n),f(a))<\varepsilon$. Luego si $n\geq m_0$, entonces $\mm(x_n,a)<\eta_0 \Rightarrow \mm(f(x_n),f(a))<\varepsilon$.\par
Por otro lado asuma que $f$ no es continua, o sea existe $\varepsilon>0$ tal que para todo $\delta$ existe un $x_\delta$ que cumple $\mm(x_\delta,a)<\delta$ y $\mm(f(x_0),f(a))\geq\varepsilon$. En particular si $\delta =\frac{1}{n}$ existe $x_n$ tal que $\mm(x_n,a)<\frac{1}{n}$ y $\mm(f(x_n),f(a))\geq \varepsilon$. Esto nos da una contradicción porque tenemos $f(x_n)$ no converge a $f(a)$.
\end{ptcbp}

Verificar continuidad es lo mismo que verificarlo para sucesiones. Recuerde además que la sucesión no puede ser escogida, debe ser arbitraria.\par
La siguiente definición nos va a servir en algún momento.
\begin{Def}
  Una función $f\colon X\to Y$ es un homeomorfismo si $f$ es biyectiva y $f$ y su inversa son continuas.
\end{Def}

Estas funciones preservan abiertos. Sin embargo, estas funciones no preservan bolas.

\begin{Def}
   Una función $f\colon X\to Y$ es una isometría si es sobreyectiva y $\mm'(f(x),f(y))=\mm(x,y)$.
\end{Def}

Será cierto que las isometrías son inyectivas? En efecto, por un argumento de kernel se obtiene el resultado.
%%https://math.stackexchange.com/questions/120539/every-isometry-is-a-homeomorphism?utm_medium=organic&utm_source=google_rich_qa&utm_campaign=google_rich_qa
\begin{Ej}
  Una isometría es un homeomorfismo.
\end{Ej}

\begin{ptcb}
Primero corroboramos que una isometría es inyectiva. Sea $f$ una isometría con $f(a)=f(b)$ entonces $\mm'(f(a),f(b))=0$. Tenemos que $\mm(a,b)=0$, pues $f$ es una isometría, y por lo tanto $a=b$.\par
Ahora vemos que $f$ es continuas. Pero esto es inmediato, suponga que $\mm(a,b)<\varepsilon$ entonces $\mm'(f(a),f(b))<\varepsilon$ para cualquier $\varepsilon>0$.\par
Entonces nos falta ver que la inversa de $f$ preserva distancias. Como $f$ es una isometría $\mm(a,b)=\mm(f(a),f(b))$ para $a,b\in X$. Como $f$ es biyectiva, existen $c,d\in Y$ tales que $a=f^{-1}(c), b=f^{-1}(d)$. De la igualdad anterior tenemos $\mm(f^{-1}(c),f^{-1}(d))=\mm(f(f^{-1}(c)),f(f^{-1}(d)))=\mm(c,d)$. Así $f^{-1}$ también es una isometría que por lo tanto es continua. \par
Concluímos que una isometría es un homeomorfismo.
\end{ptcb}

\subsubsection*{Compacidad}

Sea $f\colon\bonj{a,b}\to \bR$ continua. Qué propiedades tiene esta función? Es uniformemente continua, alcanza puntos extremos, manda intervalos en intervalos. Estos intervalos son los compactos de $\bR$.

\begin{Def}\label{def:compactoSuccs}
  Un conjunto $C\subseteq X$ es compacto si toda sucesión $(x_n)_{n\in\bN}\subseteq C$ tiene una subsucesión $(x_{n_k})_{k\in\bN}$ tal que $x_{n_k}\to c\in C$.
\end{Def}

Bajo esta definición todo compacto es cerrado pues todos los puntos de adherencia son límites de sucesiones y no hay otro lugar que caer para los puntos de adherencia que en $C$.

Note que si $x_{n_k}\to c$ entonces $c$ es un punto de adherencia de $(x_n)_{n\in\bN}$.
Ahora tome $c\in\overline{(x_n)_{n\in\bN}}$, tome $n_2> n_1$ y $x_{n_1}\in B(c,1)\cap(x_n)_{n\in\bN}$. En general $x_{n_k}\in B(c,\frac{1}{k})\cap(x_n)_{n\in\bN}$, esta sucesión converge a $c$. Ahora yo quiero que mi sucesión conste de puntos distintos. Si queremos que $x_{n_1}\neq x_{n_2}$, tome $x_{n_2}\in B(c,r)$ con $r<\min\conj{\frac{1}{2},\mm(x_{n_1},c)}$ entonces $x_{n_1}\neq x_{n_2}$.\par
En general, cuando $n_k>n_{k-1}$, $x_{n_k}\in B(c,r)\cap (x_n)_{n\in\bN}$ con $r <\min\conj{\frac{1}{k},\mm(c,x_{n_1}),\cdots,\mm(c,x_{n_{k-1}})}$.

\begin{Def}
  Decimos que una colección $\mathcal{U}=\conj{U_\alpha\colon \alpha\in A}$ de abiertos de $X$ cubre $C$ si $C\subseteq\cup_{\alpha\in A}U_\alpha$.
\end{Def}

\begin{Lem}
  Sea $C\subseteq X$ compacto y $\cU$ un cubrimiento de $C$. Entonces existe $\varepsilon >0$ tal que para todo $x\in C$ existe $\alpha\in A$ que satisface $B(x,\varepsilon)\subseteq U_\alpha$.
\end{Lem}
Aquí el $\varepsilon$ nos sirve para todos $U_\alpha$'s.
\begin{ptcbp}
Asuma que el resultado es falso, o sea que para todo epsilon puedo encontrar un x tal que $B(x,\varepsilon)$ no está contenido en todos los $U_\alpha$. Van a haber pedazos en $U_\alpha$ pero no completamente. \par
Mejor dicho, para todo $n$ existe $x_n\in C$ tal que $B(x_n,\frac{1}{n})\subsetneq U_\alpha$ para todo $\alpha\in A$. Sabemos que existe $x_{n_k}\to c_0\in C$. Existe un $\alpha$ tal que $c_0\in U_\alpha$.\par
Todavía no hemos usado la hipótesis acerca de $\cU$ como cubrimiento. Vea que como $U_\alpha$ es abierto podemos tomar $\varepsilon > 0$ tal que $B(c_0,\varepsilon)\subseteq U_\alpha$. Basta ver que las bolas anteriores las podemos meter en esta nueva bola para obtener una contradicción, pues según nuestra hipótesis por contradicción no podemos meter las bolas anteriores en un $U_\alpha$.\par
Sea $n_0$ tal que $\mm(x_n,c_0)<\frac{\varepsilon}{2}$ y $\frac{1}{n}<\frac{\varepsilon}{2}$ para todo $n\geq n_0$. Entonces $B(x_n,\frac{1}{n})\subseteq B(c_0,\varepsilon)$ Esto pues $y\in B(x_n,\frac{1}{n})\Rightarrow \mm(y,c_0)\leq \mm(y,x_n)+\mm(x_n,c_0)<\frac{1}{n}+\frac{\varepsilon}{2}<\varepsilon$.
\end{ptcbp}

\subsection{Día 4| 22-3-18}

\subsubsection*{Primera sesión de ejercicios}

\begin{Ej}[2.2.11.1.b Santiago Cambronero]
  Sea $A=\conj{(x,e^x)\colon x\in\bQ}$. Es $A$ acotado? Cerrado? Abierto? Encuentre $A^o, \overline{A}$ y $\partial(A)$.
\end{Ej}
\begin{ptcb}
Tome $(x_n)_{n\in\bN}\subseteq\bR\setminus\bQ$ y $x_n\to a$ entonces como la exponencial es continua $e^{x_n}\to e^a$. Así $(x_n,e^{x_n})\to (a,e^a)$.
\end{ptcb}

\begin{Ej}[2.2.11.7. Santiago Cambronero]
  Mostrar que $\overline{A}$ es el menor cerrado que contiene a $A$
\end{Ej}
\begin{ptcb}
Tome $B$ un cerrado tal que $A\subseteq B\subseteq \overline{A}$. Así $B^C\subseteq A^C$, existe $x\in B^C$ tal que $x\in\overline{A}$. Como $B^C$ es abierto entonces existe $r>0$ tal que $B(x,r)\subseteq B^C$. Esto implica que esta bola está contenida en $A^C$, sin embargo $x\in\overline{A}$. Eso implica que para todo $r_1$ si tomamos la bola $B(x,r_1)$ pescamos a alguien en $A$. Esto es contradictorio pues la bola está completamente contenida en el complemento.
\end{ptcb}

\begin{ptcb}
Estar en $\overline{A}$ significa que existe una sucesión de $A$ que converge al punto. En otras palabras $\exists(x_n)_{n\in\bN}\subseteq A\colon x_n\to x$. Esa misma sucesión está contenida en $B$ y como $B$ es cerrado, es igual a su clausura. Esto nos dice que $x\in B$ y por tanto $\overline{A}\subseteq B$.
\end{ptcb}

\begin{Ej}[2.2.11.7.a Santiago Cambronero]
  Si $A\subseteq B$ entonces $\overline{A}\subseteq\overline{B}$
\end{Ej}

\begin{ptcb}
Tenemos que $A\subseteq \overline{B}$. Como $\overline{B}$ es cerrado, luego $\overline{A}$ es el menor cerrado bajo inclusión que contiene a $A$. Entonces $\overline{A}\subseteq \overline{B}$.
\end{ptcb}

\begin{Ej}[2.2.11.7.b Santiago Cambronero]
 Mostrar $\overline{A\cup B}=\overline{A}\cup\overline{B}$
\end{Ej}

\begin{ptcb}
\begin{enumerate}
  \item[$``\supseteq''$] Como $\overline{A}\cup\overline{B}$ es cerrado entonces $A\cup B\subseteq \overline{A}\cup\overline{B} \Rightarrow \overline{A\cup B}\subseteq \overline{A}\cup\overline{B}$.
  \item[$``\subseteq''$] Si $x\in\overline{A\cup B}$ existe $(x_n)_{n\in\bN}\subseteq A\cup B$ tal que $x_n\to x$. Entonces la sucesión está en uno de los conjuntos. Esto significa que el límite está en uno de los dos conjuntos.
\end{enumerate}
\end{ptcb}

\begin{ptcb}
Por la definición de cerradura, como $\overline{A\cup B}$ es cerrado y $A,B\subseteq A\cup B\subseteq\overline{A\cup B}$. Entonces sus cerraduras están contenidas en la cerradura de la unión.
\end{ptcb}

\begin{Ej}[2.2.11.7.c Santiago Cambronero]
 Mostrar $\overline{A\cap B}\subseteq\overline{A}\cap\overline{B}$
\end{Ej}

\begin{ptcb}
Como $\overline{A}\cap\overline{B}$ es cerrado tenemos que $A\cap B\subseteq \overline{A}\cap\overline{B}$. Por definición de cerradura $\overline{A\cap B}\subseteq\overline{A}\cap\overline{B}$.
\end{ptcb}

\begin{Ej}[2.2.11.12 Santiago Cambronero]
 Mostrar $A\subseteq\bR$ es abierto $\iff A=\cup_i I_i$ con $I_i$ intervalos abiertos. La unión es disjunta.
\end{Ej}

\begin{ptcb}
\begin{enumerate}
  \item[$(\Leftarrow)$] $A$ sería unión de abiertos, entonces $A$ es abierto.
  \item[$(\Rightarrow)$]
 % Si $A$ es un abierto arbitrario, para $x\in\bR$ defina $I_x=\cup_{x\in I\subseteq U}I$ con $I$ intervalos abiertos. Para $x\in A$ existe $r>0$ tal que $\obonj{x-r,x+r}\subseteq A$. Además si $y$ es un racional que está en $\obonj{x-r,x+r}$ entonces $\obonj{x-r,x+r}\subseteq I_y$. Así $x\in I_y$, y como $x$ es arbitrario, cualquier $x\in A$ está en $I_q$ para $q\in A\cap\bQ$. Así $A\subseteq \cup_{q\in A\cap\bQ} I_q$. Pero $I_q\subseteq A$ para $q\in A\cap\bQ$. Entonces $A=\cup_{q\in A\cap\bQ} I_q$. Los intervalos $I_q$ son disjuntos por nuestra definición, ya que si $x\in I_p\cap I_q$ entonces la unión está metida en ambos. Así si $I_q$ es distinto a $I_p$ entonces la unión es disjunta.
\end{enumerate}
\end{ptcb}


\begin{Ej}[2.2.11.16.a Santiago Cambronero]
 Si $A$ es abierto, entonces $A\subseteq(\overline{A})^o$. Encuentre un ejemplo donde la inclusión es estricta.
\end{Ej}

\begin{ptcb}
Tome $a\in A$ entonces existe $r>0$ tal que $B(x,r)\subseteq A\subseteq\overline{A}$. Entonces $a\in (\overline{A})^o$. \par
A manera de ejemplo considere $A=\obonj{a,b}\cup\obonj{b,c}$. Entonces $\overline{A}=\bonj{a,c}$ y $(\overline{A})^o=\obonj{a,c}$.
\end{ptcb}

Otra forma usando la definición topológica del interior.
\begin{ptcb}
%Si A es abierto, al clausurarlo A barra interior es el mayor abierto contenido en A barra y como A está contenido en A barra debe estar contendio en el interior.
Si $A$ es abierto, entonces $(\overline{A})^o$ es el mayor abierto contenido en $\overline{A}$. Además $A\subseteq\overline{A}$ y como $A$ es abierto se sigue que $A\subseteq(\overline{A})^o$.
\end{ptcb}


\begin{Ej}[2.2.11.16.b Santiago Cambronero]
 Si $A$ es cerrado, entonces $\overline{(A^o)}\subseteq A$.
\end{Ej}

\begin{ptcb}
Sea $(x_n)_{n\in\bN}\subseteq A^o$ tal que $x_n\to x$. Esto significa que $x\in \overline{(A^o)}$ y así $x\in A$
\end{ptcb}

Parte c
\begin{ptcb}
beta A barra subset A barra, o sea barra alpha A = beta A barra subset A barra
\end{ptcb}

\subsection{Día 5| 3-4-18}

Retomando un ejemplo anterior, para $c\in(x_n)_{n\in\bN}$ nos interesa lo que pasa en la cola. El comportamiento de esta sucesión se determina no por una cantidad finita de puntos sino por los últimos infinitos puntos. Tome $x_{n_1}\in B(c,1)$ y $\varepsilon_2=\min\conj{\frac{1}{2},\mm(c,x_{n_1})}$. Sea $x_{n_2}\in B(c,\varepsilon_2)$. \par
Lo primero que hay que hacer es asumir que el conjunto $\conj{x_n\colon n\in\bN}$ sea infinito. Si fuera finito, la sucesión sólo toma 10 puntos. Es agarrar sucesiones constantes y cada una de estas nos da un punto de acumulación. No aproximándolo, sino repitiéndolo. Llega un momento donde puedo hacer el segundo paso finitas veces, el de escoger el $\varepsilon_2$.\par
Bajo esta nueva hipótesis con $x_{n_1}\neq c$, sea $n_2>n_1$ tal que $x_{n_2}\in B(c,\varepsilon_2)$. Iterando el proceso existe $n_k>n_{k-1}>\cdots>n_1$ tal que $x_{n_k}\in B(c,\varepsilon_k)$ con $\varepsilon_k=\min\conj{\frac{1}{k},\mm(c,x_{n_1}),\cdots,\mm(c,x_{n_k})}$. Esto me asegura el proceso de escoger una sucesión donde todos los elementos son distintos.\par
Recordamos la definición \ref{def:compactoSuccs} de conjuntos compactos.\par

\begin{nonum-Def}
  Un conjunto $C$ se dice compacto si dada $(x_n)_{n\in\bN}\subseteq C$ existe $(x_{n_k})_{k\in\bN}\subseteq C$ que converge a un punto de $C$.
\end{nonum-Def}
Por la construcción tenemos la siguiente equivalencia.

\begin{Lem}
  Un conjunto $C$ es compacto si y sólo si para cualquier $(x_n)_{n\in\bN}\subseteq C$ se cumple $\cap_{k\in\bN}\overline{\conj{x_n\colon n>k}}\neq\emptyset$.
\end{Lem}

\begin{Lem}
  Sea $C$ un conjunto compacto y $\cU=\conj{U_\alpha\colon\alpha\in A}$ un cubrimiento por abiertos de $C$. Entonces existe $\varepsilon>0$ tal que para todo $x\in C$ existe $\alpha_0\in A$ tal que $B(x,\varepsilon)\subseteq U_{\alpha_0}$.
\end{Lem}

%Uno de los probabilistas más importantes de la historia se llamaba Donald Von Holder. Sólo publicaba artículos muy importantes. Durante una conferencia con gente joven un estudiante le pregunta "profesor, cómo hace para mantenerse activo?". Donald responde "muy sencillo, después de almuerzo me voy a caminar por una hora." L amoraleja es, hacer algo una hora para llegar enteros hasta la vejez.

\begin{Th}\label{thm:BolzWeirIffCubrimientos}
  Sea $C\subseteq X$, entonces $C$ es compacto si y sólo si para todo cubrimiento por abiertos $\cU=\conj{U_\alpha\colon \alpha\in A}$ de $C$ existen $(\alpha_i)_{i\in\bonj{m}}\subseteq A$ tal que $C\subseteq\cup_{i\in\bonj{m}}U_{\alpha_i}$.
\end{Th}

\begin{ptcbp}
\begin{enumerate}
  \item[$(\Rightarrow)$] Sea $\cU=\conj{U_\alpha\colon \alpha\in A}$ tal que $C\subseteq\cup_{\alpha\in A}U_{\alpha}$. Sea $\varepsilon>0$ según el lema anterior. Tome $x_1\in C$, así existe $\alpha_1$ tal que $B(x,\varepsilon)\subseteq U_{\alpha_1}$. \par
      Si $C\subseteq U_{\alpha_1}$, estamos listos. De lo contrario tome $x_2\in C\setminus U_{\alpha_1}\subseteq C\setminus B(x_1,\varepsilon)$. Al escogerlo fuera de $U_{\alpha_1}$ ya sé que la distancia $\mm(x_1,x_2)$ es mayor a $\varepsilon$. Tome $\alpha_2$ tal que $B(x_2,\varepsilon)\subseteq U_{\alpha_2}$. En el caso de que $C\subseteq U_{\alpha_1}\cup U_{\alpha_2}$, tenemos el resultado.\par
      De lo contrario existe $x_3\in C\setminus(U_{\alpha_1}\cup U_{\alpha_2})$. En otras palabras $\mm(x_3,x_1)\geq\varepsilon, \mm(x_3,x_2)\geq\varepsilon, \mm(x_2,x_1)\geq\varepsilon$. Esto contradice que esta sucesión sea convergente, contradiciendo la condición de Cauchy.\par
      Si iteramos este proceso, podemos encontrar $x_{k+1}\in C\setminus(\cup_{i\in\bonj{k}}U_{\alpha_i})$ y $\mm(x_i,x_j)\geq\varepsilon$ para $i,j\in\bonj{k+1}, i\neq j$. Si este proceso acaba, significa que existe $m$ tal que $C\subseteq\cup_{i\in\bonj{m}}U_{\alpha_i}$. En el caso contrario construimos una sucesión $(x_n)_{n\in\bN}\subseteq C$ tal que $\mm(x_i,x_j)\geq \varepsilon$ si $i\neq j$ y esto es una contradicción.\par
      Esto pues $\varepsilon\leq\mm(x_i,x_j)\leq \mm(x_i,c)+\mm(x_j,c)$ lo que nos dice que no pueden haber subsucesiones convergentes, contradiciendo la definición de compacidad.
  \item[$(\Leftarrow)$] Asuma que $(x_n)_{n\in\bN}\subseteq C$ no tiene una subsucesión convergente. Defina $U_k=X\setminus (\overline{\conj{x_n\colon n\geq k}})$. Note que $C\subseteq\cup_{k\in\bN}U_k=X\setminus (\cap_{k\in\bN}\overline{\conj{x_n\colon n\geq k}})$.
      %Para este punto ya está la prueba
      y este conjunto es $X$. Luego, por compacidad, existen $k_1<\cdots<k_m$ tales que $C\subseteq\cup_{i\in\bonj{m}}U_{k_i}$. Al unir todos, como van creciendo, es lo mismo que poner el más grande. O sea $C\subseteq U_{k_m}=X\setminus (\overline{\conj{x_n\colon n\geq k_m}})$ y esto es una contradicción pues $x_n\in C$.
\end{enumerate}
\end{ptcbp}

En los textos se toma la siguiente deinición de compacidad.
\begin{Def}\label{def:compactoCubrimientos}
  Dado $\cU=\conj{\widetilde{U}_\alpha\colon \alpha\in A}$ con $\widetilde{U}_\alpha$ abierto respecto a $C$. Diremos que $C$ es compacto si existen $\alpha_1,\cdots,\alpha_n$ tal que $C=\cup_{i\in\bonj{m}}\widetilde{U}_{\alpha_i}$.
\end{Def}

Recuerde que como $\widetilde{U}_\alpha$ es abierto respecto a $C$, existe $U_\alpha$ abierto en $X$ tal que $\widetilde{U}_\alpha=U_\alpha\cap C$.
\begin{Ej}
  La definición anterior coincide con la definición \ref{def:compactoSuccs}.
\end{Ej}

\begin{ptcb}
Leer Munkres 203
\end{ptcb}
Veamos en $\bR$, cuáles son los conjuntos compactos? En $\bonj{0,1}$ una sucesión dentro de este conjunto tiene una subsucesión convergente por Bolzano-Weierstra{\ss}.\par
Primero veremos un resultado, las funciones continuas mandan compactos en compactos.

\begin{ptcbp}
Sea $f\colon X\to Y$ continua y $K\subseteq X$ compacto. Defina $K_1=f(K)$. Sea $\cU=\conj{U_\alpha\colon\alpha\in A}$ un cubrimiento de $K$. Así $K_1\subseteq\cup_{\alpha\in A}\Rightarrow f^{-1}(K_1)\subseteq\cup_{\alpha\in A}f^{-1}(U_\alpha)$. \par
\begin{ptcb}
Recuerde que $c\in f^{-1}(f(K))\iff f(c)\in f(K)\Rightarrow k\in K$ entonces $f(k)\in f(K)\Rightarrow K\subseteq f^{-1}(f(K))$.
\end{ptcb}
Esto nos dice que $K\subseteq f^{-1}(f(K))\subseteq\cup_{\alpha\in A}f^{-1}(U_\alpha)$. Como $K$ es compacto, existen $\alpha_1,\cdots,\alpha_{m}$ tales que $K\subseteq\cup_{i\in\bonj{m}}f^{-1}(U_{\alpha_i})\Rightarrow f(K)\subseteq\cup_{i\in\bonj{m}}f(f^{-1}(U_{\alpha_i}))$. Entonces $f(K)=K_1$ y $\cup_{i\in\bonj{m}}f(f^{-1}(U_{\alpha_i}))\subseteq\cup_{i\in\bonj{m}}U_{\alpha_i}$ nos da el resultado.
\end{ptcbp}
\begin{Ej}
  Sea $f\colon X\to Y$ y $\tilde{K}\subseteq X, K\subseteq Y$. Pruebe que $\tilde{K}\subseteq f^{-1}(f(\tilde{K}))$ y $f(f^{-1}(K))\subseteq K$. Se cumple la igualdad cuando $f$ es inyectiva y sobreyectiva respectivamente.
\end{Ej}
%No todos los elementos de $K$ tienen que ser imagenes de alguien.
\begin{Th}\label{thm:intervalosEncajados}
  Dado $(X,\mm)$ un espacio métrico. Las siguientes aseveraciones son equivalentes.
  \begin{enumerate}
    \item $X$ es compacto.
    \item Dados $\conj{F_{\alpha}\colon\alpha\in A}$ con $F_\alpha$ cerrados tal que para todos $\alpha_1,\cdots,\alpha_m$ se tiene $\cap_{i\in\bonj{m}}F_{\alpha_i}\neq\emptyset\Rightarrow\cap_{\alpha\in A}F_{\alpha}\neq\emptyset$.
  \end{enumerate}
\end{Th}

\begin{ptcbp}
Probamos $\mathit{1}.\Rightarrow\mathit{2.}$ por contradicción. Suponga que la conclusión es falso o sea existe $\conj{F_\alpha\colon\alpha\in A}$ tal que $\cap_{\alpha\in A}F_\alpha=\emptyset\Rightarrow\cup_{\alpha\in A}X\setminus F_\alpha= X$. Luego $\conj{X\setminus F_\alpha\colon\alpha\in A}$ es un cubrimiento de $X$. Así existen $\alpha_1,\cdots,\alpha_m$ tal que $X=\cup_{i\in\bonj{m}}X\setminus F_{\alpha_i}\Rightarrow\cap_{i\in\bonj{m}}F_{\alpha_i}=\emptyset$. Esto es una contradicción. \textcolor{red}{Por qué?}
\end{ptcbp}

\begin{Def}
  Un espacio $(X,\mm)$ es acotado si para todo $x_0\in X$ existe $r>0$ tal que $B(x_0,r)\supseteq X$. O equivalentemente, existe $M$ tal que $\mm(x,y)\leq M$ para todos los $x,y\in X$.\par
  De manera análoga $B\subseteq X$ es acotado si $(B,\mm_B)$ es acotado. O sea para cualquier $x_0\in X$ existe $r>0$ tal que $B\subseteq B(x_0,r)$.
\end{Def}
Será que un conjunto compacto es acotado? Si $C$ es compacto $C\subseteq\cup_{n\in\bN}B(x_0,n)$. Por compacidad existen $n_1<n_2<\cdots<n_m$ tales que $C\subseteq\cup_{n\in\bonj{m}}B(x_0,n)=B(x_0,n_m)$.

\begin{Lem}
  Si $C\subseteq X$ es compacto, entonces $C$ es cerrado y acotado.
\end{Lem}

%bigtimes adelante
Considere $C=\bigtimes_{i\in\bonj{d}}\bonj{a_i,b_i}\subseteq\bR^d$ con la métrica euclídea. Si $C$ no es compacto, existe $\cU=\conj{U_\alpha\colon\alpha\in A}$ tal que $C\subsetneq\cup_{i\in\bonj{m}}U_{\alpha_i}$ para todo $\alpha_1,\cdots,\alpha_m$.\par
INSERTAR FIG5.1
\par
Divida el rectángulo en $2^d$ rectángulos de la forma $\bigtimes_{j\in\bonj{d}}\bonj{c^i_j,d^i_j}$, $i\in\bonj{2^d}$. Donde $\bonj{c^i_j,d^i_j}$ es de la forma $\bonj{a_j,\frac{a_i+b_i}{2}}$ ó $\bonj{\frac{a_i+b_i}{2},b_j}$. Por hipótesis existe un $i_0\in\bonj{2^d}$ tal que $C_1=\bigtimes_{j\in\bonj{d}}\bonj{c_j^{i_0},d_j^{i_0}}$ tal que no puede ser cubierto por una cantidad finita de $U_\alpha$'s.
Iterando el proceso $C_k=\bigtimes_{j\in\bonj{d}}\bonj{c_j^k,d_j^k}$ con $\bonj{c_j^k,d_j^k}\supseteq\bonj{c_j^{k+1},d_j^{k+1}}$ y $|d_j^k-c_j^k|=\frac{b_j-a_j}{2^k}$. Por el teorema \ref{thm:intervalosEncajados} de los intervalos encajados existe $z_j=\cap_{k\in\bN}\bonj{c_j^k,d_j^k}\Rightarrow \cap_{k\in\bN}C_k=\conj{(z_1,\cdots,z_{algo})}$. Como $z\in C$, existe $\beta\in A\colon z\in U_\beta$. Como $U_\beta$ es abierto, existe $\varepsilon>0$ tal que $B(z,\varepsilon)\subseteq U_\beta$\par
INSERTAR FIG5.2
\par
Si $\frac{\sqrt{d}\sup_{i\in\bonj{d}}|b_i-a_i|}{2^k}<\varepsilon$ entonces $C_k\subseteq B(z,\varepsilon)\subseteq U_\beta$.

\begin{Lem}
  Si $C_1\subseteq C_2$ con $C_2$ compacto y $C_1$ cerrado entonces $C_1$ es compacto.
\end{Lem}

\begin{Ej}
  Muestre el lema anterior.
\end{Ej}

\begin{ptcb}
En efecto, sea $\cU$ un cubrimiento por abiertos de $C_1$. Por topología de subespacio, todo abierto en $\cU$ es de la forma $U\cap C_1$ con $U$ abierto en $C_2$. Tome $$\tilde{\cU}=\conj{U\subseteq C_2\colon U\,\,\,\text{es abierto y}\exists \tilde{U}\in\cU\colon U\cap C_1=\tilde{U}}$$Observe que $\tilde{\cU}$ es un cubrimiento por abiertos de $C_1$. A su vez $C_2\setminus C_1$ es abierto y asi $\tilde{\cU}\cup\conj{C_2\setminus C_1}$ es un cubrimiento por abiertos de $C_2$. Ahora como $C_2$ es compacto podemos extraer un subcubrimiento finito que también cubre $C_1$. Por lo tanto, como el cubrimiento era arbitrario, se sigue que $C_1$ es compacto.
\end{ptcb}

Como $\bigtimes_{i\in\bonj{d}}\conj{a_i,b_i}$ es compacto,  si $C$ es cerrado y acotado existe $n$ tal que $C\subseteq\bigtimes_{i\in\bonj{d}}\bonj{-n,n}$ tenemos que $C$ es compacto. Con esto probamos el teorema de Heine-Borel.
\begin{Th}[Heine-Borel]\label{thm:HeineBorel}
  Sea $C\subseteq\bR^d$ con la métrica euclídea. Entonces $C$ es compacto si y sólo si $C$ es cerrado y acotado.
\end{Th}

\subsection{Día 6| 5-4-18}

\subsubsection*{Completitud}

\begin{Def}
  Sea $(X,\mm)$ un espacio métrico y $(x_n)_{n\in\bN}\subseteq X$. Decimos que la sucesión es de Cauchy si para todo $\varepsilon>0$ existe $n_0$ tal que $\mm{x_n,x_m}<\varepsilon$ cuando $n,m\geq n_0$.
\end{Def}

Ahora al igual que en $\bR$ se cumple que:

\begin{Lem}
  Toda sucesión convergente es de Cauchy y toda sucesión de Cauchy es acotada.
\end{Lem}

\begin{Def}
  Decimos que $(X,\mm)$ es completo si toda sucesión de Cauchy en $X$ es convergente en $X$.
\end{Def}

\begin{Ex}
  Sea $X =\conj{f\colon\bonj{a,b}\to\bR\colon f\text{ es continua}}$ con la métrica $\mm_{\infty}=\sup_{x\in\bonj{a,b}}\conj{|f(x)-g(x)|}=\nm{f-g}_{\infty}$. Sea $(f_n)_{n\in\bN}$ de Cauchy. \par
  Dado $\varepsilon>0$ existe $n_0$ tal que $\nm{f_n-f_m}_{\infty}\leq\varepsilon$, ahora con $x\in\bonj{a,b}$, $|f_n(x)-f_m(x)|\leq \nm{f_n-f_m}<\varepsilon$ si $m,n\geq n_0$. Defina $f=\lim_{n\to\infty} f_n$, si $x\in\bonj{a,b}$ existe $n_1$ tal que $|f(x)-f_m(x)|<\varepsilon$ para $m\geq n$. Note que
  $$|f(x)-f_m(x)|\leq |f(x)-f_n(x)|+|f_m(x)-f_n(x)|$$
  El primer término es pequeño pues $f_n\to f$ puntualmente y el segundo término pues $(f_n)_{n\in\bN}$ es Cauchy, esto nos permite empequeñecer independiente del $x$. \par
  De esta manera, si $m,n\geq n_0$ y $m,n\geq n_1$ entonces $|f(x)-f_n(x)|\leq 2\varepsilon$ para $n\geq\max\conj{n_0,n_1}$. Esto prueba coonvergencia uniforme, lo que implica que $f$ además es continua.
\end{Ex}

Considere un subconjunto cerrado $C$ de un espacio completo $X$. Será cierto que toda sucesión convergente en $C$ converge en $C$?\par
Sea $(x_n)_{n\in\bN}\subseteq C\subseteq X$, como $X$ es completo tenemos que $x_n\to y\in X$. Como $C$ es cerrado, se tiene que $y$ está en $C$, toda sucesión convergente en $C$ converge a un límite dentro de $C$.\par
Esto prueba el lema siguiente:

\begin{Lem}
  Sea $(X,\mm)$ un espacio completo, $C\subseteq X$ con $C$ cerrado. Entonces $C$ con métrica inducida es un espacio completo. Además si $(C,\mm_C)$ es completo, entonces es cerrado en $X$.
\end{Lem}

\begin{ptcbp}
Suponga que $C$ es completo, entonces toda sucesión de Cauchy es convergente dentro $C$. De esta manera toda sucesión convergente es $C$ converge dentro de $C$ que es la definición de ser cerrado.
\end{ptcbp}

Habrá alguna relación entre completitud y compacidad?\par
Vea que $\bR$ es completo pero no es compacto. Por otra parte, si $X$ es compacto y $(x_n)_{n\in\bN}\subseteq X$ es de Cauchy existe $(x_{n_k})_{k\in\bN}\subseteq(x_n)_{n\in\bN}$ que converge. O sea $x_{n_k}\to y\in X$, entonces $\mm(x_n,y)\leq \mm(x_n,x_{n_k})+\mm(x_{n_k},y)$. Esto nos dice que $\mm(x_n,y)\to 0$ y por tanto $(x_n)_{n\in\bN}$ es convergente.

\begin{Lem}
  Sea $(X,\mm)$ un espacio compacto, entonces $(X,\mm)$ es completo.
\end{Lem}

\begin{Ej}
  Completar los detalles de la prueba anterior.
\end{Ej}

\begin{ptcb}
En efecto, sea $(x_n)_{n\in\bN}\subseteq X$ una sucesión de Cauchy. Como $X$ es compacto, por definición \ref{def:compactoSuccs}, $\exists(x_{n_k})_{k\in\bN}\subseteq (x_n)_{n\in\bN}$ tal que $x_{n_k}\xrightarrow[k\to\infty]{}\ell\in X$.\par
Sea $\varepsilon>0$, entonces como $(x_n)_{n\in\bN}$ es de Cauchy existe $M\in\bN$ tal que $\mm(x_n,x_m)<\frac{\varepsilon}{2}$ cuando $m,n>M$. Además como $x_{n_k}\xrightarrow[k\to\infty]{}\ell$ tenemos que existe $N\in\bN$ tal que $\mm(x_{n_k},x)<\frac{\varepsilon}{2}$ cuando $k>N$.
Entonces existe $K\in\bN\colon K>\max{M,N}$ tal que tenemos $n_K\geq K>N$ y por desigualdad triangular, cuando $n> K$ se cumple que
$$\mm(x_n,\ell)\leq \mm(x_n,x_{n_K})+\mm(x_{n_K},\ell)<\varepsilon$$
Por lo tanto $x_n\xrightarrow[n\to\infty]{}\ell$. Como $(x_n)_{n\in\bN}$ era arbitraria, toda sucesión de Cauchy es convergente y por lo tanto $X$ es completo.
\end{ptcb}

Aún si completitud no implica directamente compacidad, podemos encontrar una propiedad que junto a completitud nos de compacidad. A esto se le conoce como estar totalmente acotado.

\begin{Def}
  Un espacio métrico $(X,\mm)$ se dice ser totalmente acotado o precompacto si $\forall\varepsilon>0$ existen $y_1,\cdots,y_n\in X\colon X\subseteq\cup_{i\in\bonj{m}}B(y_i,\varepsilon)$.
\end{Def}

\begin{Ex}
  Tome $X=\bN$ y $\mm(m,n)= m\neq n? 1\colon 0$, la métrica discreta. Entonces $X$ es acotado pero no totalmente acotado. Esto pues el $\varepsilon$ de la definición puede ser menor a 1 y la cantidad de bolas no es contable sino finita.
\end{Ex}

\begin{Th}
  En un espacio totalmente acotado, toda sucesión posee una subsucesión de Cauchy.
\end{Th}

\begin{ptcbp}
Sea $(x_n)_{n\in\bN}\subseteq X$, entonces como $X$ es totalmente acotado existen $y_1,\cdots, y_m$ tal que
$$(x_n)_{n\in\bN}\subseteq X\subseteq\bigcup_{i\in\bonj{m}}B(y_i,1)$$
Si la sucesión no tiene infinitos puntos, el resultado es inmediato pues tenemos subsucesiones constantes. Sin perdida de generalidad podemos asumir que $\conj{x_n\colon n\in\bN}$ tiene cardinalidad infinita. Entonces existe $z_1$ tal que $B(z_1,1)$ contiene infinitos de la sucesión.\par
Sea $(x_{n,1})_{n\in\bN}\subseteq B(z_1,1)$ y $\conj{x_{n,1}\colon n\geq 1}$ tiene cardinalidad infinita. Si iteramos este proceso, tenemos que $(x_{n,k})_{n\in\bN}\subseteq B(z_k,\frac{1}{k})$ es una sucesión de cardinalidad infinita. De esta manera extraemos la subsucesión $(x_{n,k+1})_{n\in\bN}\subseteq B(z_{k+1},\frac{1}{k+1})$. Note que $\mm(x_{n,k},x_{m,k})\leq\frac{2}{k}$.\par
Tome la sucesión $(x_{k,k})_{k\in\bN}$ y note que $\mm(x_{k,k},x_{\ell,\ell})\leq\frac{2}{k}$.
\end{ptcbp}

Aplicamos un argumento diagonal, creamos muchas sucesiones y así construimos una matriz de sucesiones. De aquí yéndonos por la diagonal podemos agarrar esta.

\iffalse
Totalmente acotado nos da la cantidad finita de bolas. Tenemos infinitos señores en una cantidad finita de bolas. Hay una bola con infinitos de estos. Seguimos cubriendo el espacio, pero con bolas de radio menor $(1\to \frac{1}{2})$. Nuevamente una de estas bolas tiene una cantidad infinita de elementos. Otra vez existe una de estas bolas de medio radio que contiene infinitos señores. Seguimos sacando infinitos en bolas de tamaño un tercio, un cuarto, un quinto,... La subsucesión en el quinto paso es subsucesión de la cuarto paso que a su vez es del tercer paso y así. Pero estas subsucesiones son de los señores que ya había filtrado. \par
En algún momento llegue al paso $k$ y agarro una subsucesión infinita en una bola aún más pequeña. En el paso $k$ todos los puntos están apelotados. En el paso $\ell\geq k$ todos los del paso $\ell$ eran del paso $k$. Por la manera escogida no se repiten puntos.
\fi

La otra dirección de la implicación también es cierta.

\begin{Lem}
  Sea $(X,\mm)$ un espacio métrico. Entonces $X$ es totalmente acotado si y sólo si toda sucesión $(x_n)_{n\in\bN}\subseteq X$ posee una subsucesión de Cauchy.
\end{Lem}

\begin{ptcbp}
Suponga que $X$ no es totalmente acotado, entonecs existe $\varepsilon >0$ tal que $X$ no se puede cubrir con una cantidad finita de bolas con radio $\varepsilon$. Tome $y_1\in X$, entonces $X\subsetneq B(y_1,\varepsilon)$. Sea $y_2\in X\setminus B(y_1,\varepsilon)$. Como $X\subsetneq B(y_1,\varepsilon)\cup B(y_2,\varepsilon)$. En general tome
$$y_k\in X\setminus\left(\cup_{j\in\bonj{k-1}}B(y_j,\varepsilon)\right)$$
Entonces $\mm(y_k,y_\ell)\geq\varepsilon$, o sea la sucesión $(y_n)_{n\in\bN}$ no puede tener una subsucesión de Cauchy.
\end{ptcbp}

\begin{Th}
  Dado un espacio métrico $(X,\mm)$. Se tiene que $C$ es compacto si y sólo si completo y totalmente acotado.
\end{Th}

\subsubsection*{Compleción de un espacio}

Sea $(X,\mm)$ un espacio. Tome $(x_n)_{n\in\bN},(y_n)_{n\in\bN}\subseteq X$ sucesiones de Cauchy. Vea que
$$\mm(x_m,y_m)\leq \mm(x_m,x_n)+\mm(x_n,y_n)+\mm(y_m,y_n)$$
Y a su vez, podemos hacer lo mismo con $\mm(x_n,y_n)$. Entonces
$$|\mm(x_m,y_m)-\mm(x_n,y_n)|\leq \mm(x_n,x_m)+\mm(y_n,y_m)$$
Sabemos que dado $\varepsilon\geq 0$, existe $n_0$ tal que cuando $m,n\geq n_0$:
\begin{gather*}
  \mm(x_n,x_m)<\frac{\varepsilon}{2} \\
  \mm(y_n,y_m)<\frac{\varepsilon}{2}
\end{gather*}
Así $(\mm(x_m,y_m))_{m\in\bN}$ es una sucesión de Cauchy.\par
Definimos
$$\mm^\#((x_n)_{n\in\bN},(y_n)_{n\in\bN})=\lim_{m\to\infty}\mm(x_m,y_m)$$
Sea $(z_n)_{n\in\bN}$ de Cauchy, como $\mm(x_m,y_m)\leq \mm(x_m,z_m)+\mm(z_m,y_m)$ al tomar limites se tiene que
\begin{align*}
  \mm^\#((x_n)_{n\in\bN},(y_n)_{n\in\bN})\leq  & \mm^\#((x_n)_{n\in\bN},(z_n)_{n\in\bN}) \\
   +&\mm^\#((z_n)_{n\in\bN},(y_n)_{n\in\bN})
\end{align*}

Esto nos dice que $\mm^\#$ es una semimétrica ya que no cumple definición positiva.

\subsection{Día 7| 10-4-18}

Retomamos de la clase pasada las sucesiones. Vamos a definir clases de equivalencia para tratar bien la métrica. \par
Diremos que $(x_n)_{n\in\bN},(y_n)_{n\in\bN}$ están relacionadas si

$$\mm^\#((x_n)_{n\in\bN},(y_n)_{n\in\bN})=\lim_{m\to\infty}\mm(x_m,y_m)=0$$
Es evidente que esta relación es de equivalencia.
\begin{Ej}
  Verificar que la relación anterior en efecto es de equivalencia.
\end{Ej}

\begin{ptcb}

 La relación es reflexiva por definición de la métrica.
  $$\lim_{m\to\infty}\mm(x_m,x_m)=0$$
 Es simétrica pues la métrica también cumple simetría. El límite no cambia nada.
  $$\lim_{m\to\infty}\mm(x_m,y_m)=0\Rightarrow \lim_{m\to\infty}\mm(y_m,x_m)=0$$
Ahora suponga que
$$\lim_{m\to\infty}\mm(x_m,y_m)=0=\lim_{m\to\infty}\mm(y_m,z_m)$$
 entonces vea que por desigualdad triangular tenemos que
    $$\lim_{m\to\infty}\mm(x_m,z_m)\leq \lim_{m\to\infty}\mm(x_m,y_m)+\lim_{m\to\infty}\mm(y_m,z_m)$$
    Como ambos límites son cero, se sigue que la relación es transitiva.

\end{ptcb}

Consideramos un nuevo espacio, si $\cR$ es la relación anterior definimos:
$$X^\#=\quot{\conj{(x_n)_{n\in\bN}\subseteq X\colon (x_n)\,\text{ es de Cauchy}}}{\cR}$$

Tome $\mm^\#\colon X^\#\times X^\#\to\bR\colon (\bonj{x},\bonj{y})\mapsto\mm^\#(\bonj{x},\bonj{y})=\mm^\#((x_n)_{n\in\bN},(y_n)_{n\in\bN})$. Necesitamos verificar que $\mm^\#$ está bien definida.
\begin{ptcbp}
Sean $(x_n)_{n\in\bN}\cR(x_n')_{n\in\bN}$ y $(y_n)_{n\in\bN}\cR(y_n')_{n\in\bN}$. Como
$$\lim_{m\to\infty}\mm(x_m,y_m)=\mm^\#((x_n)_{n\in\bN},(y_n)_{n\in\bN})$$
Entonces tenemos que probar
$$\lim_{m\to\infty}\mm(x_m,y_m)=\lim_{m\to\infty}\mm(x_m',y_m')$$
Pero observe que
\begin{align*}
  \mm(x_m,y_m) &\leq \mm(x_m,x_m')+\mm(x_m',y_m) \\
  &\leq \mm(x_m,x_m')+\mm(x_m',y_m')+\mm(y_m,y_m')
\end{align*}
Por lo tanto $\lim_{m\to\infty}\mm(x_m,y_m)\leq \lim_{m\to\infty}\mm(x_m',y_m')$ entonces por simetría se cumple lo pedido.
\end{ptcbp}

\begin{Ej}
  Verifique que $(X^\#,\mm^\#)$ es un espacio métrico.
\end{Ej}

\begin{ptcb}
En efecto, tenemos que probar que $\mm^\#$ cumple ser definida positiva, simétrica y que cumple la desigualdad triangular.\par
Probamos primero definición positiva, vea que $\mm^\#$ siempre es positiva pues es el límite de términos positivos. Ahora vea que
$$\mm^\#(\bonj{(x_n)},\bonj{(y_n)})=0\iff \lim_{m\to\infty}\mm(x_m,y_m)=0 $$
Pero esta es la definición de la relación de equivalencia. O sea $(x_n)\cR (y_n)$ y por tanto $\bonj{(x_n)}=\bonj{(y_n)}$.


Por definición de $\mm^\#$ tenemos la simetría, ya que $\mm$ es simétrica.\par
Ahora considere $(x_n)_{n\in\bN},(y_n)_{n\in\bN},(z_n)_{n\in\bN}\subseteq X$. Queremos ver que
\begin{align*}
 \mm^\#((x_n)_{n\in\bN},(z_n)_{n\in\bN}) &\leq\mm^\#((x_n)_{n\in\bN},(z_n)_{n\in\bN})\\
   &\quad +\mm^\#((z_n)_{n\in\bN},(y_n)_{n\in\bN})
\end{align*}
Pero en efecto, para cualquier $n$ tenemos que
$$\mm(x_n,z_n)\leq \mm(x_n,y_n)+\mm(y_n,z_n)$$
Tomando límites obtenemos el resultado.\par
\textcolor{red}{Revisar}


\end{ptcb}
Vamos a ver que este nuevo espacio es completo y que además $X$ es denso en él.

\begin{Lem}
  Existe una inyección de $X$ en $X^\#$.
\end{Lem}
\begin{ptcbp}
Considere $i\colon X\to X^\#\colon x\mapsto\bonj{(x_n)_{n\in\bN}}$ con $x_n=x$. Esta sucesión es de Cauchy pues es constante. La función $i$ es inyectiva, tome $x\neq y$ elementos de $X$. Tenemos que
\begin{align*}
  \mm(x,y)&=\lim_{m\to\infty}\mm(x,y) \\
  &=\mm^\#((x_n)_{n\in\bN},(y_n)_{n\in\bN}) \\
   &=\mm^\#(\bonj{(x_n)_{n\in\bN}},\bonj{(y_n)_{n\in\bN}})
\end{align*}
Por lo que $\mm(x,y)=\mm^\#(i(x),i(y))$
\end{ptcbp}
Más aún, $i$ preserva distancias pero no es una isometría ya que no cumple sobreyectividad.\par
Ahora, $i(X)$ es la copia de $X$ dentro de $X^\#$. Queremos ver que este conjunto es denso dentro de $X^\#$ o sea $\overline{i(X)}=X^\#$.

\begin{ptcbp}
Considere $\underline{y}=\bonj{(y_n)_{n\in\bN}}$, dado $\varepsilon>0$ existe $n_0$ tal que $m,\ell\geq n_0$ entonces $\mm(y_m,y_\ell)<\varepsilon \Rightarrow \lim_{m\to\infty}\mm(y_m,y_\ell)\leq \varepsilon$. Pero esto es igual a $\mm^\#((y_n)_{n\in\bN},(y_\ell)_{n\in\bN})=\mm^\#(\underline{y},i(y_\ell))$. Es decir, cuando $\ell\geq n_0$ se cumple $\mm^\#(\underline{y},i(y_\ell))\leq \varepsilon$ y por lo tanto $i(X)$ es denso en $X^\#$.
\end{ptcbp}

Queremos ver que $X^\#$ es un espacio completo, es decir queremos agarrar una sucesión aquí o sea una sucesión de sucesiones. Queremos encontrar una nueva sucesión para la cual la nuestra original converge. Primero aproximamos toda sucesión con elementos de $X$, esto nos dará una nueva sucesión y esta debe converger a la que construimos.

\begin{ptcbp}
Sea $(\underline{y}^n)_{n\in\bN}\subseteq X^\#$, tome $y_n\in X$ tal que $\mm^\#(\underline{y}^n,i(y_n))<\frac{1}{n}$. Ahora vea que
\begin{align*}
  \mm(y_m,y_n) & =\mm^\#(i(y_m),i(y_n)) \\
   &\leq\mm^\#(i(y_m),\underline{y}^m)+\mm^\#(\underline{y}^m,\underline{y}^n)\\
   &+\mm^\#(\underline{y}^n,i(y_n))\\
   &\leq \frac{1}{n}+\frac{1}{m}+\mm^\#(\underline{y}^m,\underline{y}^n)
\end{align*}
Por lo que si $(\underline{y}^n)_{n\in\bN}$ es de Cauchy en $X^\#$ entonces $(y_n)_{n\in\bN}$ es Cauchy en $X$. Finalmente, vamos a probar que $(\underline{y}^n)_{n\in\bN}$ conver a $\bonj{(y_n)_{n\in\bN}}=\underline{y}$. Es decir que $\lim_{n\to\infty}(\underline{y}^n,\underline{y})=0$. Vea que
\begin{align*}
  \mm^\#(\underline{y}^n,\underline{y}) &\leq \mm^\#(\underline{y}^n,i(y_n))+\mm^\#(i(y_n),\underline{y}) \\
   &\leq \frac{1}{n}+\mm^\#(i(y_n),\underline{y})
\end{align*}

Por construcción se cumple que $\mm^\#(i(y_n),\underline{y})\to 0$ cuando $n\to\infty$.
\end{ptcbp}

\begin{Ej}
  Refine el último detalle de la prueba anterior.
\end{Ej}

\begin{ptcb}

\end{ptcb}

Lo que acabamos de probar es lo siguiente:
\begin{Th}
  Dado un espacio métrico $(X,\mm)$, existe un espacio métrico $(X^\#,\mm^\#)$ completo y una función inyectiva $i\colon X\to X^\#$ tal que $\mm(x,y)=\mm^\#(i(x),i(y))$ y se cumple que $i(X)$ es denso en $X^\#$.
\end{Th}
Ahora considere $(Y,\mm')$ completo y $j\colon X\to Y$ que satisface $\mm(x,y)=\mm'(j(x),j(y))$. Vamos a construir una función inyectiva $\theta\colon X^\#\to Y$ que preserva distancias. En cierto sentido, $X^\#$ es el espacio completo más pequeño que contiene a $X$.\par
Lo que haremos es lo siguiente, primero vamos a definir una función en todo $X^\#$ para ello la definimos primero en $i(X)$. Una vez hecho esto, como $i(X)$ es denso podemos extender la definición mediante límites. Lo que queda el límite, es lo que vale.
\begin{ptcbp}
Dado $x\in X$, definimos $\theta(i(x))=j(x)$. Ya conocemos la función $j$, pero por qué se preservan las distancias con $\theta$? Note que
\begin{align*}
\mm'(\theta(i(x)),\theta(i(y)))  &=\mm'(j(x),j(y))\\
   &=\mm(x,y)  \\
   &=\mm^\#(i(x),i(y))
\end{align*}

Es decir $\theta\colon i(X)\to Y\colon x\mapsto j(x)$ preserva distancias y por tanto inyectiva. \par
Dada $\underline{y}$, existe una sucesión $(i(y_n))_{n\in\bN}$ tal que
$$\lim_{n\to\infty}\mm^\#(\underline{y},i(y_n))=0$$
Tome $\theta(\underline{y})=\lim_{n\to\infty}(\theta(i(y_n)))=\lim_{n\to\infty}j(y_n)$. Como $j$ es continua, para que el límite exista basta que $y_n$ sea convergente. Entonces si probamos que $(y_n)_{n\in\bN}$ tenemos el resultado.\par
Note que $\mm'(j(y_n),j(y_m))=\mm^\#(i(y_n),i(y_m))=\mm(y_n,y_m)$. Como $(i(y_n))_{n\in\bN}$ es de Cauchy, tenemos que $(y_n)_{n\in\bN}$ y $(j(y_n))_{n\in\bN}$ son de Cauchy. Por lo tanto $\lim_{n\to\infty}j(y_n)$ existe.
\end{ptcbp}

\begin{Ej}
Muestre que $\theta$ está bien definida y que preserva distancias.
\end{Ej}

\begin{ptcb}
\end{ptcb}

\subsubsection*{Conexidad}

\begin{Def}
  Un espacio $(X,\mm)$ es disconexo si existen $A,B$ abiertos no vacíos tal que $X=A\cup B$ con $A\cap B=\emptyset$. El espacio $(X,\mm)$ es conexo si no es disconexo.
\end{Def}
Insertar FIG 7.1\par
Entonces ser conexo y tener $X=A\cup B$ con $A,B$ abiertos entonces $A$ es vacío o todo el espacio y $B$ vice-versa. \par

Note además, si $(X,\mm)$ es disconexo con $X=A\cup B$. Entonces $A$ es abierto y por lo tanto $X\setminus A$ es cerrado. Pero por construcción $B=X\setminus A$ y por tanto $B$ es cerrado y abierto. Análogamente $A$ es abierto y cerrado.
\begin{Lem}
  Un espacio $(X,\mm)$ es conexo si y sólo si los únicos conjuntos abiertos y cerrados son $X$ y $\emptyset$.
\end{Lem}

\begin{Def}
  Decimos que $E\subseteq X$ es conexo si $(E,\mm_E)\leq (X,\mm)$ es un espacio conexo.
\end{Def}

De forma equivalente, si $E$ es disconexo $E=A'\cup B'$ disjuntos y abiertos en $(E,\mm_E)$ no vacíos. Es decir, existen $A,B\subseteq X$ abiertos tales que $E=(E\cap A)\cup (E\cap B)$ con $A\cap E\neq \emptyset$ y $E\cap B\neq \emptyset$. Luego si $E$ es conexo, $A,B$ abiertos disjuntos con $E\subseteq A\cup B$, entonces $E\subseteq A$ ó $E\subseteq B$.

\begin{Ex}
  $E=\conj{x}$, $x\in X$ es conexo.
\end{Ex}

\begin{Ex}
  $I=\obonj{a,b}\subseteq\bR$ es conexo.
\end{Ex}

\begin{ptcb}
Sean $A,B$ abiertos disjuntos no vacíos de manera que podamos ver $I=(I\cap A)\cup(I\cap B)$. Tome $(s,t)\in(I\cap A)\times(I\cap B)$. Por definición de intervalo $\bonj{s,t}\subseteq I$.
Insertar FIG 7.2\par
Sea $u=\sup(A\cap\bonj{s,t})$ y tenemos que $u\in I$. Entonces está en $A$ ó en $B$.
\begin{enumerate}
  \item Si $u\in A$ entonces existe $\delta>0$ tal que $\obonj{u-\delta,u+\delta}\subseteq A$. Vea que $u\neq t$, $u$ está en $A$ y $t\in B$. Luego $u<t$ y por tanto existe $\delta_1>0$ tal que $\bonj{u,u+\delta_1}\subseteq A\cap\bonj{s,t}$. Esto contradice que $u$ es el $\sup$.
  \item Si $u\in B$, entoces $s<u$. Como $B$ es abierto existe $\delta_2>0$ tal que $\obonj{u-\delta_2,u+\delta_2}\subseteq B$. Entonces $\rbrack u-\delta_2, u\rbrack\subseteq \bonj{s,t}$. Finalmente $w\in A\cap \bonj{s,t}$ tal que $u-\delta_2<w\leq u$. Esto es una contradicción pues hay un punto en la intersección.
\end{enumerate}
\end{ptcb}

\begin{Lem}
  Sean $(X,\mm),(Y,\mm')$ espacios métricos y $f\colon X\to Y$ continua. Si $E\subseteq X$ es conexo entonces $f(E)$ es conexo.
\end{Lem}

\begin{ptcbp}
Sean $A,B$ abiertos disjuntos tales que
\begin{gather*}
  f(E)=(f(E)\cap A)\cup(f(E)\cap B) \\
  \Rightarrow E=(f^{-1}(A)\cap E)\cup(f^{-1}(B)\cap E)
\end{gather*}
Pero si $E$ es conexo, uno de los dos debe ser vacío y otro todo el espacio. Entonces $E\subseteq f^{-1}(A)$ ó $E\subseteq f^{-1}(B)$ lo que implica que $f(E)\subseteq A$ ó $f(E)\subseteq B$. Entonces uno de los pedazos es vacío y el otro es $f(E)$. Por lo tanto $f(E)$ es conexo.

\end{ptcbp}

\begin{Ex}
  $\bR^d$ es conexo.
\end{Ex}

\begin{ptcb}
Sean $\vx,\vy\in\bR^d$, tome $f\colon\bonj{0,1}\to\bR^d, t\mapsto (1-t)\vx+t\vy$. Esta función es continua. Entonces diremos que el segmento de recta entre $\vx,\vy$ es $f(\bonj{0,1})=\conj{\vz\in\bR^d\colon \vz=(1-t)\vx+t\vy, t\in\bonj{0,1}}=\bonj{\vx,\vy}$. \par
Ahora si $\bR^d$ fuese disconexo, existen $A$ y $B$ abiertos disjuntos no vacíos tales que $\bR^d=A\cup B$. Tome $(\vx,\vy)\in A\times B$, entonces $\bonj{\vx,\vy}=(\bonj{\vx,\vy}\cap A)\cup (\bonj{\vx,\vy}\cap B)$. Esto es una contradicción pues el segmento de recta $\bonj{\vx,\vy}$ es conexo por el lema anterior.
\end{ptcb}

\begin{Lem}
  Sea $(U_\alpha)_{\alpha\in\cA}\subseteq\cP(X)$ una colección de conjuntos conexos. Si $\cap_{\alpha\in\cA}U_\alpha\neq\emptyset$, entonces $U=\cup_{\alpha\in\cA}U_\alpha$ es conexo.
\end{Lem}

\begin{ptcbp}
Sea $x\in \cap_{\alpha\in\cA}U_\alpha$. Tome $A,B$ abiertos tales que $U=(U\cap A)\cup(U\cap B)$. Sin perdida de generalidad $x\in U\cap A$, o sea $x\in A$. Vea que $U\cap B\neq \emptyset\iff (\cup_{\alpha\in\cA}U_\alpha)\cap B\neq\emptyset$. Entonces existe $\beta_0$ tal que $U_{\beta_0}\cap B\neq \emptyset$. Además $x\in U_{\beta_0}\cap A$ pues $x$ está en la intersección. Por lo tanto $E_{\beta_0}=(E_{\beta_0}\cap A)\cup (E_{\beta_0}\cap b)$. Esto es una contradicción \textcolor{red}{por qué?}
\end{ptcbp}

\begin{Def}
Una curva o camino entre $x,y\in X$ es una función
$f\colon\bonj{0,1}\to X$ continua tal que $(f(0),f(1))=(x,y)$. Un espacio es conexo por caminos o arcoconexo si dados dos puntos $x,y\in X$, existe un camino entre $x,y$.
\end{Def}
INSERTAR FIG 7.3\par
\begin{Lem}
  Todo espacio conexo por caminos es conexo.
\end{Lem}

\begin{ptcbp}
Sea $X$ un espacio conexo por caminos pero asuma a manera de contradicción que $X$ no es conexo. Así existen $A,B\subseteq X$ abiertos, disjuntos, no vacíos de manera que $X=A\cup B$. Tome $(a,b)\in A\times B$.\par
Como $X$ es conexo por caminos, existe $\gamma\colon\bonj{0,1}\to X$ continua tal que $(\gamma(0),\gamma(1))=(a,b)$. \par
Note que $\gamma^{-1}(A)$ y $\gamma^{-1}(B)$ son disjuntos de $\bonj{0,1}$ y su unión es $\bonj{0,1}$ por definición de función. Como $\gamma$ es continua, son abiertos y como $(0,1)\in \gamma^{-1}(A)\times\gamma^{-1}(B)$. \par
Así hemos encontrado una desconexión de $\bonj{0,1}$ en abiertos, disjuntos, no vacíos. Esto es una contradicción pues este conjunto es conexo y por lo tanto nuestra suposición de que $X$ no era conexo está errada. Por lo tanto $X$ es conexo.
\end{ptcbp}

\begin{Ej}
  Si $E\subseteq\bR^d$ y $E$ es conexo y abierto, entonces $E$ es conexo por caminos.
\end{Ej}

%%https://proofwiki.org/wiki/Connected_Open_Subset_of_Euclidean_Space_is_Path-Connected
\begin{ptcb}
 Sea $a\in E$ y considere $A\subseteq E$:
 $$A=\conj{x\in E\colon\exists\gamma\in\cC(\bonj{0,1},X),(\gamma(0),\gamma(1))=(x,a)}$$

 el subconjunto de $E$ de los puntos que pueden ser unidos a $a$ por un camino.\par

  Tome $x\in A$, esto significa que $B(x,r)\subseteq E$. Observe que si $y\in B(x,r)$, existe $\gamma'$ un camino en linea recta de $x$ a $y$ en virtud de que la bola es convexa. Además como $x\in A$ tenemos que hay un camino de $a$ hacia $x$, entonces $\gamma\gamma'$ (\textcolor{red}{abuso?}) es un camino entre $y$ y $a$. O sea $y\in A$ y por tanto $B(x,r)\subseteq A$ pues $y$ era arbitrario. Por lo tanto $A$ es abierto. \par
 De manera análoga, defina $B=E\setminus A$, el conjunto de puntos que no se pueden unir a $a$ por un camino, y vea que $B$ es abierto. Para $x\in B$, considere $B(x,r)\subseteq E$. Si existe $y\in B(x,r)$ tal que hay un camino de $y$ a $a$, podríamos extender el camino hacia $x$. Por lo tanto $B(x,r)\subseteq B$ y así $B$ es abierto.\par
 Claramente $A\cap B=\emptyset$ y $E=A\cup B$, también $a\in A$ lo que nos dice que $A\neq\emptyset$. Cómo $E$ es conexo se sigue que $B$ debe ser vacío y por tanto $E=A$ y así $E$ es conexo por caminos.
\end{ptcb}
INSERTAR FIG7.4

\subsection{Día 8| 12-4-18}

\subsubsection*{Segunda sesión de ejercicios}

\begin{Ej}[2.2.11.22 Santiago Cambronero]
  Sea $E$ un espacio métrico separable y $(U_i)_{i\in I}$ familia de abiertos no vacíos, disjuntos por parejas. Mostrar que $I$ es contable
\end{Ej}

\begin{ptcb}
\textcolor{red}{Como $E$ es separable, existe $D\subseteq E$ tal que $D$ es contable. Además} \par
$D$ es denso \textcolor{red}{y contable} y por tanto para cualquier $H\subseteq E$ tal que $H$ es abierto tenemos $D\cap H\neq \emptyset$. En particular para cada $U_i$ tenemos $D\cap U_i\neq \emptyset$. O sea, existe $x_i\in D\cap U_i$ tal que $x_i\not\in U_j$ para $j\neq i$. Considere $f\colon I\to D\colon i\mapsto x_i$, $f$ es inyectiva por construcción. Entonces para cada elemento de $I$ hay uno en $D$ y por tanto $I$ es contable.
\end{ptcb}

\begin{Ej}[2.3.4.19 Santiago Cambronero]
  Sea $f,g\colon E\to E'$ continuas y $f(x)=g(x)$ para $x\in D$ donde $D$ es denso en $E$. Mostrar que $f(x)=g(x)$ para $x\in E$.
\end{Ej}

\begin{ptcb}
Sea $x\in E$ tal que $x_n\to x$ donde $(x_n)_{n\in\bN}\subseteq D$. Entonces para cualquier $x_n$ tenemos $f(x_n)=g(x_n)$. Al tomar límites tenemos que $f(x)=g(x)$ ya que $f,g$ son continuas.
\end{ptcb}

\begin{ptcb}
Vea que $D\subseteq\conj{x\in E\colon f(x)=g(x)}$ entonces $\overline{D}\subseteq \overline{\conj{x\in E\colon f(x)=g(x)}}$. Pero este conjunto es cerrado y $\overline{D}=E$. Entonces $E\subseteq\conj{x\in E\colon f(x)=g(x)}$. Por lo tanto $f$ y $g$ coinciden en $E$.
\end{ptcb}

\begin{Ej}
  Sea $E'$ completo y $f\colon D\to E'$ uniformemente continua en $D$ con $\overline{D}=E$. Demuestre que existe una única extensión de $f$ hacia $E$. Muestre además que es uniformemente continua.
\end{Ej}

\begin{ptcb}
Sea $x\in E$ y $(x_n)_{n\in\bN}\subseteq D$ tal que $x_n\to x$. Entonces tenemos $f(x)=\lim_{n\to\infty}f(x_n)$. Como $(x_n)_{n\in\bN}$ es una sucesión convergente, es de Cauchy. Ahora $f$ uniformemente continua en $D$ implica que para $x,y\in D$
$$\forall\varepsilon>0\exists\delta>0\colon \mm(x,y)<\delta \Rightarrow\mm_{E'}(f(x),f(y))<\varepsilon$$
Entonces $\exists N$ tal que cuando $m,n>N$ entonces $\mm(x_n,x_m)<\delta$ entonces $\mm_{E}'(f(x_n),f(x_m))<\varepsilon$. \par
Terminar de escribir viendo las fotos
\end{ptcb}

\begin{Ej}
  Sean $A,B\subseteq E$ disjuntos. Si $A$ es cerrado y $B$ compacto entonces $\mm(A,B)>0$. Muestre que el resultado es falso si sólo se pide que ambos sean cerrados.
\end{Ej}

\begin{ptcb}
Suponga que $\mm(A,B)=0$ entonces para todo $n\in\bN$ existen $x_n\in B$ y $y_n\in A$ tales que $\mm(x_n,y_n)<\frac{1}{n}$. Considere $(x_n)_{n\in\bN}\subseteq B\Rightarrow \exists (x_{n_k})_{k\in\bN}\subseteq(x_n)_{n\in\bN}$ que converge a $x\in B$. Como $\mm(x,y_{n_k})\leq \mm(x,x_{n_k})+\mm(x_{n_k},y_{n_k})<\tilde{\varepsilon}$. El primer término es pequeño pues la sucesión converge y el segundo por construcción. Entonces $y_{n_k}\to x$ como $A$ es cerrado entonces $x\in A$ pero esto es una contradicción pues $A$ y $B$ son disjuntos.
\end{ptcb}

\subsection{Día 9| 17-4-18}

\begin{Ex}
  Considere el conjunto $D=\bonj{-1,0}\times\conj{0}\cup\conj{(x,\sin(\frac{1}{x}))\colon x\in\bonj{0,1}}$. Veamos que $D$ es conexo.
\end{Ex}

\begin{ptcb}
Asuma que $D=A\cup B$ con $A,B$ abiertos disjuntos. Vamos a probar que alguno contiene a $(0,0)$ y por tanto contiene toda la curva.\par
  Asuma que $(0,0)\in A$. A partir de aquí $D_1=\bonj{-1,0}\times\conj{0}$ lo podemos ver como $\conj{(-t,0)\colon t\in\bonj{0,1}}$. Esto es la imagen de $\bonj{0,1}$ bajo $\varphi(t)=(-t,0)$, una parametrización continua. Como $D_1\subseteq A\cup B$ y $D_1$ es conexo, entonces $D_1\subseteq A$. Además $D_1$ es abierto y puedo meter una bola en él.\par
  Entonces vamos a mostrar que existe un $n_0$ tal que $(\frac{1}{\pi n},\sin(\pi n))=(\frac{1}{\pi n},0)\in A$ cuando $n\geq n_0$. Esto se sigue de que existe $r>0$ tal que $B((0,0),r)\subseteq A$ entonces tiene un punto de acumulación.\par
  Por último, vemos que $A$ contiene a $\conj{(x,\sin(\frac{1}{x}))\colon x\in\bonj{0,1}}$.
\end{ptcb}

\begin{Ej}
  Verifique el último paso del ejemplo anterior.
\end{Ej}


\begin{Ej}
  Muestre que no existe $\varphi\colon\bonj{0,1}\to\bR$ continua tal que $\varphi(0)=(0,0)$ y $\varphi(1)=(\frac{1}{\pi},0)$.
\end{Ej}

Leer Munkres 178
\subsubsection*{Categorías de Baire}

\begin{Def}
  Dado un espacio $(X,\mm)$, decimos que $A\subseteq X$ es denso en ninguna parte si $X\setminus\overline{A}$ es denso en $X$.
\end{Def}

Lo que esto significa es que $X\setminus\overline{A}\cap B(x,r)\neq\emptyset$. Esto es equivalente a $X\setminus\overline{A}\cap B(x,r)\neq\emptyset$ para $x\in \overline{A}$. A la vez podemos ver esto como $(\overline{A})^o=\emptyset$.\par
Esto es una forma de decir que el conjunto es muy pequeño, en el sentido que no le podemos meter bolas adentro. Qué conjuntos se pueden construir a partir de conjuntos pequeños? Por ejemplo $\bQ$ está hecho de conjuntos pequeños, pero uniéndolos obtenemos un conjunto denso.\par
Podemos construir abiertos con conjuntos pequeños? La respuesta nos la da Baire, no es posible construir abiertos a partir de conjuntos pequeños. Pero sólo en espacios completos.

\begin{Def}
  Un conjunto $A$ es de primera categoría si es unión contable de conjuntos densos en ninguna parte. Cualquier conjunto que no sea de primera catergoría, es de segunda categoría.
\end{Def}

Ahora, en un espacio completo todos los espacios son de segunda categoria. Note que sí es posible escribir conjuntos abiertos como uniones no contables de conjuntos de primera categoría, pues un conjunto es unión de los conjuntos unitarios que contienen a sus elementos.
\begin{Th}[Categorías de Baire]\label{thm:BaireCategories}
  Sea $(X,\mm)$ completo, si $D\subseteq X$ es abierto, entonces $D$ es de segunda categoría.
\end{Th}

Precisamos un lema más antes de proseguir con la prueba del teorema.

\begin{Lem}
  Sea $(X,\mm)$ completo y $G_n\subseteq X$ abierto y denso para $n\geq 1$. Entonces $\cap_{n\geq 1}G_n$ es denso.
\end{Lem}

\begin{ptcbp}
Sea $A$ un abierto, entonces existe $x_1\in A\cap G_1$ pues $G_1$ es abierto. Como $A,G_1$ son abiertos, existe $r_1>0$ tal que
$$B_1=B(x_1,r_1)\subseteq A\cap G_1$$
Al ser $B_1$ abiertos entonces podemos repetir el mismo proceso pero con $G_2$. Así existen $x_2\in X,r_2>0$ tal que $B(x_2,r_2)\subseteq B_1\cap G_2$. Sin perdida de generalidad podemos tomar $r_2<\frac{r_1}{2}$. Montando esta sucesión podemos garantizar que los puntos se van acercando más. Estamos generando una sucesión de Cauchy y como el espacio es completo, va a tener un límite.
Observe que
$$B_2=B(x_2,\frac{r_2}{2})\subseteq \overline{B_2}\subseteq A\cap G_1\cap G_2$$
Pero este procedimiento nos lleva a la cantidad finita, la completitud nos lleva a coger todos. La clausura es para asegurar que los límites caen donde deben de caer.\par
Iteramos este proceso y así existen $x_n\in X$, $r_n>0$ tales que
\begin{align*}
  B_n &=B(x_n,r_n)\quad\text{y} \\
  B_n\subseteq\overline{B_n} &\subseteq B_{n-1}\cap G_n \\
   &\subseteq A\cap G_1\cap\cdots\cap G_n
\end{align*}
Note que al agarrar el $x_n$, él está en $B_n$ que esta contenido en todos los $B_k$'s anteriores. Así $x_n\in B_i$ para $i\in\bonj{n}$. En otras palabras, si $m<n$ entonces $x_m,x_n\in B(x_m,r_m)$ y así $\mm(x_m,x_n)<r_m\leq\frac{r_1}{2^m}$. Entonces $(x_n)_{n\in\bN}$ es de Cauchy, sea $x=\lim_{n\to\infty}x_n$. Vea que
\begin{gather*}
  \conj{x_n\colon n\geq m}\subseteq B_m\Rightarrow B_m \\
  \Rightarrow x\in\overline{\conj{x_n\colon n\geq m}}\subseteq\overline{B_m}\subseteq B_{m-1}\subseteq G_{m-1}\cap A
\end{gather*}

\end{ptcbp}

De aquí retomamos el teorema \ref{thm:BaireCategories}.
%Cogemos un abiertos y una unión de cosas densas en ninguna parte y el abierto los interseca.

\begin{ptcbp}
Sean $A_n$ conjuntos densos en ninguna parte y tome $G_n=X\setminus\overline{A_n}$. Vea que $G_n$ es abierto y denso por definición. Entonces
\begin{align*}
  G\cap\bigcap_{n\in\bN}G_n\neq \emptyset &\Rightarrow G\cap\bigcap_{n\in\bN}(X\setminus\overline{A_n})\neq \emptyset\\
   &\quad=G\cap X\setminus(\bigcup_{n\in\bN}\overline{A_n})
\end{align*}
\end{ptcbp}

\subsubsection*{Arzelá-Ascoli}

%El lema de AA introduce una condición equicont tenemos una suc de funciones continua que nos permite extraer subsucesiones que convergen uniformemente

\begin{Def}
  Dada una familia $\cF=(f_\alpha)_{\alpha\in\cA}$ de funciones, $(X,\mm),(Y,\mm')$ espacios métricos y $f_\alpha\colon X\to Y$, se dice ser equicontinua en $x_0$ si para todo $\varepsilon >0$ existe $\delta>0$ tal que $\mm(x,x_0)<\delta\Rightarrow\mm'(f_\alpha(x),f_\alpha(x_0))<\varepsilon$ para $\alpha\in\cA$.\par
  Una familia es equicontinua en $X$ si es equicontinua en todo punto de $X$.
\end{Def}

Observe que el mismo $\delta$ sirve para todas las funciones.

\begin{Lem}
  Sea $(X,\mm)$ completo y $(f_n)_{n\in\bN}$ una sucesión de funciones equicontinuas. Si $D\subseteq X$ es denso y $\lim_{n\to\infty}f_n(d)$ existe para cualquier $d\in D$, entonces $f_n$ converge en $X$ a una función continua.
\end{Lem}

Para qué sirve completitud? Para probar el paso intermedio de que la sucesión es Cauchy. Por lo tanto va a existir el límite.
\begin{ptcbp}
Basta probar que $(f_n(x))_{n\in\bN}$ es de Cauchy para $x\in X$. Ahora dado $\varepsilon>0$ existe $\delta>0$ tal que para todo $n\geq 1$ se cumple
$$\mm(x,x_0)<\delta\Rightarrow\mm'(f_n(x),f_n(x_0))$$
Queremos probar que $\mm(f_n(x),f_m(x))<\varepsilon$ para $x\in X$. Nosotros sabemos esto pero para $d\in D$ el denso. Entonces la densidad nos permite meter alguien en la bola de $x_0$.\par
Tome $d\in D\cap B(x_0,\delta)$, sabemos que $(f_n(d))_{n\in\bN}$ es de Cauchy. Es decir, existe $N$ tal que cuando $m,n\geq N$ tenemos que $\mm'(f_n(d),f_m(d))<\varepsilon$. Luego cuando $m,n\geq N$
\begin{align*}
  \mm'(f_n(x_0),f_m(x_0)) &\leq \mm'(f_n(x_0),f_n(d)) \\
                         &\quad+\mm'(f_n(d),f_m(d))\\
                         &\quad+\mm'(f_m(x_0),f_m(d))<3\varepsilon
\end{align*}
Sea $f(x)=\lim_{n\to\infty}f_n(x)$ entonces $\lim_{n\to\infty}\mm'(f_n(x),f_n(x_0))=\mm'(f(x),f(x_0))$. Tomando limites obtenemos el resultado.
\end{ptcbp}

Note que no es necesario que todo el espacio sea completo no es necesario, sino que $\overline{(f_n(x))_{n\in\bN}}$ sea completo. Por ejemplo si $\overline{(f_n(x))_{n\in\bN}}$ es compacto. En $\bR$ basta que sea acotada, aquí una sucesión equicontinua y acotada cumple todo lo anterior.\par

En compactos la equicontinuidad nos garantiza que si tenemos convergencia puntual, tenemos convergencia uniforme.

\begin{Lem}
  Sea $(f_n)_{n\in\bN}$ equicontinua. Si $(f_n(x))_{n\in\bN}$ converge para $x\in K$ compacto, entonces $f_n$ converge uniformemente en $K$.
\end{Lem}

\begin{ptcbp}
Sean $x_0\in K$ y $\varepsilon>0$. Entonces existe $\delta>0$ tal que por equicontinuidad
$$\mm(x,x_0)<\delta\Rightarrow\mm'(f_n(x),f_n(x_0))<\varepsilon$$
Reescribimos esto con bolas
$$x\in B(x_0,\delta,\mm)\Rightarrow f(x)\in B(f(x_0),\varepsilon,\mm')$$

Queremos probar que la distancia entre $f_n(x_0)$ y $f(x_0)$ es menor que $\varepsilon$ arbitrario. Hay convergencia puntual pero $n\geq N(x_0)$. Como hay puntual $N$ varía, la equicontinuidad entra haciendo que la distancia entre $f_n(x_0)$ y $f(x_0)$ sea menor que \textcolor{red}{un montón de cosas}. Dos están acotadas por equicontinuidad y una por convergencia puntual. Ahora tenemos un problema pues sólo se puede dentro de las bolas. Reducimos a un número finito de bolas gracias a compacidad y esto nos da el resultado.\par

Note que $\cup_{x_0\in K}B(x_0,\delta)$ es un cubrimiento por abiertos de $K$. Entoces existen $(x_k)_{k\in\bonj{n}}$ con $K\subseteq\cup_{k\in\bonj{n}}B(x_k,\delta_{x_k})$. Además existe $N$ tal que cuando $n\geq N$ y $k\in\bonj{n}$:
$$\mm'(f_n(x_k),f(x_k))<\varepsilon$$
Si $x\in K$, entonces existe $k_0$ tal que $x\in B(x_{k_0},\delta_{x_{k_0}})$. Luego tenemos
\begin{align*}
  \mm'(f_n(x),f(x)) &\leq \mm'(f_n(x),f_n(x_{k_0})) \\
                         &\quad+\mm'(f_n(x_0),f(x_{k_0}))\\
                         &\quad+\mm'(f(x_{k_0}),f(x))<3\varepsilon
\end{align*}
Pues $\mm'(f(x_{k_0}),f(x))=\lim_{n\to\infty}\mm'(f_n(x_{k_0}),f_n(x))\leq \varepsilon$.
\end{ptcbp}

\begin{Th}[Arzelá-Ascoli]\label{lem:ArzelaAscoli}
  Sea $(f_n)_{n\in\bN}$ equicontinua con $X$ separable. Suponga que para cada $x\in X$ se cumple que $\overline{(f_n)_{n\in\bN}}$ es compacto. Entonces existe $(f_{n_k})_{k\in\bN}$ subsucesión convergente en $X$ a una función continua $f$ y la convergencia es uniforme en compactos.
\end{Th}

\begin{ptcbp}
Sea $D=(x_i)_{i\in\bN}$ denso en $X$, que existe por separabilidad. Basta probar que existe $f_{n_k}$ tal que $f_{n_k}(x_i)$ converge para todo $i\geq 1$. Sabemos que $\overline{(f_n(x_1))_{n\in\bN}}$ es compacto, entonces existe una subsucesión $f_{1,n}$ de $f_n$ tal que $f_{1,n}(x_1)$ es convergente.\par
De igual forma $(f_{1,n}(x_2))_{n\in\bN}\subseteq\overline{(f_n(x_2))_{n\in\bN}}$, entonces existe una subsucesión $f_{2,n}$ tal que $(f_{2,n}(x_2))_{n\in\bN}$ y $(f_{2,n}(x_2))_{n\in\bN}$. Al iterar este proceso existe una sucesucesión $f_{k+1,n}$ de $f_{k,n}$ tal que $f_{k+1,n}(x_{k+m}),\cdots,f_{k+1,n}(x_1)$ convergen. Ahora tome $f_{n,n}$ que es subsucesión de $(f_{k,n})_{k\in\bN}$ siempre que $n=k$. Entonces $f_{n,m}(x_\ell)$ converge si $\ell,n\geq k$. Usando los dos lemas anteriores se sigue el resultado.
\end{ptcbp}

\begin{Cor}
  Sean $f_n\colon X\to \bR$ equicontinuas tal que $X$ es separable. Si $(f_n(x))_{n\in\bN}$ es acotada para todo $x$ entonces existe $f_{n,k}$
 que converge en el sentido del teorema anterior.
 \end{Cor}

 %Ahora considere las funciones de un espacio compacto hacia $\bR$. Con la métrica infinita tenemos un espacio métrico.

 \begin{Ex}
   Sea $K\subseteq\bR^d$ compacto y $\cC(K,\bR)=\conj{f\colon K\to\bR\colon f \quad\text{ es continua}}$. Entonces $(\cC(K,\bR),\mm_{\infty})$ es un espacio métrico.
 \end{Ex}

 \begin{Lem}
   Sea $F\subseteq\cC(K,\bR)$, entonces $F$ es compacto si y sólo si $F$ es cerrado, acotado y equicontinuo.
 \end{Lem}

 \begin{ptcbp}
 \begin{enumerate}
   \item[$(\Rightarrow)$] Hay que verificar que $(f_\alpha)_{\alpha\in\cA}$ es equicontinua. Por contradicción asuma que existe $\varepsilon>0$ y $\vx_0\in K$ tal que existen $\vy_n,\alpha_n$ con
       $$\nm{\vy_n-\vx_0}<\frac{1}{n}\land |f_{\alpha_n}(\vy_n)-f_{\alpha_n}(\vx_0|\geq\varepsilon$$
       Como $(f_{\alpha_n})_{n\in\bN}\subseteq F$, entonces existe $n_k$ tal que $f_{\alpha_{n_k}}\xrightarrow[k\to\infty]{} f$ en norma infinito. Es decir
       $$\sup_{\vx\in K}|f_{\alpha_{n_k}}(\vx)-f(\vx)|\xrightarrow[k\to\infty]{} 0$$
       Pero $f_{\alpha_{n_k}}(x_0)\xrightarrow[k\to\infty]{}f(x_0)$ y $f_{\alpha_{n_k}}(y_{n_k})\xrightarrow[k\to\infty]{}f(x_0)$. Esto es una contradicción. \textcolor{red}{por qué?}
   \item[$(\Leftarrow)$] Sea $(f_n)_{n\in\bN}\subseteq F$. Como $F$ es acotado existe $M>0$ tal que $\nm{f}_\infty\leq M$ para $f\in F$. Entonces $\sup_{\vx\in K}|f_n(\vx)|\leq M$. Esto inmediatamente nos dice que $(f_n(x))_{n\in\bN}$ es acotado.
 \end{enumerate}
 \end{ptcbp}

 Veamos que las bolas cerradas no son compactas.
 Considere $\overline{B(0,1)}=\conj{f\in\cC(K,\bR)\colon \sup_{x\in K}|f(x)|\leq 1}$. Sea $f_n(x)= 0$ cuando $0\leq x\leq 1-\frac{1}{n}$ y $f(x)=nx-n-1$ si $1-\frac{1}{n}\leq x\leq 1$. Entonces $\sup_{x\in\bonj{0,1}}|f_n(x)|=1$. Pero
 $$|f_n(x)-f_n(1)|=|nx-n+1-1|=n|x-1|$$
 Siempre que $1-\frac{1}{n}\leq x\leq 1$. Por lo tanto
 \begin{align*}
   |f_n(x)-f_n(1)|<\varepsilon &\iff |x-1|\leq\varepsilon \\
    &\iff |x-1|\leq \frac{\varepsilon}{n}
 \end{align*}
 Esto contradice la independencia del $\delta$ y por tanto no es equicontinua.

 \subsection{Día 10| 19-4-18}

 \subsubsection*{Stone-Weierstrass}


 Sea $(X,\mm)$ espacio compacto y $\cC(X,\bR)$ con la métrica $\mm_{\infty}(f,g)=\nm{f-g}_\infty=\sup_{x\in X}|f(x)-g(x)|$.

 \begin{Lem}
     El espacio $\cC(X,\bR)$ es completo.
 \end{Lem}

 Vamos a estudiar condiciones para que un espacio sea denso en el espacio mencionado anteriormente. Ahora recuerde que definimos el máximo y mínimo entre dos funciones $f,g$ como
 \begin{gather*}
   f\lor g(x)=\frac{1}{2}\left(f(x)+g(x)+|f(x)-g(x)|\right) \\
    f\land g(x)=\frac{1}{2}\left(f(x)+g(x)-|f(x)-g(x)|\right)
 \end{gather*}
Ahora, qué necesitamos para que un espacio sea cerrado por máximos y mínimos. Necesitamos que sea cerrado por valores absolutes esencialmente. Vamos a tomar un espacio vectorial cerrado por valores absolutos, un retículo, y esto veremos que es denso en el espacio de funciones continuas.

\begin{Def}
  Sea $\cR\subseteq\cC(X)$, decimos que $\cR$ es un retículo si $\cR$ es un subespacio vectorial tal que $f,g\in\cR\Rightarrow f\lor g, f\land g\in\cR$.
\end{Def}

\begin{Th}
  Sea $(X,\mm)$ completo, $\cR\subseteq\cC(X)$ es un retículo que satisface que para $x,y\in X, a,b\in\bR$ existe $f\in\cR$ tal que $(f(x),f(y))=(a,b)$ entonces tenemos $\cR$ es denso en $\cC(X,\bR)$.
\end{Th}
%Dado $f\in\cC(X)$ y $\varepsilon>0$ existe $h\in\cR$ tal que $||h-f||_\infty<\varepsilon$ entonces $-\varepsilon<h(z)-f(z)<\varepsilon$ para $z\in X$.
\begin{ptcbp}
Tome $f\in\cC(X)$ y dados $x,y\in X$ existe $h_{x,y}\in\cR$ tal que
$$h_{x,y}(x)=f(x)\quad h_{x,y}(y)=f(y)$$
Fije $x\in X$. La función
$$g_x(z)=h_{x,y}(z)-f(z)$$
al evaluarla en $y$ nos da $0$. Dado $\varepsilon>0$, existe $\delta_y$ tal que
$$\mm(z,y)<\delta_y\Rightarrow|g_x(z)-g_x(y)|<\varepsilon$$
Por lo tanto tenemos
\begin{align*}
  z\in B(y,\delta_y) &\Rightarrow -\varepsilon<g_x(z)<\varepsilon \\
   &\Rightarrow h_{x,y}(z)-f(z)<\varepsilon \\
   &\Rightarrow h_{x,y}(z)<f(z)+\varepsilon
\end{align*}
Como la colección de las bolas $B(y,\delta_y)$ forma un cubrimiento por abiertos de $X$, entonces podemos reducir a $y_1,\cdots,y_m$ tales que $X\subseteq\cup_{i\in\bonj{m}}B(y_i,\delta_{y_i})$. Sea $h_x(z)=(\land_{i\in\bonj{m}}h_{x,y_i})(z)$ entonces $h_x(z)<f(z)+\varepsilon$ para $z\in X$. Esto ocurre pues $h_x(z)\leq h_{x,y_i}(z)$ en $B(y_i,\delta_{y_i})$. \par
Note que $h_x(x)=f(x)$, tome $g(z)=h_x(z)-f(z)$. Como $g(x)=0$, existe $\delta_x$ tal que
\begin{gather*}
  \mm(x,z)<\delta_x\Rightarrow |g(z)-g(x)|<\varepsilon \\
  \iff -\varepsilon<h_x(z)-f(z)<\varepsilon \\
  \iff -\varepsilon+f(z)<h_x(z),\quad z\in B(x,\delta_x)
\end{gather*}

Por compacidad existen $x_1,\cdots, x_\ell$ tales que
$X\subseteq\cup_{i\in\bonj{\ell}}B(x_i,\delta_{x_i})$.
 Esto nos dice que $-\varepsilon+f(z)<h_{x_i}(z)\leq h(z),\, z\in B(x_i,\delta_{x_i})$.
 Sea $h=\lor_{i\in\bonj{\ell}}h_{x_i}\Rightarrow f(z)-\varepsilon<h_x(z)$
 para cualquier $z\in X$. Escoja $z\in X$, tiene que existe $x_i$ tal que $h(z)=h_{x_i}(z)$ y en ese punto $h(z)=h_{x_i}(z)<f(z)+\varepsilon$. Así encontramos alguien en el retículo que está a distancia $\varepsilon$ de cualquier función.
\end{ptcbp}

\begin{Def}
  Un álgebra $A$ es un espacio vectorial con un producto $\circ\colon A^2\to A\colon (x,y)\mapsto x\circ y$ que es bilineal y asociativo. Decimos que $A$ tiene unidad si existe $\mathit{1}\in A$ tal que $\mathit{1}x=x\mathit{1}=x$. Además $A$ es un álgebra normada si $\nm{xy}\leq\nm{x}\nm{y}$ y $\nm{\mathit{1}}=1$ si $A$ tiene unidad.
\end{Def}

\begin{Ex}
  $\cC(X,\bR)$ es un álgebra normada con unidad donde $(f\circ g)(x)=f(x)g(x)$.
\end{Ex}

\begin{ptcb}
Basta ver que
$$\sup_{z\in X}|f(z)g(z)|\leq \sup_{z\in X}|f(z)|\sup_{z\in X}|g(z)|$$
Y esto es equivalente a $\nm{fg}_\infty\leq \nm{f}_\infty\nm{g}_\infty$.
\end{ptcb}

Cuál es la relación entre las álgebras y los retículos? Vamos a ver que las álgebras contienen el valor absoluto de sus elementos y así ver que es un retículo.\par
Si $A\subseteq X$ con $X$ un espacio normado y una métrica inducida por la norma. Tome dos sucesiones convergentes, entonces el producto ($\circ$ del álgebra)de ellas converge al producto de sus límites.

\begin{Ej}
  Corrobore el hecho anterior.
\end{Ej}

\begin{Lem}\label{lem:algNormadaEsReticulo}
  Sea $A\subseteq\cC(X)$ un álgebra normada con unidad. Entonces $\overline{A}$ es un retículo.
\end{Lem}

\begin{Ej}
 Verifique que $\overline{A}$ es un álgebra normada con unidad.
\end{Ej}

 \begin{ptcb}
 Hay que definir el producto en $\overline{A}$. Esto se logra con el ejercicio anterior. Hay que verificar que se cumple la desigualdad, pero esta se cumple en sucesión. Tomando límites se obtiene el resultado.
 \end{ptcb}

 Proseguimos con la prueba del lema \ref{lem:algNormadaEsReticulo}.
 \begin{ptcbp}
 Basta probar que si $h\in\overline{A}$ entonces $|h|\in\overline{A}$. Como $\nm{h}_\infty=c\Rightarrow h(z)\in\bonj{-c,c}$ para $z\in X$. Al ser $g(t)=|t|$, con $g\colon\bonj{-c,c}\to\bR$, continua, dado $n$ existe un polinomio $p_n(t)$ tal que $\nm{g-p_n}_\infty<\frac{1}{n}$. Esto es equivalente a $\sup_{z\in X}|g(z)-p_n(z)|<\frac{1}{n}$. Esto nos dice que $| |z|-p_n(z)|<\frac{1}{n}$ para $z\in\bonj{-c,c}$ entonces $| |h(y)|-p_n(h(y))|<\frac{1}{n}$ con $y\in X$.\par
 Falta ver que $p_n$ está en el álgebra, pero esto se sigue de que
\begin{gather*}
  p_n(h(y))=a_0+a_1h(y)+\cdots+a_m(h(y))^m \\
  =\left(a_0\mathit{1}+a_1h+a_2h\circ h+\cdots+a_m\underbrace{h\circ h\circ\cdots\circ h}_{m\,\,\text{veces}}\right)(y)
\end{gather*}
 \end{ptcbp}

 \begin{Th}[Stone-Weierstrass]\label{thm:StoneWeierstrass}
   Sea $(X,\mm)$ compacto y $A\subseteq \cC(X)$ un álgebra normada tal que $x\neq y\Rightarrow\exists h\in A$ tal que $h(x)\neq h(y)$ y $\mathit{1}\in A$ entonces tenemos que $A$ es denso en $\cC(X)$.
 \end{Th}

 \begin{ptcbp}
 Sabemos que $\overline{A}$ es un retículo. Falta probar que dados $x,y\in X$ y $a,b\in\bR$ existe $h_1\in A$ tal que $(h_1(x),h_1(y))=(a,b)$. Considere el sistema
 $$\alpha h(x)+\beta=a\quad \alpha h(y)+\beta =b$$
 Este sistema tiene solución pues la siguiente matriz tiene determinante no nulo.
 $$
 \begin{pmatrix}
   h(x) & 1 \\
   h(y) & 1
 \end{pmatrix}
 =
 |h(x)-h(y)|\neq 9
 $$
 \end{ptcbp}
La última condición del teorema se conoce como separación de puntos. En otras palabras tenemos la siguiente definición.

\begin{Def}
  Decimos que $h\in\cC(X)$ separa puntos si para $x\neq y$, se cumple que $h(x)\neq h(y)$.
\end{Def}

 \subsection{Día 11| 24-4-18}
 Nos interesan los números complejos pues nos interesan las funciones trigonométricas. Es más fácil ver estas funciones como las partes reales e imaginarias de $e^{iz}$.\par
 La diferencia entre el teorema siguiente y \ref{thm:StoneWeierstrass} es cerradura por conjugados complejos.
 \begin{Th}[Stone-Weierstrass]
   Sea $X$ espacio métrico compacto y $\cA\subseteq\cC(X,\bC)$ subálgebra que satisface:
   \begin{enumerate}
     \item Tiene unidad,
     \item Separa puntos: si $x\neq y$ entonces existe $f\in\cA\colon f(x)\neq f(y)$,
     \item $f\in\cA\Rightarrow \overline{f}\in\cA$
   \end{enumerate}
   Entonces $\cA$ es denso en $\cC(X,\bC)$.
 \end{Th}



 \begin{ptcbp}

 Sea $f\colon X\to\bC$ con $f=f_1+if_2$. Recuerde que $\overline{f}=f_1-if_2$. Luego se cumple que

 \begin{gather*}
   \frac{1}{2}(f+\overline{f})=f_1=\Re(f)\in\cA \\
   \frac{-i}{2}(f-\overline{f})=f_2=\Im(f)\in\cA
 \end{gather*}
 Así $f_1,f_2\in\cC(X,\bR)$. Vamos a probar que $\cA_1=\overline{\cA\cap\cC(X,\bR)}=\cC(X,\bR)$. Este conjunto es una subálgebra con unidad $\mathit{1}$ la función constante. Basta probar que separa puntos.\par
 Tome $x\neq y$, existe $f\in\cC(X,\cC)$ tal que
 $$f_1(x)+if_2(x)=f(x)\neq f(y)=f_1(y)+if_2(y)$$

 Entonces $f_1(x)\neq f_1(y)$ ó $f_2(x)\neq f_2(y)$. Por el teorema \ref{thm:StoneWeierstrass} se encuentra lo buscado. \par
 Tome $h\in\cC(X,\bC)$ con $h=h_1+ih_2$. Sabemos que existen $(f_n)_{n\in\bN},(g_n)_{n\in\bN}\subseteq\cA_1$ tales que $f_n\to h_1, g_n\to h_2$ en $\cC(X,\bR)$. Entonces $h_n=f_n+ig_n\in\cA$ converge a $h\in\cC(X,\bC)$.

 \end{ptcbp}

 \begin{Ex}
   Recuerde que $e^{iz}=\cos(z)+i\sin(z)$ y considere $p(e^{iz})=\sum_{n=-k}^{k}c_n(e^{iz})^n$. Observe que esto es $\sum_{n=-k}^{k}c_ne^{izn}$ y $e^{-izn}=\overline{e^{izn}}$ entonces
   $c_ke^{izn}+c_{-k}e^{izn}=a_k\sin(kz)+b_k\cos(kz)$ con $(a_k,b_k)=(c_k+c_{-k},i(c_k-c_{-k}))$. Por lo tanto
   $$p(e^{izn})=\sum_{k=0}^{n}a_k\sin(kz)+b_k\cos(kz)$$
   Separa puntos pues $e^{iz}\neq e^{iw}$ tomando $p(x)=x$ tenemos el resultado.
 \end{Ex}

 El ejemplo anterior nos dice que los polinomios trigonométricos son densos en $\cC(\partial B(0,1),\cC)$. Sea $X=\conj{f\colon\bonj{-\pi,\pi}\to\bR\colon f(-\pi)=f(\pi)}$, note que $\bonj{-\pi,\pi}\xrightarrow[]{e^{iz}}\partial B(0,1)$. Vea que $\partial B(0,1)$ y $X$ son isométricos.

 \subsubsection*{Tercera Sesión de Ejercicios}

 \begin{Ej}[2.4. Joseph Varilly]
   Mostrar que el complemento del conjunto de Cantor es abierto y denso.
 \end{Ej}

 \begin{ptcb}
 Considere la sucesión de los intervalos a los que les quitamos el tercio de la mitad. El conjunto de Cantor es la intersección de todos estos conjuntos, cada conjunto de la sucesión es la unión disjunta de intervalos de longitud $\frac{1}{3^i}$. Es claro que el conjunto de Cantor es cerrado pues es la intersección de uniones finitas de intervalos cerrados. \par
 Vea que $C^c$ es denso si y sólo si $C^o=\emptyset$. Así para $x\in C$ tenemos $B(x,\varepsilon)\cap C^c\neq\emptyset$. Como $x\in C$ entonces $x\in C_k$ con $k$ tal que $\frac{1}{3^k}<\varepsilon$. FIG11.1
 \end{ptcb}

 \begin{Ej}[2.9. Joseph Varilly]
   Mostrar que el conjunto $\conj{z}\cup\conj{z_n\colon n\in\bN}$ es compacto donde $z_n\to z$.
 \end{Ej}

 \begin{ptcb}
Vea que si cubrimos en el conjunto con $\varepsilon$-bolas. Alguna de todas las bolas contiene a $z$, inmediatamente contiene infinitos $z_n$'s salvo un número finito. Los otros se reparten en una bola cada una. Por lo tanto hay finitas bolas.
 \end{ptcb}

 \begin{Ej}[2.15. Joseph Varilly]
   Un subconjunto $K\subseteq\bR^n$ es compacto si y sólo si toda función en él es acotada.
 \end{Ej}

 \begin{ptcb}
Si toda función es acotada, la función $\mm(x,y)$ para $y\in K$ es acotada. Esta es la definición de ser acotado.\par
Si $K$ no fuese cerrado tome $x\in\overline{K}\setminus K$ y entoncecs $\forall\varepsilon\exists x\in K \mm(x,y)<\varepsilon$ entonces la función $\frac{1}{\mm(x,y)}>\frac{1}{\varepsilon}$ cumple ser continua pero no acotada.
 \end{ptcb}

 \begin{Ej}[2.4.6.17. Santiago Cambronero]
   Si $f\colon K\to \bR$ es continua, $K$ compacto, conexo, entonces $f(K)=\bonj{\min_{x\in K}(f(x)),\max_{x\in K}(f(x))}$.
 \end{Ej}

 \begin{ptcb}
Las funciones continuas preservan compactos. Entonces $f(K)$ es una unión de intervalos cerrados. Ahora, $f$ también preserva conexos por lo que es sólo un intervalo cerrado. Como $f$ alcanza su máximo y mínimo se deduce el resultado.
 \end{ptcb}

 \begin{Ej}[2.4.6.23. Santiago Cambronero]
   Si $E$ es compacto, toda sucesión en $E$ con un solo punto de acumulación es convergente.
 \end{Ej}

 \begin{ptcb}
Si $E$ es compacto, es completo. Para $(x_n)_{n\in\bN}\subseteq E$ existe $(x_{n_k})_{k\in\bN}\subseteq(x_n)_{n\in\bN}$ convergente. Entonces sea $x=\lim_{k\to\infty}x_{n_k}$. \par
Ahora asuma que $\neg(x_n\to x)$, así existe $(x_{n_\ell})_{\ell\in\bN}\subseteq (x_n)_{n\in\bN}$ tal que existe $\varepsilon_0>0\colon\mm(x_{n_\ell},x)>\varepsilon$. Por completitud existe $(x_{n_{\ell_m}})_{m\in\bN}$ subsucesión convergente. Entonces esta sucesión converge a $x$, pero la distancia es mayor a $\varepsilon_0$. Esto es una contradicción.
 \end{ptcb}

\subsection{Día 12| 3-5-18}

\subsubsection*{Cuarta Sesión de Ejercicios}

 \begin{Ej}[2.4.6.6. Santiago Cambronero]
   Sea $f\colon\bonj{0,1}\to\bonj{0,1}$ continua. Muestre que $f$ tiene al menos un punto fijo.
 \end{Ej}

 \begin{ptcb}
Considere la función $h(x)=f(x)-x$ y observe que $h$ es continua. Además $h(0)\geq 0$ y $h(1)\leq 0$. Por el teorema de Bolzano de los valores intermedios tenemos que $\exists c\in\bonj{0,1}\colon h(c)=0$. Luego $f(c)=c$ y por tanto $f$ tiene un punto fijo.
 \end{ptcb}

 \begin{Ej}[2.4.6.7. Santiago Cambronero]
   Sea $f\colon\bR\to\bR$ diferenciable y $|f'(x)|\leq 1$. Se concluye que $f$ es una contracción? Más aún, qué pasa si $|f'(x)|< 1$.
 \end{Ej}

 \begin{ptcb}
En el primer caso inmediatamente no, considere $f(x)=x$. Tenemos que $f'(x)=1$ para todo $x$. Luego $|f(x)-f(y)|=|x-y|$ y por tanto $f$ no es $q$-Lipschitz con $q<1$. Por lo tanto $x$ no es una contracción a pesar de cumplir la hipótesis.
\textcolor{red}{Finish}
 \end{ptcb}




\begin{Th}[Banach, 1922]
  Sea $(X,\mm)$ un espacio métrico completo y $f\colon X\to X$ una contracción. Esto es, $f$ es $\la$-Lipschitz con $0<\la<1$. Entonces $f$ tiene un único punto fijo.
\end{Th}

\begin{ptcbp}
En efecto, considere $x_0\in X$ y sea $x_{n+1}=f(x_n)$. Note que por definición de contracción y por inducción tenemos lo siguiente:
\begin{align*}
   \mm(x_{n+2},x_{n+1})&\leq \la\mm(x_{n+1},x_n)  \\
   &\leq \la^2\mm(x_{n},x_{n-1})  \\
   &\vdots  \\
   &\leq \la^n\mm(x_{n+1-n},x_{n-n})  \\
   &=\la^n\mm(x_1,x_0)
\end{align*}
Entonces para $n>m$ aplicando desigualdad triangular y el hecho anterior resulta en:
\begin{align*}
  \mm(x_n,x_m) &\leq\sum_{i=m}^{n-1}\mm(x_{i+1},x_i)  \\
   &\leq\sum_{i=m}^{n-1}\la^i\mm(x_{1},x_0)   \\
   &\leq\mm(x_1,x_0)\sum_{i=m}^{\infty}\la^i \\
   &=\mm(x_1,x_0)\left(\sum_{i=0}^{\infty}\la^i-\left(\frac{1-\la^m}{1-\la}\right)\right) \\
   &=\mm(x_1,x_0)\left(\frac{1}{1-\la}-\left(\frac{1-\la^m}{1-\la}\right)\right) \\
   &=\frac{\la^m\mm(x_1,x_0)}{1-\la}
\end{align*}
Como esta expresión tiende a $0$ pues $\la<1$ se sigue que $(x_n)_{n\in\bN}$ es de Cauchy y por tanto $x_n\to x\in X$. Ahora vea que
$$\lim_{n\to\infty}x_{n+1}=\lim_{n\to\infty}f(x_n)=f(\lim_{n\to\infty}x_{n})\Rightarrow x=f(x)$$
Así este es el punto fijo de $f$. \par
Ahora suponemos que existen $x,y$ puntos fijos de $f$. Entonces tenemos que
\begin{align*}
  \la\mm(f(x),f(y)) &=\la\mm(x,y)  \\
   &\geq\mm(f(x),f(y)) \\
   &\iff\mm(f(x),f(y))=0\iff x=y
\end{align*}
Se sigue que sólo hay un punto fijo.
\end{ptcbp}

 \begin{Ej}[2.4.6.12. Santiago Cambronero]
   Sea $E$ completo y $f\colon E\to E$. Si $f^k=\underbrace{f\circ f\circ\cdots\circ f}_{k\,\text{ veces}}$ es contractiva entonces $f$ tiene un único punto fijo.
 \end{Ej}

 \begin{ptcb}
Por el teorema de Banach, $f^k$ tiene un único punto fijo $x$. Esto nos dice que
\begin{align*}
  f^k(x)=x &\Rightarrow f(f^k(x))=f(x)  \\
   &\Rightarrow f^k(f(x))=f(x)  \\
   & \Rightarrow f(x)=x
\end{align*}

Donde la última igualdad se sigue pues $f^k$ tiene un único punto fijo. Así $x$ es único punto fijo de $f$.
 \end{ptcb}

\begin{Ej}
  Bernstein's polynomials.
\end{Ej}

\section{Parcial 2}

\subsection{Día 13| 10-5-18}

\subsubsection*{Funciones de Variación Acotada}

\begin{Def}
  Sean $f\colon\bonj{a,b}\to\bR$ y $\Gamma=\conj{x_0=a,x_1,\cdots,x_m=b}$ con $a=x_0<x_1<\cdots<x_m=b$. Definimos
  $$S(f,\Gamma)=\sum_{i\in\bonj{m}}|f(x_i)-f(x_{i-1})|$$
  y $\operatorname{Var}\bonj{f,\bonj{a,b}}=\sup_\Gamma S(f,\Gamma)$.\par
  Una función se dice de variación acotada si $\operatorname{Var}\bonj{f,\bonj{a,b}}<\infty$.
\end{Def}

InSERTAR Fig13.1

\begin{Ex}\label{ex:monotImpliesBndVar}
Sea $f\colon\bonj{a,b}\to\bR$ monótona, entonces $f$ es de variación acotada.
\end{Ex}

\begin{ptcb}
\begin{itemize}
  \item Si $f$ es creciente entonces $S(f,\Gamma)=\sum_{i\in\bonj{n}}(f(x_i)-f(x_{i+1}))=f(b)-f(a)$.
  \item En cambio para $f$ decreciente tenemos que $\operatorname{Var}\bonj{f,\bonj{a,b}}=f(a)-f(b)$.
\end{itemize}
\end{ptcb}

\begin{Ej}
  La función $f\bonj{-1,1}\to\bR$
tal que $f(x)=(x=0)?1\colon 0$ es de variación acotada.

\end{Ej}

\begin{ptcb}
Tome $\Gamma=\conj{-1,0,1}$ entonces $S(f,\Gamma)=2$.
\end{ptcb}

De hecho, $S(f,\Gamma)=2$ si $0\in\Gamma$ y $S(f,\Gamma)=0$ si $0\not\in\Gamma$.

\begin{Ej}
  La función $f(x)=(x\in\bQ\cap\bonj{0,1})?1\colon0$ entonces $\operatorname{Var}\bonj{f,\bonj{0,1}}=\infty$.
\end{Ej}

\begin{ptcb}
Considere $\Gamma=\conj{0=x_0,x_1,\cdots,x_n=1}$ con $x_k$ irracional para $k$ par y racional si $k$ es impar. La suma de $f$ respecto a $\Gamma$ estará dada por
$$S(f,\Gamma)=\sum_{i\in\bonj{n}}|f(x_{i-1})-f(x_{i})|$$
Todos los términos de esta suma son 1 por definición de $f$. Luego $\Var\bonj{f,\bonj{0,1}}=\sup_{\Gamma}|\Gamma|=n$, este término es no acotado y por tanto la variación es infinita.
\end{ptcb}

\begin{Lem}
  Si $f\colon\bonj{a,b}\to\bR$ es $\la$-Lipschitz entonces $f$ es de variación acotada.
\end{Lem}

\begin{ptcbp}
Note que
\begin{align*}
  S(f,\Gamma) &=\sum_{i\in\bonj{m}}|f(x_i)-f(x_{i-1})|  \\
   &\leq \la\sum_{i\in\bonj{m}}|x_i-x_{i-1}|  \\
   &=\la(b-a)
\end{align*}
Luego $f$ es de variación acotada.
\end{ptcbp}

\begin{Lem}
  Sea $f$ de variación acotada. Entonces $f$ es acotada.
\end{Lem}

\begin{ptcbp}
Tome $\Gamma=\conj{a,x,b}$ con $x\in\obonj{a,b}$ una partición. Luego tenemos que
$$|f(x)-f(a)|+|f(b)-f(x)|\leq\Var\bonj{f,\bonj{a,b}}$$
Entonces por desigualdad triangular al revés tenemos
\begin{align*}
  2|f(x)| &=|f(x)|+|f(x)| \\
   &\leq |f(a)|+|f(a)-f(x)|\\
   &\quad+|f(b)|+|f(b)-f(x)|  \\
   &\leq |f(a)|+|f(b)|+\Var\bonj{f,\bonj{a,b}}
\end{align*}
Por lo tanto $$|f(x)|\leq\frac{1}{2}\left(|f(a)|+|f(b)|+\Var\bonj{f,\bonj{a,b}}\right)$$
\end{ptcbp}

\begin{Th}
  Sean $f,g\colon\bonj{a,b}\to\bR$ funciones de variación acotada. Se cumple que:
  \begin{enumerate}
    \item $cf+g$ es de variación acotada para $c\in\bR$.
    \item $fg$ es de variación acotada.
    \item Si existe $\varepsilon>0$ tal que $|g(x)|>\varepsilon$ para $x\in\bonj{a,b}$. Entonces $\frac{f}{g}$ es de variación acotada.
  \end{enumerate}
\end{Th}

\begin{ptcbp}
 \quad Como $f,g$ son de variación acotada, entonces $\Var\bonj{f,\bonj{a,b}}\leq \frac{M_1}{c}$, y $\Var\bonj{g,\bonj{a,b}}\leq M_2$ con $M_1, M_2>0$. Sea $\Gamma=(x_i)_{i\in\bonj{n}}$ partición de $\bonj{a,b}$, entonces tenemos que si $h=cf+g$ la variación de $h$ es
      \begin{gather*}
        \sum_{i\in\bonj{n}}|h(x_i)-h(x_{i-1})|\\ =\sum_{i\in\bonj{n}}|cf(x_i)-cf(x_{i-1})+g(x_i)-g(x_{i-1})|\\
        \leq c\sum_{i\in\bonj{n}}|f(x_i)-f(x_{i-1})|+\sum_{i\in\bonj{n}}|g(x_i)-g(x_{i-1})|\\
        \leq c\left(\frac{M_1}{c}\right)+M_2
      \end{gather*}
  Tome $M=M_1+M_2$, de esta manera para cualquier partición se cumple que $S(h,\Gamma)=S(cf+g,\Gamma)\leq M$. Por lo tanto $h$ es de variación acotada.\par
 \quad Nuevamente, $\Var\bonj{f,\bonj{a,b}}\leq \frac{M_1}{c}$, y $\Var\bonj{g,\bonj{a,b}}\leq M_2$. Sea $h=fg$ y $\Gamma=(x_i)_{i\in\bonj{n}}$ partición de $\bonj{a,b}$.
      La suma de $h$ respecto a $\Gamma$ está dada por
      \begin{gather*}
        \sum_{i\in\bonj{n}}|h(x_i)-h(x_{i-1})|\\
        =\sum_{i\in\bonj{n}}|f(x_i)g(x_i) - f(x_{i-1})g(x_{i-1})|
      \end{gather*}
      Ahora, al término $
        |f(x_i)g(x_i) - f(x_{i-1})g(x_{i-1})|$ le sumamos cero de manera que sea $
        |f(x_i)g(x_i) - f(x_i)g(x_{i-1})+f(x_i)g(x_{i-1})-f(x_{i-1})g(x_{i-1})|$. Acotamos por desigualdad triangular y obtenemos que la expresión es menor a lo siguiente
        $$|f(x_i)| |g(x_i) - g(x_{i-1})|+|f(x_i)-f(x_{i-1})| |g(x_{i-1})|$$
        Recuerde que una función de variación acotada, es acotada. Sean $A,B>0$ respectivas cotas de $f,g$ entonces $|f(x)|\leq A,\ |g(x)|\leq B$. Luego tenemos que
        \begin{gather*}
        \sum_{i\in\bonj{n}}|h(x_i)-h(x_{i-1})|\\
        \leq B\sum_{i\in\bonj{n}}|f(x_i)-f(x_{i-1})|+A\sum_{i\in\bonj{n}}|g(x_i)-g(x_{i-1})|\\
        \leq BM_1+AM_2
      \end{gather*}
      Tomamos $M=BM_1+AM_2>0$ y así $h$ es de variación acotada.\par
      \quad Para el último apartado basta verificar que si $g$ es de variación acotada, entonces $\frac{1}{g}$ también. \par
      Como $g$ es de variación acotada, tome $\Gamma=(x_i)_{i\in\bonj{n}}$ partición de $\bonj{a,b}$ y así existe $M>0$ tal que $S(g,\Gamma)\leq M$. Consideremos $S(\frac{1}{g},\Gamma)$, tenemos que
      \begin{align*}
        \sum_{i\in\bonj{n}}\left|\frac{1}{g(x_i)}-\frac{1}{g(x_{i-1})}\right| &=\sum_{i\in\bonj{n}}\left|\frac{g(x_{i-1})-g(x_{i})}{g(x_i)g(x_{i-1})}\right|\\
        &\leq \sum_{i\in\bonj{n}}\left|\frac{g(x_{i-1})-g(x_{i})}{\varepsilon^2}\right|\\
        &=\frac{1}{\varepsilon^2}\sum_{i\in\bonj{n}}\left|g(x_{i})-g(x_{i-1})\right|\\
        &\leq\frac{M}{\varepsilon^2}
      \end{align*}
Luego $\Var\bonj{\frac{1}{g},\bonj{a,b}}\leq\frac{M}{\varepsilon^2}$ y por tanto $\frac{1}{g}$ es de variación acotada. Finalmente por el apartado anterior aplicado a $(f)(\frac{1}{g})$ obtenemos que el cociente de dos funciones de variación acotada, es de variación acotada.
\end{ptcbp}
Vamos a ver si entendemos la definición. Cuando escribimos $S(f,\Gamma)$ es $\sum_{i\in\bonj{n}}|f(x_i)-f(x_{i-1})|$. Tome $\Gamma=\conj{x_0,x_1,\cdots,x_n}$ y $y\not\in\Gamma$ con $x_{i-1}<y<x_i$ e $i<n$. Defina $\Gamma_1=\Gamma\cup\conj{y}$, entonces qué es $S(f,\Gamma)$?\par
Inserta FIG13.2\par
Vea que $S(f,\Gamma_1)=\sum_{j\in\bonj{i}}|f(x_j)-f(x_{j-1})|+\sum_{j=i+2}^n|f(x_j)-f(x_{j-1})|+|f(x_i)-f(y)|+|f(y)-f(x_{i-1})|$. Esto nos dice $S(f,\Gamma)\leq S(f,\Gamma_1)$. Por inducción podemos ver que si $\Gamma_1\subseteq\Gamma_2$ entonces $S(f,\Gamma_1)\leq S(f,\Gamma_2)$.\par
Ahora si $\bonj{c,d}\subseteq\bonj{a,b}$ y $\Gamma$ es una partición de $\bonj{c,d}$ tome $\Gamma_1=\Gamma\cup\conj{a,b}$. Esto define una partición sobre $\bonj{a,b}$.\par
FIG13.3\par
Así $S(f,\Gamma)\leq S(f,\Gamma_1)\leq\Var\bonj{f,\bonj{a,b}}$. Entonces $\Var\bonj{f,\bonj{c,d}}\leq\Var\bonj{f,\bonj{a,b}}$.\par
Con las particiones no hay que complicarse, regáleles puntos.
\begin{Lem}\label{lem:variacionSepara}
  Sea $c\in\obonj{a,b}$, entonces $\Var\bonj{f,\bonj{a,b}}=\Var\bonj{f,\bonj{a,c}}+\Var\bonj{f,\bonj{c,b}}$.
\end{Lem}
\begin{ptcbp}
Probamos dos desigualdades.\par
Tome $\Gamma$ partición de $\bonj{a,b}$, si $\Gamma=\conj{x_0,x_1,\cdots,x_m}$ entonces existe $j_0$ tal que $x_{j_0}\leq c<x_{j_{0}+1}$. Note que $\tilde{\Gamma}=\conj{x_0,x_1,\cdots,x_{j_0},c}$ particíón de $\bonj{a,c}$ y $\widehat{\Gamma}=\conj{c,x_{j_0+1},\cdots,x_m}$ es una particíón de $\bonj{c,b}$.\par
Además, partimos de una partición arbitraria que despedazamos en dos particiones que construimos.
\begin{align*}
 S(f,\Gamma)\leq S(f,\Gamma_1) &=S(f,\tilde{\Gamma})+S(f,\widehat{\Gamma})  \\
   &\leq\Var\bonj{f,\bonj{a,c}}+\Var\bonj{f,\bonj{c,b}}
\end{align*}
Por lo tanto $\Var\bonj{f,\bonj{a,b}}\leq\Var\bonj{f,\bonj{a,c}}+\Var\bonj{f,\bonj{c,b}}$.\par
Por otro lado si $\Gamma_1,\Gamma_2$ son particiones de $\bonj{a,c},\bonj{c,b}$ respectivamente entonces $\Gamma=\Gamma_1\cup\Gamma_2$ es una partición de $a,b$. Luego $S(f,\Gamma)=S(f,\Gamma_1)+S(f,\Gamma_2)$. \par
FIG 13.4\par
Esto nos dice que $S(f,\Gamma_1)+S(f,\Gamma_2)\leq\Var\bonj{f,\bonj{a,b}}$. Tomamos sup's uno por uno, dejando uno fijo y luego en otro. Obtenemos
$$\sup_{\Gamma_1}S(f,\Gamma_1)+\sup_{\Gamma_2}S(f,\Gamma_2)\leq\Var\bonj{f,\bonj{a,b}}$$
Esto nos da el resultado.
\end{ptcbp}

Qué pasa cuando nuestras funciones crecen y decrecen?
\begin{Def}
  Dado $x\in\bR$ definimos $x^+=(x\geq 0)?x\colon 0$ y $x^-=(x\leq 0)?-x\colon 0$. \par
  Entonces $|x|=x^++x^-$ y $x=x^+=x^-$.\par
  Definimos $P(f,\Gamma)=\sum_{i\in\bonj{n}}(f(x_i)-f(x_{i-1}))^+$ y $N(f,\Gamma)=\sum_{i\in\bonj{n}}(f(x_i)-f(x_{i-1}))^-$. Luego definimos $P\bonj{f,\bonj{a,b}}=\sup_\Gamma P(f,\Gamma),N\bonj{f,\bonj{a,b}}=\sup_\Gamma N(f,\Gamma)$ como la variación positiva y la variación negativa de $f$ respectivamente.
\end{Def}

De la definición anterior tenemos que:
$$
\begin{cases}
  S(f,\Gamma)=P(f,\Gamma)+N(f,\Gamma) \\
  f(b)-f(a)=P(f,\Gamma)-N(f,\Gamma)
\end{cases}
$$
Qué relación hay entre la variación positiva y la variación negativa? De la relación anterior, tomando sup's tenemos que $f(b)-f(a)=P\bonj{f,\bonj{a,b}}-N\bonj{f,\bonj{a,b}}$. Análogamente tenemos que $\Var\bonj{f,\bonj{a,b}}=P\bonj{f,\bonj{a,b}}+N\bonj{f,\bonj{a,b}}$. Esto lo probamos a continuación:
\begin{ptcbp}
Tomando cotas superiores obtenemos que
$$S(f,\Gamma)\leq P\bonj{f,\bonj{a,b}}+N\bonj{f,\bonj{a,b}}$$
Entonces en particular, esto acota al sup. Luego
$$\Var\bonj{f,\bonj{a,b}}\leq P\bonj{f,\bonj{a,b}}+N\bonj{f,\bonj{a,b}}$$
Por otro lado sabemos que $P(f,\Gamma)+N(f,\Gamma)\leq\Var\bonj{f,\bonj{a,b}}$. Sea $(\Gamma_n)_{n\in\bN}$ una sucesión de particiones de manera que
$$\lim_{n\to\infty}N(f,\Gamma_n)=N\bonj{f,\bonj{a,b}}$$
Entonces $\lim_{n\to\infty}P(f,\Gamma_n)=f(b)-f(a)+N\bonj{f,\bonj{a,b}}=P\bonj{f,\bonj{a,b}}$. Así la partición que nos sirve para aproximar $N$ nos deja aproximar $P$ por las igualdades que teníamos anteriormente. Esto concluye la desigualdad, porque entonces lo que tenemos aquí es que
$$P(f,\Gamma_n)+N(f,\Gamma_n)\leq\Var\bonj{f,\bonj{a,b}}$$
Al tomar un límite $n\to\infty$ esto resulta en $P\bonj{f,\bonj{a,b}}+N\bonj{f,\bonj{a,b}}\leq\Var\bonj{f,\bonj{a,b}}$.
\end{ptcbp}

Vamos a reducir la notación. Las indentidades anteriores se resumen en $V=P+N, f(b)-f(a)=P-N$. Sea $x\in\bonj{a,b}$, entonces
$$f(x)-f(a)=P\bonj{f,\bonj{a,x}}-N\bonj{f,\bonj{a,x}}$$
De aquí $f(x)=\left(P\bonj{f,\bonj{a,x}}+f(a)\right)-N\bonj{f,\bonj{a,x}}$. Entonces $x<y$ nos dice que
\begin{gather*}
  P\bonj{f,\bonj{a,x}}\leq P\bonj{f,\bonj{a,y}} \\
  N\bonj{f,\bonj{a,x}}\leq N\bonj{f,\bonj{a,y}}
\end{gather*}
\begin{Th}[Jordan]\label{thm:thm:BV2monotJordan}
  Sea $f\colon\bonj{a,b}\to\bR$. Entonces $f$ es de variación acotada si y sólo si es la resta de dos funciones positivas y crecientes.
\end{Th}

\begin{Ej}
  Escriba los detalles de la prueba anterior. Use el hecho de que $f(x)=\left(P\bonj{f,\bonj{a,x}}+f(a)\right)-N\bonj{f,\bonj{a,x}}$ es lo mismo que $f(x)=\left(P\bonj{f,\bonj{a,x}}+f^+(a)\right)-\left(N\bonj{f,\bonj{a,x}}+f^-(a)\right)$.
\end{Ej}

\begin{ptcb}
Suponga que $f=g-h$ con $g,h$ positivas y crecientes, esto significa que $g,h$ están acotadas por debajo por cero y además por el ejemplo \ref{ex:monotImpliesBndVar} tenemos que $g,h$ son de variación acotada. Entonces su suma es de variación acotada, luego $f$ lo es.\par
Ahora, sea $f$ de variación acotada, por la sugerencia tenemos que para todo $x\in\obonj{a,b}$ se cumple que
$$\left(P\bonj{f,\bonj{a,x}}+f^+(a)\right)-\left(N\bonj{f,\bonj{a,x}}+f^-(a)\right)$$
Tomemos como candidatos las siguientes funciones:
 \begin{gather*}
   g(x)=\left(P\bonj{f,\bonj{a,x}}+f^+(a)\right)\\
   h(x)=\left(N\bonj{f,\bonj{a,x}}+f^-(a)\right)
 \end{gather*}
\textcolor{red}{finish}

\end{ptcb}
\subsection{Día 14| 15-5-18}

De dónde sale la idea de funciones de variación acotada? Considere la integral de Riemman $\int_{a}^{b}\dd x$. Normalmente esto lleva a $b-a$, cualquiera nos diría que esa es la longitud del intervalo. Pero que pasa si medimos con $\int_{a}^{b}\dd\phi=\phi(b)-\phi(a)$. Al medir todos los pedacitos de intervalo, uno esperaría que la suma de las medidas fuera la medida de todo el intervalo. \par
Lo que necesitamos al hacer $\sum|\phi(x_i)-\phi(x_{i-1})|$. Pero quién dice que eso es finito? De aquí nace la idea de variación acotada. Para medir otras longitudes, queremos que los pedazos sean finitos.\par
Más adelante es natural pensar en variación acotada en campos como análisis harmónico y oscilación de funciones.\par
Lo último que vimos es que si $\phi$ es de variación acotada, entonces $\phi=\phi_1-\phi_2$ con $\phi_1,\phi_2$ positivas y crecientes.

\begin{Lem}
  Sea $\phi\:\bonj{a,b}\to\bR$ una función de variación acotada. Entonces $\phi$ tiene a lo sumo una cantidad contable de discontinuidades. Todas las discontinuidades son saltos o son removibles.
\end{Lem}

\begin{ptcbp}
Vea que si $\overline{x}$ es una discontinuidad de $\phi$, entonces $\overline{x}$ es una discontinuidad de $\phi_1$ ó de $\phi_2$. Lo que puede pasar es que los saltos de una se anulen con los saltos de la otra, o que una salte mucho más que la otra y entonces le queda el salto. Así, basta analizar estas funciones y probar el resultado para $\phi$ creciente. \par
Sea
$$D_k=\conj{x\in\bonj{a,b}\: \phi(x^+)-\phi(x^-)\geq\frac{1}{k}}$$
Asuma que $x_1<\cdots<x_n\in D_k$. Vamos a montar una partición del intervalo. Sea $x_{j-1}<y_j<x_{j}$ para $j\in\bonj{m}$. Entonces tenemos la siguiente partición
$$\conj{y_0=a<y_1<y_2<\cdots<y_{m+1}=b}$$
Ahora la pregunta es, cuánto es $|\phi(y_i)-\phi(y_{i-1})|\geq\frac{1}{k}$ para $j\in\bonj{m}$. Esto es mayor a $\frac{1}{k}$ pues los saltos dentro de $D_k$ son mayores a $\frac{1}{k}$ y estamos agarrando un pedazo más a cada lado. Es decir $\phi(y_j)\geq\phi(x_j^+)$ y $\phi(y_{j-1})\geq\phi(x_j^-)$. Luego tenemos que
\begin{gather*}
  \frac{1}{k}m\leq\sum_{k\in\bonj{m}}\phi(y_k)-\phi(y_{k-1})\leq\phi(b)-\phi(a) \\
  \Rightarrow m\leq k(\phi(b)-\phi(a))
\end{gather*}
Esto nos da una cota al número de discontinuidades.
\end{ptcbp}

Cuando tenemos funciones continuas, podemos aproximar la variación por la malla. Hay un $\delta$ tal que si la malla es pequeña atrapamos todos los de la malla.\par
Sea $f\:\bonj{a,b}\to\bR$ continua. Recuerde que $|\Gamma|=\sup_{j\in\bonj{m}}\conj{x_j-x_{j-1}}$.
\begin{Th}
Sea $f\:\bonj{a,b}\to\bR$ continua y de variación acotada. Entonces dado $M\leq\Var\bonj{f,\bonj{a,b}}=V$, existe $\delta>0$ tal que $$|\Gamma|<\delta\Rightarrow M\leq S(f,\Gamma)\leq V$$
\end{Th}

\begin{ptcbp}
Sea $\varepsilon>0$ tal que $M+\varepsilon<V$. Entonces existe $\Gamma_1=\conj{\tilde{x_0}<\tilde{x_1}<\cdots<\tilde{x_n}}$ tal que
$$M=\varepsilon< S(f,\Gamma_1)\leq V$$
Además, por continuidad uniforme, existe $\delta_1>0$ tal que
$$|x-x'|<\delta_1\Rightarrow|f(x)-f(x')|<\varepsilon$$
Tome $\delta_2=\frac{1}{2}\min\conj{\delta_1,|\Gamma_1|},\ |\Gamma|<\delta$. Ahora considere $\Gamma_2=\Gamma\cup\Gamma_1$, entonces tenemos
$$M+\varepsilon<S(f,\Gamma_1)\leq S(f,\Gamma_2)$$
Si $\Gamma=\conj{x_0<x_1<\cdots<x_n}$, podría darse que entre $x_j, x_{j+1}$ hubieran varios $\overline{x_i}$? Si fuese el caso que $x_j<\tilde{x_i}<\tilde{x_{i+1}}<x_{j+1}$ esto sería contradictorio con el tamaño de la malla $\Gamma$, pues entonces $|\Gamma_1|<|\Gamma|$ y escogimos $|\Gamma|$ más pequeña.\par
FIG14.2\par
Sabemos que dados $x_j,x_{j+1}$ existe a los sumo un $\tilde{x_i}$ tal que $x_j\leq\tilde{x_j}\leq x_{j+1}$. \par
FIG14.3\par
Ahora descomponemos la suma entre los intervalitos que tienen un $\tilde{x_j}$ en medio y los que no.
\begin{gather*}
  S(f,\Gamma_2)=\sum_{\substack{j\in\bonj{n}\\ \bonj{x_j,x_{j+1}}\cap\Gamma_1=\emptyset}}|f(x_j)-f(x_{j-1})|\\
  +\sum_{\substack{j\in\bonj{n}\\ \bonj{x_j,x_{j+1}}\cap\Gamma_1\neq\emptyset}}|f(x_{j+1})-f(\tilde{x_{i}})|+|f(\tilde{x_{i}})-f(\tilde{x_{i}})| \\
  \text{ver foto} \\
  \leq S(f,\Gamma)+\varepsilon
\end{gather*}

Por lo tanto tenemos que $M+\varepsilon<S(f,\Gamma_2)\leq S(f,\Gamma)+\varepsilon$ y así $M< S(f,\Gamma)$.

\end{ptcbp}

De esta manera probamos que
$$\lim_{|\Gamma|\to 0}S(f,\Gamma)=\Var\bonj{f,\bonj{a,b}}$$
Ahora si $\Gamma=\conj{a=x_0<x_1<\cdots<x_n=b}$, tenemos que
\begin{align*}
  S(f,\Gamma) & =\sum_{i\in\bonj{m}}|f(x_i)-f(x_{i-1})|\\
  &=\sum_{i\in\bonj{m}}|f'(\xi_i)|(x_i-x_{i-1})
\end{align*}
si es el caso que $f'(x)$ existe y es continua. Entonces
$$\lim_{|\Gamma|\to 0}S(f,\Gamma)=\int_{a}^{b}|f'(x)|\dd x$$

\begin{Cor}
  Sea $f\:\bonj{a,b}\to\bR$ de clase $\cC^1$. Entonces se cumple que
  \begin{gather*}
    V=\int_{a}^{b}|f'(x)|\dd x\
    P=\int_{a}^{b}(f'(x))^+\dd x\
    N=\int_{a}^{b}(f'(x))^-\dd x
  \end{gather*}
\end{Cor}

\subsubsection*{Curvas Rectificables}

Sea $\gamma\:\bonj{a,b}\to\bR^2,\ t\mapsto(f(t),g(t))$ continua. Cuanto mide la curva?\par
FIG 14.4\par
\begin{Def}
  Dado $\Gamma=\conj{a=x_0<x_1<\cdots<x_m=b}$, definimos
 \begin{gather*}
   L(\gamma,\Gamma)=\sum_{i\in\bonj{m}}\nm{\gamma(x_i)-\gamma(x_{i-1})} \\
 L(\gamma)=\sup_\Gamma\conj{L(\gamma,\Gamma)}
 \end{gather*}
 Decimos que la curva $\gamma$ es rectificable si $L(\gamma)<\infty$.
\end{Def}

Observe que tenemos las siguientes desigualdades.
\begin{gather*}
  \max\conj{|f(x_i)-f(x_{i-1})|,|g(x_i)-g(x_{i-1})|} \\
  \leq \nm{\gamma(x_i)-\gamma(x_{i-1})}\\
  \leq |f(x_i)-f(x_{i-1})|+|g(x_i)-g(x_{i-1})|
\end{gather*}

\begin{Rmk}
  Las deigualdades anteriores se siguen del comportamiento entre norma-1 y norma-2.
\end{Rmk}

\begin{Lem}
  La curva $\gamma$ es rectificable si y sólo si $f,g$ son de variación acotada.
\end{Lem}

\begin{Ej}
  Escriba los detalles de la prueba tomando sumas sobre las desigualdades anteriores y tomando sup's.
\end{Ej}

\begin{ptcb}
\begin{enumerate}
  \item[$(\Rightarrow)$] Si $\gamma$ es rectificable, tome $\Gamma=\conj{a=x_0<x_1<\cdots<x_m=b}$ partición de $\bonj{a,b}$. Entonces consideramos
      \begin{gather*}
        L(\gamma,\Gamma)=\sum_{i\in\bonj{m}}\nm{\gamma(x_i)-\gamma(x_{i-1})}\\
        =\sum_{i\in\bonj{m}}\left((f(x_i)-f(x_{i-1}))^2+(g(x_i)-g(x_{i-1}))^2\right)^{\frac{1}{2}}
      \end{gather*}
      Luego, como $L(\gamma)$ es el sup, tenemos que
      $$\sum_{i\in\bonj{m}}|f(x_i)-f(x_{i-1})|,\sum_{i\in\bonj{m}}|g(x_i)-g(x_{i-1})|\leq L(\gamma)$$
      Se sigue que $\Var\bonj{f,\bonj{a,b}},\Var\bonj{g,\bonj{a,b}}\leq L(\gamma)$. Por lo tanto $f,g$ son de variación acotada.
  \item[$(\Leftarrow)$] Ahora tenemos que
  \begin{align*}
   L(\gamma,\Gamma)& \leq \sum_{i\in\bonj{m}}|f(x_i)-f(x_{i-1})|\\
   &\quad+\sum_{i\in\bonj{m}}|g(x_i)-g(x_{i-1})|\\
   &\leq \Var\bonj{f,\bonj{a,b}}+\Var\bonj{g,\bonj{a,b}}
  \end{align*}
  Esta es una cota superior para $L(\gamma,\Gamma)$, entonces el sup está por debajo.
  $$L(\gamma)\leq \Var\bonj{f,\bonj{a,b}}+\Var\bonj{g,\bonj{a,b}}$$
  Por lo tanto $\gamma$ es rectificable.
\end{enumerate}
\end{ptcb}
\subsubsection*{La Integral de Riemann-Stieltjes}
La idea de Riemann-Stieltjes es cambie como mide el intervalo. La pregunta es si $f$ es Riemann-Stieltjes integrable respecto a la medida nueva del intervalo.\par
Sea $\phi\:\bonj{a,b}\to\bR$. Dados
\begin{enumerate}
  \item $f\:\bonj{a,b}\to\bR$ acotada, (Recuerde que la intergral de Riemann está definida para funciones acotadas.)
  \item $\Gamma=\conj{a=x_0<x_1<\cdots<x_n=b}$ partición de $\bonj{a,b}$,
  \item $\conj{\xi_i}_{i\in\bonj{n}}$ con $x_{i-1}<\xi_i<x_i$ para $i\in\bonj{n}$.
\end{enumerate}
Definimos $R(f,\Gamma,\phi,\conj{\xi_i}_{i\in\bonj{n}})=R(f,\Gamma,\phi)=\sum_{i\in\bonj{n}}f(x_i)(\phi(x_i)-\phi(x_{i-1}))$.

\begin{Def}
  Decimos que $f\bonj{a,b}\to\bR$ es Riemann-Stieltjes integrable con respecto a $\phi$. Si existe $I$ tal que dado $\varepsilon>0$ existe $\delta>0$ que satisface
  $$|\Gamma|<\delta\Rightarrow|R(f,\Gamma,\phi)-I|<\varepsilon$$
  Denotamos $I=\int_{a}^{b}f\dd\phi$.
\end{Def}
Hay un cambio fundamental entre esto y la integral de Riemann, aquí todo empieza a cambiar. \par
Defina
\begin{gather*}
  U(f,\Gamma,\phi)=\sum_{i\in\bonj{n}}M_i(\phi(x_i)-\phi(x_{i-1})) \\
  L(f,\Gamma,\phi)=\sum_{i\in\bonj{n}}m_i(\phi(x_i)-\phi(x_{i-1}))
\end{gather*}
Aquí $M_i=\sup\conj{f(x)\: x_{i-1}\leq x\leq x_i}$ y $m_i=\inf\conj{f(x)\: x_{i-1}\leq x\leq x_i}$. Las desigualdades no se dan directamente como ensanguchar y listo. Ahora no hay orden establecido entre $U,L$ y $R$ pues $\phi$ no es necesariamente monótona.\par
En general no hay orden establecido entre estas sumas. Si $\phi$ es creciente tenemos la desigualdad
$$L\leq R\leq U$$
Ya que todos los términos son positivos. En el caso de que $\phi$ sea decreciente, la desigualdad se invierte. El problema es que la función puede moverse mucho y al final no se sabe como se comportan las sumas.

\begin{Ej}
  Mostrar que $f$ es integrable respecto a $\phi$ si y sólo si dado $\varepsilon>0$ existe $\delta>0$ tal que
  $$|\Gamma|,|\Gamma'|<\delta\Rightarrow|R(f,\Gamma,\phi)-R(f,\Gamma',\phi)|<\varepsilon$$
\end{Ej}

Usando el ejercicio vamos a mostrar lo siguiente.
\begin{Lem}
  Si $f,\phi$ son discontinuas en un mismo punto entonces $f$ no es integrable respecto a $\phi$.
\end{Lem}

\begin{ptcbp}
Para el primer caso considere
$$\lim_{x\to\overline{x}^+}\phi(x)=\lim_{x\to\overline{x}^-}\phi(x)\neq\phi(\overline{x})$$
Esto es lo mismo que decir que existe $\varepsilon>0$ tal que para todo $\delta>0$ existe
\begin{enumerate}
  \item $x_1>\overline{x},\ |x_1-\overline{x}|<\frac{\delta}{2},\ |\phi(\overline{x})-\phi(x_1)|\geq \varepsilon$,
  \item $x_2<\overline{x},\ |x_2-\overline{x}|<\frac{\delta}{2},\ |\phi(\overline{x})-\phi(x_2)|\geq \varepsilon$,
\end{enumerate}
FIG 14.5\par
Además, como $f$ es discontinua, existe $\xi_\delta$ tal que
$$|\overline{x}-\xi_\delta|<\tilde{\delta},\ \text{y}\ |f(\xi_\delta)-f(\overline{x})|\geq\varepsilon$$
Aquí tenemos $\tilde{\delta}=\min\conj{\frac{\delta}{2},\frac{|\overline{x}-x_1|}{2},\frac{|\overline{x}-x_2|}{2}}$. Si se cumple que $\overline{x}<\xi_\delta<x_2$, tome $\Gamma$ una partición que contenga $x_1,x_2,\overline{x}$ y $\Gamma\cap\obonj{x_1,\overline{x}}=\Gamma\cap\obonj{\overline{x},x_2}=\emptyset$, además tomemos $R(f,\Gamma,\phi), R(f,\Gamma,\phi)$ de tal manera que en el intervalo $\bonj{\overline{x},x_2}$, $R$ toma el valor de $\xi_i=\xi_\delta$ y $R'$ toma el valor de $\xi_i'=\overline{x}$.
Las sumas anteriores son
\begin{gather*}
  R(f,\Gamma,\phi)=\sum_{i\in\bonj{n}}f(\xi_i)(\phi(y_i)-\phi(y_{i-1})) \\
  R'(f,\Gamma,\phi)=\sum_{i\in\bonj{n}}f(\xi_i')(\phi(y_i)-\phi(y_{i-1}))
\end{gather*}
Todos lo términos son iguales excepto en el punto de discontinuidad. Luego tenemos que
\begin{gather*}
  |R(f,\Gamma,\phi)-R'(f,\Gamma,\phi)|\\
  =|f(\xi_\delta)(\phi(x_2)-\phi(\overline{x}))-f(\overline{x})(\phi(x_2)-\phi(\overline{x}))|\\
 =|f(\xi_\delta)-f(\overline{x})| |\phi(x_2)-\phi(\overline{x})|\\
  \geq \varepsilon^2
\end{gather*}

\end{ptcbp}

\begin{Lem}
  Sea $f\bonj{a,b}\to\bR$ acotada y $\phi\bonj{a,b}\to\bR$. Entonces se cumple que
  \begin{enumerate}
    \item $\Gamma_1\subseteq\Gamma_2$, entonces
    \begin{gather*}
      U(f,\Gamma_2,\phi)\leq U(f,\Gamma_1,\phi)\\
      L(f,\Gamma_2,\phi)\geq L(f,\Gamma_1,\phi)
    \end{gather*}
    \item Si $\Gamma,\Gamma'$ son dos particiones entonces
    $$L(f,\Gamma,\phi)\leq U(f,\Gamma',\phi)$$
  \end{enumerate}
\end{Lem}

\begin{ptcbp}
Para el primer apartado asuma que $\Gamma_2=\Gamma_1\cup\conj{y}$.\par
FIG 14.6\par
Si $x_i<y<x_{i+1}$, entonces tenemos que
\begin{gather*}
  \sup_{\bonj{x_i,x_{i+1}}}(f(x)(\phi(x_{i+1})-\phi(x_i))) \\
  = \sup_{\bonj{x_i,x_{i+1}}}(f(x)((\phi(x_{i+1})-\phi(y))-(\phi(y)-\phi(x_i)))) \\
  \geq  \sup_{\bonj{x_i,y}}(f(x)(\phi(y)-\phi(x_i)))\\
  + \sup_{\bonj{y,x_{i+1}}}(f(x)(\phi(x_{i+1})-\phi(y))).
\end{gather*}
Por lo tanto $U(f,\Gamma_2,\phi)\leq U(f,\Gamma_1,\phi)$.\par
 El segundo apartado se sigue de la siguiente desigualdad. Si $\Gamma_1=\Gamma\cup\Gamma'$,
$$L(f,\Gamma,\phi)\leq L(f,\Gamma_1,\phi)\leq U(f,\Gamma_1,\phi)\leq U(f,\Gamma',\phi).$$
\end{ptcbp}

\begin{Th}\label{thm:primerCriterioRiemannS}
  Sea $f\:\bonj{a,b}\to\bR$ continua y $\phi\:\bonj{a,b}\to\bR$ de variación acotada. Entonces $f$ es integrable respecto a $\phi$ y además
  $$\left|\int_{a}^{b}f\dd\phi\right|\leq\sup_{x\in\bonj{a,b}}(f(x))\Var\bonj{\phi,\bonj{a,b}}$$
\end{Th}

\begin{ptcbp}
Sabemos que exsiten $\phi_1,\phi_2$ crecientes tales que $\phi=\phi_1-\phi_2$. Esto nos dice que
$$R(f,\Gamma,\phi)=R(f,\Gamma,\phi_1)-R(f,\Gamma,\phi_2)$$
De aquí tenemos que basta probar el resultado para $\phi$ creciente. Tenemos que mostar que existe $I$ tal que para todo $\varepsilon>0$ existe $\delta>0$ que satisface
$$|\Gamma|<\delta\Rightarrow|R(f,\Gamma,\phi)-I|<\varepsilon$$
Recordemos que $M_i=\sup_{x\in\bonj{x_{i-1},x_i}}f(x)=f(\eta_i)$ se alcanza por compacidad del dominio. Análogamente $m_i=\inf_{x\in\bonj{x_{i-1},x_i}}f(x)=f(\theta_i)$ con $x_{i-1}\leq \theta_i,\eta_i\leq x_i$.\par
Note que
\begin{gather*}
 0\leq  U(f,\Gamma,\phi)-L(f,\Gamma,\phi)\\
  =\sum_{i\in\bonj{n}}(f(\eta_i)-f(\theta_i))(\phi(x_i)-\phi(x_{i-1}))\\
  \leq\sum_{i\in\bonj{n}}|f(\eta_i)-f(\theta_i)|(\phi(x_i)-\phi(x_{i-1}))
\end{gather*}
Sabemos que dado $\varepsilon>0$ existe $\delta>0$ tal que
$$|x-x'|<\delta\Rightarrow|f(x)-f(x')|<\frac{\varepsilon}{2(\phi(a)-\phi(b))}$$
Luego si $|\Gamma|<\delta$, entonces $|\eta_i-\theta_i|<\delta$ y
\begin{gather*}
   0\leq  U(f,\Gamma,\phi)-L(f,\Gamma,\phi)\\
   \leq \frac{\varepsilon}{2(\phi(a)-\phi(b))}\sum_{i\in\bonj{n}}\phi(x_i)-\phi(x_{i-1})=\frac{\varepsilon}{2}
\end{gather*}
\end{ptcbp}

\subsection{Día 15| 17-5-18}

La clase pasada estabamos probando el teorema \ref{thm:primerCriterioRiemannS} y vimos que bastaba probarlo para $\phi$ creciente. Dado $\varepsilon>0$ existe $\delta>0$ tal que
$$|x-x'|<\delta\Rightarrow|f(x)-f(x')|<\frac{\varepsilon}{2(\phi(a)-\phi(b))}$$
Por qué nos servía que $\phi$ fuese creciente? Porque al esto nos garantiza el control de $R$ por $U,L$. Entonces vamos a probar que $U$'s convergen a algo y por lo tanto las $R$'s convergen también.\par
Vamos a probar que existen $I,\delta_1$ tales que
$$|\Gamma|<\delta_1\Rightarrow |U(f,\Gamma,\phi)-I|<\frac{\varepsilon}{2}$$
Cómo es que esto prueba que $R$ converge a algo? En este caso tendríamos que
\begin{align*}
  R(f,\Gamma,\phi)-I &\leq U(f,\Gamma,\phi)-I\\
  &\geq\frac{\varepsilon}{2}
\end{align*}
Tenemos la desigualdad inversa a partir de
\begin{align*}
  R(f,\Gamma,\phi)-I &\geq L(f,\Gamma,\phi)-I\\
  &\geq U(f,\Gamma,\phi)-I -\frac{\varepsilon}{2}\\
  &\geq -\varepsilon
\end{align*}
Y esto va a ocurrir siempre que $|\Gamma|<\min\conj{\delta,\delta_1}$.\par
Probamos que las $U$'s convergen.
\begin{ptcbp}
Tome $(\Gamma_n)_{n\in\bN}$ sucesión de particiones que en malla van para cero, es decir $|\Gamma_n|\xrightarrow[n\to\infty]{}0$. Cuál es el problema que tenemos cuando una sucesión de particiones van en malla hacia cero? Que las distancias de los elementos se van acercando. Pero las particiones no necesariamente tienen elementos comunes, no hay un orden. Para ellos inducimos uno a la fuerza.\par
Defina $\Gamma_k'=\cup_{i\in\bonj{k}}\Gamma_i$ y vea que $\Gamma_k\subseteq\Gamma_k'$. Entonces $\Gamma_k'\subseteq\Gamma_{k+1}'$. Cuando las particiones van decreciendo, las $U$'s se van haciendo más pequeñas. \par
Tome $I=\inf_{n\geq 1}U(f,\Gamma_n',\phi)$, por definición de $\inf$ tenemos que existe $k_0>0$ tal que
\begin{gather*}
I\leq U(f,\Gamma_{k_0}',\phi)\leq I+\frac{\varepsilon}{2}\\
\Rightarrow I\leq U(f,\Gamma_k',\phi)\leq I+\frac{\varepsilon}{2},\quad\text{ si } k>k_0
\end{gather*}
Entonces tenemos que
\begin{align*}
  U(f,\Gamma_k,\phi) & \geq U(f,\Gamma_k',\phi)\\
  &\geq I
\end{align*}
Por otro lado para obtener la otra desigualdad usamos $L$'s. Si $|\Gamma_k|<\delta$ entonces
\begin{align*}
  U(f,\Gamma_k,\phi) &\leq L(f,\Gamma_k,\phi)+\frac{\varepsilon}{2}\\
  &\leq L(f,\Gamma_k',\phi)+\frac{\varepsilon}{2}\\
  &\leq U(f,\Gamma_k,\phi)+\frac{\varepsilon}{2}\\
  & I+\varepsilon
\end{align*}
Entonces quién es el $\delta$ que nos sirve? Necesitamos dos condiciones
\begin{itemize}
  \item $|\Gamma_k|<\delta$,
  \item $k\geq k_0$
\end{itemize}
Luego  si $\delta_1=\frac{1}{2}\min\conj{\delta,|\Gamma_{k_0}'|}$ entonces tenemos que
$$|\Gamma_k|<\delta_1\Rightarrow |U(f,\Gamma_k,\phi)-I|<\varepsilon$$
El problema es que sólo estamos hablando de una sucesión en particular. Si hablamos de otra sucesión queremos que se comporte igual. Es análogo a continuidad por sucesiones.
\end{ptcbp}

\begin{Ej}
  Si $(\Gamma_n)_{n\in\bN},(\hat{\Gamma}_n)_{n\in\bN}$ son sucesiones tales que
  $$\lim_{n\to\infty}|\Gamma_n|=\lim_{n\to\infty}|\hat{\Gamma}_n|=0$$
  Entonces se cumple que
  $$\lim_{n\to\infty}U(f,\Gamma_n,\phi)=\lim_{n\to\infty}U(f,\hat{\Gamma}_n,\phi)$$
\end{Ej}

Note que si $\phi$ es creciente entonces siempre tenemos
\begin{align*}
   R(f,\Gamma_n,\phi)&\leq \sum_{i\in\bonj{n}}f(\xi_i)(\phi(x_i)-\phi(x_{i-1}))\\
   &\leq \sup_{x\in\bonj{a,b}}f(x)\sum_{i\in\bonj{n}}(\phi(x_i)-\phi(x_{i-1}))\\
   &\leq\sup_{x\in\bonj{a,b}}f(x)(\phi(b)-\phi(a)
\end{align*}
De manera análoga $\inf_{x\in\bonj{a,b}}f(x)(\phi(b)-\phi(a)\leq R(f,\Gamma_n,\phi)$. Si $f$ es RS-integrable respecto a $\phi$ tendremos que
$$\inf_{x\in\bonj{a,b}}f(x)(\phi(b)-\phi(a)\leq\int_{a}^{b}f\dd\phi\leq\sup_{x\in\bonj{a,b}}f(x)(\phi(b)-\phi(a)$$
Luego si $f$ es continua, existe $\xi\in\bonj{a,b}$ tal que
$$f(\xi)(\phi(b)-\phi(a))=\int_{a}^{b}f\dd\phi$$
\begin{Th}
  Asuma que las integrales $\int_{a}^{b}f_1\dd\phi, \int_{a}^{b}f_2\dd\phi$ existen. Entonces tenemos que $\int_{a}^{b}(cf_1+f_2)\dd\phi$ existe y es igual a $c\int_{a}^{b}f_1\dd\phi+\int_{a}^{b}f_2\dd\phi$. Además $\int_{a}^{b}f_1\dd (c\phi)=c\int_{a}^{b}f_1\dd\phi$.
\end{Th}

\begin{ptcbp}
Denotemos $I_1=\int_{a}^{b}f_1\dd\phi, I_2=\int_{a}^{b}f_2\dd\phi$. Sabemos que dado $\varepsilon>0$ existe $\delta_1,\delta_2$ tales que para particiones $\Gamma_1,\Gamma_2$ de $\bonj{a,b}$ se cumple
\begin{align*}
  |\Gamma_1|<\delta_1 &\Rightarrow |R(f_1,\Gamma_1,\phi)-I_1|<\frac{\varepsilon}{2|c|}\\
  |\Gamma_2|<\delta_2 &\Rightarrow |R(f_2,\Gamma_2,\phi)-I_2|<\frac{\varepsilon}{2}
\end{align*}
Cómo son las sumas de Riemann de $cf_1+f_2$? Necesitamos averiguar cuál es la partición.  Sea $\Gamma=\Gamma_1\cup\Gamma_2=\conj{a=x_0<x_1<\cdots<x_n=b}$ un refinamiento de $\Gamma_1,\Gamma_2$, entonces
$$R(cf_1+f_2,\Gamma,\phi)=\sum_{i\in\bonj{n}}(cf_1+f_2)(\xi_i)(\phi(x_i)-\phi(x_{i-1})).$$
Si $S_1,S_2$ son respectivamente las sumas de Riemann de $f_1, f_2$, entonces la suma de $cf_1+f_2$ es claramente $cS_1+S_2$. De esta manera
\begin{align*}
  |R(cf_1+f_2,\Gamma,\phi)-(cI_1+I_2)| &\leq\\
  |c| |R(f_1,\Gamma,\phi)-I_1|+|R(f_2,\Gamma,\phi)-I_2|\leq |c|\frac{\varepsilon}{2|c|}+\frac{\varepsilon}{2},
\end{align*}
cuando $|\Gamma|\to 0$.
\end{ptcbp}
También hay linealidad en $\phi$.

\begin{Th}\label{thm:linealRespPhi}
   Asuma que las integrales $\int_{a}^{b}f\dd\phi_1, \int_{a}^{b}f\dd\phi_1$ existen. Entonces tenemos que $\int_{a}^{b}f\dd(\phi_1+\phi_2)$ existe y es igual a $\int_{a}^{b}f\dd\phi_1+\int_{a}^{b}f\dd\phi_2$.
\end{Th}

\begin{Th}\label{thm:separarPuntosMediosIntegral}
  Asuma que $\int_{a}^{b}f\dd\phi$ existe. Si $a<c<b$ entonces
  $$\int_{a}^{b}f\dd\phi=\int_{a}^{c}f\dd\phi+\int_{c}^{b}f\dd\phi$$
\end{Th}

\begin{Ej}
  Pruebe los dos teoremas anteriores.
\end{Ej}

Procedemos con el teorema \ref{thm:linealRespPhi}.

\begin{ptcb}
En efecto, sea $\Gamma=(x_i)_{i\in\bonj{n}}$ una partición de $\bonj{a,b}$ con $x_1=a, x_n=b$. Como las integrales existen, llamémolas $I_1, I_2$ respectivamente. De esta manera, para $\varepsilon>0$, existe $\delta>0$ tal que
\begin{align*}
  |\Gamma|<\delta &\Rightarrow |R(f,\Gamma,\phi_1)-I_1|<\frac{\varepsilon}{2}\\
  &\Rightarrow |R(f,\Gamma,\phi_2)-I_1|<\frac{\varepsilon}{2}.
\end{align*}
Consideramos la suma $R(f,\Gamma,\phi_1+\phi_2)$. Vea que
\begin{align*}
  \lim_{|\Gamma|\to 0}R(f,\Gamma,\phi_1+\phi_2) &=\lim_{|\Gamma|\to 0}( R(f,\Gamma,\phi_1)+R(f,\Gamma,\phi_2))\\
  &=\lim_{|\Gamma|\to 0}( R(f,\Gamma,\phi_1))\\
  &\quad+\lim_{|\Gamma|\to 0}(R(f,\Gamma,\phi_2))\\
  &=I_1+I_2.
\end{align*}
De esta manera $R(f,\Gamma,\phi_1+\phi_2)$ converge y converge a la suma de las integrales.
\end{ptcb}

Ahora veremos la prueba del teorema \ref{thm:separarPuntosMediosIntegral}.

\begin{ptcb}

En efecto, sea $\Gamma$ una partición de $\bonj{a,b}$ con $\Gamma=(x_i)_{i\in\bonj{n}}$. Sea $R(f,\Gamma,\bonj{a,b})$ la suma de $f$ respecto a $\Gamma$. \par
Para ver que $\int_{a}^{c}f\dd\phi$ existe vamos a usar la condición de Cauchy sobre integrales. \par
Sea $\varepsilon>0$. Como $\int_{a}^{b}f\dd\phi$ existe, también existe $\delta>0$ tal que para cualesquiera $\Gamma_1,\Gamma_2$, particiones de $\bonj{a,b}$ con $|\Gamma_1|,|\Gamma_2|<\delta$ se cumple
$$|R(f,\Gamma_1,\bonj{a,b})-R(f,\Gamma_2,\bonj{a,b})|<\varepsilon.$$
Sean $\Gamma_1',\Gamma_2'$ particiones de $\bonj{a,c}$ y $\hat{\Gamma}$ partición de $\bonj{c,b}$ de manera que $\Gamma_i=\Gamma_i'\cup\hat{\Gamma}$. Tomamos los puntos intermedios de las particiones de manera que se cumple
\begin{gather*}
  R(f,\Gamma_1,\bonj{a,b})=R(f,\Gamma_1',\bonj{a,c})+R(f,\hat{\Gamma},\bonj{c,b}),\\
  R(f,\Gamma_2,\bonj{a,b})=R(f,\Gamma_2',\bonj{a,c})+R(f,\hat{\Gamma},\bonj{c,b}).
\end{gather*}
Si suponemos que $|\Gamma_1'|,|\Gamma_2'|<\delta$ y tomamos $\hat{\Gamma}$ de manera que $|\hat{\Gamma}|<\delta$, entonces se sigue que $|\Gamma_1|,|\Gamma_2|<\delta$ por los argumentos anteriores. Ahora de las igualdades anteriores tenemos
\begin{gather*}
  |R(f,\Gamma_1',\bonj{a,c})-R(f,\Gamma_2',\bonj{a,c})| \\
  \leq |R(f,\Gamma_1,\bonj{a,b})-R(f,\Gamma_2,\bonj{a,b})|\\
  <\varepsilon,\text{ cuando }|\Gamma_1'|,|\Gamma_2'|<\delta.
\end{gather*}
Es decir, existe la integral $\int_{a}^{c}f\dd\phi$. \par
La estrategia es análoga para mostrar que $\int_{c}^{b}f\dd\phi$ existe. Tomamos un par de particiones de $\bonj{c,b}$ y les regalamos puntos. Tomamos una partición suficientemente buena de $\bonj{a,c}$ y deducimos fórmulas como las anteriores.\par
Finalmente de cualquiera de las dos ecuaciones, tomando límites de norma de partición obtenemos
 $$\int_{a}^{b}f\dd\phi=\int_{a}^{c}f\dd\phi+\int_{c}^{b}f\dd\phi.$$
\end{ptcb}
\begin{Th}\label{thm:intPartesRS}
  Si $\int_{a}^{b}f\dd\phi$ existe, entonces $\int_{a}^{b}\phi\dd f$ existe. Además
  $$\int_{a}^{b}f\dd\phi+\int_{a}^{b}\phi\dd f=\phi(b)f(b)-\phi(a)f(a)$$
\end{Th}
\begin{ptcbp}
Sean $\Gamma=\conj{x_0<x_1<\cdots<x_n}$ y $x_{i-1}<\xi_i<x_i$. Entonces
$$R(f,\Gamma,\phi)=\sum_{i\in\bonj{n}}f(\xi_i)(\phi(x_i)-\phi(x_{i-1}))$$
Tome $\Gamma'=\conj{a=\xi_0<\xi_1<\cdots<\xi_n=b}$, ahora
\begin{align*}
  R(f,\Gamma,\phi) & =\sum_{i\in\bonj{n}}f(\xi_i)\phi(x_i)-\sum_{i\in\bonj{n}}f(\xi_i)\phi(x_{i-1})\\
  &=\sum_{i\in\bonj{n}}f(\xi_i)\phi(x_i)-\sum_{i\in\bonj{n-1}^*}f(\xi_{i+1})\phi(x_i)\\
  &=\sum_{i\in\bonj{n-1}}(f(\xi_{i+1})-f(\xi_{i}))\phi(x_i)\\
  &\quad +f(\xi_n)\phi(x_n)-f(\xi_1)\phi(x_0)\\
  &\quad\left(=f(\xi_n)\phi(b)-f(\xi_1)\phi(a)\right)\\
  &=-\sum_{i\in\bonj{n}^*}\phi(x_i)\left(f(\xi_{i+1})-f(\xi_i)\right)\\
  &\quad+f(\xi_n)\phi(b)-f(\xi_1)\phi(a)\\
  &\quad+\phi(x_0)(f(\xi_1)-f(\xi_0))\\
  &\quad+\phi(x_n)(f(\xi_{n+1})-f(\xi_n))
\end{align*}
Luego queremos relacionar el tamaño de las mallas, así podemos tomar límites y sacar las integrales. Note que
$$\sup_{i\in\bonj{n}}|x_i-x_{i-1}|\leq 2\sup_{i\in\bonj{n+1}}|\xi_i-\xi_{i-1}|$$
puesto que $x_i-x_{i-1}\leq\xi_{i+1}-\xi_{i-1}=(\xi_{i+1}-\xi_{i})+\xi_{i}-\xi_{i-1}$. De igual forma $|\Gamma'|\leq 2|\Gamma|$.
\end{ptcbp}

\begin{Ej}
  Si $\phi$ es diferenciable y $\int_{a}^{b}f\dd\phi$ existe, entonces $\int_{a}^{b}f\dd\phi=\int_{a}^{b}f(x)\phi'(x)\dd x$
  \end{Ej}

\begin{ptcb}
En efecto, sea $\Gamma=(x_i)_{i\in\bonj{n}}$ partición de $\bonj{a,b}$. Por el teorema del valor medio, podemos tomar $\eta_i\in\bonj{x_{i-1},x_i}$ para $i\in\bonj{n}$ de manera que
$$|\phi(x_i)-\phi(x_{i-1})|=\phi'(\eta_i)|x_i-x_{i-1}|.$$
Por definición de suma de Riemann-Stieltjes, los puntos, $(\xi_i)_{i\in\bonj{n}}$, en medio de los bloques de la partición simplemente deben cumplir estar en el intervalo $\bonj{x_{i-1},x_i}$. Así, sin perdida de generalidad tomamos $xi_i=\eta_i$ para todo $i\in\bonj{n}$. La suma será
\begin{align*}
  R(f,\Gamma,\phi) &=\sum_{i\in\bonj{n}}f(\eta_i)(\phi(x_i)-\phi(x_{i-1}))\\
  &=\sum_{i\in\bonj{n}}f(\eta_i)\phi'(\eta_i)(x_i-x_{i-1}),
\end{align*}
que por definición de suma de Riemann converge a $\int_{a}^{b}f\phi'\dd x$ cuando la malla de $|\Gamma|$ tiende a cero. Así como $R(f,\Gamma,\phi)$ converge a $\int_{a}^{b}f\dd\phi$, concluimos la igualdad.
\end{ptcb}

\subsubsection*{Medida}

Queremos medir longitudes, por ejemplo medir longitudes en $\bR$. Sea $m$ una ``medida''. Para un intervalo $\bonj{a,b}$ tendríamos $m(\bonj{a,b})=b-a$. A la vez intuitivamente $m(\rbonj{b,a})=b-a$. De aquí $m(\conj{b})=0$. \par
Si $A=\bigcup_{i\in\bonj{n}}\bonj{a_i,b_i}$ donde los intervalos no se traslapan, tendríamos que $m(A)=\sum_{i\in\bonj{n}}b_i-a_i$. Intuitivamente $m(A\cup B)=m(A)+m(B)$ dado $A\cap B=\emptyset$ y $m(\bigcup_{i\in\bonj{n}}\conj{a_i})=0$. \par
Qué pasa si $A=\bigcup_{i\in\bN}\bonj{a_i,b_i}$? Con intervalos disjuntos, la medida sería una serie. Así $m(\bQ)=0$. Para $E\subseteq\bR$ queremos aproximarlo con intervalos, pues podemos medir intervalos. Intuitivamente si un conjunto está metido en otro, la medida del grande es mayor a la del pequeño. Es decir $E\subset\bigcup_{i\in\bonj{n}}I_i$ con $I_i\cap I_j=\emptyset$ para todo $i,j$. Entonces la medida ``exterior'' de $E$ sería $m^+(E)=\inf\sum_{i\in\bN}m(I_i)$.\par
Recuerde que Arquimedes para medir areas, las aproximaba por poligonos y tomaba límites de alguna forma.\par
Si $\mathbf{1}$ es la función indicatriz, vamos a mostrar que $\int\mathbf{1}_B\dd x=m(B)$. A partir de esto $\int\mathbf{1}_{\bQ}\dd x=0$ y $\int\mathbf{1}_{\bR\setminus\bQ}\dd x=1$. \par
En general si $f_n\to f$, la pregunta que queremos responder es, cuándo se cumple $\int f_n\dd x=\int f\dd x$?


\subsection{Día 16| 22-5-18}

\subsubsection*{La medida de Lebesgue}

Sea $S=\conj{\lbonj{a,b}\:\ a<b}\cup\conj{\emptyset}\cup\conj{\rbonj{-\infty,b}\:\ b\in\bR}\cup\conj{\lbonj{a,\infty}}$ y defina la colección $\cA_1=\conj{\bigcup_{i\in\bonj{n}}I_i\:\ I_i\in S}$. Retomamos la idea de la medida exterior, así definimos
\begin{center}
\begin{itemize}
  \item $m_e(\lbonj{a,b})=b-a$,
  \item $m_e(\rbonj{-\infty,b})=\infty$,
  \item $m_e(\lbonj{a,\infty})=\infty$,
  \item Si $I_i\cap I_j=\emptyset$ entonces $m_e\left(\bigcup_{i\in\bonj{n}}I_i\right)=\sum_{i\in\bonj{n}}m_e(I_i)$ siempre que $I_i\in S$.
\end{itemize}
\end{center}
Hay que mostrar que esta noción está bien definida. Si $\lbonj{a,b}=\bigcup_{i\in\bonj{k}}I_i$ con $I_j$'s disjuntos, entonces $I_i=\lbonj{a_i,b_i}$ con $a_i<a_j$ si $i<j$. Como $\bonj{a_1,b_1}\cap\lbonj{a_2,b_2}=\emptyset$ entonces $a_2\geq b_1$. Al ser $\lbonj{a,b}$ conexo tenemos que $a_2=b_1$. Por lo tanto $b_i=a_{i+1}$. \par
Entonces dado $a_1=a, b_k=b$ tenemos una suma telescópica.
$$\sum_{i\in\bonj{k}}m(I_i)=\sum_{i\in\bonj{k}}b_i-a_i=b-a$$
Por otro lado si ocurre que $I_i\cap I_{i+1}\neq\emptyset$ entonces sólo tenemos $b_i>a_{a+1}$ luego $b_i-a_i>a_{i+1}-a_i$ y entonces $\sum_{i\in\bonj{n}}b_i-a_i\geq b-a$. \par
Sin embargo vea que esto no nos basta para aproximar. La medida exterior de un punto sería la medida de $\lbonj{a,a}$ o sea 0. Luego la medida de $\bN$ debería de ser cero, pero $\bN\subseteq\lbonj{a,\infty}$ y esto tiene medida infinita.\par
Arreglamos este problema al considerar uniones infinitas.
\begin{Def}
  Definimos la medida exterior de un conjunto $E\subseteq\bigcup_{i\in\bN}I_i$ para $I_i\in S$ como
  $$m_e(E)=\inf\sum_{i\in\bN}m(I_i)$$
  En esta definición $m(I_i)=b_i-a_i$ si $I_i=\lbonj{a,b}$ y $m(\emptyset)=0$.
\end{Def}

Note que si $E_1\subseteq E_2$, entonces $m_e(E_1)\leq m_e(E_2)$.

\begin{Ej}
  Pruebe el hecho anterior.
\end{Ej}

\begin{Lem}
  La medida exterior de un intervalo es su longitud. Es decir $m_e(\lbonj{a,b})=b-a$.
\end{Lem}

\begin{ptcbp}
Sea $E=\lbonj{a,b}$ entonces $E\subseteq\lbonj{a,b}$, luego $m_e(E)\leq b-a$. Es menor igual porque la medida exterior es el $\inf$.\par
Vamos a convertir $E$ en un compacto y los $I_i$'s abiertos por topología usual. Luego por compactificación tendremos un cubrimiento finito.\par
Si $E\subseteq\cup_{i\in\bN}I_i$, tome $\varepsilon>0$ tal que $a<b-2\varepsilon$. Es decir $\bonj{a,b-\varepsilon}\subseteq E\subseteq\cup_{i\in\bN}I_i$.\par
El siguiente problema es que los $I_i$'s no son abiertos. En vez de hacerlos más pequeños, los alargamos. Sea $I_i^*\in S$ tal que $I_i\subseteq(I_i^*)^o$ y $m(I_i^*)\leq (1+\varepsilon)m(I_i)$. \par
FIG 16.1\par
Entonces $\bonj{a,b-\varepsilon}\subseteq\cup_{i\in\bN}(I_i^*)^o$,
$$\Rightarrow\exists k_0\:\quad\bonj{a,b-\varepsilon}\subseteq\cup_{i\in\bonj{k_0}}(I_i^*)^o\subseteq\cup_{i\in\bonj{k_0}}I_i^*$$
Así $\bonj{a,b-2\varepsilon}\subseteq\cup_{i\in\bonj{k_0}}(I_i^*)^o$. Aquí lo que tenemos es que
\begin{align*}
 b-a-2\varepsilon&\leq \sum_{i\in\bonj{k_0}}m(I_i^*)\\
 &\leq (1+\varepsilon) \sum_{i\in\bonj{k_0}}m(I_i)\\
 &\leq (1+\varepsilon) \sum_{i\in\bN}m(I_i)\\
 \Rightarrow\frac{b-a-2\varepsilon}{1+\varepsilon} &\leq \sum_{i\in\bN}m(I_i)
\end{align*}
Por lo tanto $\frac{b-a-2\varepsilon}{1+\varepsilon} \leq m_e(\lbonj{a,b})$. Por definición de inf obtenemos el resultado.
\end{ptcbp}
%Parecería que debe funcionar, por qué debemos probarlo?
Mientras mantengamos el control de qué tanto están variando los conjuntos vamos a poder aproximar.\par
Sean $E_k,E\subseteq\bR$ tales que $E\subseteq\bigcup_{k\in\bN} E_k$. Vamos a probar que
$$m_e(E)\leq\sum_{k\in\bN}m_e(E_k)$$
\begin{ptcbp}
Sin perdida de generalidad $m_e(E_k)<\infty$. De lo contrario la desigualdad es trivial. Entonces existe $E_k\cup_{i\in\bN}I_i^k$, cubrimiento, tal que $\sum_{i\in\bN}m(I_i^k)\leq m_e(E_k)+\frac{\varepsilon}{2^k}$. Por definición de $\inf$, al sumar este $\varepsilon$ agarramos a uno de los $I_i^k$'s.\par
Tomando la unión sobre $k$ tenemos que
\begin{align*}
 E&\subseteq\cup_{k\in\bN}\cup_{i\in\bN}I_i^k\\
 \Rightarrow m_e(E)&\leq \sum_{k,i\in\bN}m(I_i^k)\\
 &\leq\sum_{k\in\bN}\left(m(E_k)+\frac{\varepsilon}{2^k}\right)
\end{align*}
Por lo tanto $m_e(E)\leq\left(\sum_{k\in\bN}m_e(E_k)\right)+\varepsilon$.
\end{ptcbp}
No se complique con quien es quien, la idea siempre es aproximar. \par
Formalmente el lema es el siguiente.
\begin{Lem}
  Si $E\subseteq\bigcup_{k\in\bN} E_k$ entonces $m_e(E)\leq\sum_{k\in\bN}m_e(E_k)$.
\end{Lem}

\begin{Ex}
  Vamos a medir algunos conjuntos.
  \begin{enumerate}
  \begin{multicols}{2}
    \item $m_e(\conj{a})$,
    \item $m_e(\bonj{a,b})$,
    \item $m_e(\obonj{a,b})$,
    \item $m_e(\rbonj{a,b})$.
    \end{multicols}
  \end{enumerate}
\end{Ex}

\begin{ptcb}
\begin{enumerate}
  \item Como $\conj{a}\subseteq\lbonj{a,a+\varepsilon}$, entonces $m_e(\conj{a})\leq\varepsilon$ para todo $\varepsilon>0$. Luego $m_e(\conj{a})=0$.
  \item Como $\lbonj{a,b}\subseteq\bonj{a,b}$ entonces
  $$b-a=m_e(\lbonj{a,b})\leq m_e(\bonj{a,b})$$
  Además $\bonj{a,b}=\lbonj{a,b}\cup\conj{b}$, de aquí
  $$m_e(\bonj{a,b})\leq m_e(\lbonj{a,b})+m_e(\conj{b})=b-a$$
\end{enumerate}
\end{ptcb}

\begin{Ej}
  Pruebe los casos que faltan del ejemplo anterior.
\end{Ej}
Queremos pasarnos a más dimensiones. Defina $S_d=\conj{\bigtimes_{i\in\bonj{d}}I_i\:\ I_i\in S}$. note que $\bigtimes_{i\in\bonj{d}}\lbonj{a_i,b_i}\in S_d$.\par
FIG 16.2
\begin{Def}
  Dado $E\subseteq\bR^d$ con $E\subseteq\bigcup_{i\in\bN}I_i$ dados $I_i\in S_d$, definimos
  $$m_e(E)=\inf\sum_{i\in\bN}m(I_i)$$
  Aquí $m(\bigtimes_{i\in\bonj{d}}I_i)=\prod_{i\in\bonj{d}}m(I_i)$. Es decir $m(\bigtimes_{i\in\bonj{d}}\lbonj{a_i,b_i})=\prod_{i\in\bonj{d}}(b_i-a_i)$.
\end{Def}

\begin{Lem}
  Si $E_1\subseteq E_2$, entonces $m_e(E_1)\leq m_e(E_2)$.
\end{Lem}

\begin{Lem}
  $m_e(\bigtimes_{i\in\bonj{d}}\lbonj{a_i,b_i})=\prod_{i\in\bonj{d}}(b_i-a_i)$
\end{Lem}

\begin{Lem}
Si $E\subseteq\bigcup_{k\in\bN}E_k$, entonces $m_e(E)\leq\sum_{k\in\bN}m_e(E_k)$.
\end{Lem}

\begin{Ej}
  Pruebe los lemas anteriores adaptando las pruebas de una dimensión.
\end{Ej}

\begin{Ex}
  Vamos a medir $\bonj{a,b}\times\conj{c}\subseteq\bR^2$

\end{Ex}

\begin{ptcb}
Observe que
\begin{gather*}
  m_e(\bonj{a,b}\times\conj{c})\\
  \leq \lbonj{a,b+\varepsilon}\times\lbonj{c,c+\varepsilon}\\
  \varepsilon(b-a+\varepsilon)
\end{gather*}
Luego este conjunto tiene medida cero.
\end{ptcb}

\begin{Lem}
  Sea $E\subseteq\bR^d$, entonces existe $G$ abierto tal que $E\subseteq G$ y
  $$m_e(E)\leq m_e(G)\leq m_e(E)+\varepsilon$$
  Es decir $m_e(E)=\inf_{E\subseteq G}m_e(G)$ con $G$ abierto.
\end{Lem}

La prueba empieza por aproximarlo con cajas. Recuerde que ahora estamos en varias dimensiones y debemos considerar cajas.
\begin{ptcbp}
Sea $\varepsilon>0$, entonces existe $I_k\in S_d$ tal que $E\subseteq\cup_{k\in\bN}I_k$ y $m_e(E)\leq \sum_{k\in\bN}m(I_k)\leq m_e(E)+\frac{\varepsilon}{2}$. Entonces sea $I_k^*\in S_d$ tal que $I_k\subseteq(I_k^*)^o$ con $m(I_k^*)\leq m(I_k)+\frac{\varepsilon}{2^{k+1}}$. \par
El $G$ que vamos a tomar será $G=\cup_{k\in\bN}(I_k^*)^o$. Luego
\begin{align*}
  m_e(G) &\leq \sum_{k\in\bN}m_e((I_k^*)^o)\\
  &\leq \sum_{k\in\bN}m_e(I_k^*)\\
  &\leq \sum_{k\in\bN}m_e(I_k)+\frac{\varepsilon}{2^{k+1}}\\
  &\leq m_e(E)+\varepsilon
\end{align*}
\end{ptcbp}

Esto nos permite cambiarnos a conjuntos abiertos. No es que sea muy ventajoso, la intuición geométrica viene dada más por las cajas que por abiertos cualesquiera.

\begin{Def}
  Un conjunto $G$ se dice ser $G_\delta$ si existen $(G_k)_{k\in\bN}$ abiertos, tales que $G=\bigcap_{k\in\bN}G_k$.
\end{Def}

Si $m_e(E)<\infty$, entonces existe $G_k$ tal que
$$m_e(E)\leq m_e(G_k)\leq m_e(E)+\frac{1}{k}$$
Luego si $H=\cap_{k\in\bN}G_k$ que contiene a $E$, tenemos que
\begin{align*}
  m_e(E)\leq m_e(H) &\leq m_e(G_k)\\
  &\leq m_e(E)+\frac{1}{k}
\end{align*}

Esto nos da el siguiente lema.

\begin{Lem}
  Si $E\subseteq\bR^d$, existe $H$ un conjunto $G_\delta$ tal que $m_e(E)=m_e(H)$.
\end{Lem}

Los conjuntos que vamos a medir, son los que intuitivamente vamos a poder medir.

\begin{Def}
  Sea $E\subseteq\bR^d$, decimos que $E$ es medible si dado $\varepsilon>0$ existe $G\subseteq\bR^d$ abierto tal que $E\subseteq G$ y $m(G\setminus E)<\varepsilon$.
\end{Def}

FIG16.3\par
Todos los conjuntos son medibles? Si $m_e(E)<\infty$, existe $G$ tal que $m_e(G)\leq m_e(E)+\varepsilon$. Note que $m_e(G)=m_e((G\setminus E)\cup E)$. De aquí tenemos que
$m_e(G)\leq m_e(G\setminus E)+m_e(E)$. Esta desigualdad no es comparable con la anterior. Luego no todo conjunto es medible.

\begin{Ex}
  Cualquier intervalo de $\bR$ es medible.
\end{Ex}

\begin{ptcb}
Considere $\lbonj{a,b}\subseteq\obonj{a-\varepsilon,b}=G$. Entonces tenemos que
$$m_e(G\setminus\lbonj{a,b})=m_e(\obonj{a-\varepsilon,a})=\varepsilon$$
\end{ptcb}

\begin{Ej}
  Pruebe los demás casos con una prueba análoga.
\end{Ej}

\begin{Lem}
  La unión de conjuntos medibles es un conjunto medible. Sea $(E_i)_{i\in\bN}$ con $E_i$ medible, entonces $\bigcup_{i\in\bN}E_i$ es medible.
\end{Lem}

\begin{ptcbp}
Sea $\varepsilon>0$, entonces existe $G_i$ abierto tal que $m_e(G_i\setminus E_i)<\frac{\varepsilon}{2^i}$. Entonces como $G=\bigcup_{k\in\bN}G_k$ es abierto, buscamos $m_e(G\setminus\bigcup_{i\in\bN}E_i)$.
\begin{gather*}
  m_e\left(G\setminus\bigcup_{i\in\bN}E_i\right)\\
  =m_e\left(\bigcup_{i\in\bN}G_i\setminus\bigcup_{i\in\bN}E_i\right)\\
  \leq m_e\left(\bigcup_{i\in\bN}G_i\setminus E_i\right)\leq \sum_{i\in\bN}m_e(G_i\setminus E_i)=\varepsilon
\end{gather*}
\end{ptcbp}

\begin{Lem}
  Sean $I_k\subseteq\bR^d$ con $I_k=\bigtimes_{i\in\bonj{d}}\bonj{a_i^k,b_i^k}$ tal que $I_i^o\cap I_j^o=\emptyset$. Entonces $m_e(\bigcup_{k\in\bonj{m}}I_k)=\sum_{k\in\bonj{m}}m_e(I_k)$.
\end{Lem}

\begin{Ej}
  Pruebe el lema anterior.
\end{Ej}

\begin{Lem}\label{lem:medSeparaDistMayorACero}
  Sean $E_1,E_2\subseteq\bR^d$ tal que $\mm(E_1,E_2)>0$. Entonces
  $$m_e(E_1\cup E_2)=m_e(E_1)+m_e(E_2)$$
\end{Lem}

\begin{ptcbp}

Sea $\varepsilon>0$, entonces existen $I_k\in S_d$ tales que
$$E_1\cup E_2\subseteq\bigcup_{k\in\bN}I_k$$
Con $m_e(E_1\cup E_2)\leq\sum_{k\in\bN}m(I_k)\leq m_e(E_1\cup E_2)+\varepsilon$.\par
FIG 16.4\par
Dividiendo son $I_k$'s en cajas más pequeñas podemos asumir que $E_1\cap I_j\neq\emptyset\Rightarrow E_2\cap I_j=\emptyset$. Luego observe que
$$\bN=\conj{j\in\bN\:\ E_j\cap E_1}\cup\conj{j\in\bN\:\ I_j\cap E_2\neq\emptyset}$$
Denotemos estos conjuntos como $\Lambda_1, \Lambda_2$. Luego
$$\bigcup_{k\in\bN}I_k=\bigcup_{k\in\Lambda_1}I_k\cup\bigcup_{k\in\Lambda_2}I_k $$
Ahora $E_1\subseteq\bigcup_{k\in\Lambda_1}I_k$ pues si $e\in E_1$ entonces $e\in \bigcup_{k\in\Lambda_1}I_k$ ó bien $e\in \bigcup_{k\in\Lambda_2}I_k$. Como $I_i\cap E_1=\emptyset$ cuando $i\in\Lambda_2$, tenemos que $e\in\bigcup_{k\in\Lambda_1}I_k$.\par
Finalmente tenemos que
\begin{gather*}
  m_e(E_1)\leq \sum_{k\in\Lambda_1}m(I_k)\\
  m_e(E_2)\leq \sum_{k\in\Lambda_2}m(I_k)
\end{gather*}
\begin{align*}
  \Rightarrow m_e(E_1)+m_e(E_2) & \leq \sum_{k\in\bN} m(I_k)\\
&\leq m_e(E_1\cup E_2)+\varepsilon
\end{align*}
\end{ptcbp}

\subsection{Día 17| 24-5-18}

Por el momento sabemos que la unión de conjuntos medibles es medible. Vamos a ver los complementos de medibles también lo son. Para ello, veremos que los cerrados lo son y así vamos a ver un criterio de medición respecto a cerrados. Usando esto vamos a probar que la medida de la unión de dos conjuntos disjuntos es la suma de las medidas.

\begin{Lem}\label{ej:parcial1I2018Ej7}
  Sea $G\subseteq\bR^d$ abierto, entonces $G=\bigcup_{k\in\bN}I_k$ con $I_k=\bigtimes_{i\in\bN}\bonj{a_i^k,b_i^k}$ y $I_k^o\cap I_\ell^o=\emptyset$ para $k\neq\ell$.
\end{Lem}

\begin{Ej}
  Observe que este lema es el séptimo ejercicio del primer examen. Demuestrelo.
\end{Ej}

\begin{Th}
  Sea $F\subseteq\bR^d$ cerrado, entonces $F$ es medible.
\end{Th}
\begin{ptcbp}
Primero asumamos que $F$ es acotado como una hipótesis extra. Dado $\varepsilon>0$ existe $G$ abierto tal que $m_e(G)\leq m_e(F)+\varepsilon$.\par
FIG17.1\par
Como $G\setminus F$ es abierto, entonces $G\setminus F=\bigcup_{k\in\bN}I_k$ con $I_k$ iguales que en el lema \ref{ej:parcial1I2018Ej7}. Note que $\bigcup_{k\in\bonj{m}}I_k\cap F=\emptyset$ porque están en el complemento. Como ambos son compactos entonces se cumple que $\mm(\bigcup_{k\in\bonj{m}}I_k,F)>0$. Aplicamos el lema \ref{lem:medSeparaDistMayorACero} tenemos que
\begin{align*}
m_e(F)+m_e(\bigcup_{k\in\bonj{m}}I_k) &=  m_e(\bigcup_{k\in\bonj{m}}I_k\cup F)\\
&\leq  m_e(G)\\
\Rightarrow m_e(\bigcup_{k\in\bonj{m}}I_k) &\leq  m_e(G)-m_e(F)\leq\varepsilon\\
\Rightarrow\sum_{k\in\bonj{m}} m_e(I_k)\leq\varepsilon\\
\Rightarrow\sum_{k\in\bN} m_e(I_k)\leq\varepsilon\\
\end{align*}
Finalmente $m_e(G\setminus F)=m_e(\bigcup_{k\in\bN}I_k)\leq \sum_{k\in\bN} m_e(I_k)\leq\varepsilon$.\par
Cómo nos quitamos de en cima el hecho de que $F$ no sea acotado? Hay forma de escribir $F$ como unión de conjuntos medibles?\par
Note que $F=\bigcup_{k\in\bN}F\cap\overline{B(0,k)}$, estos conjuntos son compactos y por lo anterior son medibles. Finalmente la unión infinita también es medible. Por lo tanto $F$ es medible.
\end{ptcbp}

La estrategia para probar que algo es medible es despedazarlo en objetos que ya sabemos que son medibles. De hecho, esta misma estrategia se usará para probar que los complentos son medibles.

\begin{Lem}
  Sea E medible, entonces su complemento es medible.
\end{Lem}

\begin{ptcbp}
Dado $k$, existe $G_k$ tal que $m_e(G_k\setminus E)\leq \frac{1}{k}$. Cómo relaciono aproximaciones de $E$ por las del complemento?\par
 Nosotros sabemos que si $E\subseteq G_k$ entonces $E^c\supseteq G_k^c$, tomar complementos invierte la inclusión. Este conjunto es un cerrado y por tanto medible. Podemos trabajar con esto.\par
Tome $H=\bigcup_{k\in\bN}G_k^c$, inmediatamente $H\subseteq E^c$ y es medible. A quién no hemos logrado pescar? A los que están en $E^c\setminus H$. Vamos a considerar $m_e(E^c\setminus H)$.
\begin{align*}
  m_e(E^c\setminus H) & =m_e\left(E^c\cap\left(\bigcap_{k\in\bN}G_k\right)\right)\\
  & =m_e(E^c\cap G_k)\\
  & =m_e(G_k\setminus E)\leq\frac{1}{k}\\
\end{align*}
Por lo tanto $F=H\cup(E^c\setminus H)$ y $m_e(E^c\setminus H)$. Por lo tanto $F$ es medible.
\end{ptcbp}

En la prueba anterior usamos el siguiente hecho.

\begin{Lem}
  Suponga que $Z\subseteq\bR^d$ tiene medida exterior cero. Entonces $Z$ es medible.
\end{Lem}

\begin{ptcbp}
Dado $\varepsilon>0$ existe $G\supseteq Z$ tal que
$$m_e(G)\leq m_e(Z)+\varepsilon=\varepsilon$$
Pero $m_e(G)\geq m_e(G\setminus Z)$ y así $m_e(G\setminus Z)\leq \varepsilon$. Se concluye que $Z$ es medible.
\end{ptcbp}

Sean $E_k\subseteq\bR^d$ medibles. Podemos empezar a jugar con operaciones entre conjuntos.
$$\bigcap_{k\in\bN}E_k=\left(\bigcup_{k\in\bN}E_k^c\right)^c$$
Entonces inmediatamente las intersecciones son medibles. Las intersecciones finitas también tomando $E_k=\emptyset$ salvo finitos $k$'s. De igual manera la resta de medibles es medible.

\begin{Def}
  Sea $\Om$ un conjunto y $\cF\subseteq\cP(\Om)$. Decimos que $\cF$ es una $\sigma$-álgebra sobre $\Om$ si se cumplen las siguientes condiciones.
  \begin{enumerate}
    \item $\Om\in\cF$,
    \item $E\in\cF\Rightarrow E^c\in\cF$,
    \item Si $E_k\in \cF$ para todo $k\in\bN$, entonces $\bigcup_{k\in\bN} E_k\in \cF$.
  \end{enumerate}
\end{Def}

\begin{Ex}
  Los siguientes son ejemplos de $\sigma$-álgebras sobre $\Om$.
  \begin{enumerate}
    \item $\cF=\conj{\emptyset,\Om}$,
    \item $\cF=\cP(\Om)$.
  \end{enumerate}
  Si $\Om=\bR^d$ el conjunto de conjuntos medibles es una $\sigma$-álgebra sobre $\bR^d$.
\end{Ex}

\begin{Def}
  Sea $\cA\subseteq\cP(\Om)$, definimos
  $$\sg(\cA)=\gen(\cA)=\bigcap_{\cA\subseteq\cF}\cF$$
  A este conjunto se le llama la $\sg$-álgebra generada por $\cA$. Donde $\cF$ es una $\sg$-álgebra. Diremos que esta es la $\sg$-álgebra más pequeña que contiene a $\cA$.
\end{Def}

\begin{Ej}
    Corrobore que en efecto $\gen(\cA)$ es una $\sg$-álgebra.
\end{Ej}

\begin{Def}
  Si $\tau$ es una topología sobre $\bR^d$, entonces la $\sg$-álgebra $\gen(\tau)=\cB$ se conoce como la $\sg$-álgebra Boreliana. Note inmediatamente que $\cB\subseteq\cM$, donde $\cM$ es el conjunto de conjuntos medibles de $\bR^d$.
\end{Def}

Finalmente vamos a verificar el hecho que buscabamos desde el principio. La medida de la unión es la suma de las medidas. Empezamos con un lema.

\begin{Lem}
  Sea $E\subseteq\bR^d$, entonces $E$ es medible si y sólo si dado $\varepsilon>0$ existe $F\subseteq E$ cerrado tal que $m_e(E\setminus F)<\varepsilon$.
\end{Lem}

\begin{ptcbp}
Si $E$ es medible, entonces $E^c$ es medible. Dado $\varepsilon>0$ existe $G\supseteq E^c$ tal que $m_e(G\setminus E^c)<\varepsilon$.
\begin{align*}
  G\setminus E^c &=G\cap(E^c)^c\\
   &=G\cap E\\
    &=(G^c)^c\cap E\\
\end{align*}
De esta manera tome $F=G^c$ y entonces $m_e(E\setminus F)<\varepsilon$.
\end{ptcbp}

\begin{Ej}
  Prueba la otra dirección del lema anterior.
\end{Ej}

\begin{Def}\label{def:medidaDeLebesgue}
  Si $E$ es medible, definimos la medida de Lebesgue de $E$ como $m_e(E)$. Denotamos simplemente $m_e\equiv m$.
\end{Def}

\begin{Th}
  Sea $(E_k)_{k\in\bN}\subseteq\cM$. Si $E_k\cap E_\ell=\emptyset$ si $k\neq \ell$ entonces
  $$m\left(\bigcup_{k\in\bN}E_k\right)=\sum_{k\in\bN}m(E_k)$$
\end{Th}

Cada $E_k$ se puede meter en un $G_k$. Pero al aproximarlos por arriba tenemos problemas de traslapes. Para ello aproximamos por abajo con cerrados por dentro. Satisface buenas aproximaciones y que son disjuntos.

\begin{ptcbp}
Asumamos como una hipótesis extra que $E_k$ es acotado para todo $k\in\bN$. Teniendo esto sea $F_k\subseteq E_k$ cerrado. Dado $\varepsilon>0$ tenemos que
$$m(E_k\setminus F_k)<\frac{\varepsilon}{2^k}$$
Sabemos que todos estos conjuntos $(F_i)_{i\in\bonj{m}}$ son compactos y disjuntos. Luego $\mm(F_i,F_j)>0$ para $i\neq j$.
\begin{align*}
  m\left(\bigcup_{i\in\bonj{m}}F_i\right) & =\sum_{i\in\bonj{m}} m(F_i)\\
   &\leq m\left(\bigcup_{i\in\bN}F_i\right) \\
   &\leq m\left(\bigcup_{k\in\bN}E_k\right) \\
\end{align*}
Por lo tanto $\sum_{k\in\bN} m(F_k)\leq m\left(\bigcup_{k\in\bN}E_k\right)$. Finalmente
\begin{align*}
  m(E_k) &=m(F_k\setminus(E_k\setminus F_k))\\
  &\leq m(F_k)+m(E_k\setminus F_k)\\
  \Rightarrow m(E_k)-\frac{\varepsilon}{2^k}&\leq m(F_k)\\
  \Rightarrow \sum_{k\in\bN} m(E_k)- \varepsilon &\leq \sum_{k\in\bN} m(F_k)\\
  &\leq m\left(\bigcup_{k\in\bN}E_k\right)
\end{align*}
Para deshacernos de la condición de acotación, consideramos diferencias de bolas.
\begin{align*}
  E_k & =(E_k\cap B(0,1))\\
  &\quad \cup\left(\bigcup_{m\in\bN}\bonj{B(0,m+1)\setminus B(0,m)}\cap E_k\right)
\end{align*}

Tomando unión disjunta sobre $k,m$ entonces obtenemos un conjunto medible por álgebra de conjuntos medibles.
\end{ptcbp}

\subsection{Día 18| 29-5-18}

Recordemos que si $E_k$ es medible y $E_k\cap E_\ell=\emptyset$, para $k\neq \ell$ entonces $m(\bigcup_{k\in\bN}E_k)=\sum_{k\in\bN}m(E_k)$. \par
Sean $I_k=\bigtimes_{i\in\bonj{d}}\bonj{a_i^k,b_i^k}$ tal que $I_k^o\cap I_\ell^o=\emptyset$ para $k\neq\ell$ entonces
\begin{align*}
  m\left(\bigcup_{k\in\bN}I_k^o\right)&=\sum_{k\in\bN}m(I_k^o)\\
  &=\sum_{k\in\bN}m(I_k)
\end{align*}
Ahora acotamos con
\begin{align*}
   m\left(\bigcup_{k\in\bN}I_k^o\right)&\leq m\left(\bigcup_{k\in\bN}I_k\right)\\
   &\leq \sum_{k\in\bN}m(I_k)
\end{align*}
Luego $ m\left(\bigcup_{k\in\bN}I_k\right)=\sum_{k\in\bN}m(I_k)$. Este resultado es muy importante pues todo abierto se puede descomponer en estas cajas. Esto es en virtud de que la topología producto está generada por elementos básicos $I_k$'s.\par
Ahora asuma que $E_1\subseteq E_2$ conjuntos medibles. Entonces $E_2=E_1\cup (E_2\setminus E_1)$, luego $m(E_2)=m(E_1)+m(E_2\setminus E_1)$. Si $m(E_1)<\infty$ entonces $m(E_2\setminus E_1)=m(E_2)-m(E_1)$.\par
Veamos otras cosas que podemos hacer con esto, considere una sucesión de conjuntos medibles $(E_k)_{k\in\bN}$ con $E_k\subseteq E_{k+1}$. Entonces podemos separar la unión  en conjuntos disjuntos.
$$\bigcup_{k\in\bN}E_k=E_1\cup\bigcup_{i\in\bN\setminus\conj{1}}E_i\setminus E_{i-1}$$
FIG18.1\par
Luego
\begin{align*}
  m\left(\bigcup_{k\in\bN}E_k\right) &=m(E_1)+\sum_{i\in\bN\setminus\conj{1}}m(E_i\setminus E_{i-1})\\
  &=m(E_1)+\sum_{i\in\bN\setminus\conj{1}}m(E_i)-m(E_{i-1})
\end{align*}
Esto ocurre siempre que $m(E_i)<\infty$ para todo $i\in\bN$. Luego la suma que nos queda es una telescópica.
\begin{align*}
   m(\bigcup_{k\in\bN}E_k) &=m(E_1)+\lim_{i\to\infty}m(E_i)-m(E_1)\\
   &=\lim_{i\to\infty}m(E_i)
\end{align*}
Por otro lado si $E_k\supseteq E_{k+1}$ consideramos
$$E_1=\left(\bigcup_{k\in\bN}E_k\setminus E_{k+1}\right)\cup\left(\bigcap_{k\in\bN}E_k\right)$$
Tomando medidas y asumiendo que todos tienen medida finita resulta en
\begin{align*}
  m(E_1) &=\left(\sum_{k\in\bN}m(E_k\setminus E_{k+1})\right)+m\left(\bigcap_{k\in\bN}E_k\right)\\
  &=\left(\sum_{k\in\bN}m(E_k)-m(E_{k+1})\right)+m\left(\bigcap_{k\in\bN}E_k\right)
\end{align*}
Esta suma vuelve a ser telescópica.
\begin{gather*}
  \Rightarrow m(E_1)=m(E_1)-\lim_{k\to\infty}m(E_k)+m\left(\bigcap_{k\in\bN}E_k\right)\\
  \Rightarrow m\left(\bigcap_{k\in\bN}E_k\right)=\lim_{k\to\infty}m(E_k)
\end{gather*}
Note que era suficiente asumir que $E_1$ tiene medida finita. Sin embargo esto refleja como se iría pensando el problema. Además si uno prescinde de esta suposición, entonces consideramos $E_k=\lbonj{k,\infty}$ que tiene medida infinita. Su intersección es vacía y por tanto no se cumple el resultado.\par
Las disquisición anterior se resume en el teorema siguiente.

\begin{Th}\label{thm:mCadenasAscDec}
  Sea $(E_k)_{k\in\bN}$ una sucesión de conjuntos medibles.
  \begin{enumerate}
    \item Si $E_k\subseteq E_{k+1}$ para todo $k\in\bN$, entonces $m(\bigcup_{k\in\bN}E_k)=\lim_{i\to\infty}m(E_i)$.
    \item Si existe $i_0$ tal que $m(E_{i_0})<\infty$ y $E_k\supseteq E_{k+1}$, entonces $m\left(\bigcap_{k\in\bN}E_k\right)=\lim_{k\to\infty}m(E_k)$.
  \end{enumerate}
\end{Th}
Nos planteamos la pregunta, qué ocurre cuando $E_k$ no es medible? Si asumimos que $(E_k)_{k\in\bN}$ una sucesión de conjuntos no medibles tal que $E_k\subseteq E_{k+1}$ y $m_e(E_k)<\infty$. Sea $H_k$ un conjunto $G_\delta$ tal que $m_e(E_k)=m(H_k)$ con $E_k\subseteq H_k$. Definimos $G_k=\bigcup_{i\in\bN\setminus\bonj{k-1}}H_i\subseteq H_k$,note que $E_k\subseteq E_{k+\ell}\subseteq H_{k+\ell}$, cuál es la ventaja de hacer esto?
\begin{gather*}
  E_k\subseteq G_k \subseteq H_k\\
  \Rightarrow m_e(E_k)\leq m(G_k)\leq m(H_k)
\end{gather*}
Pero recuerde que $m_e(E_k)=m(H_k)$. Los $G_k$'s crecen y por lo tanto la medida de la unión es $\lim_{k\to\infty}m(G_k)$, por el teorema anterior. Por las desigualdades anterior, tenemos que $m_e(E_k)=m(G_k)$. Finalmente tenemos una sucesión de números positivos crecientes, o sea, convergen al sup.
$$m_e(E_k)\leq m_e\left(\bigcup_{k\in\bN}E_k\right)\leq \lim_{k\to\infty}m_e(E_k)$$

\subsubsection*{Caracterización de medibilidad}

\begin{Th}\label{thm:medEsGDoFSmasomenosMedCero}
  Las siguientes condiciones son equivalentes.
  \begin{enumerate}
    \item $E$ es un conjunto medibles,
    \item $E=H\setminus Z$, con $H$ un conjunto $G_\delta$ y $m_e(Z)=0$,
    \item $E=H\cup Z$, con $H$ un conjunto $F_\sg$ y $m_e(Z)=0$.
  \end{enumerate}
\end{Th}

\begin{ptcbp}
Vea que los últimos dos apartados nos dan el primero por álgebra de conjuntos medibles. Veamos que $\mathit{1.}\Rightarrow\mathit{3.}$.\par
Sea $E$ medible, dado $k\in\bN$ existe $F_k\subseteq E$ cerrado tal que $m_e(E\setminus F_k)<\frac{1}{k}$. Tome $H=\bigcup_{k\in\bN}F_k$, entonces
\begin{align*}
  m(E\setminus H) &=m\left(E\setminus\left(\bigcup_{k\in\bN}F_k\right)\right)\\
  &\leq m(E\setminus F_k)\leq \frac{1}{k}
\end{align*}
Esto significa que $F_k$ tiene medida cero y por lo tanto $m(E\setminus H)=0$.
\end{ptcbp}

\begin{Ej}
  Escriba con detalle el apartado que falta.
\end{Ej}

\begin{Th}
  Las siguientes condiciones son equivalentes.
  \begin{enumerate}
    \item $E$ es un conjunto medibles,
    \item Dado $\varepsilon>0$, existen $S,N_1,N_2$ tales que $E=(S\cup N_1)\setminus N_2$ con
        \begin{itemize}
          \item $S=\bigcup_{k\in\bN}I_k$, $I_k\in S_d$ (cajas abiertas a la derecha y cerradas por la izquierda),
          \item $m_e(N_1)<\varepsilon$,
          \item $m_e(N_2)<\varepsilon$.
        \end{itemize}
  \end{enumerate}
\end{Th}

\begin{Ej}
  Pruebe el teorema anterior.
\end{Ej}

\begin{Th}[Carathéodory]
  Sea $E\subseteq\bR^d$, entonces $E$ es medible si y sólo si para cualquier $A\subseteq\bR^d$ se cumple
  $$m_e(A)=m_e(A\cap E)+m_e(A\setminus E)$$
\end{Th}
En general de forma trivial tendríamos una desigualdad, sin embargo un conjunto medible parte cualquier conjunto en dos pedazos sin problemas.

\begin{ptcbp}
Sabemos que existe $H$ un conjunto $G_\delta$ tal que $A\subseteq H$ y $m_e(A)=m(H)$ pues $H$ es medible. Pero en virtud de esto mismo, tenemos que $m(H)=m((H\cap E)\cup (H\setminus E))$.
\begin{align*}
  m_e(A) &=m_e(H)=m(H)\\
  &m((H\cap E)\cup (H\setminus E))\\
  &m(H\cap E)+m(H\setminus E)\\
  &\geq m_e(A\cap E)+m_e(A\setminus E)
\end{align*}
Como tenemos la otra desigualdad inmediatamente, se cumple la igualdad.\par
Primero vamos a asumir otra hipótesis. Pensemos que pasa cuando la medida exterior es finita? Que podemos trabajar con álgebra de conjuntos sin problemas.\par
Asuma que $m_e(E)<\infty$, sabemos que existe $H$ un conjunto $G_\delta$ que contiene a $E$ tal que $m_e(E)=m_e(H)$. Le aplicamos la hipótesis a $H$.
\begin{align*}
  m_e(H) & m_e(H\cap E)+m_e(H\setminus E)\\
  &= m_e(E)+m_e(H\setminus E)
\end{align*}
Por lo tanto $0=m_e(H\setminus E)$, si definimos $Z=H\setminus E$, entonces $E=H\setminus Z$ con $m_e(Z)=0$ y $H$ un conjunto $G_\delta$. Por la caracterización \ref{thm:medEsGDoFSmasomenosMedCero} tenemos que $E$ es medible.\par
Tenemos dos opciones, despedazar nuestro conjunto en cosas fáciles de trabajar o afinar la prueba para evitar sumar infinitos.\par
Por ahora, despedazamos. Tome $m_e(E)=\infty$ y $E_k=E\cap B(0,k)\subseteq E_{k+1}$. Lo que vamos a hacer es repetir la prueba con un pequeño cambio para los $E_k$'s. \par
Sea $H_k$ un conjunto $G_\delta$ que contiene a $E_k$ tal que $m_e(H_k)=m_e(E_k)$. Nuevamente por hipótesis tenemos que
\begin{align*}
  m_e(H_k) &=m_e(H_k\cap E)+m_e(H_k\setminus E)\\
  &\geq m_e(E_k)+m_e(H_k\setminus E)
\end{align*}
Esto pues $E_k\subseteq H_k$, $E_k\subseteq E$. Por lo tanto $m(H_k\setminus E)=0$. Tome $H=\bigcup_{k\in\bN}H_k\supseteq\bigcup_{k\in\bN}E_k=E$. Finalmente
\begin{align*}
  m_e(H\setminus E) & =m_e\left(\bigcup_{k\in\bN}H_k\setminus E\right)\\
  &\leq \sum_{k\in\bN}m(H_k\setminus E)=0
\end{align*}

\end{ptcbp}
La moraleja es, no trabaje con $A$! Trabaje con alguien que se comporta bien, $H$ un medible.
El resultado a continuación es muy técnico. Es de esos lemas que le salva a uno la vida cuando todo se pone rudo, es muy útil ya que le ahorra a uno mucho trabajo.
\begin{Th}
  Sea $E\subseteq\bR^d$, entonces existe $H$ un conjunto $G_\delta$ que contiene a $E$ tal que
  $$m_e(E\cap M)=m(H\cap M)$$
  para cualquier conjunto $M$ medible.
\end{Th}
Como en la prueba anterior vamos a asumir primero medida finita.

\begin{ptcbp}
Asuma que $m_e(E)<\infty$ porque el $H$ que nos sirve es el clásico conjunto $G_\delta$ que siempre sirve. Existe $H$ un conjunto $G_\delta$ que contiene a $E$ tal que
$$m_e(E)=m_e(H)=m(H)$$
Sea $M$ medible, luego
\begin{gather*}
  m_e(E)=m_e(E\cap M)+m_e(E\setminus M)\\
  m(H)=m(H\cap M)+m(H\setminus M)
\end{gather*}
Note que $m_e(E\cap M)\leq m(H\cap M)$ y $m(E\setminus M)\leq m(H\setminus M$. Por lo tanto $m_e(E\cap M)=m(H\cap M)$.\par
Qué pasa cuando las medidas no son finitas? Sea $E= \bigcup_{k\in\bN}E_k$, con $E_k\subseteq E_{k+1}$ y $m_e(E_k)<\infty$. Sabemos que existe $H_k$ un conjunto $G_\delta$ tal que
$$m_e(E_k\cap M)=m_e(H_k\cap M)$$
para cualquier $M$ medible. Queremos poner $E$ en vez de $E_k$, si pudieramos tomar límites estaríamos listos. El problema es que como los $H_k$'s no necesariamente están ordenados, un límite no necesariamente será un $H$. Los vamos a forzar a ser ordenados, pero esto será unión de intersecciones y así no necesariamente será un conjunto $G_\delta$.\par
Sea $V_k=\bigcap_{j\geq k}H_j\subseteq H_k$, además usando la técnica de la vez anterior tenemos que $E_k\subseteq V_k\subseteq H_k$.%ver el gather anterior
 Observe que
\begin{gather*}
  E_k\cap M\subseteq V_k\cap M\subseteq H_k\cap m\\
  \Rightarrow m_e(E_k\cap M)\leq m_e(V_k\cap M)\leq m(H_k\cap M)
\end{gather*}
Pero tenemos $m_e(E_k\cap M)=m(H_k\cap M)$. Entonces $v_k\subseteq V_{k+1}$, con $V_k$ un conjunto $G_\delta$ tal que $m_e(V_k\cap M)=m_e(E_k\cap M)$. Al tomar limites se obtiene
$$m_e(\left(\bigcup_{k\in\bN}V_k\right)\cap M)=m_e(E\cap M)$$
A este tipo de conjuntos se les llama $G_{\sg\delta}$.\par
Como $H_1=\bigcup_{k\in\bN}V_k$, entonces $H_1$ es medible tal que $m(H_1\cap M)=m_e(E\cap M)$. Sea $H$ un conjunto $G_\delta$ tal que $H_1=H\setminus Z$ con $Z$ de medida cero. Finalemente tenemos
\begin{align*}
  m(H\cap M) &=m((H\setminus Z)\cap M)+m((H\cap Z)\cap M)\\
  &=m((H\setminus Z)\cap M)=m_e(H_1\cap M)
\end{align*}
\end{ptcbp}

\subsubsection*{El conjunto de Cantor}

El conjunto de Cantor es un ejemplo maravilloso en el sentido que contradice toda intuición posible. Sabemos que este conjunto es denso por ninguna parte, no contable, y no contiene intervalos. Vamos a ver que este conjunto es de medida cero.\par
A recordar, tomamos una sucesión $(C_k)_{k\in\bN}$ de manera que
\begin{gather*}
  C_0=\bonj{0,1}\\
  C_1=\bonj{0,\frac{1}{3}}\cup\bonj{\frac{2}{3},1}\\
  C_2=\bonj{0,\frac{1}{9}}\cup\bonj{\frac{2}{9},\frac{1}{3}}\cup\bonj{\frac{2}{3},\frac{7}{9}}\cup\bonj{\frac{8}{9},1}
\end{gather*}
En general $C_{k+1}$ se obtiene de $C_k$ dividiendo cada intervalo en tres intervalos iguales y removiendo el tercio del medio.\par
El conjunto de Cantor es $C=\bigcap_{k\in\bN}C_k$ que es cerrado. Proseguimos a medirlo.

 $$m(C_{k+1}) =\frac{2}{3}m(C_k)=\left(\frac{2}{3}\right)^{k+1}$$

Luego $m(C)=\lim_{k\to\infty}m(C_k)=0$. Así el conjunto de Cantor es denso en ninguna parte y de medida cero. La próxima lección probaremos que el conjunto de Cantor es no numerable construyendo una biyección de $C$ en $\bonj{0,1}$.


\subsection{Día 19| 31-5-18}

\subsubsection*{Quinta Sesión de Ejercicios}

\begin{Ej}[2.4 Wheeden \& Zygmund]
  Sea $(f_k)_{k\in\bN}$ una sucesión de funciones de variación acotada en $\bonj{a,b}$ con una cota uniforme que converge puntualmente a $f$. Muestre que $f$ es de variación acotada sobre $\bonj{a,b}$ con la misma cota que las $f_k$. De un contraejemplo cuando la cota no es uniforme.
\end{Ej}

\begin{ptcbp}
Sea $\Gamma=\conj{a=x_0,x_1,\cdots,x_n=b}$ particíón de $\bonj{a,b}$, queremos probar que $\sum_{i\in\bonj{n}}|f(x_i)-f(x_{i-1})|$ está acotado. \par
Sea $k_0\in\bN$ tal que
$$|f(x_i)-f_k(x_i)|<\frac{\varepsilon}{n},\quad k>k_0$$
Usamos desigualdad triangular dos veces.
\begin{align*}
  \sum_{i\in\bonj{n}}|f(x_i)-f(x_{i-1})| &\leq \sum_{i\in\bonj{n}}|f(x_i)-f_k(x_{i})|\\
  &\quad +\sum_{i\in\bonj{n}}|f_k(x_i)-f_k(x_{i-1})|\\
  &\quad +\sum_{i\in\bonj{n}}|f_k(x_{i-1})-f(x_{i-1})|\\
  &\leq \varepsilon+\varepsilon+M
\end{align*}
Tomamos $M'=2\varepsilon+M$ y así $f$ es de variación acotada.\par
Ahora para el contraejemplo tomamos $f(x)=x\sin(\frac{1}{x})$ que no es de variación acotada en $\bonj{0,1}$/ Sin embargo considera siguiente sucesión de funciones.
$$
f_n(x)=
  \begin{cases}
    x\sin(\frac{1}{x}), & \mbox{ si } \frac{1}{n}\leq x\\
    0, & \text{ si }\frac{1}{n} > x
  \end{cases}
$$
Todos los términos se comportan bien, sin embargo la función como tal no. La idea es tomar una función que no es de variación acotada y despedazarla.

\end{ptcbp}
Podemos formular otro contraejemplo.

\begin{ptcb}
Sea $H\subseteq\bonj{0,1}$ contable, $H=\conj{x_i}_{i\in\bonj{n}}$. Definimos la siguiente sucesión de funciones.
\begin{gather*}
  f_0(x) = 1,\quad\forall x\in\bonj{0,1}\\
  f_1(x)=
  \begin{cases}
    0, & \mbox{ si } x=x_1\\
    f_0(x), & \mbox{ si no}
  \end{cases}
\end{gather*}
En general tenemos que

$$
f_{k+1}(x)=
  \begin{cases}
    0, & \mbox{ si } x=x_{k+1}\\
    f_{k}(x), & \mbox{ si no}
  \end{cases}
$$
\textcolor{red}{finish, ver foto}
\end{ptcb}

\begin{Ej}[2.5 Wheeden \& Zygmund]
  Suponga que $f$ es finita en $\bonj{a,b}$ y de variación acotada en $\bonj{a+\varepsilon,b}$ para todo $\varepsilon>0$ con $\Var\bonj{f,\bonj{a+\varepsilon,b}}\leq M$. Muestre que $f$ es de variación acotada en $\bonj{a,b}$. Se cumple que $\Var\bonj{f,\bonj{a,b}}\leq M$? Si no, qué hipótesis garantiza esto?
\end{Ej}

\begin{ptcbp}
Sea $\Gamma=\conj{a=x_1,x_2,\cdots,x_n=b}$ particíón de $\bonj{a,b}$. Tome $x_2=a+\varepsilon$, ahora escrbimos la suma de $f$ respecto a $\Gamma$.
\begin{align*}
  S(f,\Gamma,\bonj{a,b}) &=|f(a)-f(x_2)|\\
  &\quad +\sum_{i\in\bonj{n}\setminus\conj{1}}|f(x_i)-f(x_{i-1})|\\
  &\leq |f(a)-f(x_2)|+\Var\bonj{f,\bonj{a+\varepsilon,b}}\\
  &\leq |f(a)|+|f(x_1)-f(b)+f(b)|+M\\
  &\leq |f(a)|+|f(b)|+2M
\end{align*}
Con que haya un salto en $a$, no es posible obtener la cota. Es necesario que $f$ sea continua en $a$.
\end{ptcbp}

\begin{Ej}[2.9 Wheeden \& Zygmund]
  Sea $\cC$ una curva parametrizada por $(\varphi(t),\psi(t))$ con $t\in\bonj{a,b}$.
  \begin{enumerate}
    \item Suponga que $\varphi,\psi$ son continuas y de variación acotada, entonces $L(\gamma)=\lim_{|\Gamma|\to 0}L(\gamma,\Gamma)$.
        \item Si $\varphi,\psi$ son continuamente diferenciables, muestre que $L(\gamma)=\int_{a}^{b}\left((\varphi'(t))^2+(\psi'(t))^2\right)^{\frac{1}{2}}$.
  \end{enumerate}
\end{Ej}

\begin{ptcbp}
\begin{enumerate}
  \item En efecto, sea $\Gamma_1=(x_i)_{i\in\bonj{n}}$ partición de $\bonj{a,b}$ tal que
  $$L(\gamma)-\ell(\gamma,\Gamma_1)<\varepsilon.$$
  En virtud de que $\varphi,\psi$ son continuas en $\bonj{a,b}$, son uniformemente continuas. Así, podemos tomar un $\delta >0$ tal que
  $$\max\conj{|\varphi(\alpha)-\varphi(\beta)|,\psi(\alpha)-\psi(\beta)|}<\varepsilon,$$
  cuando $|\alpha-\beta|<\delta$. Tome ahora $\Gamma_2=(y_i)_{i\in\bonj{m}}$ otra partición de $\bonj{a,b}$ de malla menor que $\delta$. Sea $\Gamma=\Gamma_1\cup\Gamma_2=(t_i)_{i\in\bonj{m'}}$ un refinamiento con $m'>m$. Así tenemos que nuestra primera desigualdad nos da
  \begin{align*}
    L(\gamma)-\ell(\gamma,\Gamma_1)<\varepsilon & \Rightarrow L(\gamma)-\varepsilon<\ell(\gamma,\Gamma_1)\\
    &\Rightarrow L(\gamma)-\varepsilon<\ell(\gamma,\Gamma).
  \end{align*}
  Ahora, por definición de longitud de curva tenemos que
  \begin{align*}
    \ell(\gamma,\Gamma) &=\sum_{i\bonj{m'}}\sqrt{|\Delta\varphi|^2+|\Delta\psi|^2}\\
    &=\sum_{i\bonj{m'}}\sqrt{2\varepsilon^2}=\varepsilon m'\sqrt{2}.
  \end{align*}
  De manera análoga obtenemos que $\ell(\gamma,\Gamma_2)=\varepsilon m\sqrt{2}$. Así
  \begin{align*}
    |L(\gamma)-\ell(\gamma,\Gamma_1)| &\leq |L(\gamma)-\ell(\gamma,\Gamma)|\\
    &\quad+|\ell(\gamma,\Gamma)-\ell(\gamma,\Gamma_1)|\\
    &=\varepsilon+\varepsilon\sqrt{2}(m'-m).
  \end{align*}
  Como tenemos que la diferencia está acotada por un múltiplo de $\varepsilon$ concluimos que $\lim_{|\Gamma_1|\to 0}\ell(\gamma,\Gamma_1)=L(\gamma)$.
  \item Tomamos ahora $\Gamma$, una partición uniforme de $\bonj{a,b}$. Entonces por definición de integral de Riemann tenemos que
      \begin{align*}
        L(\gamma) &=\sum_{i\in\bonj{n}}\frac{1}{n}\sqrt{|\Delta\varphi|^2+|\Delta\psi|^2}\\
        &=\int_{a}^{b}\sqrt{|\varphi'(t)|^2+|\psi'(t)|^2}\dd t.
      \end{align*}
\end{enumerate}
\end{ptcbp}

\subsection{Día 20|5-6-18}

Sea $x\in\bonj{0,1}$, entonces podemos representarlo en base 3 como
$$x=\sum_{i\in\bN}\frac{n_i}{3}=\frac{n_1}{3}+\frac{n_2}{3^2}+\frac{n_3}{3^3}+\cdots$$
 donde $n_i\in\conj{0,1,2}$. Observe que $\frac{1}{3}=\sum_{i\geq 2}\frac{2}{3^i}$.
\begin{Ej}
  Existe más de una expansión en base 3 si y sólo $x=\frac{k}{3^n}$ con $k,n\in\bN$.
\end{Ej}

Para evitar el problema de varias expansiones, si $x$ es tal que cumple las siguientes condiciones
\begin{enumerate}
  \item $n_k=1$ y $n_{k+\ell}=0$ para todo $\ell\in\bN$,
  \item $n_k=0$ y $n_{k+\ell}=2$ para todo $\ell\in\bN$.
\end{enumerate}
Entonces tomaremos otra expansión.\par
Ahora si $x\in\bonj{0,\frac{1}{3}}$ debe ocurrir que $n_1$ es 0. Lo más grande que puede pasar es 0 en el primer término y 2 en todos los demás. \par
Luego $x\in\bonj{\frac{2}{3},1}$ ocurre si $n_1=0$ y $n_2=0$ y $x\in\bonj{\frac{2}{9},\frac{1}{3}}$ si y sólo si $n_1=0, n_2=2$. Para los demás conjuntos tenemos que $x\in\bonj{\frac{2}{3},\frac{7}{9}}\iff n_1=2,n_2=0$ y $x\in\bonj{\frac{8}{9},1}\iff n_1=2,n_2=2$. En general tenemos el siguiente hecho.

\begin{Ej}
  $x\in C_k$ si y sólo si $n_i\in\conj{0,2}$ para todo $i\in\bonj{k}$.
\end{Ej}

Esto nos permite deducir el siguiente lema.
\begin{Lem}
  $x\in C$, el conjunto de Cantor si y sólo si $n_k\in\conj{0,2}$ para todo $k\in\bN$.
\end{Lem}

Considere ahora la función
$$\Phi\: C\to\bonj{0,1},\ x=\sum_{i\in\bN}\frac{n_i}{3^i}\mapsto\sum_{i\in\bN}\left(\frac{n_i}{2}\right)\frac{1}{2^i}$$
Esto nos da una biyección entre el conjunto de Cantor y $\bonj{0,1}$. Por lo tanto $C$ es no contable.\par

Volvemos a la pregunta de que si todos los conjuntos son medibles. La respuesta es no, pero no solo vamos a ver un ejemplo, vamos a ver que todo conjunto de medida exterior cero contienen un conjunto no medible.

\subsubsection*{El conjunto de Vitali}

El conjunto de Vitali no es medible, pero para probar esto necesitamos un lema. El lema \textit{técnico}.

\begin{Lem}[Vitali, 1905]\label{lem:LemaTecVitali}
  Sea $E\subseteq\bR$ medible con $m(E)>0$. Entonces el conjunto
  $$E-E=\conj{e-f,\ e,f\in E}$$
  contiene un intervalo que contiene a cero.
\end{Lem}

Qué es lo que nos dice esto? Si un conjunto no contiene un intervalo entonces debe ocurrir que alguna de las hipótesis no es cierta. O sea, el conjunto o no es medible o tiene medida cero.

\begin{ptcbp}
Dado $m(E)>0$, entonces para $\varepsilon>0$, existe $G\supseteq E$ abierto tal que
$$m(E)\leq m(G)< (1+\varepsilon)m(E)$$
Al ser $G$ abierto tenemos que
$$G=\bigcup_{k\in\bN}\bonj{a_k,b_k}$$
con $\obonj{a_i,b_i}\cap\obonj{a_j,b_j}=\emptyset$ para $i\neq j$. Llame $I_k=\bonj{a_k,b_k}$, luego tome $E_k=E\cap I_k$. Entonces ocurre lo siguiente.
\begin{align*}
  m(G) &=\sum_{k\in\bN}m(I_k)\\
  m(E) &=m\left(\bigcup_{k\in\bN}E_k\right)\\
  & =\sum_{k\in\bN}m(E_k)
\end{align*}
Luego existe $k_0$ tal que $m(I_{k_0})\leq m(E_{k_0})(1+\varepsilon)$. Por lo menos en uno la desigualdad no da vuelta.\par
Suponga que $m(I_{k_0})=\alpha$, vamos a probar que si $\varepsilon=\frac{1}{3}$ entonces el intervalo $B(\alpha,\frac{1}{2})\subseteq E_{k_0}-E_{k_0}\subseteq E-E$. \par
Sea $d\in B(\alpha,\frac{1}{2})$, supongamos $(d+E_{k_0})\cap E_{k_0}=\emptyset$ a manera de contradicción, entonces
$m((d+E_{k_0})\cup E_{k_0})=2m(E_{k_0})$. \par
Pero $E_{k_0}\subseteq I_{k_0}$, entonces $(d+I_{k_0})\cup I_{k_0}$ es un intervalo.\par
FIG 20.1\par
Como $|d|<\frac{m(I_{k_0})}{2}$, entonces tenemos
$$
(d+I_{k_0})\cup I_{k_0}
=\begin{cases}
   \bonj{a_k,b_k+d}, & \mbox{cuando } d>0,\\
    \bonj{a_k+d,b_k}, & \mbox{cuando } d\leq 0
 \end{cases}
$$
Así tenemos
\begin{align*}
  m((d+I_{k_0})\cup I_{k_0}) &= d+m(I_{k_0})\\
  &< \frac{1}{2}m(I_{k_0})+m(I_{k_0})
  &= \frac{3}{2}m(I_{k_0})
\end{align*}
Juntando esto con los hechos anteriores resulta en:
$$2m(E_{k_0})<\frac{3}{2}m(I_{k_0})\Rightarrow (1+\frac{1}{3})m(E_{k_0})<m(I_{k_0})$$
Luego si $d\in B(\alpha,\frac{1}{2})$ tenemos que $(d+E_{k_0})\cap E_{k_0}\neq \emptyset$. Entonces $d\in E_{k_0}-E_{k_0}$ y por lo tanto $B(\alpha,\frac{1}{2})\subseteq E_{k_0}-E_{k_0}\subseteq E-E$.
\end{ptcbp}

Para definir el conjunto de Vitali consideramos la relación $x\sim y\iff x-y\in\bQ$. Esta relación es de equivalencia. Note que podemos expresar cada clase como $\bonj{x}=\conj{x+q\: q\in\bQ}$. Así como las clases de equivalencia particionan $\bR$ tenemos que $\bR=\bigcup_{x\in\bR}\bonj{x}$ y la cantidad de clases es no contable.\par
Por el axioma de elección podemos tomar un representante de cada clase. Sea $E$ el conjunto formado por estos elementos.\par
Note que $(E-E)\cap\bQ=\conj{0}$, entonces en virtud del lema \ref{lem:LemaTecVitali} debe ocurrir $m_e(E)=0$ ó que $E$ no sea medible. Pero $\bigcup_{q\in\bQ}E+q = \bR$.
$$\Rightarrow m(\bR)\leq\sum_{q\in\bQ}m_e(E+q)$$
Es decir, si $m_e(E)=0\Rightarrow m_e(E+q)=0\Rightarrow m(\bR)=0$. Pero esto es imposible, entonces debe ocurrir que $E$ no es medible.\par
Ahora sea $A$ un conjunto tal que $m_e(A)>0$. Podemos despedazar $A$ en la unión de $A_q$'s con $A_q=A\cap(E+q)$. Así $A=\bigcup_{q\in\bQ}A_q$. Luego $(A_q-A_q)\cap\bQ=\conj{0}$, entonces debe ocurrir que $m_e(A_q)=0$ ó $A_q$ no es medible.\par
Como tenemos $m_e(A)\leq\sum_{q\in\bQ}m_e(A_q)$, existe $q_0$ tal que $m_e(A_{q_0})\neq 0$. Es decir, existe $q_0$ tal que $A_{q_0}$ no es medible.

\begin{Lem}
  Sea $A\subseteq\bR$ tal que $m_e(A)>0$. Entonces existe $B\subseteq A$ tal que $B$ no es medible.
\end{Lem}

\section{Parcial 3}

Ya sabemos qué son conjuntos medibles. Ahora vamos a ver qué son funciones medibles. Estas funciones son la integrables en el sentido de Lebesgue.

\begin{Def}
  Sea $f\: E\to\bR\cup\conj{\infty}\cup\conj{-\infty}$. Decimos que $f$ es medible si $\conj{x\in E\: f(x)\geq a}=f^{-1}(\obonj{a,\infty})$ es medible para cualquier $a\in\bR$.
\end{Def}

Vamos a denotar $\conj{x\in E\: f(x)\geq a}$ como $\conj{f>a}$.
\begin{Rmk}
  Note que $E=\bigcup_{a\in\bR}\conj{f>a}\cup\conj{f=-\infty}$. Si $f$ es medible, entonces $E$ es medible si y sólo si ${f=-\infty}$ es medible.
\end{Rmk}

De ahora en adelante el dominio de nuestra función es un conjunto medible.

\begin{Lem}
  Sea $f\: E\to\overline{bR}$. Las siguientes afirmaciones son equivalentes.
  \begin{enumerate}
    \item $f$ es medible,
    \item $\conj{f\geq a}$ es medible para todo $a\in\bR$,
    \item $\conj{f< a}$ es medible para todo $a\in\bR$,
    \item $\conj{f\leq a}$ es medible para todo $a\in\bR$.
  \end{enumerate}
\end{Lem}

\begin{ptcbp}
Observe que $\conj{f>a}=\conj{ f\leq a}^c=\bigcup_{n\in\bN}\conj{f\geq a+\frac{1}{n}}$. También podemos escribir $\conj{f\geq a}=\bigcap_{n\in\bN}\conj{f>a-\frac{1}{n}}$.\par
Sin importar con el conjunto que empecemos, podemos escribir los demás con álgebra de conjuntos.
\end{ptcbp}

\begin{Ej}
  Refine los detalles de esta prueba.
\end{Ej}

\begin{Ex}
  Veremos algunos ejemplos de funciones medibles.
  \begin{enumerate}
    \item Sea $f\: E\to\bR$ continua, entonces $f^{-1}(\obonj{a,\infty})$ es abierto.
    \item Si $A$ es medible,  $f(x)=\mathbf{1}_A(x)$ cumple
    $$
    \conj{\mathbf{1}_A>a}
    =\begin{cases}
       \emptyset, & \mbox{si } a>1 \\
       A, & \mbox{si } a\in\bonj{0,1} \\
       E, & \mbox{si } a<0.
     \end{cases}
    $$
  \end{enumerate}
\end{Ex}

\begin{Rmk}
  Sea $f\: E\to\overline{\bR}$ medible, entonces
  \begin{align*}
    \conj{f=\infty} &=\bigcap_{n\in\bN}\conj{f\geq n}\\
    \conj{a\leq f<b} &=\conj{a\leq f}\cap\conj{f<b}\\
    \conj{f=a} &=\bigcap_{n\in\bN}\conj{a\leq f\leq a+\frac{1}{n}}
  \end{align*}
\end{Rmk}

Qué pasa si sólo tenemos $\Om\subseteq\bR$ denso y numerable con $\conj{f>a}$ medible para todo $a\in\Om$? Tome $\alpha\in\bR$, entonces existe $(a_n)_{n\in\bN}\subseteq\Om$ decreciente que converge a $\alpha$. Entonces $\conj{f>\alpha}=\bigcup_{k\in\bN}\conj{f>a_k}$.\par
En particular, esto nos dice que si una función es medible en $\bQ$, es medible $\bR$.

\begin{Lem}
  Sea $f\: E\to\bR$, entonces $f$ es medible si y sólo si $f^{-1}(G)$ es medible para todo $G$ abierto.
\end{Lem}

\begin{ptcbp}
\begin{enumerate}
  \item[$(\Leftarrow)$] En particular $\obonj{a,\infty}$ es abierto, entonces $f^{-1}(\obonj{a,\infty})$ es medible y así $f$ es medible.
  \item[$(\Rightarrow)$] Sea $G$ un abierto, entonces $G=\bigcup_{k\in\bN}\obonj{a_k,b_k}$.
  \begin{align*}
    f^{-1}(G) &=\bigcup_{k\in\bN}f^{-1}(\obonj{a_k,b_k})\\
     &=\bigcup_{k\in\bN}\conj{a_k<f<b_k}
  \end{align*}
  Por álgebra de conjuntos medibles tenemos el resultado.
\end{enumerate}
\end{ptcbp}

\subsection{Día 21| 7-6-18}

\subsubsection*{Sexta Sesión de Ejercicios}
TO DO:\textcolor{red}{write } 2.21, 2.22

\begin{Ej}[2.11  Wheeden \& Zygmund]
  Muestre que $\int_{a}^{b}f\dd\phi$ existe si y sólo si
  $$\forall\varepsilon>\exists\delta>0\: |R(f,\Gamma,\phi)-R(f,\Gamma',\phi)|<\varepsilon,$$
  cuando $|\Gamma|,|\Gamma'|<\delta$.
\end{Ej}

\begin{ptcbp}
En efecto, si la integral $I$ existe, tenemos que si $\Gamma$ es una partición de $\bonj{a,b}$, entonces dado $\varepsilon>0$, existe $\delta>0$ tal que
$$|R(f,\Gamma,\phi)-I|<\frac{\varepsilon}{2},\ |\Gamma|<\delta.$$
De manera análoga, si $\Gamma'$ es otra partición de $\bonj{a,b}$ tal que $\max\conj{|\Gamma|,|\Gamma'|}<\delta$ implica $|R(f,\Gamma',\phi)-I|<\frac{\varepsilon}{2}$. Luego por desigualdad triangular
\begin{align*}
  |R(f,\Gamma,\phi)-R(f,\Gamma',\phi)| &\leq |R(f,\Gamma,\phi)-I|\\
  &\quad+|R(f,\Gamma',\phi)-I|\\
  &<2\frac{\varepsilon}{2}=\varepsilon.
\end{align*}

Sea $\varepsilon>0$, tome $P_n$ una partición uniforme de $\bonj{a,b}$ y considere la sucesión $R(f,P_n,\phi)$. Por hipótesis existe $\delta>0$ tal que $|R(f,\Gamma,\phi)-R(f,\Gamma',\phi)|<\varepsilon$,
  cuando $|\Gamma|,|\Gamma'|<\delta$. \par
  Tome $N\in\bN$ tal que $\frac{b-a}{N}<\delta$. De esta manera, cuando $m,n>N$ tendremos
  $$|P_n|=\frac{b-a}{n}<\frac{b-a}{N}<\delta.$$
  De manera análoga $|P_m|<\delta$. Por hipótesis $|I_n-I_m|<\varepsilon$ y así $(I_n)_{n\in\bN}$ es una sucesión de Cauchy de números reales que converge.\par
  Sea $I$ el límite de la sucesión, vamos a mostrar que $I=int_{a}^{b}f\dd\phi$. Pero en efecto, como existe $\eta\in\bN$ tal que $|I_n-I|<\frac{\varepsilon}{2}$ y existe $\eta'\in\bN$ tal que $\frac{b-a}{\eta'}<\delta$ entonces
  \begin{align*}
    \max\conj{|\Gamma|,|P_n|}<\delta &\Rightarrow |R(f,\Gamma,\phi)-R(f,P_n,\phi)|\\
    &\quad =|R(f,\Gamma,\phi)-I_n|<\frac{\varepsilon}{2},\ n\geq\eta'.
  \end{align*}
  Entonces si tomamos $N'\geq\max\conj{\eta,\eta'}$ tendremos que si $|\Gamma|<\delta$ entonces
  \begin{align*}
     |R(f,\Gamma,\phi)-I| &\leq |R(f,\Gamma,\phi)-I_n|+|I_n-I|\\
     &<2\frac{\varepsilon}{2}=\varepsilon.
  \end{align*}
 O sea que $R(f,\Gamma,\phi)$ converge a $I=int_{a}^{b}f\dd\phi$ cuando la norma de la partición tiende a cero.
\end{ptcbp}

\begin{Ej}[2.15  Wheeden \& Zygmund]
  Sea $f$ continua y $\phi$ de variación acotada en $\bonj{a,b}$. Muestre que $$\psi(x)=\int_{a}^{x}f\dd\phi,$$
  es de variación acotada en $\bonj{a,b}$.\par
  Pruebe que si $g\in\cC(\bonj{a,b})$ entonces
  $$\int_{a}^{b}g\dd\psi=\int_{a}^{b}gf\dd\phi.$$
\end{Ej}

\begin{ptcbp}
En efecto sea $\Gamma=(x_i)_{i\in\bonj{n}}$ una partición de $\bonj{a,b}$. Considere la suma de $\psi$ respecto a $\Gamma$. Tenemos que
\begin{align*}
  S(\psi,\Gamma) &=\sum_{i\in\bonj{n}}|\psi(x_i)-\psi(x_{i-1})|\\
  &=\sum_{i\in\bonj{n}}\left|\int_{x_{i-1}}^{x_i}f\dd\phi\right|\\
  &\leq \sum_{i\in\bonj{n}}\sup_{\bonj{x_{i-1},x_i}}|f|\Var\bonj{\phi,\bonj{x_{i-1},x_i}}\\
  &\leq M(\phi(b)-\phi(a)).
\end{align*}
Esto nos dice que existe una cota de $S(\psi,\Gamma)$ de manera que $\psi$ es de variación acotada.\par
Vamos a asumir que $\psi,\phi$ son crecientes para usar el teorema del valor medio. Luego arreglaremos esto usando el teorema de Jordan ya que ambas funciones son de variación acotada.\par
Ahora considere $\Gamma$ nuevamente y $\int_{x_{i-1}}^{x_i}f\dd\phi$. Por el teorema del valor medio podemos tomar $\eta_i\in\bonj{x_{i-1},x_i}$ tal que
$$\int_{x_{i-1}}^{x_i}f\dd\phi=f(\eta_i)(\phi(x_i)-\phi(x_{i-1}).$$
Vea que la suma de Riemann-Stieltjes de $g$ respecto a $\psi$ tomando $(\eta_i)_{i\in\bonj{n}}$ como puntos intermedios es
\begin{align*}
  R(g,\Gamma,\psi) &=\sum_{i\in\bonj{n}}g(\eta_i)(\psi(x_{i-1})-\psi(x_i))\\
  &=\sum_{i\in\bonj{n}}g(\eta_i)\int_{x_{i-1}}^{x_i}f\dd\phi)\\
  &=\sum_{i\in\bonj{n}}g(\eta_i)f(\eta_i)(\phi(x_i)-\phi(x_{i-1})\\
  &=R(gf,\Gamma,\phi).
\end{align*}
Así tomando límites sobre la norma de la partición tenemos
\begin{gather*}
  \lim_{|\Gamma|\to 0}R(g,\Gamma,\psi)=\lim_{|\Gamma|\to 0}R(gf,\Gamma,\phi)\\
  \int_{a}^{b}g\dd\psi=\int_{a}^{b}gf\dd\phi.
\end{gather*}
Esto nos da el resultado para funciones crecientes. Ahora aplicamos el teorema de Jordan, toda función de variación acotada se puede escribir como la diferencia de dos funciones crecientes. Tome
$$\psi=\psi_1-\psi_2,\quad \phi=\phi_1-\phi_2,$$
y así obtenemos
\begin{align*}
  \int_{a}^{b}g\dd\psi &=\int_{a}^{b}g\dd(\psi_1-\psi_2)\\
  &=\int_{a}^{b}g\dd\psi_1-\int_{a}^{b}g\dd\psi_2)\\
  &=\int_{a}^{b}gf\dd\phi_1-\int_{a}^{b}gf\dd\phi_2)\\
  &=\int_{a}^{b}gf\dd(\phi_1-\phi_2)\\
  &=\int_{a}^{b}gf\dd\phi.
\end{align*}
Por lo tanto obtenemos el resultado.
\end{ptcbp}

\begin{Ej}[2.16  Wheeden \& Zygmund]
  Suponga que $\phi$ es de variación acotada en $\bonj{a,b}$ y $f$ es acotada y continua salvo en un número finito de puntos donde hay discontinuidades de saltos. Suponga que $\phi$ es continua en todo punto donde $f$ es discontinua. Muestre que $\int_{a}^{b}f\dd\phi$ existe.
\end{Ej}

\begin{ptcbp}
En efecto, sea $\varepsilon>0$ y $M\in\bR$ tal que $|f(x)|\leq M$ para $x\in\bonj{a,b}\setminus D$. Donde $D$ es el conjunto de discontinuidades de $f$.\par
Considere $I_j=\bonj{a_j,b_j}$ intervalos que cubren $D$ tales que
$$\sum_{j\in\bonj{n}}\phi(b_j)-\phi(a_j)<\varepsilon.$$
Podemos asumir que los puntos de $D$ están en el interior de los $I_j$ sin perdida de generalidad. Considere ahora el conjunto
$$K=\bonj{a,b}\setminus\bigcup_{j\in\bonj{n}}I_j,$$
este conjunto es compacto. Así $f$ es uniformemente continua en $K $ y por tanto existe $\delta>0$ tal que
$$|x-y|<\delta\Rightarrow|f(x)-f(y)|<\varepsilon,\quad x,y\in K.$$
Tome $\Gamma$ una partición de $\bonj{a,b}$ que no conste de puntos de discontinuidad y regálele los puntos en $(a_j)_{j\in\bonj{n}},(b_j)_{j\in\bonj{n}}$. Suponga que $|\Gamma|<\delta$, así
\begin{align*}
  U(f,\Gamma,\phi)-L(f,\Gamma,\phi) &=\sum_{x_i\neq a_j,b_j}(M_i-m_i)(\phi(x_i)-\phi(x_{i-1})\\
  &\quad+\sum_{i\in\bonj{n}}(M_i-m_i)(\phi(b_i)-\phi(a_i))\\
  &\leq\varepsilon((\phi(b)-\phi(a))+2M).
\end{align*}
Así como las sumas superiores e inferiores están cerca, concluimos que la integral existe.
\end{ptcbp}

\subsection{Día 22| 12-6-18}

Refresquemos la memoria. Sea $f\: E\to\overline{\bR}$, con $E$ un medible. La función se dice medible si el conjunto
$$\conj{x\in E\: f(x)>a}$$
es medible. Esta definición la manipulamos para cambiar la desigualdad a no estricta y hacia el otro lado.\par
Además tenemos una caracterización de medibilidad, una función es medible si y sólo si manda abiertos de vuelta en medibles.

\begin{Lem}
  Sea $f\: E\to\bR$ medible y $\phi\:\bR\to\bR$ continua. Entonces $\phi f$ es medible.
\end{Lem}

Cuál es el problema si $f$ toma valores infinitos? Habría que definir que es $\phi(\infty)$, podría ser fácil se los límites existen. Pero si $\phi$ oscila, cómo lidiamos con esto?

\begin{ptcbp}
Por la caracterización, tome $G$ un abierto, entonces
$$(\phi f)^{-1}(G)=f^{-1}(\phi^{-1}(G))$$
es un conjunto medible pues $\phi^{-1}(G)$ es un abierto.
\end{ptcbp}

\begin{Lem}
  Sea $f\:E\to\overline{\bR}$ una función medible. Si $g$ es tal que
  $$\conj{x\in E\: f(x)\neq g(x)}$$
  tiene medida cero, entonces $g$ es medible y $m(\conj{f>a})=m(\conj{g>a})$.
\end{Lem}

\begin{ptcbp}
Sea $a\in\bR$, entonces
$$\conj{g>a}=\left(\conj{g>a}\cap\conj{f=g}\right)\cup \left(\conj{g>a}\cap\conj{f\neq g}\right).$$
Note que $m_e \left(\conj{g>a}\cap\conj{f\neq g}\right)=0$, pues es subconjunto de un conjunto de medida cero y así $ \left(\conj{g>a}\cap\conj{f\neq g}\right)$ es medible.\par
Además $\conj{g>a}\cap\conj{f=g}=\conj{f>a}\cap\conj{f=g}$ es medible pues $f$ es medible y $\conj{f=g}^c$ es medible.
\end{ptcbp}

Recuerde que cuando queremos probar que alguien es medible, hay que despedazarlo en conjuntos que ya sabemos que son medibles.\par
El lema anterior nos permite modificar el lema que le precede.

\begin{Lem}
  Sea $f\: E\to\overline{\bR}$ medible y $\phi\:\bR\to\bR$ continua. Si $m(\conj{|f|=\infty})$, entonces $\phi f$ es medible.
\end{Lem}

\begin{ptcbp}
Sea
$$g=\begin{cases}
      f, & \mbox{si } |f|\neq\infty \\
      0, & \mbox{si } |f|=\infty
    \end{cases}$$
    Entonces $\phi g$ es medible y $m(\conj{\phi g\neq \phi f})=0$.
\end{ptcbp}

\begin{Ex}
  Sea $f\: E\to\overline{\bR}$ con $m(\conj{|f|=\infty})=0$. Entonces
  $$|f|, e^{cf}, e^{c|f|},f^+,f^-, \sin(f)$$
  son medibles.
\end{Ex}

\begin{Ej}
  Pruebe que las funciones anteriores son medibles usando los lemas precedentes.
\end{Ej}

\begin{Lem}
   Sea $f\: E\to\overline{\bR}$ medbile, entonces $cf, c+f$ son medibles para todo $c\in\bR$.r
\end{Lem}

\begin{Lem}
  Sean $f,g\: E\to\overline{\bR}$ medibles. Entonces $\conj{f>g}$ es medible.
\end{Lem}

\begin{ptcbp}
Note que
\begin{align*}
  \conj{f>g} &=\bigcup_{q\in\bQ}\conj{f>q>g} \\
   &=\bigcup_{q\in\bQ} \conj{f>q}\cap\conj{g\geq q}^c.
\end{align*}
\end{ptcbp}

\begin{Lem}
  Sean $f,g\: E\to\overline{\bR}$ medibles. Si $f+g$ está bien definida en $E$ entonces $f+g$ es medible.
\end{Lem}

\begin{ptcbp}
Si $f+g$ está bien definida, entonces $\conj{f+g}>a=\conj{f>a-g}$,
\end{ptcbp}

De ahora en adelante vamos a trabajar con varias funciones. Siempre que no se mecione el dominio se debe asumir que es el mismo. De lo contrario la suma no va a estar bien definida, es una condición necesaria pero no suficiente.

\begin{Lem}
  Sea $(f_n)_{n\in\bN}$ una sucesión de funciones tal que $f_n\: E\overline{\bR}$ es medible para todo $n$. Entonces se cumple que
  \begin{enumerate}
    \item $\sup_{n\in\bN}f_n, \inf_{n\in\bN}f_n$ son medibles,
    \item $\limsup_{n\to\infty}f_n, \liminf_{n\to\infty}f_n$ son medibles,
    \item Si $\lim_{n\to\infty}f_n=f$ casi por doquier, entonces $f$ es medible.
  \end{enumerate}
\end{Lem}

\begin{ptcbp}
Note que
$$\conj{\sup_{n\in\bN}f_n>a}=\bigcup_{n\in\bN}\conj{f_n>a}.$$
Además $-\sup_{n\in\bN}(-f_n)=\inf_{n\in\bN}f_n$. Por definición tenemos además
\begin{gather*}
  \limsup_{n\to\infty}f_n=\inf_{n\in\bN}\sup_{k\geq n}f_k, \\
  \liminf_{n\to\infty}f_n=\sup_{n\in\bN}\inf_{k\geq n}f_k.
\end{gather*}
\end{ptcbp}

\begin{Def}
  Sea $\phi\: E\to\bR$, decimos que $\phi$ es simple si
  $$\phi(x)=\sum_{k\in\bonj{m}}a_k\mathbf{1}_{A_k}(x),$$
  con $A_k\subseteq E$ y $a_k\in\bR$.
\end{Def}

\begin{Ej}
  Sea $\phi\: E\to\bR$ simple, entonces existe $(b_i)_{i\in\bonj{\ell}}\subseteq \bR$ y $(B_i)_{i\in\bonj{\ell}}\subseteq\cP(E)$ tales que
  \begin{enumerate}
    \item $B_i\cap B_j=\emptyset$ para $i\neq j$,
    \item $\phi(x)=\sum_{i\in\bonj{\ell}}b_i\mathbf{1}_{B_i}(x)$.
  \end{enumerate}
\end{Ej}

\begin{Rmk}
  $E=\bigcup_{i\in\bonj{\ell}}$ si se incluye el conjunto $\conj{\phi=0}$.
\end{Rmk}

Esta es la clave para la integral de Lebesgue. Vamos a definir la integral de funciones simples, después definimos la integral para funciones positivas. Después de eso vamos a aproximar funciones por funciones simples.

\begin{Ej}
  $\phi$ es medible si y sólo si $B_i$ es medible para todo $i\in\bonj{\ell}$.
\end{Ej}

Vamos a empezar a aproximar funciones por simples.\par
Sea $f\: E\to\overline{\bR}$ con $f>0$. Sea $j\in\bonj{k2^k}$, defina
$$
f_k(x)
=
\begin{cases}
  \frac{j-1}{2^k}, & \mbox{si } \frac{j-1}{2^k}\leq f(x)\leq\frac{j}{2^k} \\
  k, & \mbox{si } f\geq k.
\end{cases}
$$
INSERTAR FIG22.1\par
Note que $0\leq f(x)\leq k$, entonces
$$|f(x)-\frac{j-1}{2^k}|\leq \frac{1}{2^k}$$
si $f(x)\in\lbonj{\frac{j-1}{2^k},\frac{j}{2^k}}$. Es decir
$$|f(x)-f_k(x)|<\frac{1}{2^k}.$$
De hecho si $f(x)\neq\infty$, existe $k_0=k_0(x)$ tal que
$$|f(x)-f_k(x)|<\frac{1}{2^k},\quad\text{si} k>k_0.$$
Por otro lado $f(x)\geq k$, entonces $f(x)\geq f_k(x)\geq k$. Luego si $f(x)=+\infty$, entonces $\lim_{k\to\infty}f_k(x)=+\infty$.\par

Note que podemos representar $f_k$ con racionales diádicos,

$$f_k(x)=\sum_{j\in\bonj{k2^k}}\frac{j-1}{2^k}\mathbf{1}_{B_j}(x)+k\mathbf{1}_{\conj{f\geq k}}(x)$$
si $B_i=\conj{\frac{j-1}{2^k}\leq f\leq\frac{j}{2^k}}$. Note que si $f$ es medible, entonces $f_k$ es medible.\par
Además\par
INSERTAR FIG 22.2\par
Si $\frac{2j-2}{2^{k+1}}\leq f\leq\frac{2j-1}{2^{k+1}}$, entonces $f_k(x)=f_{k+1}(x)$. Por otro lado si $\frac{2j-1}{2^k}\leq f<\frac{2j}{2^{k+1}}$, entonces $f_{k+1}(x)=f_k(x)+\frac{1}{2^{k+1}}$.\par
Lo que acabamos de probar es lo siguiente.

\begin{Th}
  Sea $f\: E\to\lbonj{0,\infty}$. Entonces existe una sucesión $f_k$ de funciones simples tal que
  \begin{enumerate}
    \item $\lim_{k\to\infty}f_k=f$,
    \item $f_k\leq f_{k+1}$.
  \end{enumerate}
  Si $f$ es medible, las funciones $f_k$ se pueden tomar medibles.
\end{Th}

El siguiente teorema es contraintuitivo. Uno pensaría que es falso y que no sirve, sin embargo este teorema es un \emph{cañon}.

\begin{Th}[Severini-Egorov, 1910-1911]\label{thm:Egorov}
  Sea $(f_k)_{k\in\bN}$ una sucesión de funciones medibles con dominio $E$ tal que
  \begin{enumerate}
    \item $f_k\xrightarrow[k\to\infty]{} f$ casi por doquier,
    \item $m(E)<\infty$,
    \item $m(\conj{|f(x)|=\infty})=0$.
  \end{enumerate}
  Entonces para todo $\varepsilon>0$ existe $F\subseteq E$ un cerrado que satisface
  \begin{enumerate}
    \item $m(E\setminus F)<\varepsilon$, y
    \item $f_k\xrightarrow[k\to\infty]{} f$ uniformemente en $F$.
  \end{enumerate}
\end{Th}

Antes de probar el teorema, vamos a probar un lema.

\begin{Lem}
  Bajo las hipótesis del teorema de Egorov, dados $\varepsilon,\eta>0$ existen $F\subseteq E$ cerrados y $k_0\geq 1$ tal que
  \begin{enumerate}
    \item $m(E\setminus F)<\eta$,
    \item $|f(x)-f_k(x)|<\varepsilon$ si $k\geq k_0$ y $x\in F$.
  \end{enumerate}
\end{Lem}
Observe que aquí $\varepsilon$ es fijo por lo que no nos da convergencia uniforme. Tenemos que producir un $f$ aparte.
\begin{ptcbp}
Considere
$$E_m=\bigcap_{k\geq m}\conj{|f-f_k|<\varepsilon},$$
entonces tenemos dos cosas
\begin{enumerate}
  \item $E_m\subseteq E_{m+1}$,
  \item $E\supseteq\bigcup_{m\in\bN}E_m\supseteq\tilde{E}$.
\end{enumerate}
Donde $\tilde{E}=\conj{x\in E\: \lim_{k\to\infty}f_k}$ \par
\textcolor{red}{pedir fotos y completar}.\par
Dado $\eta>0$, existe $k_0$ tal que
$$m(E\setminus E_m)<\frac{\eta}{2},\quad m\geq k_0. $$
Tome $F\subseteq E_{k_0}$ cerrado tal que $m(E_{k_0}\setminus F)<\frac{\eta}{2}$. Entonces $m(E\setminus F)<\eta$ y $|f(x)-f_k(x)|<\varepsilon$ cuando $k\geq k_0$.
\end{ptcbp}

Procedemos con el teorema de Egorov.

\begin{ptcbp}
Dado $\varepsilon>0$ y $m\geq 0$, existe $F_m\subseteq E$ y $k_m$ tal que
$$m(E\setminus F_m)<\frac{\varepsilon}{2^m}$$
y $|f(x)-f_k(x)|<\frac{1}{m}$ para $k\geq k_m$, $x\in F_m$. Tome $F=\bigcap_{m\in\bN}F_m$, entonces $F$ es cerrado. Note que
\begin{align*}
  m(E\setminus F) &=m\left(\bigcup_{m\in\bN} E\setminus F_m\right)\\
  &\leq \sum_{m\in\bN}m(E\setminus F_m)<\varepsilon.
\end{align*}
Sea $\eta>0$, tome $\frac{1}{m}<\eta$. Esto nos dice que $|f(x)-f_k(x)|<\frac{1}{m}<\eta$ para cualquier $k\geq k_m$ para $x\in F$. Entonces esto nos da convergencia uniforme en $F$.
\end{ptcbp}

\subsubsection*{El Teorema de Lusin}

\begin{Def}
  Dada $f\: E\to\bR$, decimos que $f$ tiene la propiedad $C$ si dado $\varepsilon>0$, existe $F\subseteq E$ cerrado tal que
  \begin{enumerate}
    \item $m(E\setminus F)<\varepsilon$.
    \item La función $f$ es continua en $F$.
  \end{enumerate}
\end{Def}

\begin{Th}
  Sea $\phi\: E\to\bR$ medible y simple. Entonces $\phi$ satisface la propiedad $C$.
\end{Th}

\begin{ptcbp}
Sabemos que existen $(b_i)_{i\in\bonj{\ell}},(B_i)_{i\in\bonj{\ell}}$ tal que
$$\phi(x)=\sum_{i\in\bonj{\ell}}b_i\mathbf{1}_{B_i}(x),$$
ya que $\phi$ es simple. Además $B_i\cap B_j=\emptyset$, $B_i$ medible y $E=\bigcup_{i\in\bonj{\ell}}B_i$. Sea $f_i\subseteq B_i$, cerrado, tal que $m(B_i\setminus F_i)<\frac{\varepsilon}{\ell}$.\par
Tome $F=\bigcup_{i\in\bonj{m}}F_i$, entonces
\begin{align*}
  m(E\setminus F) &=m(\bigcup_{i\in\bonj{\ell}}B_i\setminus F)\\
  &\leq \sum_{i\in\bonj{\ell}}m(B_i\setminus F)\\
  &\leq \sum_{i\in\bonj{\ell}}m(B_i\setminus F_i)\\
  &<\varepsilon.
\end{align*}

Sea $(x_n)_{n\in\bN}\subseteq F$, tal que $x_n\to x_0\in F$. Entonces existe $\ell_0$ tal que $x_0\in F_{\ell_0}$. Sabemos que $(x_n)_{n\in\bN}$ es acotada pues es convergente. Si asumimos que $\ell\neq \ell_0$ tal que $(x_n)_{n\in\bN}\cap F_\ell$ es infinita, entonces $(x_n)_{n\in\bN}\cap F_\ell=(x_{n_k})_{k\in\bN}$, entonces\par
\textcolor{red}{UHHHH Aglo}\par
Luego existe \textcolor{blue}{algo}, como $x_{n_{k_s}}\in F_\ell$, tenemos que $x_0\in F_\ell$. Pero esto es contradictorio. Entonces $(x_n)_{n\in\bN}\cap F_\ell$ es finito para todo $\ell\neq\ell_0$.\par
Luego existe $k_0$ tal que $x_n\in F_{\ell_0}$ para $n\geq k_0$. Entonces $\phi(x_n)=\phi(x_0)$ cuando $n\geq k_0$.
\end{ptcbp}

\subsection{Día 23| 14-6-18}
\subsubsection*{Séptima Sesión de Ejercicios}

\begin{Ej}[3.1  Wheeden \& Zygmund]
  Suponga que $b>1$ es entero y tome $x\in\bonj{0,1}$. Muestre que existe $(c_k)_{k\in\bN}\subseteq\bonj{b-1}^*$ tal que $x=\sum_{k\in\bN}\frac{c_k}{b^k}$. Muestre que la expansión es única excepto cuando $x=\frac{c}{b^k}$ con $c\in\bonj{b^k-1}$.
\end{Ej}

\begin{ptcbp}
En efecto, podemos tomar $c_1$ como el mayor entero tal que $\frac{c_1}{b}\leq x$. De manera inductiva podemos tomar $c_n$, el mayor entero tal que $\sum_{k\in\bonj{n}}\frac{c_k}{b^k}\leq x$. Tomamos el límite hacia infinito para obtener $$\sum_{k\in\bN}\frac{c_k}{b^k}\leq x.$$
Por maximalidad de $c_n$ tenemos que
$$\sum_{k\in\bonj{n}}\frac{c_k}{b^k}\leq x<\sum_{k\in\bonj{n-1}}\frac{c_k}{b^k}+\frac{c_n+1}{b^n}$$
para todo $n\in\bN$. Si restamos la suma de la izquierda obtenemos la desigualdad
$$x-\sum_{k\in\bonj{n}}\frac{c_k}{b^k}<-\frac{c_n}{b^n}+\frac{c_n+1}{b^n}=\frac{1}{b^n},\quad n\in\bN.$$
Nuevamente tomamos límites para obtener
$$x-\sum_{k\in\bN}\frac{c_k}{b^k}\leq 0\Rightarrow x\leq \sum_{k\in\bN}\frac{c_k}{b^k}.$$
Esto nos permite concluir que $x=\sum_{k\in\bN}\frac{c_k}{b^k}$.\par
Procedemos con unicidad, suponga que $x=\sum_{k\in\bN}\frac{c_k}{b^k}=\sum_{k\in\bN}\frac{d_k}{b^k}$ y sea $n=\min\conj{k\in\bN\:\ c_k\neq d_k}$. Es decir $c_n\neq d_n$ pero $c_k=d_k$ para $k<n$. Ahora tenemos que
$$\frac{c^n-d^n}{b^n}=\sum_{k\in\bonj{n}^c}\frac{d_k-c_k}{b^k}$$
y de aquí note que
$$\frac{1}{b^n}\leq |\frac{c_n-d_n}{b^n}|=\sum_{k\in\bonj{n}^c}\frac{|d_k-c_k|}{b^k}\leq\sum_{k\in\bonj{n}^c}\frac{b-1}{b^k}=\frac{1}{b^n}.$$
Hay desigualdad estricta si $|d_m-c_m|\neq b-1$.\par
\textcolor{red}{finish}
\end{ptcbp}

\begin{Ej}[3.7  Wheeden \& Zygmund]
Si $(I_k)_{k\in\bonj{n}}$ es una colección de intervalos que no se traslapan, entonces $\bigcup_{k\in\bonj{n}}I_k$ es medible y $m(\bigcup_{k\in\bonj{n}}I_k)=\sum_{k\in\bonj{n}}m(I_k)$.
\end{Ej}

\begin{ptcbp}
En efecto, tenemos que $\bigcup_{k\in\bonj{n}}I_k$ es una unión contable de conjuntos medibles. Ya sabemos que $m\left(\bigcup_{k\in\bonj{n}}I_k\right)\leq\sum_{k\in\bonj{n}}m(I_k)$. Así basta probar la otra desigualdad.\par
Primero, recuerde que $I_k=I_k^o\cup\partial I_k$, donde la frontera tiene medida cero. Tenemos la desigualdad $m(I_k)\leq m(I_k^o)+m(\partial I_k)=m(I_k^o)$ y como $I_k^o\subseteq I_k$ obtenemos $m(I_k)=m(I_k^o)$.\par
 En general tenemos $\bigcup_{k\in\bonj{n}}I_k^o\subseteq\bigcup_{k\in\bonj{n}}I_k$ por lo que $m\left(\bigcup_{k\in\bonj{n}}I_k^o\right)\leq m\left(\bigcup_{k\in\bonj{n}}I_k\right)$. Queremos la otra desigualdad, pero en efecto
 \begin{align*}
   m\left(\bigcup_{k\in\bonj{n}}I_k\right) &=m\left(\bigcup_{k\in\bonj{n}}(I_k^o\cup\partial I_k)\right)\\
   &\leq m\left(\bigcup_{k\in\bonj{n}}I_k^o\right)+m\left(\bigcup_{k\in\bonj{n}}\partial I_k\right)\\
   &\leq  m\left(\bigcup_{k\in\bonj{n}}I_k^o\right)+\sum_{k\in\bonj{n}}m\left(\partial I_k)\right)\\
   &=m\left(\bigcup_{k\in\bonj{n}}I_k^o\right).
 \end{align*}
 De esta manera $m\left(\bigcup_{k\in\bonj{n}}I_k\right)=m\left(\bigcup_{k\in\bonj{n}}I_k^o\right)$.\par
 Ahora sea $\varepsilon>0$. Tome $(F_k)_{k\in\bonj{n}}$ una colección de conjuntos $F_\sigma$ tales que $F_k\subseteq I_k^o$ y $m(I_k^o\setminus F_k)<\frac{\varepsilon}{n}$, para todo $k\in\bonj{n}$. Vea que $F_k\cap F_\ell =\emptyset$ y cada $F_k$ es compacto por el lema de Heine-Borel. Además tenemos que $\mm(F_k,F_\ell)>0$ cuando $k\neq \ell$ y $m(I_k^o)\leq m(F_k)+m(I_k\setminus F_k)=m(F_k)+\frac{\varepsilon}{n}$. Por lo que tenemos
 \begin{align*}
   \sum_{k\in\bonj{n}}m(I_k) &=\sum_{k\in\bonj{n}}m(I_k^o)\leq \sum_{k\in\bonj{n}}m(F_k)+\varepsilon\\
   &=m\left(\bigcup_{k\in\bonj{n}}F_k\right)+\varepsilon\\
   &\leq m\left(\bigcup_{k\in\bonj{n}}I_k\right)+\varepsilon.
 \end{align*}
 Luego $\sum_{k\in\bonj{n}}m(I_k)\leq m\left(\bigcup_{k\in\bonj{n}}I_k\right)$. Por lo tanto obtenemos la igualdad.
\end{ptcbp}


\begin{Ej}[3.9  Wheeden \& Zygmund]
 Suponga que $(E_k)_{k\in\bN}$ es una sucesión de conjuntos tal que $\sum_{k\in\bN}m_e(E_k)<\infty$. Muestre que $\limsup_{k\to\infty}E_k$ tiene medida cero. También el $\liminf_{k\to\infty} E_k$.
\end{Ej}

\begin{ptcbp}

Sea $\varepsilon>0$. Como $\lim_{n\to\infty}\sum_{k\in\bonj{n}}m_e(E_k)=\sum_{k\in\bN}m_e(E_k)<\infty$ , entonces $\exists N\in\bN$ tal que si $n\geq N$ entonces
$$\sum_{k\in\bonj{n}^c}m_e(E_k)=\sum_{k\in\bN}m_e(E_k)-\sum_{k\in\bonj{n}}m_e(E_k)<\varepsilon.$$
Sabemos que
\begin{gather*}
  \limsup E_k=\bigcap_{\ell\in\bN}\left(\bigcup_{k\in\bonj{\ell}^c}E_k\right)\subseteq\bigcup_{k\in\bonj{N}^c}E_k.
\end{gather*}
Tomando medidas tenemos
\begin{align*}
  m_e(\limsup E_k) & \leq m_e\left(\bigcup_{k\in\bonj{N}^c}E_k\right)\\
  &\leq\sum_{k\in\bonj{n}^c}m_e(E_k)<\varepsilon.
\end{align*}
De esta manera $\limsup E_k$ tiene medida cero. Como $\liminf E_k\subseteq\limsup E_k$ el resultado se sigue.
\iffalse
existe $N\in\bN$ tal que $\sum_{k\in\bonj{N-1}^c}m_e(E_k)<\varepsilon$. De esta manera
$$\limsup_{k\to\infty} E_k=\bigcap_{j\in\bN}\left(\bigcup_{k\in\bonj{j-1}^c}E_k\right)\subseteq\bigcup_{k\in\bonj{m-1}^c}E_k.$$
Luego tenemos que
\begin{align*}
  m_e(\limsup_{k\to\infty} E_k) &<m_e\left(\bigcup_{k\in\bN}E_k\right)\\
                            &\leq\sum_{k\in\bN}m_e(E_k)\\
                            &<\varepsilon.
\end{align*}
Por tanto  $\limsup_{k\to\infty} E_k$ tiene medida cero. Como el $\limsup$ le gana al $\liminf$ tenemos el resultado.
\fi
\end{ptcbp}

\begin{Ej}[3.13  Wheeden \& Zygmund]
 Defina la medida interior de $E$ como
 $$\sup\conj{m(F)\:\ F\subseteq E, \text{ cerrado }}.$$
 Muestre que
 \begin{enumerate}
   \item $m_i(E)\leq m_e(E)$,
   \item Si $m_e(E)<\infty$ entonces $E$ es medible si y sólo si $m_i(E)=m_e(E)$.
 \end{enumerate}
\end{Ej}

\begin{ptcbp}
\begin{enumerate}
  \item En efecto, sea $F\subseteq E$ un cerrado. Entonces $m(F)\leq m_e(E)$, por lo que $m_e(E)$ es cota superior de $m(F)$ y así $m_i(E)\leq m_e(E)$.
  \item Primero suponga que $E$ es medible, tome $F\subseteq E$ cerrado tal que $m_e(E)-m(F)=m_e(E\setminus F)<\varepsilon$. Inmediatamente por definición de $\sup$ tenemos que $\sup(m(F))=m_e(E)$.\par
      Ahora si $m_i(E)=m_e(E)$ entonces existe $F\subseteq E$ cerrado tal que $m_e(E)-m(F)<\frac{\varepsilon}{2}$. Además existe un $G$ abierto que contiene a $E$ y $m(G)-m_e(E)<\frac{\varepsilon}{2}$. Sumando lo anterior tenemos que
      \begin{align*}
        m(G)-m(F)<\varepsilon &\Rightarrow m(G\setminus F)<\varepsilon\\
        &\Rightarrow m_e(G\setminus F)<\varepsilon\\
        &\Rightarrow m_e(E\setminus F)<\varepsilon.
      \end{align*}
      Por tanto $E$ es medible.
 \end{enumerate}
\end{ptcbp}

\begin{Ej}[3.15  Wheeden \& Zygmund]
 Si $E$ es medible y $A\subseteq E$, muestre que
 $$m(E)=m_i(A)+m_e(E\setminus A).$$
\end{Ej}

\begin{ptcbp}
En efecto, tomemos $F\subseteq A$ un cerrado. Podemos invertir este contenido con complemento de manera que $F^c\supseteq A^c$ donde el complemento se toma respecto a $E$. Como $F$ es cerrado, $F^c$ es abierto y por tanto medible. Luego
$$m(F)+m_e(A^c)\leq m(F)+m(F^c)=m(E).$$
Si tomamos el $\sup$ sobre $F\subseteq A$ cerrados, obtenemos que
$$m_i(A)+m_e(A^c)\leq m(E).$$
De manera análoga, considere $G\supseteq A^c$ un abierto. Tomando complementos tenemos $G^c\subseteq A$ con $G^c$ cerrado y así
$$m_i(A)+m(G)\geq m(G^c)+m(G)=m(E).$$
Ahora tomamos $\inf$ sobre los abiertos que contienen a $A^c$ de manera que obtenemos
$$m_i(A)+m_e(A^c)\geq m(E).$$
Concluimos el resultado pues tenemos las dos desigualdades.
\end{ptcbp}

\subsection{Día 24| 19-6-18}
\subsubsection*{Octava Sesión de Ejercicios}

\begin{Ej}[3.12  Wheeden \& Zygmund]
Sean $E_1,E_2\subseteq\bR$, muestre que $E_1\times E_2\subseteq\bR^2$ es medible y $m(E_1\times E_2)=m(E_1)m(E_2)$. \par
Como sugerencia use caracterización de medibilidad.
\end{Ej}

\begin{Ej}[3.20  Wheeden \& Zygmund]
Muestre que existe una colección $(E_k)_{k\in\bN}$ de conjuntos disjuntos tales que
$$m_e\left(\bigcup_{k\in\bN}E_k\right)<\sum_{k\in\bN}m_e(E_k),$$
con desigualdad estricta.\par
Como sugerencia considere un conjunto no medible en $\bonj{0,1}$ cuyas traslaciones por racionales son disjuntas. Si consideramos todas las traslaciones por $r\in\bQ\cap\bonj{0,1}$ y usamos el ejercicio 3.18 obtendremos el resultado.
\end{Ej}

\begin{ptcbp}
En efecto, siguiendo la sugerencia sea $E$ el conjunto de Vitali. \par
Primero note que para cualquier real $x$, existe $r\in\bQ$ tal que $x-r\in\bonj{0,1}$. De esta manera, cada clase de equivalencia (de la relación por la cual se define el conjunto de Vitali) tiene un representante en $\bonj{0,1}$.\par
Por el axioma de elección podemos asumir que $E\subseteq\bonj{0,1}$. Ahora sea $(r_k)_{k\in\bN}$ una enumeración de $\bQ\cap\bonj{0,1}$.\par
Defina $$E_k=\conj{x+r_k\: x\in E}=E+r_k,$$ y observe que si tenemos $z\in E_k\cap E_\ell$ entonces existen $x,y\in E$ tales que
$$x+r_k=y+r_\ell=z.$$
Pero esto nos dice que $x-y\in\bQ$ y esto es imposible pues significaría que $x\sim y$  así $\bonj{x}=\bonj{y}$. Por construcción sabemos que solo hay un representante por cada clase en $E$. La única otra posibilidad es que $r_k=r_\ell$, lo que nos lleva a concluir que los conjuntos $E_k$'s son disjuntos.\par
Ahora por invariancia de traslación tenemos
$$m_e(E_k)=m_e(E)>0.$$
Finalmente observe que $\bigcup_{k\in\bN}E_k\subseteq \bonj{0,2}$, pues lo más que van a trasladar los $r_k$ un conjunto va a ser por 1. Entonces
$$m_e\left(\bigcup_{k\in\bN}E_k\right)\leq 2.$$
Sin embargo,
$$\sum_{k\in\bN}m_e(E_k)=\sum_{k\in\bN}m_e(E)\to\infty.$$
Por lo tanto
$$m_e\left(\bigcup_{k\in\bN}E_k\right)<\sum_{k\in\bN}m_e(E_k).$$
\end{ptcbp}

\begin{Ej}[3.23  Wheeden \& Zygmund]%%http://jupiter.math.nctu.edu.tw/~sheu/Quiz1Sol.pdf
Suponga que $Z\subseteq\bR$ tiene medida cero. Muestre que $\conj{x^2\:x\in Z}$ es de medida cero.
\end{Ej}

\begin{ptcbp}
En efecto, vamos a despedazar $Z$ en bolas. Considere $Z_n=Z\cap B(0,n)$ con $n\in\bN$. Entonces tenemos
$$Z=\bigcup_{n\in\bN}Z_n,\quad\text{ y }\quad m(Z_n)=0.$$
Luego, existe intervalos $I_{k,n}=\bonj{a_k,b_k}$ tales que $Z_n\subseteq\bigcup_{k\in\bN}I_{k,n}$ con $\sum_{k\in\bN}m(I_k)<\varepsilon$. En particular podemos suponer $I_{k,n}\subseteq B(0,n)$. Ahora, queremos ver cómo se comportan $a_k^2, b_k^2$. Observe que para $x\in I_k$ tenemos
$$
\begin{cases}
  a_k^2\leq x^2\leq b_k^2, & \mbox{cuando }\ 0\leq a_k\leq b_k. \\
  0\leq x^2\leq\max\conj{a_k^2,b_k^2}, & \mbox{cuando }\ a_k\leq0\leq b_k. \\
  b_k^2\leq x^2\leq a_k^2, & \mbox{cuando}\ a_k\leq b_k\leq 0.
\end{cases}
$$
Observe que en el segundo caso, podemos ver $I_k$ como $\bonj{a_k,0}\cup\bonj{0,b_k}$. Así, siempre podremos tomar los casos $0\leq a_k\leq b_k$ o $a_k\leq b_k\leq 0$.\par
Vamos a trabajar con nuevos intervalos $J_k$'s que contengan a los $x^2$ de manera que
\begin{align*}
 m(J_k)&=|a_k^2-b_k^2|\\
 &=|a_k-b_k| |a_k+b_k|\\
 &\leq m(I_k)(|a_k|+|b_k|)\leq m(I_k)(2n).
\end{align*}
De esta manera obtenemos
\begin{align*}
  m_e(\conj{x^2\: x\in Z_n}) & \leq m_e\left(\bigcup_{k\in\bN}J_k\right)\\
  &\leq \sum_{k\in\bN}m(J_k)\\
  &\leq 2n\sum_{k\in\bN}m(I_k)<2n\varepsilon.
\end{align*}
Esto nos dice que $\conj{x^2\: x\in Z_n}$ tiene medida cero para todo $n\in\bN$. Finalmente, podemos ver $\conj{x^2\: x\in Z}$ como una unión de conjuntos de medida cero de manera tal que
\begin{gather*}
  \conj{x^2\: x\in Z}=\bigcup_{n\in\bN}\conj{x^2\: x\in Z_n},\\
  \Rightarrow m_e(\conj{x^2\: x\in Z})\leq\sum_{n\in\bN}m_e(\conj{x^2\: x\in Z_n})=0.
\end{gather*}
Por lo tanto $\conj{x^2\:x\in Z}$ es de medida cero.
\end{ptcbp}

\subsubsection*{El Teorema de Lusin}

Recuerde que una función $f\: E\to\overline{\bR}$ se dice tener la propiedad $C$ si para todo $\varepsilon>0$, existe $F\subseteq E$ tal que
\begin{enumerate}
  \item $m(E\setminus F)<\varepsilon,$
  \item $f$ es continua en $F$.
\end{enumerate}

\begin{Th}[Lusin]
  Sea $f\:E\to\bR$, con $E$ medible. Entonces $f$ es medible si y sólo si $f$ tiene la propiedad $C$.
\end{Th}

\begin{ptcbp}
Suponga que $f$ es medible, luego existe una sucesión $(f_n)_{n\in\bN}$ de funciones medibles simples tal que $f_n\to f$.\par
Cuándo es que la conergencia puntual y continuidad implican convergencia uniforme? Hay que devolvernos al teorema de Egorov. Primero asumimos que $m(E)<\infty$. Sea $F_n$ cerrado tal que $m(E\setminus F_n)<\varepsilon$ y $f_n$ es continua en $F_n$.\par
Por el teorema \ref{thm:Egorov} de Egorov, existe $F$ tal que
$$m(E\setminus F)<\varepsilon$$
y $f_n\to f$ uniformemente. Sea
$$\tilde{F}=F\cap\bigcap_{n\in\bN}F_n$$
 y queremos que $E\setminus\tilde{F}$ de medida pequeña. Observe que
 \begin{align*}
   m(E\setminus\tilde{F}) &\leq m(E\setminus F)+\sum_{n\in\bN}m(E\setminus F_n)\\
   &<\varepsilon.
 \end{align*}
 Además en $\tilde{F}$, $f$ es límite de funciones continuas de manera uniforme. Por tanto tenemos el resultado por Egorov.\par
 El problema es que todavía tenemos la medida finita, eso no es parte de las hipótesis. Para eso, vamos a despedazar en trozos de medida finita. Si $m(E)=\infty$, considere
 $$E_n=E\cap\conj{x\: n\leq|x|<n+1},\ n\geq 0.$$
 Sabemos que existen $F_n\subseteq E_n$ de manera que
 $$m(E_n\setminus F_n)<\frac{\varepsilon}{2^n},$$
 y $f$ es continua en $F_n$. El truco es mostrar que $f$ es continua en $\bigcup_{n\in\bN} F_n$ y que este conjunto es cerrado. Llamemos a esta unión $\hat{F}$, vamos a probar las dos cosas al mismo tiempo. Utilizamos continuidad sucesional y cerradura por sucesiones.\par
 Sea $(x_n)_{n\in\bN}\subseteq \hat{F}$ tal que $x_n\to x$. Sabemos que existe $M\in\bN$ tal que $|x_n|\leq M$ para todo $n\in\bN$.  Luego
 $$(x_n)_{n\in\bN}\subseteq \bigcup_{n\in\bonj{M}}F_n$$
 que es cerrado. Así $x\in \bigcup_{n\in\bonj{M}}F_n\subseteq \hat{F}.$\par
 Ahora suponga que $f$ tiene la propiedad $C$. Sea $F_j$ cerrado tal que
 $$m(E\setminus F_j)<\frac{1}{j},$$
 y $f$ continuaen $F_j$ para todo $j$. Tome $A=\bigcup_{j\in\bN}F_j$. Cuanto es $m(E\setminus A)$? Observe que esta medida es menor que cualquier cosa, es decir
 $$m(E\setminus A)\leq m(E\setminus F_j)<\frac{1}{j}.$$
 Así $m(E\setminus A)=0$. Finalmente
 \begin{gather*}
   \conj{x\in E\: f(x)> a}\\
   =\bigcup_{j\in\bN}\conj{x\in F_j\: f(x)>a}\cup\conj{x\in E\setminus A\: f(x)>a}.
 \end{gather*}
\end{ptcbp}

\subsubsection*{Convergencia en medida}

Esta convergencia es la convergencia más debil. Muchas convergencias implican estas, pero esta no implica nada.

\begin{Def}
  Sean $f,f_n\: E\to\overline{\bR}$ finitas casi por doquier. Decimos que $f_n$ converge a $f$ en medida si
  $$\forall\varepsilon>0\lim_{n\to\infty}m\conj{|f_n-f|>\varepsilon}=0.$$
  Denotamos $f_n\xrightarrow[]{m}f$
\end{Def}

Esto dice que el conjunto donde no converge, en medida va para cero. Vamos a empezar a estudiar modos de convergencia.

\begin{Th}
  Sean $f,f_n\: E\to\overline{\bR}$ medibles y finitas casi por doquier. Si $f_n\to f$ casi por doquier en $E$ y $m(E)<\infty$ entonces $f_n\xrightarrow[]{m} f$ en $E$.
\end{Th}

\begin{ptcbp}
Esencialmente queremos usar el teorema \ref{thm:Egorov} de Egorov. Sean $\eta, \varepsilon$, hay que mostar que existe $n_0$ tal que $$m\conj{|f_n-f|>\varepsilon}<\eta,$$
cuando $n\geq n_0$. Sea $F\subseteq E$ tal que $m(E\setminus F)$ y $f_n\to f$ uniformemente en $F$. Entonces existe $n_0$ tal que
$$|f_n(x)-f(x)|<\varepsilon,\quad n\geq n_0, x\in F.$$
Luego $\conj{x\in E\:|f_n(x)-f(x)|>\varepsilon}\subseteq E\setminus F,$
cuando $n\geq n_0$.

\end{ptcbp}

Vamos a ver una sucesión que converge en medida pero no puntualmente. Considere la sucesión
\begin{gather*}
  I_0=\ind_{\bonj{0,1}},\\
  I_{1,1}=\ind_{\bonj{0,\frac{1}{2}}}, I_{1,2}=\ind_{\bonj{\frac{1}{2},1}},\\
  I_{2,i}=\ind_{\bonj{\frac{i-1}{2^2},\frac{i}{2^2}}}, i\in\bonj{4}.
\end{gather*}
En general $I_{n,i}=\ind_{\bonj{\frac{i-1}{2^n},\frac{i}{2^n}}}$ con $i\in\bonj{2^n}$. Esta sucesión no puede converger puntualmente, es cero o uno en todo lado. Pero en medida, la medida de los intervalos es $\frac{1}{2^n}$ que va para cero. Sin embargo, así como está presentada no es una sucesión!\par
Usamos el orden
$$(m,i)\leq(n,j)\iff (m\leq n)\lor ((m=n)\land i\leq j),$$
tenemos una sucesión que no converge en ningún punto pero converge a cero en medida.\par

Convergencia en medida nunca implica convergencia puntual. Sin embargo convergencia puntual implica convergencia en medida, pero sólo en medida finita. Veamos un contraejemplo, una sucesión que converge puntualmente pero no converge en medida.\par
Considere la sucesión
$$f_n=\ind_{\bonj{-n,n}}\to 1,$$
pero $1-f_n=\ind_{\bonj{-n,n}^c}=\ind_{\bonj{-\infty,-n}}+\ind_{\bonj{n,\infty}}$. Es decir $f_n$ no converge en medida a $1$.

\begin{Th}
  Sea $f_n\xrightarrow[]{m}f$ en $E$. Entonces existe $n_j$ de manera que $f_{n_j}\to f$ casi por doquier.
\end{Th}

\begin{ptcbp}
Dado $j$, existe $n_j$ tal que
$$m\conj{|f_n-f|>\frac{1}{j}}\leq\frac{1}{2^j},\ n\geq n_j.$$
Sin perdida de generalidad $n_j\leq n_{j+1}$, entonces
$$m\conj{|f_{n_j}-f|>\frac{1}{j}} < \frac{1}{2^j}.$$
Tome $E_j=\conj{|f_{n_j}-f|>\frac{1}{j}}$, y
$$H_m=\bigcup_{j\in\bonj{m-1}^c}\conj{|f_{n_j}-f|>\frac{1}{j}}.$$
Entonces $H_m\supseteq H_{m+1}$ y
\begin{align*}
  m(H_m) &\leq \sum_{j\in\bonj{m-1}^c}m\conj{|f_{n_j}-f|>\frac{1}{j}}\\
  &=\sum_{j\in\bonj{m-1}^c}\frac{1}{2^j}=\frac{1}{2^{m-1}}.
\end{align*}
Luego tome $Z=\bigcup_{m\in\bN}H_m$ tiene medida cero.\par
Si $x\in E\setminus Z=\bigcup_{m\in\bN}E\setminus H_m$. Esto nos dice que existe $m_0$ tal que $x\in E\setminus H_{m_0}$.
\begin{gather*}
  \Rightarrow x\in\bigcap_{j\in\bonj{m-1}^c} E\setminus\conj{|f_{n_j}-f|>\frac{1}{j}},\\
  \Rightarrow x\in\bigcap_{j\in\bonj{m-1}^c} \conj{|f_{n_j}-f|\leq\frac{1}{j}},\\
  \Rightarrow f_{n_j}\xrightarrow[j\to\infty]{}f.
\end{gather*}
\end{ptcbp}
\begin{Def}
  Decimos que $(f_n)_{n\in\bN}$ es Cauchy en medida si para todo $\varepsilon,\eta>0$ existe $N\in\bN$ tal que
  $$m\conj{|f_n-f_m|>\varepsilon}<\eta,$$
  para $n,m\geq N$.
\end{Def}

Inmediatamente convergencia en medida implica Cauchy en medida.

\begin{Lem}
  Una sucesión de Cauchy en medida es convergente en medida.
\end{Lem}

\begin{ptcbp}
Sabemos que existe $n_j$ tal que
$$m\conj{|f_n-f_m|>\frac{1}{2^j}}<\frac{1}{2^j},$$
si $n,m\geq n_j$. Sin perdida de generalidad suponga que $n_j<n_{j+1}$ y tome
\begin{gather*}
  E_j=\conj{|f_{n_{j+1}}-f_{n_j}|>\frac{1}{2^j}}\\
  H_m=\bigcup_{j\in\bonj{m-1}^c} E_j.
\end{gather*}
Entonces $m(H_m)\leq \frac{1}{2^{m-1}}$, tome $x\in E\setminus H_m$. Así
$$|f_{n_{j+1}}-f_{n_j}|<\frac{1}{2^j},\quad j\geq m.$$
Luego la serie $\sum_{j\in\bonj{m-1}^c}f_{n_{j+1}}-f_{n_j}$ converge absolutamente y uniformemente por el $M$-test de Weierstrass. Así existe $f$ el límite de $f_{n_j}$ uniforme en $E\setminus H_m$.\par
Qué tenemos que probar ahora? Que $f_{n_j}$ converge a $f$ en medida. Tome $x\in E\setminus H_m$ y $k\geq m$.
\begin{align*}
  |f_{n_k}(x)-f_{n_m}(x)| &\leq\sum_{j\in\bonj{k-1}\setminus\bonj{m-1}}|f_{n_{j+1}}(x)-f_{n_j}(x)|\\
  &\leq\sum_{j\in\bonj{k-1}\setminus\bonj{m-1}}\frac{1}{2^j}<\frac{1}{2^{m-1}}.
\end{align*}
 Luego $|f(x)-f_{n_m}(x)|\leq\frac{1}{2^{m-1}}$ si $x\in E\setminus H_m$. \par
Hay que mostrar que dado $\varepsilon,\eta>0$, existe $N\in\bN$ tal que
$$m\conj{|f-f_k|>\varepsilon}<\eta,\quad k\geq N.$$
Si $\frac{1}{2^{m-1}}<\varepsilon,\eta$, entonces
$$\conj{|f_k-f|>\varepsilon}\subseteq H_m.$$
Es decir $m\conj{|f_k-f|>\varepsilon}\leq m(H_m)\leq \frac{1}{2^{m-1}}<\eta$.
\end{ptcbp}

\subsubsection*{La función de Cantor}

INSERTAR FIG 24.1 función de Cantor\par
Sea $$D_k=\bonj{0,1}\setminus C_k=\bigcup_{j\in\bonj{2^k-1}}I_j,$$
con $I_j=\obonj{\alpha_j,\beta_j}$ y $0<\alpha_1<\beta_1<\cdots<\alpha_{2^k-1}<\beta_{2^k-1}<1$. Definimos
$$
f_n(x)
=\begin{cases}
  0, &\text{ cuando }\ x=0,\\
   \frac{j}{2^n}, &\text{ si}\ x\in\bonj{\alpha_i,\beta_i},\\
   \text{lineal entre}, &(\beta_j,\frac{j}{2^n}),(\alpha_{j+1},\frac{j+1}{2^n}),\\
   1, &\text{ si }\ x=1.
 \end{cases}
$$
Entonces $f_n$ es creciente, continua y $(f(0),f(1))=(0,1)$. Note que
$$D_{k+1}=\bigcup_{j\in\bonj{2^k-1}}I_j\cup\bigcup_{i\in\bonj{2^k}}\obonj{\tilde{\alpha}_i,\tilde{\beta}_i},$$
con $0<\tilde{\alpha}_i<\tilde{\beta}_i<\alpha_1$, $\beta_{2^k-1}<\tilde{\alpha}_{2^k}<\tilde{\beta}_{2^k}<1$ y $\beta_{j-1}<\tilde{\alpha}_{j}<\tilde{\beta}_{j}<\alpha_j$.\par
Entonces
\begin{gather*}
  f_{k+1}(x)=\frac{2j}{2^{k+1}},\quad x\in\bonj{\alpha_i,\beta_i},\\
  f_{k+1}(x)=\frac{2j-1}{2^{k+1}},\quad x\in\bonj{\tilde{\alpha}_i,\tilde{\beta}_i}.
\end{gather*}
Además $|f_k(x)-f_{k-1}(x)|<\frac{1}{2^{k}}$. Entonces $\sum_{k}f_{k+1}-f_k$ converge uniformemente, luego $f_k\to f$ uniformemente. Luego la función $f$ es continua, creciente, $f(0)=0, f(1)=1$ y
$$f_k(x)=f(x),\quad\text{ en }\ D_k.$$
Luego si $x\in D_k$, $f'(x)=0$.

\subsection{Día 25| 21-6-18}

\subsubsection*{La integral de Lebesgue}

Sea $f\: E\to\overline{\bR}$, con $f$ positiva. Vamos a definir la curva de la función.

\begin{Def}
  La gráfica o curva de una función es
  $$G(f,E)=\conj{(x,f(x))\in E\times\bR\: f(x)<\infty}.$$
  Ahora los puntos bajo esta curva serán llamados
  $$R(f,E)=\conj{(x,y)\in E\times\bR\: 0\leq y\leq f(x)}.$$
\end{Def}
Observe que $G\subseteq R$, pues $R$ es la curva y toda el area debajo. También tenemos que $R(f,E)\subseteq\bR^{d+1}$ cuando $E\subseteq\bR^d$. Procedemos a definir la integral como el área bajo la curva.

\begin{Def}
  Si $R(f,E)$ es medible, entonces definimos
  $$\int_{E}f\dd m=m(R(f,E)).$$
  Si $R(f,E)$ es de medida finita la integral existe.
\end{Def}

\begin{Lem}\label{lem:posYMedImplicaAreaMedible}
  Sea $f\: E\to\overline{\bR}$, con $f$ positiva y medible. Entonces $R(f,E)$ es medible.
\end{Lem}

Cómo probamos esto? Empezamos con indicadores, si no sirve, tratamos con simples. Vamos a despedazar esto en lemas.

\begin{Lem}
  Sea $f=a\ind_A$ con $A$ medible, entonces $R(f,E)$ es medible.
\end{Lem}

\begin{ptcbp}
Tenemos que el conjunto en cuestión es
$$R(f,E)=\conj{(x,y)\in E\times\bR\: 0\leq y\leq a\ind_A},$$
pero esto se puede describir como
$$\conj{(x,y)\: x\in A,\ 0\leq y\leq a}\cup\conj{(x,y)\: x\in E\setminus A, y=0}.$$
Entonces recordando el ejercicio 4 del examen parcial 2, podemos ver estos conjuntos como
$$A\times\bonj{0,a}\cup E\setminus A\times\conj{0}.$$
Vamos a probar que
$$A\times\bonj{a,b}\text{ es medible }\iff A\text{ es medible}.$$
Además $m(A\times\bonj{a,b})=(b-a)m(A)$.\par
\begin{enumerate}
  \item Si $A=\bigtimes_{i\in\bonj{d}}\bonj{a_i,b_i}$, entonces $A\times\bonj{a,b}$ es una caja $d+1$ dimensional. Este conjunto es medible y la medida es el producto de todas las longitudes.
  \item Si $A$ es un abierto $G$ entonces podemos expresar
  $$G=\bigcup_{k\in\bN}I_k,\quad I_k^o\cap I_\ell^o=\emptyset,$$
  con $I_k=\bigtimes_{i\in\bonj{d}}\bonj{a_i(k),b_i(k)}$. Entonces
  $G\times\bonj{a,b}=\left(\bigcup_{k\in\bN}I_k\right)\times\bonj{a,b}$ es medible y tenemos que
  \begin{align*}
    m(G\times\bonj{a,b}) &=\sum_{k\in\bN}m(I_k\times\bonj{a,b})\\
    &=(b-a)\sum_{k\in\bN}m(I_k)\\
    &=(b-a)m(G).
  \end{align*}
  \item Si $Z\subseteq E$ es de medida cero, dado $\varepsilon>0$ existe $G\supseteq Z$ abierto tal que $m(G\setminus Z)=m(G)<\varepsilon$.\par
      Entonces
      \begin{align*}
        m(Z\times\bonj{a,b}) &\leq m(G\times\bonj{a,b})\\
        &=\varepsilon(b-a).
      \end{align*}
  \item Sea $A$ un conjunto $G_\delta$, entonces $$A=\bigcap_{j\in\bN}G_j,$$ con $G_j$ abierto para $j\in\bN$. Cómo trabajamos con sucesiones decrecientes? Forzándolo!\par
      Defina $$\tilde{G}_j=\bigcap_{i\in\bonj{j}}G_i\supseteq A,$$
      un abierto. Tenemos $\tilde{G}_j\supseteq\tilde{G}_{j+1}$. Observe que
      \begin{align*}
        A\times\bonj{a,b} &=\left(\bigcap_{j\in\bN}G_j\right)\times\bonj{a,b}\\
        &= \left(\bigcap_{j\in\bN}\tilde{G}_j\right)\times\bonj{a,b}\\
        &= \bigcap_{j\in\bN}\left(\tilde{G}_j\times\bonj{a,b}\right),\\
      \end{align*}
      es medible. Si $A$ tiene medida finita, entonces
       \begin{align*}
        m(A\times\bonj{a,b}) &=m\left(\bigcap_{j\in\bN}\left(\tilde{G}_j\times\bonj{a,b}\right)\right)\\
        &=\lim_{j\to\infty}m\left(\tilde{G}_j\times\bonj{a,b}\right)\\
        &=(b-a)\lim_{j\to\infty}m\left(\tilde{G}_j\right)\\
        &=m(A)(b-a).
      \end{align*}
      Finalmente si $A=H\setminus Z$ con $H$ un $G_\delta$ y $Z$ de medida cero,
      $$A\times\bonj{a,b}=(H\times\bonj{a,b})\setminus(Z\times\bonj{a,b}),$$
      con $Z\times\bonj{a,b}$ un conjunto de medida cero.\par
      Todo esto sirvió para medida finita, pero esto se puede arreglar facilmente para medida infinita. Si $m(A)=\infty$, sea
      $$A_k=A\cap (B(0,k)\setminus B(0,k-1)).$$
      Entonces despedazamos $A$ e anillos y así los $A_k$ son disjuntos por parejas. De esta manera
      $$A\times\bonj{a,b}=\bigcup_{k\in\bN}\left(A_k\times\bonj{a,b}\right).$$
\end{enumerate}
Recordemos que
$$R(a\ind_A,E)=(A\times\bonj{0,a})\cup((E\setminus A)\times\conj{0}),$$
entonces $\int_{E}a\ind_A\dd m=am(A)$.
\end{ptcbp}

El paso que seguiría sería simples, pero vamos tomar un pequeño desvío. Cuánto debería de medir solamente la curva de la función? Intuitivamente debería ser cero. Esencialmente la integral va a ser lo mismo con o sin la curva de la función.

\begin{Lem}
  Sea $f\: E\to\overline{\bR}$ medible, positiva. Entonces $m(G(f,E))=0$.
\end{Lem}

El truco de esta prueba va para el bolsillo de trucos.

\begin{ptcbp}
Sea $\varepsilon>0$. Recordemos que
 $$G(f,E)=\conj{(x,f(x))\in E\times\bR\: f(x)<\infty}.$$
 Ahora definimos
 $$E_k=\conj{x\in E\: k\varepsilon\leq f(x)\leq(k+1)\varepsilon},$$
 entonces podemos cubrir $G(f,E)$ como
 $$G(f,E)\subseteq\bigcup_{k\in\bN}\left(E_k\times\bonj{k\varepsilon,(k+1)\varepsilon}\right).$$
 El truco aquí es \emph{no se agarre con cosas complicadas de medir. Aproxímelas!}\par
 Ahora si $E$ es de medida finita, entonces
 \begin{align*}
   m_e(G(f,E)) &\leq\sum_{k\in\bN}m(E_k\times\bonj{k\varepsilon,(k+1)\varepsilon})\\
   &\leq \varepsilon\sum_{k\in\bN}m(E_k)=\varepsilon m(E).
 \end{align*}
 Esto nos dice que la gráfica tiene medida cero. Resta ver qué pasa con medida infinita de $E$.
\end{ptcbp}

\begin{Ej}
  Complete el caso de $m(E)=\infty$ de la prueba anterior despedazando $E$ en anillos.
\end{Ej}

Ahora volvemos al lema \ref{lem:posYMedImplicaAreaMedible}. Ya tenemos las herramientas necesarias para probarlo.

\begin{ptcbp}
Como $f$ es positiva y medible, existe $(f_k)_{k\in\bN}$ una sucesión de funciones simples tal que
\begin{enumerate}
  \item $0\leq f_k(x)\leq f_{k+1}(x)$ para todo $x\in E$,
  \item $f_k\xrightarrow[]{k\to\infty}f$.
\end{enumerate}
Entonces podemos expresar
\begin{align*}
  R(f,E) &=\conj{(x,y)\: x\in E,\ 0\leq y\leq f(x)}\\
  &=\conj{(x,y)\: x\in E,\ 0\leq y< f(x)}\cup G(f,E).
\end{align*}
Pero por nuestras funciones simples tenemos
\begin{gather*}
\conj{(x,y)\: x\in E,\ 0\leq y< f(x)}\\
=\bigcup_{k\in\bN}\conj{(x,y)\: x\in E,\ 0\leq y< f_k(x)}.
\end{gather*}
La igualdad se cumple pues si $y<f(x)$ en algún punto debe haber un $f_k(x)$ muy cercano a $f(x)$, más que $y$. La otra dirección es inmediata de nuestra condición $\mathit{1}$ de la sucesión.\par
Recuerde que las funciones simples se pueden escribir como
$$f_k(x)=\sum_{i\in\bonj{m_k}}a_{i,k}\ind_{A_{i,k}},$$
con $\bigcup_{i\in\bonj{m_k}}A_{i,k}= E$ donde los conjuntos son disjuntos por parejas. Entonces tenemos que
\begin{gather*}
  \conj{(x,y)\: x\in E,\ 0\leq y< f_k(x)}\\
  =\bigcup_{i\in\bonj{m_k}}\conj{(x,y)\:\ x\in A_{i,k},\ 0\leq y\leq a_{i,k}},
\end{gather*}

es un conjunto medible. Se cumple que
\begin{align*}
  m(R(f,E)) &=\sum_{i\in\bonj{m_k}}m(A_{i,k}\times\bonj{0,a_{i,k}})\\
  &=\sum_{i\in\bonj{m_k}}a_{i,k}m(A_{i,k}).
\end{align*}
\end{ptcbp}

Esto ya nos permite calcular la integral de funciones simples.
\begin{Cor}
  Si $f(x)=\sum_{i\in\bonj{m_k}}a_i\ind_{A_i},$
  es una función simple con $A_i\cap A_j$ disjuntos, entonces
  $$\int_{E}f\dd m=\sum_{i\in\bonj{m}}a_im(A_i).$$
\end{Cor}

Veremos propiedades fáciles de la integral de Lebesgue.

\begin{Th}
  Sean $f,g\: E\to\overline{\bR}$ medibles, positivas. Se cumplen las siguientes propiedades
  \begin{enumerate}
    \item Si $g(x)\leq f(x)$ para $x\in E$, entonces $\int_{E}g\dd m\leq \int_{E}f\dd m$.
    \item Si $E_1\subseteq E_2\subseteq E$, entonces $\int_{E_1}f\dd m\leq \int_{E_2}f\dd m$.
    \item Si $\int_{E}f\dd m<\infty$ entonces $m(\conj{f=\infty})=0$.
  \end{enumerate}
\end{Th}

\begin{ptcbp}
\begin{enumerate}
  \item En efecto, si $g(x)\leq f(x)$ entonces $R(g,E)\subseteq R(f,E)$ y así $m(R(g,E))\leq m(R(f,E))$.
  \item Como $E_1\subseteq E_2$, entonces $R(f,E_1)\subseteq R(f,E_2)$. El resto de la prueba sigue igual que el apartado anterior.
  \item Sea $E_1\subseteq E$ y considere $f\ind_{E_1}$. Entonces
  $$\int_{E}f\ind_{E_1}\dd m=\int_{E_1}f\dd m,$$
  pues podemos ver $R(f\ind_{E_1},E)$ como
  \begin{gather*}
    \conj{(x,y)\: x\in E,\ 0\leq y\leq f\ind_{E_1}}\\
    =(E\setminus E_1\times\conj{0})\cup\conj{(x,y)\: x\in E_1,\ 0\leq y\leq f(x)}.
  \end{gather*}
  Así tome $E_1=\conj{f=\infty}$, entonces
  \begin{align*}
    \ind_{E_1}\leq f &\Rightarrow n\ind_{E_1}\leq f\\
    &\Rightarrow n(m(E_1))\leq\int_{E}f\dd m,\quad\forall n\in\bN,\\
    &\Rightarrow m(E_1)=0.
  \end{align*}

\end{enumerate}
\end{ptcbp}

\begin{Th}[Convergencia Monótona de Lebesgue]\label{thm:MCTLebesguePositive}
  Sean $f,f_k\: E\to\overline{\bR}$ medibles, positivas para $k\in\bN$. Si se cumple
\begin{enumerate}
  \item $0\leq f_k(x)\leq f_{k+1}(x)$ para todo $x\in E$,
  \item $f_k\xrightarrow[]{k\to\infty}f$,
\end{enumerate}
entonces $$\int_{E}f\dd m=\lim_{k\to\infty}\int_{E}f_k\dd m.$$
\end{Th}

\begin{ptcbp}
Recuerde que
$$R(f,E)=\conj{(x,y)\: x\in E,\ 0\leq y< f(x)}\cup G(f,E).$$
Tenemos que
\begin{align*}
  \int_{E}f\dd m &=m(\conj{(x,y)\: x\in E,\ 0\leq y< f(x)})\\
  &=m\left(\bigcup_{k\in\bN}\conj{(x,y)\: x\in E,\ 0\leq y< f_k(x)}\right).
\end{align*}
Note que
\begin{gather*}
  \conj{(x,y)\: x\in E,\ 0\leq y< f_k(x)}\\
  \subseteq\conj{(x,y)\: x\in E,\ 0\leq y< f_{k+1}(x)},
\end{gather*}
entonces
\begin{align*}
\int_{E}f\dd m&=\lim_{k\to\infty}m(\conj{(x,y)\: x\in E,\ 0\leq y< f_k(x)})\\
&=\lim_{k\to\infty}\int_{E}f_k\dd m.
\end{align*}
\end{ptcbp}

\begin{Rmk}
Esta propiedad ni siquiera se puede enunciar para la integral de Riemann. Que sea posible intercambiar el límite con la integral es una de las propiedades más importantes de la integral de Lebesgue.
\end{Rmk}

Sean $(E_k)_{k\in\bN}$ tal que $E$ es la unión disjunta de estos conjuntos.\par
Entonces
\begin{gather*}
  R(f,E)=\bigcup_{k\in\bN} R(f,E_k)\\
  \Rightarrow m(R(f,E))=\sum_{k\in\bN}m(R(f,E_k))\\
  \Rightarrow \int_{E}f\dd m=\sum_{k\in\bN}\int_{E_k}f\dd m.
\end{gather*}

En pocas palabras, podemos despedazar nuestro dominio en cuantos pedazos querramos de manera que la integral sobre el dominio va a ser la integral sobre los pedazos.

\subsection{Día 26| 28-6-2018}

Recuerde que la clase pasada definimos la integral. Cuando $f\:E\to\bR$ es medible y positiva, entonces
$$R(f,E) = \conj{(x,y)\: x\in E, 0\leq y\leq f(x)},$$
es medible y $$\int_Ef\dd m=m(R(f,E)).$$
Si $\phi=\sum_{i\in\bonj{n}}a_i\ind_{A_i}$ con $A_i\cap A_j=\emptyset$, entonces
$$\int_E\phi\dd m=\sum_{i\in\bonj{n}}a_im(A_i).$$
También vimos el teorema \ref{thm:MCTLebesguePositive} de convergencia monótona, si $0\leq f_k\leq f_{k+1}$ tal que $f_k\to f$, entonces $$\int_Ef\dd m=\lim_{n\to\infty}\int_Ef_k\dd m.$$
Ahora si $\phi$ es simple como antes y $\phi\leq f$, entonces $\int_E\phi\dd m\leq \int_Ef\dd m$.

\begin{Lem}
  Sea $f\: E\to\bR$ medible y positiva. Entonces
  $$\int_Ef\dd m=\sup\conj{\int_E\phi\dd m\: \phi\text{ simple},\ \phi\leq f}.$$
\end{Lem}

\begin{ptcbp}
Dada $f\geq 0$, tome
$$E_k^j=\conj{\frac{j-1}{2^k}\leq f\leq\frac{j}{2^k}},\quad j\in\bonj{k2^k},$$
y tome $E_k^0=\conj{f\geq k}$. Luego
$$\phi_k=k\ind_{E_k^0}+\sum_{j\in\bonj{k2^k}}\frac{j-1}{2^k}\ind_{E_j^k},$$
entonces $\phi_k\leq f$, $\phi_k\leq \phi_{k+1}$ y $\phi_k\to f$. Note que estas son las hipótesis del teorema de convergencia monótona. Así tendremos
$$\int_Ef\dd m=\lim_{k\to\infty}\int_E\phi_k\dd m.$$
\end{ptcbp}

Observe que estos conjuntos $E_j^k$ son una partición del espacio. Nos preguntamos ahora, qué pasa cuando integramos sobre un conjunto de medida cero? Cuánto vale $\int_Ef\dd m$? Estamos integrando sobre un pedazo que es esencialmente vacío. Cuando no tenemos idea de como entrarle a un problema como estos, empezamos con indicadores, nos pasamos a simples y de alguna manera tomamos límites. En este caso sale de una vez con funciones simples.\par
Sea $(\varphi_k)_{k\in\bN}$ una sucesión creciente de funciones simples tal que $\varphi_k\to f$. Entonces
$$\int_E\varphi_k\dd m=\sum_{i\in\bonj{n}}a_im(A_i),$$
si $\varphi_k=\sum_{i\in\bonj{n}}a_i\ind_{A_i}$. Como $A_i\subseteq E$, entonces $m(A_i)=0$. Así $\int_E\varphi_k\dd m=0$, en otras palabras
$$\int_Ef\dd m=\lim_{k\to\infty}\int_E\varphi_k\dd m=0.$$
Esto prueba el lema a continuación.
\begin{Lem}\label{lem:conjMedCeroImpliesIntCero}
  Sea $f\:E\to\bR$ medible y positiva. Si $m(E)=0$, entonces $\int_Ef\dd m=0$.
\end{Lem}

Ahora si $f\equiv 0$ casi por doquier, definimos $Z=\conj{f>0}$. Luego $Z$ tiene medida cero y
$$\int_Ef\dd m=\int_Z f\dd m+\int_{E\setminus Z}f\dd m=0,$$
pues $f\equiv 0$ en $E\setminus Z$.\par
Cómo funcionan este tipo de resultados? Si $f,g\geq 0$ y $f\geq g$ casi por doquier, cómo probaríamos que la integral de $f$ le gana a la de $g$ cuando la desigualdad es casi por doquier?\par
Sea $Z=\conj{f<g}$ con medida cero, entonces
\begin{align*}
  \int_Ef\dd m & =\int_{E\setminus Z}f\dd m+\int_Zf\dd m\\
  &=\int_{E\setminus Z}f\dd m\geq \int_{E\setminus Z}g\dd m\\
  &=\int_{E\setminus Z}g\dd m+\int_Z g\dd m=\int_{E}g\dd m.
\end{align*}

\subsubsection*{La Desigualdad de Markov}

Sea $\alpha\geq 0$, entonces
\begin{gather*}
  \alpha\ind_{\conj{f\geq \alpha}}\leq f\ind_{\conj{f\geq \alpha}},\\
  \Rightarrow \alpha m(\conj{f\geq \alpha})\leq \int_{\conj{f\geq \alpha}}f\dd m\leq \int_{E}f\dd m.
\end{gather*}
Esta inocente desigualdad tiene grandes aplicaciones. Vea que si $\int_{E}f\dd m=0$ y $n\geq 0$, entonces
\begin{gather*}
  \frac{1}{n}m\left(\conj{f\geq \frac{1}{n}}\right)\leq \int_Ef\dd m=0,\\
  \Rightarrow m\left(\conj{f\geq \frac{1}{n}}\right)=0,\ n\in\bN.
\end{gather*}
Pero vea que $\conj{f\geq 0}=\bigcup_{n\in\bN}\conj{f\geq \frac{1}{n}}$. Es decir $m(\conj{f\geq 0})=0.$ Acabamos de probar el siguiente lema.

\begin{Lem}
  Sea $f\geq 0$ y medible. Entonces
  $$\int_Ef\dd m=0 \iff f\equiv 0\text{ casi por doquier}.$$
\end{Lem}

Finalmente vamos a probar que la integral es lineal. Porque si tenemos una integral que no es un operador lineal no nos sirve de mucho.

\begin{Lem}\label{lem:LebesgueIntLineal}
  Sean $f,g\:E\to\bR$ medibles, positivas. Si $c>0$, entonces
  $$\int_E(cf+g)\dd m=c\int_E f\dd m+\int_Eg\dd m.$$
\end{Lem}

\begin{ptcbp}
Sea $(f_k)_{k\in\bN}$ una sucesión creciente de funciones simples que convergen a $f$. Si tenemos
$f_k=\sum_{i\in\bonj{\ell}}a_i\ind_{A_i}$, con $E=\bigcup_{i\in\bonj{\ell}}A_i$ y $A_i$ disjuntos por parejas. Vea que
\begin{align*}
  cf_k &=\sum_{i\in\bonj{\ell}}ca_i\ind_{A_i}\\
  \Rightarrow\int_Ecf_k\dd m&=\sum_{i\in\bonj{\ell}}ca_im(A_i)\\%%wus going on
  &=c\sum_{i\in\bonj{\ell}}a_im(A_i)=c\int_Ef_k\dd m.
\end{align*}
Así siempre podemos sacar la constante $c$ siempre que sean simples. Como $(cf_k)_{k\in\bN}$ es una sucesión de funcioens simples y crecientes con $cf=\lim_{k\to\infty}cf_k$ tenemos

\begin{align*}
  \int_Ecf\dd m &=\lim_{k\to\infty}\int_Ecf_k\dd m\\
  &=c\lim_{k\to\infty}\int_Ef_k\dd m\\
  &=c\int_Ef\dd m.
\end{align*}

Ahora queremos probar la parte de la suma. Nuevamente tome $(g_k)_{k\in\bN}$ una sucesión creciente de funciones simples que converge a $g$. Ahora
$$g_k=\sum_{i\in\bonj{m}}b_i\ind_{B_i},$$
con $E=\bigcup_{i\in\bonj{m}}B_i$ y $B_i$ disjuntos dos a dos. Note que
$$f_k+g_k=\sum_{i\in\bonj{\ell}}a_i\ind_{A_i}+\sum_{j\in\bonj{m}}b_j\ind_{B_j}.$$
Despedazamos nuestros $A$'s en intersecciones de $B$'s. Este truco de intersecarlos todos nos debe servir para jugar con varias simples.\par
Como

\begin{align*}
  \ind_{A_i} & =\ind_{A_i\cap\bigcup_{k\in\bonj{m}}B_j}\\
  &=\ind_{\bigcup_{k\in\bonj{m}}B_j\cap A_i}\\
  &=\sum_{j\in\bonj{m}}\ind_{B_j\cap A_i}.
\end{align*}

Así tenemos que
$$f_k+g_k=\sum_{i\in\bonj{\ell}}\sum_{j\in\bonj{m}}(a_i+b_j)\ind_{A_i\cap B_j},$$
lo que nos dice que la integral de las simples es
$$\int_Ef_k+g_k\dd m= \sum_{i\in\bonj{\ell}}\sum_{j\in\bonj{m}}(a_i+b_j)m(A_i\cap B_j).$$
Lo que esperaríamos es que la suma de las $b_j$'s es que nos de la integral de $g$. Ahora tenemos
\begin{align*}
   \sum_{i\in\bonj{\ell}}\sum_{j\in\bonj{m}}b_jm(A_i\cap B_j)&=\sum_{j\in\bonj{m}}b_j\left(\sum_{i\in\bonj{\ell}}m(A_i\cap B_j)\right)\\
   &=\sum_{j\in\bonj{m}}b_jm\left(\bigcup_{i\in\bonj{\ell}}A_i\cap B_j\right).
\end{align*}
Pero en virtud de que $\bigcup_{i\in\bonj{\ell}}A_i\cap B_j=B_j$ ya que los $A_i$ son partición, tenemos el resultado.
\end{ptcbp}

\begin{Ej}
  Muestre el otro caso, los detalles son análogos.
\end{Ej}

Asuma ahora que $0\leq f\leq g$, entonces
\begin{gather*}
  \int_E\left((g-f)+f\right)\dd m=\int_Eg\dd m\\
  =\int_E(g-f)\dd m+\int_Ef\dd m=\int_Eg\dd m.
\end{gather*}
Si además tenemos $\int_Ef\dd m<\infty$, entonces
$$\int_E(g-f)\dd m=\int_Eg\dd m-\int_Ef\dd m.$$
Más aún podemos considerar $(f_k)_{k\in\bN}$ una sucesión de funciones positivas, medibles. Tome
$$\int_E\left(\sum_{k\in\bN}f_k\right)\dd m,$$
si $S_n=\sum_{k\in\bonj{n}}f_k$, entonces $(S_n)_{n\in\bN}$ es un sucesión creciente tendremos por el teorema \ref{thm:MCTLebesguePositive} de convergencia monótona tendremos:
\begin{align*}
 \int_E\left(\sum_{k\in\bN}f_k\right)\dd m&=\lim_{n\to\infty}\int_E S_n\dd m\\
 &=\lim_{n\to\infty}\sum_{k\in\bonj{n}}\int_E f_n\dd m\\
 &=\sum_{k\in\bN}\int_E f_k\dd m.
\end{align*}

La gran preegunta aquí es, podemos intercambiar el límite con la integral? Es decir, supongamos que $(f_k)_{k\in\bN}$ es una sucesión funciones solamente positiva con $f_k\to f$ casi por doquier. Entonces será cierto que se cumple
$$\lim_{k\to\infty}\int_Ef_k\dd m=\int_E f\dd m?$$
La respuesta a esta pregunta no es necesariamente cierta. Considere la sucesión
$$
f_k=\begin{cases}
      k, & \mbox{si } 0\leq x\leq \frac{1}{k}\\
      0, & \text{ cuando } \frac{1}{k}\leq x.
    \end{cases}
$$
Entonces $f_k\to 0$, pero $\int_Ef_k\dd m=1$.

\begin{Lem}[Fatou]\label{lem:FatouPositive}
  Sea $(f_k)_{k\in\bN}$ una sucesión de funciones medibles positivas. Entonces
  $$\int_E\liminf_{k\to\infty}f_k\dd m\leq \liminf_{k\to\infty}\int_Ef_k\dd m.$$
\end{Lem}
El secreto de esto es lo siguiente. Recordar la definición de $\liminf$!

\begin{ptcbp}
Note que
\begin{align*}
  \liminf_{k\to\infty}f_k &= \sup_{n\in\bN}\inf_{k\geq n}f_k\\
  &= \sup_{n\in\bN}g_n,
\end{align*}
con $g_n =\inf_{k\geq n}f_k$.
A mayor $n$ menos cosas tenemos el $\inf$ y por tanto va a ser más grande. Entonces $g_n\to\liminf_{k\to\infty}f_k$. Luego
$$\int_E\liminf_{k\to\infty}f_k\dd m=\lim_{n\to\infty}\int_E g_n\dd m,$$
pero
$$g_n=\inf_{k\geq n}f_k\leq f_\ell,\quad \ell\geq n.$$
Ahora integramos para obtener
\begin{gather*}
  \int_Eg_n\dd m\leq \int_E f_\ell\dd m,\quad \ell\geq n,\\
  \Rightarrow\int_Eg_n\dd m\leq \inf_{\ell\geq n}\int_Ef_\ell\dd m.
\end{gather*}
Cuando tiramos $n\to \infty$ entonces
\begin{gather*}
  \int_Eg_n\dd m\to\int_E\liminf_{k\to\infty}f_k\dd m\\
  \inf_{\ell\geq n}\int_Ef_\ell\dd m\to\sup_{n\in\bN}\inf_{\ell\geq n}\int_Ef_\ell\dd m\\
  =\liminf_{k\to\infty}\int_Ef_k\dd m.
\end{gather*}
Esto nos da el resultado que buscamos.
\end{ptcbp}

El siguiente resultado es de los más poderosos.
\begin{Th}[Convergencia Dominada]\label{thm:DCTLebesguePositive}
  Sea $(f_k)_{k\in\bN}$ una sucesión de funciones medibles, positivas que convergen a $f$. Si existe $g$ positiva y medible tal que:
  \begin{enumerate}
    \item $0\leq f_k\leq g$ casi por doquier,
    \item $\int_Eg\dd m<\infty$.
  \end{enumerate}
  Entonces $$\lim_{k\to\infty}\int_Ef_k\dd m=\int_Ef\dd m.$$
\end{Th}

\begin{ptcbp}
Sabemos que
\begin{align*}
  \int_Ef\dd m &=\int_E\lim_{k\to\infty}f_k\dd m\\
  &=\int_E\liminf_{k\to\infty}f_k\dd m\\
  &\leq \liminf_{k\to\infty}\int_E f_k\dd m.
\end{align*}
Donde la última desigualdad se sigue del Lema de Fatou. Vamos a volcar la desigualdad usando el hecho que $\inf(A)=-\sup(-A)$. \par
Considere $g_k=g-f_k\geq 0$. Entonces
$$\int_E\liminf_{k\to\infty}g_k\dd m=\int_E(g-f)\dd m,$$
pero tenemos
\begin{align*}
  \int_E\liminf_{k\to\infty}g_k\dd m&\leq \liminf_{k\to\infty}\int_Eg_k\dd m\\
  &=\liminf_{k\to\infty}\left(\int_Eg\dd m-\int_Ef_k\dd m\right)\\
  &=\int_Eg\dd m-\limsup_{k\to\infty}\int_Ef_k\dd m.
\end{align*}
Finalmente tenemos que
\begin{gather*}
  \int_Eg\dd m-\int_Ef\dd m\leq \int_Eg\dd m-\limsup_{k\to\infty}\int_Ef_k\dd m\\
  \Rightarrow\limsup_{k\to\infty}\int_Ef_k\dd m\leq \int_Ef\dd m.
\end{gather*}
\end{ptcbp}

\subsection{Día 27| 3-7-2018}

\subsubsection*{Integración de funciones medibles}

Ya habíamos probado propiedades de integrales de funciones positivas. Hoy vamos a generalizar con integrales de funciones medibles. Vamos a ver que las funciones Riemann integrables son continuas casi por doquier redondeando toda la materia.\par
Tomemos cualquier función $f\: E\to\overline{\bR}$, recuerde que
\begin{gather*}
  f^+=\max\conj{0,f},\\
  f^-=\max\conj{0,-f},
\end{gather*}
y $|f|=f^++f^-$. También tenemos que $f=f^+-f^-$ y así si $\int_Ef^+\dd m <\infty$ ó $\int_Ef^-\dd m <\infty$ entonces definimos
$$\int_Ef\dd m=\int_Ef^+\dd m -\int_Ef^-\dd m. $$
\begin{Def}
  Decimos que $f$ es integrable si ambas integrales son finitas. Denotamos $f\in L(E)$.
\end{Def}
Observe que esto es equivalente a que la suma de las partes de la integral sea finita, pero observe que
$$\int_E(f^++f^-)\dd m <\infty \iff \int_E|f|\dd m <\infty.$$

Observe que
\begin{align*}
  \left|\int_Ef\dd m\right| &=  \left|\int_Ef^+\dd m-\int_Ef^-\dd m\right|\\
   &\leq \int_Ef^+\dd m+\int_Ef^-\dd m\\
   =\int_E|f|\dd m.
\end{align*}
Las propiedades básicas son las siguientes.
\begin{Th}
  Sean $f,g\:E\to\overline{\bR}$ medibles tales que $\int_Ef\dd m,\int_Eg\dd m$ existen. Entonces se cumple lo siguiente:
  \begin{enumerate}
    \item Si $g\leq f$, entonces $\int_Eg\dd m\leq\int_Ef\dd m$.
    \item Si $F\subseteq E$, entonces $\int_Ff\dd m$ existe si $F\subseteq E$ es medible.
    \item Si $E=\bigcup_{i\in\bN}E_i$ y $E_i\cap E_j=\emptyset$ para $i,j\in\bN$, entonces
    $$\int_Ef\dd m=\sum_{i\in\bN}\int_{E_i}f\dd m.$$
    \item Si $E$ es un conjunto de medida cero, entonces $\int_Ef\dd m =0$.
    \item Si $f=g$ casi por doquier, entonces$\int_Ef\dd m =\int_Eg\dd m$.
  \end{enumerate}
\end{Th}

\begin{ptcbp}
Veamos el primer apartado, suponga $g\leq f$. Tenemos lo siguiente
\begin{gather*}
   f^+=\max\conj{0,f}\geq \max\conj{0,g}=g^+,\\
  f^-=\max\conj{0,-f}\leq\max\conj{0,-g}=g^-,
\end{gather*}
entonces tendremos
\begin{gather*}
  \int_Ef^+\dd m\geq\int_Eg^+\dd m,\\
   \int_Ef^-\dd m\leq\int_Eg^-\dd m.
\end{gather*}
Sumando tenemos el resultado. \par
Para el segundo apartado, recuerde que la integral existe cuando alguna de las partes existe. Si existe en la parte grande (o sea, nos da un número) entonces en la parte pequeña también.\par
Si $\int_Ef^+\dd m<\infty$, entonces $\int_Ff^+\dd m<\infty$. Es análogo para $f^-$.\par
En el tercer apartado, sabemos que valen las fórmulas
\begin{gather*}
\int_Ef^+\dd m=\sum_{i\in\bN}\int_{E_i}f^+\dd m\\
\int_Ef^-\dd m=\sum_{i\in\bN}\int_{E_i}f^-\dd m
\end{gather*}
Si ambas series son números estamos listos. Con solo que uno sea un número estámos porque nada más da infinito. Como al menos una de las sumas es finita se tiene el resultado.\par
Para el cuarto apartado, separamos nuevamente en partes positivas y negativas. El resultado se vale para esas funciones por lo que la integral entera nos va a dar $0-0=0$. El quinto apartado es similar usando el primer y cuarto apartado, despedazamos el conjunto en donde son iguales y en donde no lo son. En donde no ocurre es un conjunto de medida cero.
\end{ptcbp}

\begin{Th}
  Sean $f,g\:E\to\overline{\bR}$ medibles tales que $\int_Ef\dd m,\int_Eg\dd m$ existen. Entonces se cumple lo siguiente:
  \begin{enumerate}
    \item $\int_Ecf\dd m=c\int_E f \dd m$.
    \item Si $f,g\in L(E)$ entonces
    $$\int_E(f+g)\dd m=\int_Ef\dd m+\int_Eg\dd m$$
  \end{enumerate}
\end{Th}

\begin{ptcbp}
Como no importa el caso cuando $c=0$, vemos que pasa en los demás casos. El caso fácil es $c>0$, así suponemos $c<0$. Tenemos que
\begin{align*}
  (cf)^+ &=\max\conj{0,cf}\\
  &=\max\conj{0,c(-f)}\\
  =-cf^-.
\end{align*}
Con un razonamiento análogo $(cf)^-=cf^+$. Luego, sumando se obtiene
\begin{align*}
  \int_Ecf\dd m &=\int_E(-c)f^-\dd m-\int_E(-c)f^+\dd m\\
  &=-c\int_Ef^-\dd m+c\int_Ef^+\dd m\\
  &=c\int_Ef\dd m.
\end{align*}
Ahora por definición tendríamos que
$$\int_E(f+g)\dd m=\int_E(f+g)^+\dd m-\int_E(f+g)^-\dd m.$$
El problema aquí es, cuál es la relación entre $(f+g)^+$ y $f^+, g^+$? No podemos encontrar una relación, hay demasiadas posibilidades. Queremos despedazar conjuntos de manera que podamos controlar el signo de la función.\par
Defina
\begin{gather*}
  E_1=\conj{f\geq 0,g\geq 0},\quad E_2=\conj{f\leq 0,g\leq 0},\\
  E_3=\conj{f\geq 0,g<0, f+g\geq 0}, \\
  E_4=\conj{f\geq 0,g<0, f+g< 0},\\
  E_5=\conj{f< 0,g\geq 0, f+g\geq 0},\\
  E_6=\conj{f< 0,g\geq 0, f+g< 0}.
\end{gather*}
Note que en $E_3$ tenemos que
\begin{align*}
  \int_{E_3}f\dd m &=\int_{E_3}(f+g)+(-g)\dd m\\
   &=\int_{E_3}(f+g)\dd m+ \int_{E_3}(-g)\dd m\\
   &=\int_{E_3}(f+g)\dd m- \int_{E_3}g\dd m.
\end{align*}
En la última expresión pudimos sacar el $(-1)$ por el primer apartado y entonces tenemos que ya todas las integrales existen. Pasamos a sumar $\int_{E_3}g\dd m$ y obtener el resultado sobre $E_3$. La prueba de $E_4$ es análoga, nada más cambiamos un par de detalles, $f$ por $-g$.
\begin{align*}
  \int_{E_4}(-g)\dd m &=\int_{E_4}-(f+g)+f\dd m\\
   &=\int_{E_4}-(f+g)\dd m+ \int_{E_4}f\dd m\\
   &=-\int_{E_4}(f+g)\dd m- \int_{E_4}f\dd m.
\end{align*}
Nuevamente pasando un par de términos de cada lado tenemos el resultado.
\end{ptcbp}

\begin{Ej}
  Resuelva los demás casos, no aplique un argumento de simetría. Piénselo de manera intuitiva cambiando a ver quienes son positivos y negativos.
\end{Ej}

Observe que el conjunto $L(E)$ es un espacio vectorial normado. Definimos un norma sobre $L(E)$ como $$\nm{f}_L=\int_E|f|\dd m.$$
Esta es la compleción de un espacio de funciones continuas. Es importante ver que quienes son las compleciones de ciertos espacios porque sino estamos trabajando con sólo cosas abstractas.\par
Ahora, veamos un contraejemplo de cuando la suma de las integrales no funciona.
\begin{Ex}
  Considere $f=\ind_{\lbonj{n,\infty}}$, $g=-\ind_{\lbonj{n+1,\infty}}$. Entonces $f+g=\ind_{\lbonj{n,n+1}}$ y las integrales $\int_\bR f\dd m$, $\int_\bR g\dd m$ no están definidas. Sin embargo $$\int_\bR (f+g)\dd m=1.$$
\end{Ex}
Ahora, si $\phi\in L(E)$ y $f\:E\to\overline{\bR}$ es medible tal que $\phi\leq f$ y $\int_Ef\dd m$ existe, cuánto vale $\int_E(f-\phi)\dd m$?\par
Si asumimos $f\not\in L(E)$ vea una cosa, si $f$ no es integrable una de las partes es infinita. Primero, la parte negativa de $f$ no puede ser infinita. Note que $f^-\leq\phi^-$. Entonces
$$0\leq \int_Ef^-\dd m\leq \int_E\phi^-\dd m<\infty.$$
Luego $\int_Ef^+\dd m=\infty$, y así
$$\int_Ef^+\dd m-\int_E\phi\dd m=\infty.$$
Esto porque a infinito le estamos restando un número. Ahora, $f-\phi$ puede ser integrable? Si $f-\phi$ fuese integrable y le sumamos $\phi$ quedaría $f$ que no es integrable. Pero la suma de integrabes lo es. Por lo que $f-\phi$ no puede ser integrable. \par
Como $f-\phi$ es positiva, entonces $\int_E(f-\phi)\dd m=\int_E(f-\phi)^+\dd m=\infty$. De aquí tenemos
$$\int_E(f-\phi)\dd m=\int_Ef\dd m-\int_E\phi\dd m.$$

\subsubsection*{Teoremas de convergencia}

\begin{Th}[Convergencia Monótona]\label{thm:MCTLebesgue}
  Sea $(f_k)_{k\in\bN}$ una sucesión de funciones medibles tales que $f_k\to f$ en $E$. Si existe $\phi\in L(E)$ tal que $\phi\leq f_k\leq f_{k+1}$. Entonces se cumple
  $$\lim_{k\to\infty}\int_Ef_k\dd m=\int_E f\dd m.$$
\end{Th}

El teorema de convergencia monótona \ref{thm:MCTLebesguePositive} es el mismo sólo que en vez con funciones positivas y $\phi\equiv 0$.

\begin{Lem}[Fatou]\label{lem:Fatou}
  Sea $(f_k)_{k\in\bN}$ una sucesión de funciones medibles tal que $\phi\leq f_k$ con $\phi\in L(E)$. Entonces se cumple que
  $$\int_E\liminf_{k\to\infty}f_k\dd m\leq\liminf_{k\to\infty}\int_Ef\dd m.$$
\end{Lem}

Finalmente si $|f_k|\leq \phi$ con $\phi\in L(E)$, entonces
\begin{gather*}
  0\leq f_k+\phi\leq 2\phi,\\
  0\leq \phi-f_k\leq 2\phi.
\end{gather*}
Por Fatou tenemos que
\begin{gather*}
  \Rightarrow \int_E\liminf_{k\to\infty}f_k\dd m\leq\liminf_{k\to\infty}\int_E f_k\dd m,\\
  \Rightarrow \int_E\limsup_{k\to\infty}f_k\dd m\geq\limsup_{k\to\infty}\int_E f_k\dd m,
\end{gather*}
respectivamente de ambas desigualdades.

\begin{Th}[Convergencia Dominada]\label{thm:DCTLebesgue}
  Sea $(f_k)_{k\in\bN}$ una sucesión de funciones medibles tal que $f_k\to f$ casi por doquier y $|f_k|\leq \phi$ con $\phi\in L(E)$. Entonces se cumple
  $$\lim_{k\to\infty}\int_E f_k\dd m=\int_E f\dd  m.$$
\end{Th}

\subsubsection*{Riemann y Lebesgue}

\begin{Th}
  Sea $f\:\bonj{a,b}\to\bR$ acotada. Entonces $\cR(\bonj{a,b})\subseteq L(\bonj{a,b})$, además
  $$\int_{\bonj{a,b}}f\dd m=\int_{a}^{b}f\dd x.$$
\end{Th}

\begin{ptcbp}
Sea $(\Gamma_k)_{k\in\bN}$ una colección de particiones de $\bonj{a,b}$ tales que $\Gamma_k\subseteq\Gamma_{k+1}$. Si
$$\Gamma_k=\conj{0=x_k(0)<x_k(1)<\cdots<x_k(m_k)=b},$$
entonces buscamos aplicar el criterio de Darboux. Veamos las sumas superiores y las inferiores.
\begin{gather*}
  L_k=\left(\sum_{i\in\bonj{m_k-1}}(\inf_{\bonj{x_k(i-1),x_k(i)}}f)\ind_{\lbonj{x_k(i-1),x_k(i)}}\right)\\
  \quad+(\inf_{\bonj{x_k(i-1),x_k(i)}}f)\ind_{\bonj{x_k(m_{k-1}),x_k(m_k)}},\\
  U_k=\left(\sum_{i\in\bonj{m_k-1}}(\sup_{\bonj{x_k(i-1),x_k(i)}}f)\ind_{\lbonj{x_k(i-1),x_k(i)}}\right)\\
  \quad+(\sup_{\bonj{x_k(i-1),x_k(i)}}f)\ind_{\bonj{x_k(m_{k-1}),x_k(m_k)}}.
\end{gather*}
Entonces $L_k\leq L_{k+1}\leq f\leq U_{k+1}\leq U_k$ para todo $k\in \bN$. Además $|L_k|,|U_k|\leq M$ con $M=\sup_{\bonj{a,b}}f$.\par
Sean $U=\lim_{k\to\infty}U_k$, $L=\lim_{k\to\infty}L_k$, así $L\leq f\leq U$. Ahora como
\begin{gather*}
  \lim_{k\to\infty}\int_{\bonj{a,b}} L_k\dd m=\lim_{k\to\infty}\int_{\bonj{a,b}} U_k\dd m\\
  \Rightarrow \int_{\bonj{a,b}}U\dd m=\int_{\bonj{a,b}}L\dd m=\int_{a}^{b}f\dd x.
\end{gather*}
Esto es porque al integral las $U_k, L_k$ nos quedan las sumas superiores e inferiores de Darboux. como $U\geq L$ y $\int_{\bonj{a,b}}(U-L)\dd m=0$ entonces $U=L$ casi por doqueir. Esto significa que $U=L=f$ casi por doquier.
\end{ptcbp}

\begin{Th}
  Sea $f\: \bonj{a,b}\to\bR$ acotada. Entonces $f\in\cR(\bonj{a,b})$ si y sólo si $f$ es continua casi por doquier.
\end{Th}

\begin{ptcbp}
Bajo la notación de la prueba anterior, suponga que $f\in\cR(\bonj{a,b})$. Definamos
\begin{align*}
  Z &=\conj{L=U=f}^c\cup\bigcup_{k\in\bN}\Gamma_k,\\
  &=\conj{L\neq U}\cup\conj{L\neq f}\cup\conj{f\neq U}\cup\bigcup_{k\in\bN}\Gamma_k.
\end{align*}
Vea que $Z$ es el conjunto donde $f$ debería de ser discontinua. Sea $x\not\in Z$ entonces si fuese que $f$ no es continua en $x$ existiría $\varepsilon>0$ tal que para todo $\delta>0$ existe un $x_\delta$ que satisface
$$|x-x_\delta|<\varepsilon,\ \text{ y }\ |f(x)-f(x_\delta)|\geq \varepsilon.$$
Dado $k$, existen $x_k(i-1),x_k(i)$ tales que $x\in\obonj{x_k(i-1),x_k(i)}$. Entonces
\begin{gather*}
  L_k(x)=\inf_{\bonj{x_k(i-1),x_k(i)}}f,\\
  U_k(x)=\sup_{\bonj{x_k(i-1),x_k(i)}}f.
\end{gather*}

Como $x\in\obonj{x_k(i-1),x_k(i)}$ existe $x_\delta$ tal que $x_\delta\in\obonj{x_k(i-1),x_k(i)}$ y
\begin{gather*}
  \varepsilon\leq |f(x_\delta)-f(x)|\leq U_k(x)-L_k(x),\\
  \Rightarrow \varepsilon\leq U_k(x)-L_k(x).
\end{gather*}
Esto dice que $f$ no es integrable fuera de $Z$.\par
Veamos la otra dirección, supongamos que
$$Z=\conj{x\: f\ \text{ no es continua en } x}\cup\conj{a,b}.$$
Si $x\not\in Z$ dado $\varepsilon>0$ existe $\delta>0$ tal que
$$|x-y|<\delta\Rightarrow|f(x)-f(y)|<\varepsilon.$$
Note que $y\in\obonj{x-\delta,x+\delta}$, entonces $f(y)\geq f(x)-\varepsilon$ y $f(y)\leq f(x)+\varepsilon$.\par
INSERTAR fig27.1\par
Sea $k_0$ tal que $|\Gamma_k|<\frac{\delta}{2}$ si $k\geq k_0$. Si tenemos $x\in\bonj{x_k(i-1),x_k(i)}$, entonces
\begin{gather*}
  \sup_{\bonj{x_k(i-1),x_k(i)}}f=U_k(x)\leq f(x)+\varepsilon,\\
  \inf_{\bonj{x_k(i-1),x_k(i)}}f=L_k(x)\geq f(x)-\varepsilon.
\end{gather*}
Luego tenemos que $\forall k\geq k_0$ tenemos
$$\varepsilon\geq U_k-f,f-L_k\geq 0,$$
luego $\lim_{k\to\infty}U_k=\lim_{k\to\infty}L_k=f$. Aplicamos el teorema de convergencia dominada \ref{thm:DCTLebesgue} para obtener
$$\lim_{k\to\infty}\int_{\bonj{a,b}}U_k\dd m=\lim_{k\to\infty}\int_{\bonj{a,b}}L_k\dd m=\int_{\bonj{a,b}}f\dd m.$$
\end{ptcbp}

\subsection{Día 28| 5-7-18}
\subsubsection*{Novena Sesión de Ejercicios}

\begin{Ej}[4.5  Wheeden \& Zygmund]
Muestre con un ejemplo que la composición de funciones medibles y finitas no necesariamente es medible.
\end{Ej}
A manera de sugerencia tome una ``inversa'' de la función de Cantor y una indicadora de un conjunto de medida cero adecuado.
\begin{ptcbp}
Queremos un conjunto $\conj{gf=1}$ no medible. Este conjunto es
$$(f^{-1}g^{-1})\bonj{\conj{1}}.$$
Definimos $g=\ind_A$ de manera que el conjunto anterior sea $f^{-1}(A)$. Si seguimos la sugerencia y decimos que $F$ es la inversa, entonces tendríamos que este conjunto sería $F(A)$.\par
Si $F$ es la función de Cantor entonces vamos a construir
$$f^{-1}(x)=\inf\conj{y\in\bonj{0,1}\: F(y)=x}.$$
Esta función está bien definida sobre el conjunto de Cantor salvo los extremos derechos. Si $C$ es el conjunto de Cantor y $B$ el de los extremos derechos, entonces $m(F(C\setminus B))>0$. Luego existe $D\in F(C\setminus B)$ un conjunto no medible. Entonces tome $\tilde{D}=F^{-1}(D)$, note que $\tilde{D}\subseteq C\setminus B \subseteq C$. Para nuestros efectos, tomemos $g=\ind_{\tilde{D}}$.
\end{ptcbp}
\textcolor{red}{checkup}

\begin{Ej}[4.15  Wheeden \& Zygmund]
  Sea $(f_k)_{k\in\bN}$ una sucesión de funciones medibles con $f_k\: E\to\overline{\bR}$ y $E$ es de medida finita. Si $|f_k(x)|\leq M(x)<\infty$ para $x\in E,\ k\in\bN$, entonces dado $\varepsilon>0$ existe $F\subseteq E$ cerrado y $M\in\bR$ tal que $m(E\setminus F)<\varepsilon$ y $|f_k(x)|\leq M$ para $x\in F$ para todo $k\in\bN$.
\end{Ej}

\begin{ptcbp}
Considere los conjuntos
$$E_m\:= \conj{x\in E\:\ \forall k\in\bN\: |f_k(x)|<m}=\bigcap_{k\in\bN}\conj{f_k<m},$$
vea que $E_m\subseteq E_{m+1}$ y $\bigcup_{m\in\bN}E_m=E$. También para $x\in E$, $M(x)\leq m\Rightarrow x\in E_m$. Entonces existe $m_0$ tal que $m(E\setminus E_{m_0})<\frac{\varepsilon}{2}$. Sea $F\subseteq E_{m_0}$ cerrado tal que $m(E_{m_0}\setminus F)<\frac{\varepsilon}{2}$. Entonces como
\begin{align*}
  E\setminus F & E\setminus E_{m_0}\cup E_{m_0}\setminus F\\
  \Rightarrow m(E\setminus F)&= \text{medidas de arriba...}<\varepsilon.
\end{align*}
Y tenemos $M=m_0$ la cota uniforme de $F$.
\end{ptcbp}

\subsection{Día 29| 27-7-18}
\subsubsection*{Décima Sesión de Ejercicios}

\begin{Ej}[4.19  Wheeden \& Zygmund]
Sea $I=\bonj{0,1}$ y $f\: I^2\to \bR$ continua en cada variable por separado. Muestre que $f$ es medible como función de $(x,y)$. Si en cambio asumimos que $f$ es continua en $x$ para todo $y$ fijo pero arbitrario, ¿todavía se cumple la conclusión?
\end{Ej}

\begin{ptcbp}
Considere una partición uniforme de $I$ en $n$ subintervalos y definamos
$$f_n(x,y)=f\left(\frac{k}{n},y\right),\quad x\in\lbonj{\frac{k}{n},\frac{k+1}{n}},\ k\in\bonj{n-1}^*.$$
Como por hipótesis $f$ es continua en $\frac{k}{n}$ para $y$ fijo pero arbitrario, por definción se sigue que dado $\varepsilon>0$, existe $\delta=\delta(y)$ tal que
$$\left|x-\frac{k}{n}\right|<\delta\Rightarrow\left|f(x,y)-f\left(\frac{k}{n},y\right)\right|<\varepsilon.$$
Ahora para cada $n>\frac{1}{\delta}$, cuando $x\in\lbonj{\frac{k}{n},\frac{k+1}{n}}$ tenemos que
$$\left|x-\frac{k}{n}\right|\leq\frac{1}{n}<\delta\Rightarrow\left|f(x,y)-f\left(\frac{k}{n},y\right)\right|<\varepsilon.$$
Lo que implica que $f_n\to f$ puntualmente. Ahora queremos ver que $f_n$ es medible pues esto nos mostrará que $f$ es medible. Primero vamos a ver que
$$\conj{f_n>a}=\bigcup_{k\in\bonj{n-1}^*}\left(\lbonj{\frac{k}{n},\frac{k+1}{n}}\times A\right).$$
Donde $A\:=\conj{y\in I\: f(\frac{k}{n},y)>a}$, este conjunto es medible en virtud de que $f$ es continua en $y$. Si tenemos que $(u,v)\in\conj{f_n>a}$ entonces $f_n(u,v)=f\left(\frac{k}{n},v\right)>a$ con $p\in\lbonj{\frac{k}{n},\frac{k+1}{n}}$ para algún $k$. Inmediatamente esto nos dice que $v\in A$ por lo que $(u,v)$ está en el producto cartesiano y por tanto en la unión.\par
Por otro lado si $(u,v)\in\bigcup_{k\in\bonj{n-1}^*}\left(\lbonj{\frac{k}{n},\frac{k+1}{n}}\times A\right)$ entonces existe algún $k_0$ tal que $u\in\lbonj{\frac{k}{n},\frac{k+1}{n}}$ y $f(\frac{k_0}{n},v)=f_n(u,v)>a$. De esta manera $(u,v)\in\conj{f_n>a}$.\par
De esta manera terminos de ver la igualdad de los conjuntos. Ahora, queremos ver que los conjuntos de la unión son medibles. Pero en efecto, como $f$ es continua en cada variable por separado, es medible y por tanto los conjuntos $A$ son medibles para cada $k$. Luego $\conj{f_n>a}$ es una unión de conjuntos medibles que por tanto es medible. Se sigue que $f_n$ es medible y por tanto $f$ es medible.
\textcolor{red}{finish segundo apto.}
\end{ptcbp}

\begin{Ej}[5.2  Wheeden \& Zygmund]
Muestre que el teorema de convergencia monótona es falso cuando omitimos la hipótesis $\phi\in L(E)$. Muestre que el teorema de convergencia uniforme es falso cuando omitimos la hipótesis $m(E)<\infty$.
\end{Ej}


\begin{Ej}[5.3  Wheeden \& Zygmund]
Sea $(f_k)_{k\in\bN}$ una sucesión de funciones positivas y medibles sobre $E$. Si $f_k\to f$ y $f_k\leq f$ casi por doquier, entonces muestre que $\int_Ef_k\dd m\to\int_Ef\dd m$.
\end{Ej}

\begin{Ej}[5.4  Wheeden \& Zygmund]
Sea $f\in L(\obonj{0,1})$, muestre que $x^kf(x)\in L(\obonj{0,1})$ para todo $k\in\bN$. Además $\int_{0}^{1}x^kf(x)\dd x\to 0$.
\end{Ej}

\begin{Ej}[5.5  Wheeden \& Zygmund]
Con el teorema de Egorov, pruebe el teorema de convergencia acotada.
\end{Ej}

\begin{Th}[Convergencia Acotada]\label{thm:BoundedCvg}
    Sea $(f_k)_{k\in\bN}$ una sucesión de funciones medibles con $f_k\to f$ casi por doquier en $E$. Si $E$ tiene medida finita y existe $M\in\bR$ tal que $|f_k|\leq M$ casi por doquier, entonces $\int_Ef_k\dd m\to \int_Ef\dd m$.
\end{Th}
%%%FERNANDO
\begin{ptcbp}
Buscamos aplicar el teorema de convergencia uniforme. Tenemos que $|f_k|\leq M$ lo que implica que $\int_Ef_k\dd m$ es finita. Luego $f_k\to f\Rightarrow |f|\leq M$.\par
Por el teorema \ref{thm:Egorov} de Egorov, existe $F\subseteq E$ con $m(E\setminus F)<\varepsilon$ donde $f_k\to f$ uniformemente en $F$. Entonces $f_k\in L(F)$ y así $\int_Ff_k\dd m\to\int_Ff\dd m$. Para el resto del conjunto tenemos que
\begin{align*}
  \left|\int_{E\setminus F}f-f_k\dd m\right| &\leq \int_{E\setminus F}|f-f_k|\dd m\\
  &\leq \int_{E\setminus F}|f|+|f_k|\dd m\\
  &\leq 2M\varepsilon
\end{align*}
\end{ptcbp}

\begin{Ej}[5.6  Wheeden \& Zygmund]\label{ej:derivarSignoIntegral}
Considere $f\:I^2\to\bR$ una función que cumple:
  %\begin{multicols}{3}
\begin{itemize}
  \item Para todo $x$, $f$ es integrable respecto a $y$.
  \item $\pdv{f}{x}\ \hspace{-1.5mm}(x,y)$ es acotada como función de dos variables.
\end{itemize}
 %\end{multicols}
 Muestre que $\pdv{f}{x}(x,y)$ es medible como función de $y$ para $x$ fijo pero arbitrario y que
 $$\dv{x}\int_{0}^{1}f(x,y)\dd y=\int_{0}^{1}\pdv{x}f(x,y)\dd y.$$
\end{Ej}
\iffalse
%%%FERNANDO
\begin{ptcbp}
Por definición
$$\pdv{f}{x}(x,y)=\lim_{h\to 0}\frac{f(x+h,y)-f(x,y)}{h}.$$
Considere ahora la sucesión $\left(\frac{1}{n}\right)_{n\in\bN}$. De esta manera con
\end{ptcbp}
\fi

\begin{Ej}[5.9  Wheeden \& Zygmund]
Sea $p>0$ y $\lim_{k\to\infty}\int_E|f-f_k|^p\dd m=0$. Muestre que $f_k\xrightarrow[]{m}f$.
\end{Ej}

%%%FRANCISCO
\begin{ptcbp}
Sea $\varepsilon>0$ y considere una partición de $E$ a definir
$$\conj{|f-f_k|>\varepsilon}\cup\conj{|f-f_k|\leq\varepsilon}=E.$$
Entonces
\begin{align*}
  \int_E|f-f_k|^p\dd m &\geq \varepsilon^pm(\conj{|f-f_k|>\varepsilon})+Am(\conj{|f-f_k|\leq\varepsilon})\\
  &\geq \varepsilon^pm(\conj{|f-f_k|>\varepsilon})
\end{align*}
\end{ptcbp}
\end{multicols}
\end{document} 