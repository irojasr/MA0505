\documentclass[utf8]{beamer}

\mode<presentation>
{
  \usetheme{Warsaw}
  \setbeamercovered{transparent}
}


\usepackage{amsfonts,mathtools,amssymb}
\usepackage[spanish]{babel}
\usepackage{times}
\usepackage[T1]{fontenc}
\usepackage[shortlabels]{enumitem}
\usepackage{tikz}
\usepackage{physics}

\title[MA0505]{MA0505 - An\'alisis I}
\subtitle{Lecci\'on XXII: Riemann y Lebesgue IV}

\author{Pedro M\'endez\inst{1}}

\institute[Universidad de Costa Rica] % (optional, but mostly needed)
{
  \inst{1}%
  Departmento de Matem\'atica Pura y Ciencias Actuariales\\
  Universidad de Costa Rica
  }

\date[I-2021] {Semestre I, 2021}

%%%%%%%%% === Theorems and suchlike === %%%%%%%%%%%%%%

\theoremstyle{plain}
\newtheorem{Th}{Teorema}               %%% Theorem 1.1.1
\newtheorem{Tmon}{Teoremón}
\newtheorem{Prop}{Proposición}         %%% Proposition 1.1.2
\newtheorem{Lem}{Lema}                 %%% Lemma 3
\newtheorem{Cor}{Corolario}            %%% Corollary 4

\theoremstyle{definition}
\newtheorem{Def}{Definición}           %%% Definition 5
\newtheorem{Ex}{Ejemplo}               %%% Example 6
\newtheorem{Ej}{Ejercicio}             %%% Ejercicio 7
\newtheorem{Hec}[Th]{Hecho}            %%% Hecho 1.1.8

\theoremstyle{remark}
\newtheorem{Rmk}[Th]{Observación}      %%%Remark 1.1.9
\newtheorem*{nonum-Rmk}{Observación}         %%% No number Fact
\newtheorem*{Notn}{Notaci\'on}        %% Notaciones
\newtheorem*{Warn}{Advertencia}       %% Advertencias

\numberwithin{equation}{section}

%% Accomodations 

\makeatletter
\def\moverlay{\mathpalette\mov@rlay}
\def\mov@rlay#1#2{\leavevmode\vtop{%
   \baselineskip\z@skip \lineskiplimit-\maxdimen
   \ialign{\hfil$\m@th#1##$\hfil\cr#2\crcr}}}
\newcommand{\charfusion}[3][\mathord]{
    #1{\ifx#1\mathop\vphantom{#2}\fi
        \mathpalette\mov@rlay{#2\cr#3}
      }
    \ifx#1\mathop\expandafter\displaylimits\fi}
\makeatother

% Greek letters:

\newcommand{\al}{\alpha}                %% short for  \alpha
\newcommand{\bt}{\beta}                 %% short for  \beta
\newcommand{\Dl}{\Delta}                %% short for  \Delta
\newcommand{\dl}{\delta}                %% short for  \delta
\newcommand{\eps}{\varepsilon}          %% short for  \varepsilon
\newcommand{\Ga}{\Gamma}                %% short for  \Gamma
\newcommand{\ga}{\gamma}                %% short for  \gamma
\newcommand{\La}{\Lambda}               %% short for  \Lambda
\newcommand{\la}{\lambda}               %% short for  \lambda
\newcommand{\kp}{\kappa}                %% short for  \kappa
\newcommand{\Om}{\Omega}                %% short for  \Omega
\newcommand{\om}{\omega}                %% short for  \omega
\newcommand{\Sg}{\Sigma}                %% short for  \Sigma
\newcommand{\sg}{\sigma}                %% short for  \sigma
\newcommand{\te}{\theta}                %% short for  \theta
\newcommand{\vf}{\varphi}               %% short for  \varphi
\newcommand{\ze}{\zeta}                 %% short for  \zeta

%Boldface letters

\newcommand{\bC}{\mathbb{C}}    %%% números complejos
\newcommand{\bN}{\mathbb{N}}    %%% números naturales
\newcommand{\bP}{\mathbb{P}}        %% números enteros positivos
\newcommand{\bQ}{\mathbb{Q}}    %%% números racionales
\newcommand{\bR}{\mathbb{R}}    %%% números reales
\newcommand{\bS}{\mathbb{S}}    %%% esfera
\newcommand{\bZ}{\mathbb{Z}}    %%% números enteros

%Script letters:

\newcommand{\cA}{\mathcal{A}}           %% formas diferenciales
\newcommand{\cB}{\mathcal{B}}           %% una base vectorial
\newcommand{\cC}{\mathcal{C}}           %% otra base vectorial
\newcommand{\cF}{\mathcal{F}}           %% espacio de Fock
\newcommand{\cL}{\mathcal{L}}           %% operadores lineales
\newcommand{\cM}{\mathcal{M}}           %% multiplicadores
\newcommand{\cN}{\mathcal{N}}           %% funciones nulas
\newcommand{\cP}{\mathcal{P}}           %% una particion
\newcommand{\cR}{\mathcal{R}}           %% funciones representativas
\newcommand{\cS}{\mathcal{S}}           %% funciones de Schwartz


%Brackets

\newcommand{\bonj}[1]{\left\lbrack#1\right\rbrack}
\newcommand{\obonj}[1]{\left\rbrack#1\right\lbrack}
\newcommand{\rbonj}[1]{\left\rbrack#1\right\rbrack}
\newcommand{\lbonj}[1]{\left\lbrack#1\right\lbrack}
\newcommand{\snm}[1]{\|#1\|}           %small norma
\newcommand{\nm}[1]{\left\|#1\right\|} %norma pegadita
\newcommand{\pnm}[1]{\biggl|\biggl|#1\biggr|\biggr|}
\newcommand{\set}[1]{\{\,#1\,\}}    %% set notation
\newcommand{\floor}[1]{\lfloor#1\rfloor} %% mayor entero <= x
\newcommand{\Set}[1]{\biggl\{\,#1\,\biggr\}} %% set notation (large)
\newcommand\quot[2]{
        \mathchoice
            {% \displaystyle
                \text{\raise1ex\hbox{$#1$}\Big/\lower1ex\hbox{$#2$}}%
            }
            {% \textstyle
                {^{ #1}/_{ #2}}
            }
            {% \scriptstyle
                {^{ #1}/_{ #2}}
            }
            {% \scriptscriptstyle
                {^{ #1}/_{ #2}}
            }
    }
\newcommand*\squot[2]{{^{ #1}/_{ #2}}}%%%small quotient

%Symbols 

\newcommand{\x}{\times}
\renewcommand{\geq}{\geqslant}          %% mayor o igual (variante)
\newcommand{\hookto}{\hookrightarrow}     %% inclusion arrow
\newcommand{\isom}{\simeq}              %% isomorfismo
\renewcommand{\l}{\ell}                   %% ele cursiva
\renewcommand{\leq}{\leqslant}          %% menor o igual (variante)
\newcommand{\less}{\setminus}           %% set difference
\newcommand{\To}{\Rightarrow}
\newcommand{\ov}{\overline}
\newcommand{\un}{\underline}
\newcommand{\del}{\partial}
\newcommand{\ind}{\mathbf{1}}       %%%indicator function

%%% Small fractions in displays:

\newcommand{\half}{{\mathchoice{\nhalf}{\thalf}{\shalf}{\shalf}}} %%display text script script^2
\newcommand{\happi}{{\tfrac{\pi}{2}}} %% small fraction  \pi/2
\newcommand{\quarter}{\tfrac{1}{4}} %% small fraction  1/4
\newcommand{\nhalf}{\frac{1}{2}}
\newcommand{\shalf}{{\scriptstyle\frac{1}{2}}} %% tiny fraction 1/2
\newcommand{\thalf}{{\tfrac{1}{2}}} %% small fraction  1/2
\newcommand{\third}{\tfrac{1}{3}}   %% small fraction  1/3 %Hay que renew porque mathabx toma second y third como x'' y x''' por ejemplo

\newcommand{\ihalf}{{\tfrac{i}{2}}} %% small fraction  i/2

\newcommand{\suci}{_{i=1}^\infty} %% diminutivo
\newcommand{\sucj}{_{j=1}^\infty} %% diminutivo
\newcommand{\suck}{_{k=1}^\infty} %% diminutivo
\newcommand{\sucl}{_{\l=1}^\infty} %% diminutivo
\newcommand{\sucm}{_{m=1}^\infty} %% diminutivo
\newcommand{\sucn}{_{n=1}^\infty} %% diminutivo

\newcommand*{\Cdot}{{\raisebox{-0.25ex}{\scalebox{1.5}{$\cdot$}}}}      %% cdot más grande
\renewcommand{\.}{\Cdot}                %% producto escalar
\newcommand{\cupdot}{\charfusion[\mathbin]{\cup}{\.}}
\newcommand{\bigcupdot}{\charfusion[\mathbin]{\bigcup}{\.}}
\DeclareMathOperator{\Var}{Var}     %%%variance

\begin{document}

\begin{frame}
  \titlepage
\end{frame}

\begin{frame}{Agenda}
  \tableofcontents
  % You might wish to add the option [pausesections]
\end{frame}


% Structuring a talk is a difficult task and the following structure
% may not be suitable. Here are some rules that apply for this
% solution: 

% - Exactly two or three sections (other than the summary).
% - At *most* three subsections per section.
% - Talk about 30s to 2min per frame. So there should be between about
%   15 and 30 frames, all told.

% - A conference audience is likely to know very little of what you
%   are going to talk about. So *simplify*!
% - In a 20min talk, getting the main ideas across is hard
%   enough. Leave out details, even if it means being less precise than
%   you think necessary.
% - If you omit details that are vital to the proof/implementation,
%   just say so once. Everybody will be happy with that.

\section{La Relación entre las Integrales}

\subsection{Recordatorio de Particiones}
\begin{frame}{Particiones}
Sea $f:[a,b]\to\bR$ acotada y 
$$\Ga_k=\set{a=x_1^k<x_2^k<\dots<x_{m_k}^k=b}.$$
Consideremos 
\begin{enumerate}[a)]
  \item Si $m_i=\inf_{\bonj{x_{i-1},x_i}}f(x)$, entonces 
  $$\l_k(x)=\sum_{i=1}^{n-1}m_i\ind_{\lbonj{x_{i-1},x_i}}+m_n\ind_{\bonj{x_{n-1},x_n}}.$$
  \item Si $M_i=\sup_{\bonj{x_{i-1},x_i}}f(x)$, entonces
  $$u_k(x)=\sum_{i=1}^{n-1}M_i\ind_{\lbonj{x_{i-1},x_i}}+M_n\ind_{\bonj{x_{n-1},x_n}}.$$
\end{enumerate}
\end{frame}

\begin{frame}
  Note que 
  $$\l_k(x)\leq f(x)\leq u_k(x).$$
  Recordemos que si $\Ga_k\subseteq\Ga_{k+1}$, entonces 
  \begin{enumerate}[(i)]
    \item $\l_k\leq\l_{k+1}$.
    \item $u_k\geq u_{k+1}$.
  \end{enumerate}
  Si $|f(x)|\leq M$ para $x\in[a,b]$, entonces $|\l_k|,|u_k|\leq M$ en $[a,b]$. Y también
  $$
  \begin{cases}\displaystyle
    \l=\lim_{k\to\infty}\l_k\\
    \displaystyle u=\lim_{k\to\infty}u_k
  \end{cases}
  \overbrace{\To}^{\text{(TCD)}}
  \begin{cases}\displaystyle
    \lim_{k\to\infty}\int\limits_{\bonj{a,b}}\l_k(x)\dd x=\int\limits_{\bonj{a,b}}\l(x)\dd x.\\
    \displaystyle\lim_{k\to\infty}\int\limits_{\bonj{a,b}}u_k(x)\dd x=\int\limits_{\bonj{a,b}}u(x)\dd x.
  \end{cases}
  $$
\end{frame}

\begin{frame}
  Luego tenemos que
  \begin{align*}
    &\int\limits_{\bonj{a,b}}u(x)\dd x=\int\limits_{\bonj{a,b}}\l(x)\dd x\\
    \iff&\int\limits_{\bonj{a,b}}(u(x)-\l(x))\dd x=0\\
    \iff&u=\l\ \text{c.p.d.}
  \end{align*}
  Como 
  $$\int\limits_{\bonj{a,b}}\l_k(x)\dd x=\sum_{i=1}^{n}m_i(x_i-x_{i-1}),\ \int\limits_{\bonj{a,b}}u_k(x)\dd x=\sum_{i=1}^{n}M_i(x_i-x_{i-1}),$$
  entonces concluimos que $u=\l$ c.p.d. si y sólo si las sumas inferiores y superiores convergen al mismo punto.
\end{frame}

\subsection{El Teorema Clave}
\begin{frame}
  \begin{Th}\label{th:riemannIffContAE}
    Sea $f:[a,b]\to\bR$ acotada. Entonces son equivalentes
    \begin{enumerate}[(i)]
      \item $f$ es Riemann-integrable.
      \item $f$ es continua c.p.d.
    \end{enumerate}
  \end{Th}
  Comenzamos asumiendo que $f$ es Riemann-integrable. Tomemos 
  $$Z=\set{\l\neq f}\cap\set{\l\neq u}\cap\set{f\neq u}\cap\bigcup\suck\Ga_k.$$
  Como $f$ es Riemann-integrable, entonces $m(Z)=0$. 
\end{frame}

\begin{frame}{Continuamos la Prueba}
  Si fuese que $x\not\in Z$, entonces 
  $$u(x)=\l(x)=f(x),\ x\not\in\Ga_k,\ k\geq 1.$$
  Supongamos a manera de contradicción que $f$ no es continua en $x$. Entonces existe un $\eps>0$ tal que para $\dl>0$, existe $x_\dl$ tal que 
  $$|x_\dl-x|<\dl,\ \text{y}\ |f(x)-f(x_\dl)|>\eps.$$
\end{frame}

  \begin{frame}{Continuamos la Prueba}
  Dado $k$, existen $x_i^k,x_{i+1}^k$ tales que $x\in\obonj{x_i^k,x_{i+1}^k}$. Tomemos $\dl_0$ tal que 
  $$\obonj{x-\dl,x+\dl}\subseteq\obonj{x_i^k,x_{i+1}^k},\ \dl\leq \dl_0.$$
  Entonces 
  \begin{align*}
    x_\dl\in\obonj{x_i^k,x_{i+1}^k}&\To \eps\leq M_i-m_i\\
    &\To u_k(x)-\l_k(x)>\eps, \text{para todo }k.
  \end{align*}
\end{frame}

\begin{frame}{La Otra Dirección}
  Asumamos que $f$ es continua c.p.d. Sea entonces 
  $$Z=\set{x:\ f(x)\ \text{es discontinua en }x}$$
  y tomemos $x\in Z\cup\set{a,b}$. Dado $\eps>0$, existe $\dl>0$ tal que si $y\in\obonj{a,b}$, entonces
  $$|x-y|<\dl\To|f(x)-f(y)|<\frac{\eps}{2}.$$
\end{frame}

\begin{frame}{Continuamos la Prueba}
  Tome ahora $|\Ga_k|<\frac{\dl}{2}$. Sea $i$ tal que $x\in \lbonj{x_{i-1},x_i}$ para $1\leq i\leq n$. Como $\lbonj{x_{i-1},x_i}\subseteq\obonj{x-\dl,x+\dl}$ tenemos que 
  $$f(x)-\frac{\eps}{2}\leq f(x)<f(x)+\frac{\eps}{2},\ y\in \lbonj{x_{i-1},x_i}.$$
  Luego 
  $$f(x)-\frac{\eps}{2}<m_i\leq M_i\leq f(x)+\frac{\eps}{2}$$
  y así concluimos que si $|\Ga_k|\xrightarrow[k\to\infty]{}0$, existe $k_0$ tal que 
  $$k\geq k_0\To |u_k(x)-f(x)|,|\l_k(x)-f(x)|<\frac{\eps}{2}.$$
\end{frame}

\begin{frame}{Concluimos la Prueba}
  Lo anterior nos dice que 
  $$\l_k(x)\xrightarrow[k\to\infty]{}f(x),\ u_k(x)\xrightarrow[k\to\infty]{}f(x).$$
  Por el T.C.D. se cumple que 
  \begin{align*}
    &\lim_{k\to\infty}\int\limits_{\bonj{a,b}}\l_k(x)\dd x=\int\limits_{\bonj{a,b}}f(x)\dd x.
    &\lim_{k\to\infty}\int\limits_{\bonj{a,b}}u_k(x)\dd x=\int\limits_{\bonj{a,b}}f(x)\dd x.
  \end{align*}
  En conclusión las integrales de Riemann y Lebesgue concuerdan.
\end{frame}
\section*{Resumen}

\subsection*{Qu\'e vimos hoy}

\begin{frame}{Resumen}

  % Keep the summary *very short*.
  \begin{itemize}
  \item El teorema \ref{th:riemannIffContAE} que nos da la equivalencia entre integrabilidad de Riemann y continuidad casi por doquier.
  \end{itemize}
  
\end{frame}

\subsection*{Ejercicios a trabajar}
\begin{frame}{Ejercicios}
    
  \begin{itemize}
    \item
      Lista 22
      \begin{itemize}
      \item No hay ejercicios para hoy.
      \end{itemize}
    \end{itemize}
  
\end{frame}


% All of the following is optional and typically not needed. 
\appendix
\section<presentation>*{\appendixname}
\subsection<presentation>*{Lectura adicional}

\begin{frame}[allowframebreaks]
  \frametitle<presentation>{Lecturas adicionales}
    
  \begin{thebibliography}{10}
    
  \beamertemplatebookbibitems
  % Start with overview books.

  \bibitem{CambroNotas}
    S.Cambronero.
    \newblock {\em Notas MA0505}.
    \newblock 20XX.

    \bibitem{NachoNotas}
    I.Rojas
    \newblock {\em Notas MA0505}.
    \newblock 2018.
 
  \end{thebibliography}
  
\end{frame}
%% 6 - 2:10:52 48
%% 7 - 1:17:44 44
%% 8 - 27:37 69
%% 9 - 1:07:47 86 + 59:38 40 approx 2 h c 7 min
%% 10 - 2:22:39 63
%% 11 - Approx 2h c 2 min
%% 12 - Motiv y figs 1:00:17.53, en tot: 2:22:23.43
%% 13 - 1:18:03.51
%% 14 - 2:05:13.72
%% 15 - 1:13:41.49
%% 16 - 46:11.67
%% 17 - 44:18.85
%% 18 - 1:03:50.64
%% 19 - 51:57.37
%% 20 - 30:52.52
%% 21 - 1:06:30.84
%% 22 - 26:47.73
\end{document}


