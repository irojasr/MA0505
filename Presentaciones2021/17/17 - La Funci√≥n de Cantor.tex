\documentclass[utf8]{beamer}

\mode<presentation>
{
  \usetheme{Warsaw}
  \setbeamercovered{transparent}
}


\usepackage{amsfonts,mathtools,amssymb}
\usepackage[spanish]{babel}
\usepackage{times}
\usepackage[T1]{fontenc}
\usepackage[shortlabels]{enumitem}
\usepackage{tikz}
\usepackage{physics}

\title[MA0505]{MA0505 - An\'alisis I}
\subtitle{Lecci\'on XVI: Egorov y Lusin}

\author{Pedro M\'endez\inst{1}}

\institute[Universidad de Costa Rica] % (optional, but mostly needed)
{
  \inst{1}%
  Departmento de Matem\'atica Pura y Ciencias Actuariales\\
  Universidad de Costa Rica
  }

\date[I-2021] {Semestre I, 2021}

%%%%%%%%% === Theorems and suchlike === %%%%%%%%%%%%%%

\theoremstyle{plain}
\newtheorem{Th}{Teorema}               %%% Theorem 1.1.1
\newtheorem{Tmon}{Teoremón}
\newtheorem{Prop}{Proposición}         %%% Proposition 1.1.2
\newtheorem{Lem}{Lema}                 %%% Lemma 3
\newtheorem{Cor}{Corolario}            %%% Corollary 4

\theoremstyle{definition}
\newtheorem{Def}{Definición}           %%% Definition 5
\newtheorem{Ex}{Ejemplo}               %%% Example 6
\newtheorem{Ej}{Ejercicio}             %%% Ejercicio 7
\newtheorem{Hec}[Th]{Hecho}            %%% Hecho 1.1.8

\theoremstyle{remark}
\newtheorem{Rmk}[Th]{Observación}      %%%Remark 1.1.9
\newtheorem*{nonum-Rmk}{Observación}         %%% No number Fact
\newtheorem*{Notn}{Notaci\'on}        %% Notaciones
\newtheorem*{Warn}{Advertencia}       %% Advertencias

\numberwithin{equation}{section}

%% Accomodations 

\makeatletter
\def\moverlay{\mathpalette\mov@rlay}
\def\mov@rlay#1#2{\leavevmode\vtop{%
   \baselineskip\z@skip \lineskiplimit-\maxdimen
   \ialign{\hfil$\m@th#1##$\hfil\cr#2\crcr}}}
\newcommand{\charfusion}[3][\mathord]{
    #1{\ifx#1\mathop\vphantom{#2}\fi
        \mathpalette\mov@rlay{#2\cr#3}
      }
    \ifx#1\mathop\expandafter\displaylimits\fi}
\makeatother

% Greek letters:

\newcommand{\al}{\alpha}                %% short for  \alpha
\newcommand{\bt}{\beta}                 %% short for  \beta
\newcommand{\Dl}{\Delta}                %% short for  \Delta
\newcommand{\dl}{\delta}                %% short for  \delta
\newcommand{\eps}{\varepsilon}          %% short for  \varepsilon
\newcommand{\Ga}{\Gamma}                %% short for  \Gamma
\newcommand{\ga}{\gamma}                %% short for  \gamma
\newcommand{\La}{\Lambda}               %% short for  \Lambda
\newcommand{\la}{\lambda}               %% short for  \lambda
\newcommand{\kp}{\kappa}                %% short for  \kappa
\newcommand{\Om}{\Omega}                %% short for  \Omega
\newcommand{\om}{\omega}                %% short for  \omega
\newcommand{\Sg}{\Sigma}                %% short for  \Sigma
\newcommand{\sg}{\sigma}                %% short for  \sigma
\newcommand{\te}{\theta}                %% short for  \theta
\newcommand{\vf}{\varphi}               %% short for  \varphi
\newcommand{\ze}{\zeta}                 %% short for  \zeta

%Boldface letters

\newcommand{\bC}{\mathbb{C}}    %%% números complejos
\newcommand{\bN}{\mathbb{N}}    %%% números naturales
\newcommand{\bP}{\mathbb{P}}        %% números enteros positivos
\newcommand{\bQ}{\mathbb{Q}}    %%% números racionales
\newcommand{\bR}{\mathbb{R}}    %%% números reales
\newcommand{\bS}{\mathbb{S}}    %%% esfera
\newcommand{\bZ}{\mathbb{Z}}    %%% números enteros

%Script letters:

\newcommand{\cA}{\mathcal{A}}           %% formas diferenciales
\newcommand{\cB}{\mathcal{B}}           %% una base vectorial
\newcommand{\cC}{\mathcal{C}}           %% otra base vectorial
\newcommand{\cF}{\mathcal{F}}           %% espacio de Fock
\newcommand{\cL}{\mathcal{L}}           %% operadores lineales
\newcommand{\cM}{\mathcal{M}}           %% multiplicadores
\newcommand{\cN}{\mathcal{N}}           %% funciones nulas
\newcommand{\cP}{\mathcal{P}}           %% una particion
\newcommand{\cR}{\mathcal{R}}           %% funciones representativas
\newcommand{\cS}{\mathcal{S}}           %% funciones de Schwartz


%Brackets

\newcommand{\bonj}[1]{\left\lbrack#1\right\rbrack}
\newcommand{\obonj}[1]{\left\rbrack#1\right\lbrack}
\newcommand{\rbonj}[1]{\left\rbrack#1\right\rbrack}
\newcommand{\lbonj}[1]{\left\lbrack#1\right\lbrack}
\newcommand{\snm}[1]{\|#1\|}           %small norma
\newcommand{\nm}[1]{\left\|#1\right\|} %norma pegadita
\newcommand{\pnm}[1]{\biggl|\biggl|#1\biggr|\biggr|}
\newcommand{\set}[1]{\{\,#1\,\}}    %% set notation
\newcommand{\floor}[1]{\lfloor#1\rfloor} %% mayor entero <= x
\newcommand{\Set}[1]{\biggl\{\,#1\,\biggr\}} %% set notation (large)
\newcommand\quot[2]{
        \mathchoice
            {% \displaystyle
                \text{\raise1ex\hbox{$#1$}\Big/\lower1ex\hbox{$#2$}}%
            }
            {% \textstyle
                {^{ #1}/_{ #2}}
            }
            {% \scriptstyle
                {^{ #1}/_{ #2}}
            }
            {% \scriptscriptstyle
                {^{ #1}/_{ #2}}
            }
    }
\newcommand*\squot[2]{{^{ #1}/_{ #2}}}%%%small quotient

%Symbols 

\newcommand{\x}{\times}
\renewcommand{\geq}{\geqslant}          %% mayor o igual (variante)
\newcommand{\hookto}{\hookrightarrow}     %% inclusion arrow
\newcommand{\isom}{\simeq}              %% isomorfismo
\renewcommand{\l}{\ell}                   %% ele cursiva
\renewcommand{\leq}{\leqslant}          %% menor o igual (variante)
\newcommand{\less}{\setminus}           %% set difference
\newcommand{\To}{\Rightarrow}
\newcommand{\ov}{\overline}
\newcommand{\un}{\underline}
\newcommand{\del}{\partial}
\newcommand{\ind}{\mathbf{1}}       %%%indicator function

%%% Small fractions in displays:

\newcommand{\half}{{\mathchoice{\nhalf}{\thalf}{\shalf}{\shalf}}} %%display text script script^2
\newcommand{\happi}{{\tfrac{\pi}{2}}} %% small fraction  \pi/2
\newcommand{\quarter}{\tfrac{1}{4}} %% small fraction  1/4
\newcommand{\nhalf}{\frac{1}{2}}
\newcommand{\shalf}{{\scriptstyle\frac{1}{2}}} %% tiny fraction 1/2
\newcommand{\thalf}{{\tfrac{1}{2}}} %% small fraction  1/2
\newcommand{\third}{\tfrac{1}{3}}   %% small fraction  1/3 %Hay que renew porque mathabx toma second y third como x'' y x''' por ejemplo

\newcommand{\ihalf}{{\tfrac{i}{2}}} %% small fraction  i/2

\newcommand{\suci}{_{i=1}^\infty} %% diminutivo
\newcommand{\suck}{_{k=1}^\infty} %% diminutivo
\newcommand{\sucl}{_{\l=1}^\infty} %% diminutivo
\newcommand{\sucm}{_{m=1}^\infty} %% diminutivo
\newcommand{\sucn}{_{n=1}^\infty} %% diminutivo

\newcommand*{\Cdot}{{\raisebox{-0.25ex}{\scalebox{1.5}{$\cdot$}}}}      %% cdot más grande
\renewcommand{\.}{\Cdot}                %% producto escalar
\newcommand{\cupdot}{\charfusion[\mathbin]{\cup}{\.}}
\newcommand{\bigcupdot}{\charfusion[\mathbin]{\bigcup}{\.}}
\DeclareMathOperator{\Var}{Var}     %%%variance

\begin{document}

\begin{frame}
  \titlepage
\end{frame}

\begin{frame}{Agenda}
  \tableofcontents
  % You might wish to add the option [pausesections]
\end{frame}


% Structuring a talk is a difficult task and the following structure
% may not be suitable. Here are some rules that apply for this
% solution: 

% - Exactly two or three sections (other than the summary).
% - At *most* three subsections per section.
% - Talk about 30s to 2min per frame. So there should be between about
%   15 and 30 frames, all told.

% - A conference audience is likely to know very little of what you
%   are going to talk about. So *simplify*!
% - In a 20min talk, getting the main ideas across is hard
%   enough. Leave out details, even if it means being less precise than
%   you think necessary.
% - If you omit details that are vital to the proof/implementation,
%   just say so once. Everybody will be happy with that.

\section{El Teorema de Egorov}

\begin{frame}{El Teorema}
  Sean $f_k:E\to\bR$ medibles tales que $f_k\xrightarrow[k\to\infty]{}f$ c.p.d. en $E$. Sabemos que no se puede esperar en general que $f_k\to f$ uniformemente en compactos.
  \begin{Th}[Egorov]\label{th:Egorov}
    Sea $E\subseteq\bR^d$ con $m(E)<\infty$. Si $|f(x)|\neq\infty$ c.p.d. en $E$ entonces dado $\eps>0$, existe $F_\eps\subseteq E$ cerrado que satisface:
    \begin{enumerate}[(i)]
      \item $m(E\less F_\eps)<\eps$.
      \item $f_k\xrightarrow[k\to\infty]{} f$ uniformemente en $F_\eps$.
    \end{enumerate}
  \end{Th}
\end{frame}

\begin{frame}{Un Comentario}
  \begin{itemize}
    \item Sea $E=\bR^d$ y $f_k=\ind_{B(0,k)}(x)$, entonces $f_k\xrightarrow[k\to\infty]{}1$ puntualmente.
    \item Si $f$ no es acotado, existe $x$ tal que $|1-f_k(x)|=1$.
    \item Si $F$ es cerrado y $m(\bR^d\less F)\neq\infty$ entonces $F$ no es acotado.
  \end{itemize}
\end{frame}

\begin{frame}{Un Lema Previo}
  Antes de probar el teorema \ref{th:Egorov} de Egorov, vamos a probar el siguiente lema.
  \begin{Lem}\label{lem:tecEgorov}
Dado $\eps>0$ y $\eta>0$, entonces existen $F\subseteq E$ un cerrado y $k_0=k_0(\eps,\eta)\geq 0$ que satisfacen:
\begin{enumerate}[(i)]
  \item $m(E\less F)<\eta$.
  \item $|f_k(x)-f(x)|<\eps$ si $x\in F$ y $k\geq k_0$.
\end{enumerate} 
  \end{Lem}
\end{frame}

\begin{frame}{Prueba del Lema}
  Llamemos 
  $$\tilde E=\set{x\in E:\ \lim_{k\to\infty}f_k=f,\ |f|<\infty}.$$
  Sean $\eps>0$ y $\eta>0$, definamos
  $$E_m=\bigcap_{k=m}^\infty\set{|f-f_k|<\eps}=\set{x\in E:\ k>k_0\To |f(x)-f_k(x)|<\eps}.$$
  Entonces $E_m$ es medible y $E_m\subseteq E_{m+1}$. 
  \begin{Ej}\label{ej:pruebaLemEgorov}
Muestre que $\bigcup\sucm E_m=\tilde{E}$.
  \end{Ej}
\end{frame}

\begin{frame}{Continuamos la Prueba}
 \begin{itemize}
   \item Luego $m(\tilde{E})=\lim_{m\to\infty}m(E_m)$. Es decir
   $$\lim_{m\to\infty}m(\tilde{E}\less E_m)=\lim_{m\to\infty}(m(\tilde{E})-m(E_m))=0.$$
  \item Sea $k_0$ tal que $k\geq k_0\To m(E\less E_k)<\frac{\eta}{2}$. Si $x\in E_{k_0}$ entonces tenemos que 
  $$k\geq k_0\To |f(x)-f_k(x)|<\eps.$$
  \item Finalmente tomemos $F$ cerrado, $F\subseteq E_{k_0}$ que satisfaga $m(E_{k_0}\less F)<\frac{\eta}{2}$. Entonces
  $$m(E\less F)\leq m(E\less E_{k_0})+m(E_{k_0}\less F)<\eta.$$
 \end{itemize}
\end{frame}

\begin{frame}{Prueba del Teorema de Egorov}
  Dado $\eps>0$ existen $F_m\subseteq E$ cerrados y $\kp_m^\eps$ tales que 
  \begin{enumerate}[(i)]
    \item $m(E\less F_m)<\frac{\eps}{2^m}$.
    \item $|f(x)-f_k(x)|<\frac1m$ para $k\geq \kp_m^\eps$ y $x\in F_m$.
  \end{enumerate}
  Sea $F=\bigcap\sucm F_m$. Entonces $F$ es cerrado y 
  $$m(E\less F)=m\left(E\cap\bigcup\sucm F_m^c\right)\leq \sum\sucm m(E\less F_m)<\eps.$$
  Además si $x\in F$, dado $m\geq 1$ existe $\kp_m$ tal que
  $$k\geq\kp_m\To|f(x)-f_k(x)|<\frac1m.$$
\end{frame}

\section{El Teorema de Lusin}

\begin{frame}{Lusin}
  \begin{Def}\label{def:propC}
Una función $f:E\to\bR$ tiene la \alert{propiedad $C$} en $E$ si dado $\eps>0$, existe un $F\subseteq E$ tal que 
\begin{enumerate}
  \item $m(E\less F)<\eps$
  \item $f$ es continua relativa a $F$. Es decir $f:F\to\bR$ es continua.
\end{enumerate}
  \end{Def}
  En este caso si $\set{x_n}\sucn\subseteq F$ con $x_n\xrightarrow[n\to\infty]{}x\in F$, entonces $\lim_{n\to\infty}f(x_n)=f(x)$. Vamos a mostrar que esta condición es equivalente a la medibilidad de $f$.
\end{frame}

\begin{frame}{Funciones Simples}
  \begin{Lem}\label{lem:SimpleToC}
    Sea $\phi$ una función simple y medible. Entonces $\phi$ tiene la propiedad $C$.
  \end{Lem}
  \begin{itemize}
    \item Sea $\phi(x)=\sum_{k=1}^m b_k\ind_{B_k}(x)$ con $B_i\cap B_j=\emptyset$ y $b_i\neq b_j$ si $i\neq j$.
    \item Dado $\eps>0$, tome $F_\l\subseteq E_\l$ cerrado tal que $m(E_\l\less F_\l)<\frac\eps m$.
    \item Entonces $F=\bigcupdot_{\l=1}^m F_\l$ es cerrado y $m(E\less F)<\eps$.
  \end{itemize}
\end{frame}

\begin{frame}{Continuamos la Prueba}
  \begin{itemize}
  \item Sea $\set{x_n}\sucn\subseteq F$ tal que $x_n\xrightarrow[n\to\infty]{}y F_{\l_0}$.
  \item Dado $\l\neq \l_0$, entonces $\set{x_n}\sucn\cap F_\l$ es finito.
  \item En caso contrario, existe $\set{x_{n_k}}\suck\subseteq F_\l$ con $x_{n_k}\xrightarrow[n_k\to\infty]{}y$. Es decir $y\in F_\l$, lo que nos lleva a una contradicción.
  \item Por lo tanto $\exists k_0$ tal que $x_n\in F_{\l_0}$ para $n\geq k_0$. Entonces $\phi(x_n)=a_{\l_0}=\phi(y)$.
\end{itemize}
\end{frame}

\begin{frame}{El Teorema}
  \begin{Th}[Lusin]\label{th:Lusin}
    Sea $f:E\to\bR$ con $E$ medible. Entonces $f$ es medible si y sólo si $f$ tiene la propiedad $C$ en $E$.
  \end{Th}
  Sea $f:E\to\bR$ medible, entonces existe $\set{f_n}\sucn$ una sucesión de funciones simples y medibles tales que $\lim_{n\to\infty}f_n(x)=f(x)$ c.p.d. en $E$.\par 
  Sabemos que existe $F_k\subseteq E$ cerrado que satisface:
  \begin{enumerate}
    \item $m(E\less F_k)<\frac{\eps}{2^{k+1}}$.
    \item $f_k$ es continua relativa a $F_k$.  
  \end{enumerate}
\end{frame}

\begin{frame}{El Caso de Medida Finita}
  Si $m(E)<\infty$, entonces por el teorema de Egorov existe $F_0\subseteq E$ tal que 
  $$m(E\less F_0)<\frac{\eps}{2},\ f_k\xrightarrow[k\to\infty]{}f\ (\text{unif. en }F_0).$$
  Si tomamos $F=F_0\cap\bigcap\suck F_k$, entonces 
  $$m(E\less F)\leq m(E\less F_0)+\sum\suck m(E\less F_k)<\eps.$$
  Como $f_k\xrightarrow[k\to\infty]{}f$ unif. en $F$, tenemos que $f$ es continua en $F$.
\end{frame}

\begin{frame}{El Caso de Medida Infinita}
  Si $m(E)=\infty$, llamemos 
  $$E_k=E\cap \set{x\in\bR^d:\ k-1\leq |x|<k}.$$
  Ahora si $\set{x_k}\suck\subseteq F$ con $x_n\xrightarrow[n\to\infty]{}y\in F$, entonces existe $k_0$ tal que 
  $$k\geq k_0\To |x_k|\leq |y|+1.$$
  Si $m\geq |y|+1$, entonces $x_k\in E_m$ para $k\geq k_0$ y por un argumento anterior, existen $k_1,\l_1$ tales que 
  $$n\geq k_1\To x_n\in F_{\l_1}.$$
\end{frame}

\begin{frame}{La Otra Dirección}\label{ej:dirContrariaLusin}
  Si $f$ poseé la propiedad $C$, entonces para $k\geq 1$ existe $F_k\subseteq E$ que satisface
  \begin{enumerate}[(i)]
    \item $m(E\less F_k)<\frac1k$.
    \item $f$ es continua relativa a $F_k$.
  \end{enumerate}
  Sea $H=\bigcup\suck F_k$, entonces $H\subseteq E$ y $m(E\less H)=0$ (\alert{ejercicio}). Finalmente 
  \begin{gather*}
    \set{x\in E:\ f(x)>a}\\
    =\set{x\in H:\ f(x)>a}\cup \set{x\in Z:\ f(x)>a}\\
    =\bigcup\suck\set{x\in F_k:\ f(x)>a} \cup\set{x\in Z:\ f(x)>a}
  \end{gather*}
  es un conjunto medible.
\end{frame}
\section*{Resumen}

\subsection*{Qu\'e vimos hoy}

\begin{frame}{Resumen}

  % Keep the summary *very short*.
  \begin{itemize}
  \item El teorema \ref{th:Egorov} de Egorov y el lema \ref{lem:tecEgorov} para probarlo.
  \item La propiedad $C$ \ref{def:propC} y el lema \ref{lem:SimpleToC} que garantiza que las funciones simples cumplen dicha propiedad
  \item El teorema \ref{th:Lusin} de Lusin
  \end{itemize}
  
\end{frame}

\subsection*{Ejercicios a trabajar}
\begin{frame}{Ejercicios}
    
  \begin{itemize}
    \item
      Lista 16
      \begin{itemize}
      \item El ejercicio \ref{ej:pruebaLemEgorov} en medio de la prueba del lema previo a Egorov.
      \item El ejercicio \ref{ej:dirContrariaLusin} en la dirección contraria del teorema de Lusin.
      \end{itemize}
    \end{itemize}
  
\end{frame}


% All of the following is optional and typically not needed. 
\appendix
\section<presentation>*{\appendixname}
\subsection<presentation>*{Lectura adicional}

\begin{frame}[allowframebreaks]
  \frametitle<presentation>{Lecturas adicionales}
    
  \begin{thebibliography}{10}
    
  \beamertemplatebookbibitems
  % Start with overview books.

  \bibitem{CambroNotas}
    S.Cambronero.
    \newblock {\em Notas MA0505}.
    \newblock 20XX.

    \bibitem{NachoNotas}
    I.Rojas
    \newblock {\em Notas MA0505}.
    \newblock 2018.
 
  \end{thebibliography}
  
\end{frame}
%% 6 - 2:10:52 48
%% 7 - 1:17:44 44
%% 8 - 27:37 69
%% 9 - 1:07:47 86 + 59:38 40 approx 2 h c 7 min
%% 10 - 2:22:39 63
%% 11 - Approx 2h c 2 min
%% 12 - Motiv y figs 1:00:17.53, en tot: 2:22:23.43
%% 13 - 1:18:03.51
%% 14 - 2:05:13.72
%% 15 - 1:13:41.49
%% 16 - 46:11.67
\end{document}


