\documentclass[utf8]{beamer}

\mode<presentation>
{
  \usetheme{Warsaw}
  \setbeamercovered{transparent}
}


\usepackage{amsfonts,mathtools,amssymb}
\usepackage[spanish]{babel}
\usepackage{times}
\usepackage[T1]{fontenc}


\title[MA0505]{MA0505 - An\'alisis I}
\subtitle{Lecci\'on VIII: Categorías de Baire}

\author{Pedro M\'endez\inst{1}}

\institute[Universidad de Costa Rica] % (optional, but mostly needed)
{
  \inst{1}%
  Departmento de Matem\'atica Pura y Ciencias Actuariales\\
  Universidad de Costa Rica
  }

\date[I-2021] {Semestre I, 2021}

%%%%%%%%% === Theorems and suchlike === %%%%%%%%%%%%%%

\theoremstyle{plain}
\newtheorem{Th}{Teorema}               %%% Theorem 1.1.1
\newtheorem{Tmon}{Teoremón}
\newtheorem{Prop}{Proposición}         %%% Proposition 1.1.2
\newtheorem{Lem}{Lema}                 %%% Lemma 3
\newtheorem{Cor}{Corolario}            %%% Corollary 4

\theoremstyle{definition}
\newtheorem{Def}{Definición}           %%% Definition 5
\newtheorem{Ex}{Ejemplo}               %%% Example 6
\newtheorem{Ej}{Ejercicio}             %%% Ejercicio 7
\newtheorem{Hec}[Th]{Hecho}            %%% Hecho 1.1.8

\theoremstyle{remark}
\newtheorem{Rmk}[Th]{Observación}      %%%Remark 1.1.9
\newtheorem*{nonum-Rmk}{Observación}         %%% No number Fact
\newtheorem*{Notn}{Notaci\'on}        %% Notaciones
\newtheorem*{Warn}{Advertencia}       %% Advertencias

\numberwithin{equation}{section}

% Greek letters:

\newcommand{\al}{\alpha}                %% short for  \alpha
\newcommand{\bt}{\beta}                 %% short for  \beta
\newcommand{\Dl}{\Delta}                %% short for  \Delta
\newcommand{\dl}{\delta}                %% short for  \delta
\newcommand{\eps}{\varepsilon}          %% short for  \varepsilon
\newcommand{\Ga}{\Gamma}                %% short for  \Gamma
\newcommand{\ga}{\gamma}                %% short for  \gamma
\newcommand{\kp}{\kappa}                %% short for  \kappa
\newcommand{\La}{\Lambda}               %% short for  \Lambda
\newcommand{\la}{\lambda}               %% short for  \lambda
\newcommand{\Om}{\Omega}                %% short for  \Omega
\newcommand{\om}{\omega}                %% short for  \omega
\newcommand{\Sg}{\Sigma}                %% short for  \Sigma
\newcommand{\sg}{\sigma}                %% short for  \sigma
\newcommand{\Te}{\Theta}                %% short for  \Theta
\newcommand{\te}{\theta}                %% short for  \theta
\newcommand{\ups}{\upsilon}             %% short for  \upsilon
\newcommand{\vf}{\varphi}               %% short for  \varphi
\newcommand{\ze}{\zeta}                 %% short for  \zeta

%Boldface letters

\newcommand{\bC}{\mathbb{C}}    %%% números complejos
\newcommand{\bN}{\mathbb{N}}    %%% números naturales
\newcommand{\bP}{\mathbb{P}}        %% números enteros positivos
\newcommand{\bQ}{\mathbb{Q}}    %%% números racionales
\newcommand{\bR}{\mathbb{R}}    %%% números reales
\newcommand{\bS}{\mathbb{S}}    %%% esfera
\newcommand{\bZ}{\mathbb{Z}}    %%% números enteros

%Script letters:

\newcommand{\cA}{\mathcal{A}}           %% formas diferenciales
\newcommand{\cB}{\mathcal{B}}           %% una base vectorial
\newcommand{\cC}{\mathcal{C}}           %% otra base vectorial
\newcommand{\cD}{\mathcal{D}}           %% funciones de prueba
\newcommand{\cE}{\mathcal{E}}           %% un modulo proyectivo
\newcommand{\cF}{\mathcal{F}}           %% espacio de Fock
\newcommand{\cG}{\mathcal{G}}           %% funtor de Gelfand
\newcommand{\cH}{\mathcal{H}}           %% espacio de Hilbert
\newcommand{\cI}{\mathcal{I}}           %% un funtor de inclusion
\newcommand{\cJ}{\mathcal{J}}           %% otro funtor
\newcommand{\cK}{\mathcal{K}}           %% otro espacio de Hilbert
\newcommand{\cL}{\mathcal{L}}           %% operadores lineales
\newcommand{\cM}{\mathcal{M}}           %% multiplicadores
\newcommand{\cN}{\mathcal{N}}           %% funciones nulas
\newcommand{\cO}{\mathcal{O}}           %% funciones de crec-to lento
\newcommand{\cP}{\mathcal{P}}           %% una particion
\newcommand{\cR}{\mathcal{R}}           %% funciones representativas
\newcommand{\cQ}{\mathcal{Q}}           %% otra particion
\newcommand{\cS}{\mathcal{S}}           %% funciones de Schwartz
\newcommand{\cT}{\mathcal{T}}           %% una topologia
\newcommand{\cU}{\mathcal{U}}           %% cubrimiento abierto
\newcommand{\cV}{\mathcal{V}}           %% vecindarios
\newcommand{\cW}{\mathcal{W}}           %% grupo de Weyl


%Brackets

\newcommand{\bonj}[1]{\left\lbrack#1\right\rbrack}
\newcommand{\obonj}[1]{\left\rbrack#1\right\lbrack}
\newcommand{\rbonj}[1]{\left\rbrack#1\right\rbrack}
\newcommand{\lbonj}[1]{\left\lbrack#1\right\lbrack}
\newcommand{\snm}[1]{\|#1\|}           %small norma
\newcommand{\nm}[1]{\left\|#1\right\|} %norma pegadita
\newcommand{\pnm}[1]{\biggl|\biggl|#1\biggr|\biggr|}
\newcommand{\set}[1]{\{\,#1\,\}}    %% set notation
\newcommand{\floor}[1]{\lfloor#1\rfloor} %% mayor entero <= x
\newcommand{\Set}[1]{\biggl\{\,#1\,\biggr\}} %% set notation (large)
\newcommand\quot[2]{
        \mathchoice
            {% \displaystyle
                \text{\raise1ex\hbox{$#1$}\Big/\lower1ex\hbox{$#2$}}%
            }
            {% \textstyle
                {^{ #1}/_{ #2}}
            }
            {% \scriptstyle
                {^{ #1}/_{ #2}}
            }
            {% \scriptscriptstyle
                {^{ #1}/_{ #2}}
            }
    }
\newcommand*\squot[2]{{^{ #1}/_{ #2}}}%%%small quotient

%Symbols 

\renewcommand{\geq}{\geqslant}          %% mayor o igual (variante)
\newcommand{\hookto}{\hookrightarrow}     %% inclusion arrow
\newcommand{\isom}{\simeq}              %% isomorfismo
\renewcommand{\l}{\ell}                   %% ele cursiva
\renewcommand{\leq}{\leqslant}          %% menor o igual (variante)
\newcommand{\less}{\setminus}           %% set difference
\newcommand{\To}{\Rightarrow}
\newcommand{\ov}{\overline}
\newcommand{\un}{\underline}
\newcommand{\del}{\partial}

%%% Small fractions in displays:

\newcommand{\half}{{\mathchoice{\nhalf}{\thalf}{\shalf}{\shalf}}} %%display text script script^2
\newcommand{\happi}{{\tfrac{\pi}{2}}} %% small fraction  \pi/2
\newcommand{\quarter}{\tfrac{1}{4}} %% small fraction  1/4
\newcommand{\nhalf}{\frac{1}{2}}
\newcommand{\shalf}{{\scriptstyle\frac{1}{2}}} %% tiny fraction 1/2
\newcommand{\thalf}{{\tfrac{1}{2}}} %% small fraction  1/2
\newcommand{\third}{\tfrac{1}{3}}   %% small fraction  1/3 %Hay que renew porque mathabx toma second y third como x'' y x''' por ejemplo

\newcommand{\ihalf}{{\tfrac{i}{2}}} %% small fraction  i/2

\newcommand{\sucm}{_{m=1}^\infty} %% diminutivo
\newcommand{\suck}{_{k=1}^\infty} %% diminutivo
\newcommand{\sucn}{_{n=1}^\infty} %% diminutivo

\begin{document}

\begin{frame}
  \titlepage
\end{frame}

\begin{frame}{Agenda}
  \tableofcontents
  % You might wish to add the option [pausesections]
\end{frame}


% Structuring a talk is a difficult task and the following structure
% may not be suitable. Here are some rules that apply for this
% solution: 

% - Exactly two or three sections (other than the summary).
% - At *most* three subsections per section.
% - Talk about 30s to 2min per frame. So there should be between about
%   15 and 30 frames, all told.

% - A conference audience is likely to know very little of what you
%   are going to talk about. So *simplify*!
% - In a 20min talk, getting the main ideas across is hard
%   enough. Leave out details, even if it means being less precise than
%   you think necessary.
% - If you omit details that are vital to the proof/implementation,
%   just say so once. Everybody will be happy with that.

\section{La Definición de Categorías}

\begin{frame}{Conjuntos Densos en Ninguna Parte}
  \begin{Def}\label{def:nowhereDense}
    Dado un espacio métrico $(X,d)$, diremos que $A\subseteq X$ es \alert{denso en ninguna parte} si $(\ov A)^c$ es denso.
  \end{Def}
 Note que $(\ov A)^c$ es denso si y sólo si $(\ov A)^c\cap B(x,r)\neq \emptyset$ para $x\in X$ y $r>0$. Esto es equivalente a que $(\ov A)^\circ=\emptyset$.
\end{frame}

\begin{frame}{Primera y Segunda Categoría}
  \begin{Def}\label{def:cat1}
    \begin{itemize}
      \item Un conjunto $A$ es de \alert{primera categoría} ó \alert{magro} si $A$ es una unión contable de conjuntos densos en ninguna parte.
      \item Un conjunto es de \alert{segunda categoría} si no es de primera categoría.
    \end{itemize}
    
  \end{Def}
\end{frame}

\section{El Teorema}
\begin{frame}{El Teorema de Categorías}
  \begin{Th}[Categorías de Baire]\label{thm:cats}
    Sea $(X,d)$ un espacio completo. Si $G\subseteq X$ es un abierto no vacío, entonces $G$ es de segunda categoría.
  \end{Th}
  Antes de probar el teorema, probaremos el siguiente resultado.
  \begin{Th}[Baire]\label{thm:Baire}
    Sea $(X,d)$ completo. Si $\set{G_n}\sucn$ es una sucesión de conjuntos abiertos y densos en $X$, entonces $\bigcap\sucn G_n$ es denso en $X$.
  \end{Th} 
\end{frame}

\begin{frame}{Prueba del Teorema de Baire}
  Sea $A\subseteq X$ un abierto. 
  \begin{itemize}
    \item Como $G_1$ es denso, existe $x_1\in G_1$ tal que $x_1\in A\cap G_1$.
    \item Como $A\cap G_1$ es abierto, existe $r>0$ tal que 
    $$B(x_1,r_1)\subseteq A\cap G_1.$$
    \item Sea $B_1=B\left(x_1,\frac{r_1}{2}\right)$, entonces $\ov B_1\subseteq A\cap G_1$.
  \end{itemize}
\end{frame}

\begin{frame}{Continuamos la Prueba}
  \begin{itemize}
    \item De igual forma, al ser $G_2$ denso y abierto, existen $x_2\in G_2$ y un $r_2>0$ tales que 
     $$B(x_2,r_2)\subseteq B_1\cap G_2\subseteq A\cap G_1\cap G_2.$$
     \item Sea $B_2=B\left(x_2,\frac{r_2}{2}\right)$, entonces 
     $$\ov B_2\subseteq B_1\cap G_1\subseteq A\cap G_1\cap G_2.$$
     \item Iterando, existe $x_n\in G_n$ y $r_n<\frac{r_{n-1}}{2}\leq \frac{r_1}{2^n}$ tal que 
     $$B(x_n,r_n)\subseteq B_{n-1}\cap G_n\subseteq A\cap \bigcap\sucn G_i.$$
  \end{itemize}
\end{frame}

\begin{frame}{Continuamos la Prueba}
  \begin{itemize}
    \item Sea así $B_n=B\left(x_n,\frac{r_n}{2}\right)$, entonces
    $$\ov B_n\subseteq B_{n-1}\cap G_n.$$
    \item Si $n\geq m$ entonces $B_n\subseteq B_m$
    $$\To x_n,x_m\in B_m\To d(x_n,x_m)<\frac{r_1}{2^m}.$$
    \item Así $\set{x_n}\sucn$ es de Cauchy, sea $x=\lim x_n$.
    \item Como $\set{x_n}\sucn\subseteq B_m$, tenemos que $x\in\ov B_m$.
  \end{itemize} 
\end{frame}

\begin{frame}{Terminamos la Prueba}
  
  Por tanto $x\in A\cap\bigcap_{i=1}^m G_i$ para $m\geq 1$, lo que nos dice que 
  $$x\in A\cap\bigcap_{i=1}^\infty G_i.$$
  Concluimos que $\bigcap_{i=1}^\infty G_i$ es denso.\par 
  Con este resultado ya podemos probar el Teorema de Categorías.
\end{frame}

\begin{frame}{Prueba del Teorema de Categorías}
  \begin{itemize}
    \item Sea $G$ un abierto y asumamos que es de primera categoría.
    \item Así existen $A_n$ densos en ninguna parte tal que 
     $$G=\bigcup\sucn A_n.$$
    \item Llamemos $G_n=(\ov A_n)^c$, entonces $G_n$ es abierto y denso.
    \item Luego $\bigcap\sucn (\ov{A}_n)^c$ es denso.
  \end{itemize}
\end{frame}

\begin{frame}{Terminamos la Prueba}
  Como
  $$\left(\ov{\bigcup\sucn A_n}\right)^c=\left({\bigcup\sucn \ov A_n}\right)^c=\bigcap\sucn (\ov A_n)^c$$
  es un conjunto denso, entonces 
  $$G\cap \left(\ov{\bigcup\sucn A_n}\right)^c\neq\emptyset$$
  lo que nos lleva a una contradicción.
\end{frame}

\section*{Resumen}

\subsection*{Qu\'e vimos hoy}
\begin{frame}{Resumen}

  % Keep the summary *very short*.
  \begin{itemize}
  \item La definición \ref{def:nowhereDense} de denso por ninguna parte.
  \item La definición \ref{def:cat1} de conjuntos magros.
  \item El teorema \ref{thm:cats} de las Categorías de Baire.
  \item El teorema \ref{thm:Baire} de Baire para probar el teorema de cateogorías.
  \end{itemize}
  
\end{frame}
\iffalse
\subsection*{Ejercicios a trabajar}
\begin{frame}{Ejercicios}
    
  \begin{itemize}
    \item
      Lista 8
      \begin{itemize}
      \item 
      \end{itemize}
    \end{itemize}
  
\end{frame}

\fi
% All of the following is optional and typically not needed. 
\appendix
\section<presentation>*{\appendixname}
\subsection<presentation>*{Lectura adicional}

\begin{frame}[allowframebreaks]
  \frametitle<presentation>{Lecturas adicionales}
    
  \begin{thebibliography}{10}
    
  \beamertemplatebookbibitems
  % Start with overview books.

  \bibitem{CambroNotas}
    S.Cambronero.
    \newblock {\em Notas MA0505}.
    \newblock 20XX.

    \bibitem{NachoNotas}
    I.Rojas
    \newblock {\em Notas MA0505}.
    \newblock 2018.
 
  \end{thebibliography}
  
\end{frame}
%% 6 - 2:10:52 48
%% 7 - 1:17:44 44
%% 8 - 27:37 69
\end{document}


