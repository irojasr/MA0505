\documentclass[utf8]{beamer}

\mode<presentation>
{
  \usetheme{Warsaw}
  \setbeamercovered{transparent}
}


\usepackage{amsfonts,mathtools,amssymb}
\usepackage[spanish]{babel}
\usepackage{times}
\usepackage[T1]{fontenc}
\usepackage[shortlabels]{enumitem}
\usepackage{tikz}
\usepackage{physics}

\title[MA0505]{MA0505 - An\'alisis I}
\subtitle{Lecci\'on X: Variación Acotada}

\author{Pedro M\'endez\inst{1}}

\institute[Universidad de Costa Rica] % (optional, but mostly needed)
{
  \inst{1}%
  Departmento de Matem\'atica Pura y Ciencias Actuariales\\
  Universidad de Costa Rica
  }

\date[I-2021] {Semestre I, 2021}

%%%%%%%%% === Theorems and suchlike === %%%%%%%%%%%%%%

\theoremstyle{plain}
\newtheorem{Th}{Teorema}               %%% Theorem 1.1.1
\newtheorem{Tmon}{Teoremón}
\newtheorem{Prop}{Proposición}         %%% Proposition 1.1.2
\newtheorem{Lem}{Lema}                 %%% Lemma 3
\newtheorem{Cor}{Corolario}            %%% Corollary 4

\theoremstyle{definition}
\newtheorem{Def}{Definición}           %%% Definition 5
\newtheorem{Ex}{Ejemplo}               %%% Example 6
\newtheorem{Ej}{Ejercicio}             %%% Ejercicio 7
\newtheorem{Hec}[Th]{Hecho}            %%% Hecho 1.1.8

\theoremstyle{remark}
\newtheorem{Rmk}[Th]{Observación}      %%%Remark 1.1.9
\newtheorem*{nonum-Rmk}{Observación}         %%% No number Fact
\newtheorem*{Notn}{Notaci\'on}        %% Notaciones
\newtheorem*{Warn}{Advertencia}       %% Advertencias

\numberwithin{equation}{section}

%% Accomodations 

\makeatletter
\def\moverlay{\mathpalette\mov@rlay}
\def\mov@rlay#1#2{\leavevmode\vtop{%
   \baselineskip\z@skip \lineskiplimit-\maxdimen
   \ialign{\hfil$\m@th#1##$\hfil\cr#2\crcr}}}
\newcommand{\charfusion}[3][\mathord]{
    #1{\ifx#1\mathop\vphantom{#2}\fi
        \mathpalette\mov@rlay{#2\cr#3}
      }
    \ifx#1\mathop\expandafter\displaylimits\fi}
\makeatother

% Greek letters:

\newcommand{\al}{\alpha}                %% short for  \alpha
\newcommand{\bt}{\beta}                 %% short for  \beta
\newcommand{\Dl}{\Delta}                %% short for  \Delta
\newcommand{\dl}{\delta}                %% short for  \delta
\newcommand{\eps}{\varepsilon}          %% short for  \varepsilon
\newcommand{\Ga}{\Gamma}                %% short for  \Gamma
\newcommand{\ga}{\gamma}                %% short for  \gamma
\newcommand{\La}{\Lambda}               %% short for  \Lambda
\newcommand{\la}{\lambda}               %% short for  \lambda
\newcommand{\Om}{\Omega}                %% short for  \Omega
\newcommand{\om}{\omega}                %% short for  \omega
\newcommand{\Sg}{\Sigma}                %% short for  \Sigma
\newcommand{\sg}{\sigma}                %% short for  \sigma
\newcommand{\te}{\theta}                %% short for  \theta
\newcommand{\vf}{\varphi}               %% short for  \varphi
\newcommand{\ze}{\zeta}                 %% short for  \zeta

%Boldface letters

\newcommand{\bC}{\mathbb{C}}    %%% números complejos
\newcommand{\bN}{\mathbb{N}}    %%% números naturales
\newcommand{\bP}{\mathbb{P}}        %% números enteros positivos
\newcommand{\bQ}{\mathbb{Q}}    %%% números racionales
\newcommand{\bR}{\mathbb{R}}    %%% números reales
\newcommand{\bS}{\mathbb{S}}    %%% esfera
\newcommand{\bZ}{\mathbb{Z}}    %%% números enteros

%Script letters:

\newcommand{\cA}{\mathcal{A}}           %% formas diferenciales
\newcommand{\cB}{\mathcal{B}}           %% una base vectorial
\newcommand{\cC}{\mathcal{C}}           %% otra base vectorial
\newcommand{\cF}{\mathcal{F}}           %% espacio de Fock
\newcommand{\cL}{\mathcal{L}}           %% operadores lineales
\newcommand{\cM}{\mathcal{M}}           %% multiplicadores
\newcommand{\cN}{\mathcal{N}}           %% funciones nulas
\newcommand{\cP}{\mathcal{P}}           %% una particion
\newcommand{\cR}{\mathcal{R}}           %% funciones representativas
\newcommand{\cS}{\mathcal{S}}           %% funciones de Schwartz


%Brackets

\newcommand{\bonj}[1]{\left\lbrack#1\right\rbrack}
\newcommand{\obonj}[1]{\left\rbrack#1\right\lbrack}
\newcommand{\rbonj}[1]{\left\rbrack#1\right\rbrack}
\newcommand{\lbonj}[1]{\left\lbrack#1\right\lbrack}
\newcommand{\snm}[1]{\|#1\|}           %small norma
\newcommand{\nm}[1]{\left\|#1\right\|} %norma pegadita
\newcommand{\pnm}[1]{\biggl|\biggl|#1\biggr|\biggr|}
\newcommand{\set}[1]{\{\,#1\,\}}    %% set notation
\newcommand{\floor}[1]{\lfloor#1\rfloor} %% mayor entero <= x
\newcommand{\Set}[1]{\biggl\{\,#1\,\biggr\}} %% set notation (large)
\newcommand\quot[2]{
        \mathchoice
            {% \displaystyle
                \text{\raise1ex\hbox{$#1$}\Big/\lower1ex\hbox{$#2$}}%
            }
            {% \textstyle
                {^{ #1}/_{ #2}}
            }
            {% \scriptstyle
                {^{ #1}/_{ #2}}
            }
            {% \scriptscriptstyle
                {^{ #1}/_{ #2}}
            }
    }
\newcommand*\squot[2]{{^{ #1}/_{ #2}}}%%%small quotient

%Symbols 

\newcommand{\x}{\times}
\renewcommand{\geq}{\geqslant}          %% mayor o igual (variante)
\newcommand{\hookto}{\hookrightarrow}     %% inclusion arrow
\newcommand{\isom}{\simeq}              %% isomorfismo
\renewcommand{\l}{\ell}                   %% ele cursiva
\renewcommand{\leq}{\leqslant}          %% menor o igual (variante)
\newcommand{\less}{\setminus}           %% set difference
\newcommand{\To}{\Rightarrow}
\newcommand{\ov}{\overline}
\newcommand{\un}{\underline}
\newcommand{\del}{\partial}

%%% Small fractions in displays:

\newcommand{\half}{{\mathchoice{\nhalf}{\thalf}{\shalf}{\shalf}}} %%display text script script^2
\newcommand{\happi}{{\tfrac{\pi}{2}}} %% small fraction  \pi/2
\newcommand{\quarter}{\tfrac{1}{4}} %% small fraction  1/4
\newcommand{\nhalf}{\frac{1}{2}}
\newcommand{\shalf}{{\scriptstyle\frac{1}{2}}} %% tiny fraction 1/2
\newcommand{\thalf}{{\tfrac{1}{2}}} %% small fraction  1/2
\newcommand{\third}{\tfrac{1}{3}}   %% small fraction  1/3 %Hay que renew porque mathabx toma second y third como x'' y x''' por ejemplo

\newcommand{\ihalf}{{\tfrac{i}{2}}} %% small fraction  i/2

\newcommand{\suci}{_{i=1}^\infty} %% diminutivo
\newcommand{\suck}{_{k=1}^\infty} %% diminutivo
\newcommand{\sucm}{_{m=1}^\infty} %% diminutivo
\newcommand{\sucn}{_{n=1}^\infty} %% diminutivo

\newcommand*{\Cdot}{{\raisebox{-0.25ex}{\scalebox{1.5}{$\cdot$}}}}      %% cdot más grande
\renewcommand{\.}{\Cdot}                %% producto escalar
\newcommand{\cupdot}{\charfusion[\mathbin]{\cup}{\.}}
\newcommand{\bigcupdot}{\charfusion[\mathbin]{\bigcup}{\.}}
\DeclareMathOperator{\Var}{Var}     %%%variance

\begin{document}

\begin{frame}
  \titlepage
\end{frame}

\begin{frame}{Agenda}
  \tableofcontents
  % You might wish to add the option [pausesections]
\end{frame}


% Structuring a talk is a difficult task and the following structure
% may not be suitable. Here are some rules that apply for this
% solution: 

% - Exactly two or three sections (other than the summary).
% - At *most* three subsections per section.
% - Talk about 30s to 2min per frame. So there should be between about
%   15 and 30 frames, all told.

% - A conference audience is likely to know very little of what you
%   are going to talk about. So *simplify*!
% - In a 20min talk, getting the main ideas across is hard
%   enough. Leave out details, even if it means being less precise than
%   you think necessary.
% - If you omit details that are vital to the proof/implementation,
%   just say so once. Everybody will be happy with that.

\section{Funciones de Variación Acotada}

\subsection{Motivación}

\begin{frame}{Dos Problemas}
 Veremos dos problemas que muestran la necesidad de extendar las funciones que se pueden abordar por métodos usuales como el uso de la integral de Riemann.
\end{frame}

\begin{frame}{Longitud de una Curva}
Considere $\ga:[a,b]\to\bR^2$ con $\ga(t)=(\ga_1(t),\ga_2(t))$. Inicialmente nuestra función no es continua ni acotada. \par 
\begin{center}
  ¿Cuál es la longitud de la curva?
\end{center}
\begin{figure}
  \tikzset{every picture/.style={line width=0.75pt}} %set default line width to 0.75pt    
  \begin{tikzpicture}[x=0.75pt,y=0.75pt,yscale=-1,xscale=1]
    %uncomment if require: \path (0,300); %set diagram left start at 0, and has height of 300
    
    %Curve Lines [id:da8584735824612737] 
    \draw    (62,170) .. controls (158,95) and (135,254) .. (199,191) ;
    %Curve Lines [id:da9070985913366083] 
    \draw    (199,191) .. controls (319,90) and (286,237) .. (326,207) ;
    %Curve Lines [id:da55587343535453] 
    \draw    (326,207) .. controls (414,149) and (457,130) .. (474,210) ;
    %Straight Lines [id:da358668358532094] 
    \draw [color={rgb, 255:red, 208; green, 2; blue, 27 }  ,draw opacity=1 ]   (62,170) -- (153,193) ;
    %Straight Lines [id:da3836975443018966] 
    \draw [color={rgb, 255:red, 208; green, 2; blue, 27 }  ,draw opacity=1 ]   (153,193) -- (203,187) ;
    %Straight Lines [id:da12640039782348245] 
    \draw [color={rgb, 255:red, 208; green, 2; blue, 27 }  ,draw opacity=1 ][fill={rgb, 255:red, 208; green, 2; blue, 27 }  ,fill opacity=1 ]   (203,187) -- (266,155) ;
    %Straight Lines [id:da517062228478639] 
    \draw [color={rgb, 255:red, 208; green, 2; blue, 27 }  ,draw opacity=1 ]   (266,155) -- (326,207) ;
    %Straight Lines [id:da23425723851504388] 
    \draw [color={rgb, 255:red, 208; green, 2; blue, 27 }  ,draw opacity=1 ]   (326,207) -- (430.33,156.33) ;
    %Straight Lines [id:da9838323731219862] 
    \draw [color={rgb, 255:red, 208; green, 2; blue, 27 }  ,draw opacity=1 ]   (430.33,156.33) -- (474,210) ;
    
    % Text Node
    \draw (43.33,171.4) node [anchor=north west][inner sep=0.75pt]    {$\gamma ( a)$};
    % Text Node
    \draw (117.33,196.07) node [anchor=north west][inner sep=0.75pt]    {$\gamma ( t_{1})$};
    % Text Node
    \draw (167.33,166.07) node [anchor=north west][inner sep=0.75pt]    {$\gamma ( t_{2})$};
    % Text Node
    \draw (268,133.4) node [anchor=north west][inner sep=0.75pt]    {$\gamma ( t_{3})$};
    % Text Node
    \draw (328,210.4) node [anchor=north west][inner sep=0.75pt]    {$\gamma ( t_{4})$};
    % Text Node
    \draw (394.67,135.07) node [anchor=north west][inner sep=0.75pt]    {$\gamma ( t_{5})$};
    % Text Node
    \draw (436.67,211.07) node [anchor=north west][inner sep=0.75pt]    {$\gamma ( b)$};
    
    
    \end{tikzpicture}
    
\end{figure}
\end{frame}

\begin{frame}{Primeras Ideas}
  Podemos aproximar la curva por segmentos.
  \begin{itemize}
    \item Dados $a=t_0<t_1<t_2<\dots<t_n=b$, consideramos los segmentos $[\ga(t_i),\ga(t_{i+1})]$ con $0\leq i\leq n-1$ donde $P=\set{t_0,t_1,\dots,t_n}$.
    \item Tomando particiones cada vez más finas esperaríamos que la longitud de las curvas poligonales aproximen la longitud de la curva. ¿Será cierto que el límite es finito?
    \item Note que estos límites no pueden ser aproximados por la integral de Riemann a menos que $\ga_1,\ga_2$ sean diferenciables.
  \end{itemize}
\end{frame}

\begin{frame}{Componentes Diferenciables}
  En el caso que $\ga_1,\ga_2$ sean diferenciables, vale el teorema del valor medio:
  $$(\ga(t_{i+1})-\ga(t_i))=\ga'(\xi_i)(t_{i+1}-t_i).$$
  En general no es claro que el conjunto $L(\Ga,\rho)$ es acotado.
\end{frame}

\begin{frame}{Masa de un Alambre}
  Un problema similar surge a la hora de medir la masa de un alambre. Aquí el área transversal es $A$.

  \begin{figure}
    \begin{tikzpicture}[x=0.75pt,y=0.75pt,yscale=-1,xscale=1]
      %uncomment if require: \path (0,241); %set diagram left start at 0, and has height of 241
      
      %Flowchart: Direct Access Storage [id:dp5786229900826654] 
      \draw   (210.68,98) -- (79.29,98) .. controls (59.76,98) and (39.03,84.57) .. (33,68) .. controls (26.97,51.43) and (37.92,38) .. (57.45,38) -- (188.84,38)(235.14,68) .. controls (241.17,84.57) and (230.22,98) .. (210.68,98) .. controls (191.15,98) and (170.42,84.57) .. (164.39,68) .. controls (158.36,51.43) and (169.31,38) .. (188.84,38) .. controls (208.38,38) and (229.11,51.43) .. (235.14,68) ;
      %Straight Lines [id:da5893074994090735] 
      \draw    (75.71,100.35) -- (76,110.75) ;
      \draw [shift={(75.64,98)}, rotate = 88.4] [color={rgb, 255:red, 0; green, 0; blue, 0 }  ][line width=0.75]      (0, 0) circle [x radius= 3.35, y radius= 3.35]   ;
      %Straight Lines [id:da19694637791321545] 
      \draw    (107.21,100.35) -- (107.5,110.75) ;
      \draw [shift={(107.14,98)}, rotate = 88.4] [color={rgb, 255:red, 0; green, 0; blue, 0 }  ][line width=0.75]      (0, 0) circle [x radius= 3.35, y radius= 3.35]   ;
      %Flowchart: Direct Access Storage [id:dp15114335913071741] 
      \draw  [dash pattern={on 4.5pt off 4.5pt}] (185.5,186.32) -- (166.91,186.32) .. controls (164.15,186.32) and (161.91,172.89) .. (161.91,156.32) .. controls (161.91,139.75) and (164.15,126.32) .. (166.91,126.32) -- (185.5,126.32)(190.5,156.32) .. controls (190.5,172.89) and (188.26,186.32) .. (185.5,186.32) .. controls (182.73,186.32) and (180.49,172.89) .. (180.49,156.32) .. controls (180.49,139.75) and (182.73,126.32) .. (185.5,126.32) .. controls (188.26,126.32) and (190.5,139.75) .. (190.5,156.32) ;
      %Curve Lines [id:da5804114816457824] 
      \draw    (146,106.25) .. controls (127.38,152.8) and (125.09,154.69) .. (155.12,154.28) ;
      \draw [shift={(157,154.25)}, rotate = 539.1] [color={rgb, 255:red, 0; green, 0; blue, 0 }  ][line width=0.75]    (10.93,-3.29) .. controls (6.95,-1.4) and (3.31,-0.3) .. (0,0) .. controls (3.31,0.3) and (6.95,1.4) .. (10.93,3.29)   ;
      %Straight Lines [id:da8081791899844226] 
      \draw    (166.64,185.5) -- (167,198.68) ;
      %Straight Lines [id:da7670770097216806] 
      \draw    (185.64,185) -- (186,197.75) ;
      
      % Text Node
      \draw (77.64,101.4) node [anchor=north west][inner sep=0.75pt]    {$t_{0}$};
      % Text Node
      \draw (109.14,101.4) node [anchor=north west][inner sep=0.75pt]    {$t_{1}$};
      % Text Node
      \draw (151.14,188.78) node [anchor=north west][inner sep=0.75pt]    {$t_{i}$};
      % Text Node
      \draw (187.5,189.72) node [anchor=north west][inner sep=0.75pt]    {$t_{i+1}$};
      
      
      \end{tikzpicture}
  \end{figure}
  
\end{frame}

\begin{frame}{Análisis del Problema}
  En este caso podemos considerar 
  $$\sum_{i=1}^{n-1}\rho(t_i)A(t_i,t_{i+1}),$$
  donde $\rhd(t_i)$ es la densidad del alambre en el punto
  $t_i$ y $A(t_i,t_{i+1})$ es el area del segmento $t_i$ a $t_{i+1}$. Es decir 
  $$A(t_i,t_{i+1})\approx A\nm{\ga(t_i)-\ga(t_{i+1})}.$$
  La masa se aproxima por 
  $$\sum_{i=1}^{n-1}\rho(t_i)A\nm{\ga(t_i)-\ga(t_{i+1})}.$$
\end{frame}

\subsection{La Definición de la Variación Acotado}

\begin{frame}{Variación Acotada}
  Antes de explorar una nueva definición de integral que nos permite abordar estos problemas, vamos a centrarnos en el problema de las funciones tales que existe un $M\geq 0$ con 
  $$\sum_{i=1}^{n-1}|f(t_i)-f(t_{i+1})|\leq M$$
  para cualquier partición.
\end{frame}

\begin{frame}{Variación Acotada}
  \begin{Def}\label{def:varAcot}
    Decimos que $f:[a,b]\to\bR$ es de \alert{variación acotada} si existe un $M> 0$ tal que para toda partición 
    $$\Ga=\set{t_0=a<t_1<\dots<t_n=b}$$
    se tiene que 
    $$S(f,\Ga):=\sum_{i=1}^{n-1}|f(t_i)-f(t_{i+1})|\leq M.$$
    En otras palabras $f$ es variación acotada si vale que $$\sup_\Ga S(f,\Ga)<\infty.$$
  \end{Def}
\end{frame}

\begin{frame}{Ejemplos de Funciones de Variación Acotada}
  \newcounter{saveenumi}
  \begin{enumerate}
    \item Si $f$ es creciente, entonces
    $$\sum_{i=1}^{n-1}|f(t_i)-f(t_{i+1})|=\sum_{i=1}^{n-1}f(t_i)-f(t_{i+1})=f(b)-f(a).$$
    Análogamente si $f$ es decreciente $S(f,\Ga)=f(a)-f(b)$. Por lo tanto \alert{toda función monótona es de variación acotada}.
    \setcounter{saveenumi}{\value{enumi}}
  \end{enumerate}
\end{frame}

\begin{frame}{Ejemplos de Funciones de Variación Acotada}
  
  \begin{enumerate}
  \setcounter{enumi}{\value{saveenumi}}    
\item Tomemos $\phi:[0,1]\to\bR$ dada por 
$$\phi(x)=\begin{cases}
  0,\ x\in\bQ\\
  1,\ x\not\in\bQ
\end{cases}$$
y sea $x_n$ tal que $x_{2k}\in\bQ$ para $0\leq k\leq n$ mientras que $x_{2k+1}\in\bR\less\bQ$ para $0\leq k\leq n-1$. Tomamos $x_0=0$ y $x_{2n}=1$. Así vale que 
$$S(\phi,\Ga)=\sum{i=0}^{2n-1}|\phi(x_{i+1})-\phi(x_i)|=n$$
y por tanto $\phi$ \alert{no es de variación acotada}.
  \end{enumerate}
\end{frame}

\begin{frame}{La Variación Total}
  Dada $f:[a,b]\to\bR$ una función de variación acotada, definimos 
  $$\Var(f,[a,b])=\sup_{\Ga}S(f,\Ga).$$
  Sea $\Ga_1=\set{x_0=a<x_1<\dots<x_n=x}$ con $a<x<b$, entonces $\Ga=\Ga_1\cup\set{b}$ es una partición de $[a,b]$. Si $x_{n+1}=b$, 
  $$S(f,\Ga)=\sum_{i=1}^{n-1}|f(t_{x+1})-f(x_i)|=S(f,\Ga_1)+|f(b)-f(x)|.$$
  Por lo tanto $f:[a,x]\to\bR$ es de variación acotada y 
  $$\Var(f,[a,x])\leq\Var(f,[a,b]).$$
  Además
  \begin{align*}
    &|f(a)-f(x)|+|f(x)-f(b)\leq\Var(f,[a,b])\\
    \To &2|f(x)|\leq \Var(f,[a,b])+|f(a)|+|f(b)|.
  \end{align*}
\end{frame}
  
\begin{frame}{Propiedades}
  \begin{Lem}\label{lem:propsVA}
    Sean $f,g:[a,b]\to\bR$ de variación acotada. Entonces 
    \begin{enumerate}[(i)]
      \item $cf+g$ es de variación acotada para $c\in\bR$.
      \item $fg$ es de variación acotada.
      \item Si existe $\eps>0 $ tal que $|g(x)|\geq\eps$ para $x\in[a,b]$, entonces $\frac{1}{g}$ es de variación acotada.
      \item $f,g$ son acotadas.
    \end{enumerate}
  \end{Lem}
  La prueba de este lema es un \alert{ejercicio}.
\end{frame}

\begin{frame}{Subintervalos y la Variación}
  Usando los argumentos expuestos se prueba que si $\Ga_1$ es una partición de $[a_1,b_1]$ con $a<a_1<b_1<b$. %%Qué? 
  Tome $\Ga=\Ga_1\cup\set{a,b}$, entonces
\begin{align*}
  &S(f,\Ga_1)\leq S(f,\Ga)\leq\Var\bonj{f,[a,b]}\\
  \To&\Var\bonj{f,[a_1,b_1]}\leq \Var\bonj{f,[a,b]}.
\end{align*}

\end{frame}

\begin{frame}{Subintervalos y la Variación}
  \begin{itemize}
    \item Sea $a<c<b$, considere $\Ga$ una partición de $[a,b]$ con 
    $$\Ga=\set{x_0=a<x_1<\dots<x_n=x}.$$
    \item Tome $x_{m_0}$ tal que $x_{m_0}\leq c<x_{m_0+1}$.
    \item Defina $\Ga'=\Ga\cup\set{c}$ y
    \begin{align*}
      \Ga_1&=\set{x_0=a<x_1<\dots\leq x_{m_0}\leq c}\\
      \Ga_2&=\set{c\leq x_{m_0+1}<\dots<b=x_n}.
    \end{align*}
    
  \end{itemize}

\end{frame}

\begin{frame}{Subintervalos y la Variación}
  Entonces
    \begin{align*}
      S(f,\Ga)&\leq S(f,\Ga')=S(f,\Ga_1)+S(f,\Ga_2)\\
      &\leq\Var\bonj{f,[a,c]}+\Var\bonj{f,[c,b]} 
    \end{align*}
    Por lo tanto 
    $$\Var\bonj{f,[a,b]}\leq \Var\bonj{f,[a,c]}+\Var\bonj{f,[c,b]} $$
    Por otro lado si $\Ga_1$ es una partición de $[a,c]$ y $\Ga_2$ es una partición de $[c,b]$, entonces $\Ga=\Ga_1\cup\Ga_2$ es una partición de $[a,b]$. Es decir 
    $$S(f,\Ga_1)+S(f,\Ga_2)=S(f,\Ga)\leq\Var\bonj{f,[a,b]}.$$
\end{frame}

\begin{frame}{Subintervalos y la Variación}
  Así tenemos que 
\begin{align*}
  &\sup_{\Ga_1}S(f,\Ga_1)+S(f,\Ga_2)\leq\Var\bonj{f,[a,b]}\\
  \To&\sup_{\Ga_1}S(f,\Ga_1)+\sup_{\Ga_2}S(f,\Ga_2)\leq\Var\bonj{f,[a,b]}\\
  \To&\Var\bonj{f,[a,c]}+\Var\bonj{f,[c,b]}\leq\Var\bonj{f,[a,b]}.
\end{align*}
\begin{Lem}\label{lem:SubVar}
  Sea $f:[a,b]\to\bR$ de variación acotada. Entonces 
  $$\Var\bonj{f,[a,b]}=\Var\bonj{f,[a,c]}+\Var\bonj{f,[c,b]}.$$
\end{Lem}
\end{frame}

\subsection{Variación con Signo}

\begin{frame}
  Dado $x\in\bR$ definimos
  $$x^+=\begin{cases}
    x,\ x\geq 0\\
    0,\ x<0
  \end{cases}\quad
  x^-=\begin{cases}
    -x,\ x\leq 0\\
    0,\ x>0
  \end{cases}$$
  y así $x^+,x^-$ son positivas. Además $x^++x^-=|x|$ y $x^+-x^-=x$. Dada $f:[a,b]\to\bR$ y $\Ga=\set{x_0=a<x_1<\dots<x_n=b}$, una partición, consideremos
  \begin{align*}
    &P(f,\Ga)=\sum_{i=1}^{n-1}(f(t_i)-f(t_{i+1}))^+\\
    &N(f,\Ga)=\sum_{i=1}^{n-1}(f(t_i)-f(t_{i+1}))^-
  \end{align*}
\end{frame}

\begin{frame}{Observaciones}
  Note que 
  \begin{itemize}
    \item $S(f,\Ga)=N(f,\Ga)+P(f,\Ga)$.
    \item $-N(f,\Ga)+P(f,\Ga)\leq \sum_{i=1}^{n-1}(f(t_i)-f(t_{i+1}))\leq f(b)-f(a)$.
  \end{itemize}
  Definimos 
  $$P(f,[a,b])=\sup_\Ga P(f,\Ga),\ N(f,[a,b])=\sup_\Ga N(f,\Ga).$$
  Entonces
  $$S(f,\Ga)=N(f,\Ga)+P(f,\Ga)\leq N(f,[a,b])+P(f,[a,b])$$
  y por lo tanto 
  $$\Var(f,[a,b]))\leq N(f,[a,b])+P(f,[a,b]).$$
\end{frame}

\begin{frame}{El Otro Lado}
  $$f(a)+P(f,\Ga)=f(b)+N(f,\Ga),$$
  entonces
  $$f(a)+P(f,[a,b])=f(b)+N(f,[a,b]).$$
  Además 
  $$N(f,\Ga)+P(f,\Ga)=S(f,\Ga)\leq \Var(f,[a,b]).$$
  Entonces 
  \begin{align*}
    &f(a)-f(b)+2P(f,\Ga)=S(f,\Ga)\\
    \To&f(a)-f(b)+2P(f,[a,b])=\Var(f,[a,b])\\
    \To&N(f,[a,b])+P(f,[a,b])=\Var(f,[a,b]).
  \end{align*}
\end{frame}

\begin{frame}{Descomposición}
  \begin{Lem}\label{lem:descompVarPN}
    Sea $f:[a,b]\to\bR$ de variación acotada. Entonces vale que
    \begin{itemize}
      \item $P(f,[a,b])-N(f,[a,b])=f(b)-f(a)$.
      \item $P(f,[a,b])+N(f,[a,b])=\Var(f,[a,b])$.
    \end{itemize}
  \end{Lem}
Equivalentemente se tiene que 
\begin{itemize}
  \item $P(f,[a,b])=\half\set{\Var(f,[a,b])+f(b)-f(a)}$.
  \item $N(f,[a,b])=\half\set{\Var(f,[a,b])-f(b)+f(a)}$.
\end{itemize}
\end{frame}

\subsection{Caracterización}

\begin{frame}
  Note que 
  $$f(x)-f(a)=P(f,[a,x])-N(f,[a,x]).$$
  Como $P(f,[a,x])$ y $N(f,[a,x])$ son funciones crecientes de $x$ y positivas, entonces 
  $$f(x)=P(f,[a,x])+f(a)-N(f,[a,x]).$$
  \begin{Th}\label{thm:caracVA}
  Sea $f:[a,b]\to\bR$ una función. Entones $f$ es de variación acotada si y sólo si $f$ es la resta de dos funciones crecientes y positivas.
  \end{Th}
\end{frame}

\begin{frame}{Discontinuidades}
  \begin{Th}\label{thm:contablesDiscont}
    Sea $f:[a,b]\to\bR$ de variación acotada. Entonces $f$ tiene a lo sumo un número contable de discontinuidades. Toda discontinuidad es un salto o es removible. 
  \end{Th}
  Dado que $f=f_1-f_2$ con tales funciones positivas y crecientes, basta analizar las discontinuidades de funciones positivas y crecientes.
\end{frame}

\begin{frame}{Prueba del Teorema}
  \begin{itemize}
    \item Considere 
    $$D_n=\Set{x\in[a,b]:\ f(x_+)-f(x_-)\geq\frac1k}$$
    donde $f(x_+)=\lim_{y\to x_+}f(y)$ y $f(x_-)=\lim_{y\to x_-}f(y)$.
    \item De esta forma, si $x_0<x_1<\dots< x_m\in D_n$, existe $y_i,z_i$ tales que para $i=1,\dots,n$
    \begin{align*}
      &a=y_0\leq x_0<z_0=y_1<x_1,\\
      &x_{i-1}<z_{i-1}=y_i<x_i,\\
      &x_m\leq z_m=b.
    \end{align*}
  \end{itemize}
\end{frame}

\begin{frame}{Prueba del Teorema}
 \begin{itemize}
   \item Entonces
   \begin{align*}
     &=f(y_{i+1})-f(y_i)\\
     f(x_i^+)-f(x_i^-)&\leq f(z_i)-f(y_i) %%Hmmm
   \end{align*}
   \item Es decir 
   $$\frac mk\leq \sum\sucm f(z_i)-f(y_i)=f(b)-f(a)$$
   y así concluimos que $|D_n|<\infty$.
 \end{itemize}
\end{frame}

\begin{frame}
  \begin{Th}\label{thm:lastHoy}
    Sea $f:[a,b]\to\bR$ de variación acotada y continua. Entonces dado $M<\Var(f,[a,b])=V$, existe $\dl>0$ que satisface
    $$|\Ga|<\dl\To M< S(f,\Ga)\leq V.$$
  \end{Th}
  Sea $\mu>0$ tal que $M+\mu<V$. Sea $\Ga_1$ una partición que satisface $M+\mu<S(f,\Ga_1)$ con 
  $$\Ga_1=\set{\tilde{x}_0=a<\tilde{x}_1<\dots<\tilde{x}_k=b}.$$
  Al ser $f$ uniformemente continua, existe $\dl>0$ tal que 
  $$|x-y|<\dl\To |f(x)-f(y)|<\frac{\mu}{2(k+1)}.$$
\end{frame}

\begin{frame}{Continuamos la Prueba}
  \begin{itemize}
    \item Sea $\Ga$ una partición tal que 
    $$|\Ga|<\half\min\set{\dl,|\Ga_1|}.$$
    \item Si $\Ga=\set{x_0=a<x_1<\dots<x_n=x}$, tome $\Ga_2=\Ga\cup\Ga_1$. 
  \item Vamos a mostrar que $S(f,\Ga_2)-\mu<S(f,\Ga)$. 
  \item Entonces
  $$M<S(f,\Ga_1)-\mu\leq S(f,\Ga_2)-\mu\leq S(f,\Ga).$$
\end{itemize}
\end{frame}

\begin{frame}{Continuamos la Prueba}
  Sean $\set{\tilde{x}_i,\dots,\tilde{x}_{i_\l}}=\Ga_2\less\Ga$, entonces existe $1\leq j\leq m$ tal que 
  $$x_i<\tilde{x}_{i_k}<x_{i+1}.$$
  Note que $1\leq \l\leq k+1$ y además
\begin{align*}
  S(f,\Ga)=&\sum_{\set{1\leq j\leq m:\ \Ga_1\cap]x_{j-1},x_j[\neq\emptyset}}|f(x_j)-f(x_{j-1})|=\Sg''\\
  +&\sum_{\set{1\leq j\leq m:\ \Ga_1\cap]x_{j-1},x_j[=\emptyset}}|f(x_j)-f(x_{j-1})|=\Sg'
\end{align*}
\end{frame}

\begin{frame}{Terminamos la Prueba}
  Como $\Ga_2=\set{x_0<x_1<\dots<x_m}\cup\set{\tilde{x}_{i_1},\dots,\tilde{x}_{i_m}}$, se tiene que 
  \begin{align*}
    S(f,\Ga_2)&=\Sg'+\sum_{\set{1\leq j\leq m:\ \Ga_1\cap]x_{i-1},x_i[\neq\emptyset}}|f(x_j)-f(x_{i_k})|+|f(x_{i_k})-f(x_{j-1})|\\
    &\leq \Sg'+\frac{2\mu}{2(k+1)}(k+1)\\
    &\leq \Sg'+\mu\\
    &\leq S(f,\Ga)+\mu,
  \end{align*}
\end{frame}

\begin{frame}{Un Corolario}
  \begin{Cor}\label{cor:intsYVar}
Sea $f:[a,b]\to\bR$ tal que $f'$ es continua en $[a,b]$. Entonces 
\begin{itemize}
  \item $\Var(f,[a,b])=\int\limits_a^b|f'(x)|\dd x$.
  \item $P(f,[a,b])=\int\limits_a^b(f'(x))^+\dd x$.
  \item $N(f,[a,b])=\int\limits_a^b(f'(x))^-\dd x$
\end{itemize}
  \end{Cor}
  La prueba de este resultado es un \alert{ejercicio}.
\end{frame}
\section*{Resumen}

\subsection*{Qu\'e vimos hoy}

\begin{frame}{Resumen}

  % Keep the summary *very short*.
  \begin{itemize}
  \item La definición \ref{def:varAcot} de una función de variación acotada.
  \item El lema \ref{lem:propsVA} sobre propiedades de las funciones de variación acotada. 
  \item El lema \ref{lem:SubVar} sobre subintervalos y el comportamiento de la variación.
  \item El lema \ref{lem:descompVarPN} sobre la variación positiva y negativa.
  \item El teorema \ref{thm:caracVA} que caracteriza las funciones de variación acotada.
  \item El teorema \ref{thm:contablesDiscont} sobre las discontinuidades de una función de variación acotada. 
  \item El teorema \ref{thm:lastHoy} sobre las sumas. %Qué nombre?
  \item El corolario \ref{cor:intsYVar} sobre las integrales y la variación.
  \end{itemize}
  
\end{frame}

\subsection*{Ejercicios a trabajar}
\begin{frame}{Ejercicios}
    
  \begin{itemize}
    \item
      Lista 10
      \begin{itemize}
      \item La prueba del lema \ref{lem:propsVA} sobre funciones de variación acotada.
      \item La prueba del corolario \ref{cor:intsYVar} sobre integrales y variación.
      \end{itemize}
    \end{itemize}
  
\end{frame}


% All of the following is optional and typically not needed. 
\appendix
\section<presentation>*{\appendixname}
\subsection<presentation>*{Lectura adicional}

\begin{frame}[allowframebreaks]
  \frametitle<presentation>{Lecturas adicionales}
    
  \begin{thebibliography}{10}
    
  \beamertemplatebookbibitems
  % Start with overview books.

  \bibitem{CambroNotas}
    S.Cambronero.
    \newblock {\em Notas MA0505}.
    \newblock 20XX.

    \bibitem{NachoNotas}
    I.Rojas
    \newblock {\em Notas MA0505}.
    \newblock 2018.
 
  \end{thebibliography}
  
\end{frame}
%% 6 - 2:10:52 48
%% 7 - 1:17:44 44
%% 8 - 27:37 69
%% 9 - 1:07:47 86 + 59:38 40 approx 2 h c 7 min
%% 10 - 2:22:39 63
\end{document}


