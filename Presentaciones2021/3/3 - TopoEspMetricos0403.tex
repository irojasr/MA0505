\documentclass[utf8]{beamer}

\mode<presentation>
{
  \usetheme{Warsaw}
  \setbeamercovered{transparent}
}


\usepackage{amsfonts,mathtools,amssymb}
\usepackage[spanish]{babel}
\usepackage{times}
\usepackage[T1]{fontenc}

\title[MA0505]{MA0505 - An\'alisis I}
\subtitle{Lecci\'on III}

\author{Pedro M\'endez\inst{1}}

\institute[Universidad de Costa Rica] % (optional, but mostly needed)
{
  \inst{1}%
  Departmento de Matem\'atica Pura y Ciencias Actuariales\\
  Universidad de Costa Rica
  }

\date[I-2021] {Semestre I, 2021}

%%%%%%%%% === Theorems and suchlike === %%%%%%%%%%%%%%

\theoremstyle{plain}
\newtheorem{Th}{Teorema}               %%% Theorem 1.1.1
\newtheorem{Tmon}{Teoremón}
\newtheorem{Prop}{Proposición}         %%% Proposition 1.1.2
\newtheorem{Lem}{Lema}                 %%% Lemma 3
\newtheorem{Cor}{Corolario}            %%% Corollary 4

\theoremstyle{definition}
\newtheorem{Def}{Definición}           %%% Definition 5
\newtheorem{Ex}{Ejemplo}               %%% Example 6
\newtheorem{Ej}{Ejercicio}             %%% Ejercicio 7
\newtheorem{Hec}[Th]{Hecho}            %%% Hecho 1.1.8

\theoremstyle{remark}
\newtheorem{Rmk}[Th]{Observación}      %%%Remark 1.1.9
\newtheorem*{nonum-Rmk}{Observación}         %%% No number Fact
\newtheorem*{Notn}{Notaci\'on}        %% Notaciones
\newtheorem*{Warn}{Advertencia}       %% Advertencias

\numberwithin{equation}{section}

% Greek letters:

\newcommand{\al}{\alpha}                %% short for  \alpha
\newcommand{\bt}{\beta}                 %% short for  \beta
\newcommand{\Dl}{\Delta}                %% short for  \Delta
\newcommand{\dl}{\delta}                %% short for  \delta
\newcommand{\eps}{\varepsilon}          %% short for  \varepsilon
\newcommand{\Ga}{\Gamma}                %% short for  \Gamma
\newcommand{\ga}{\gamma}                %% short for  \gamma
\newcommand{\kp}{\kappa}                %% short for  \kappa
\newcommand{\La}{\Lambda}               %% short for  \Lambda
\newcommand{\la}{\lambda}               %% short for  \lambda
\newcommand{\Om}{\Omega}                %% short for  \Omega
\newcommand{\om}{\omega}                %% short for  \omega
\newcommand{\Sg}{\Sigma}                %% short for  \Sigma
\newcommand{\sg}{\sigma}                %% short for  \sigma
\newcommand{\Te}{\Theta}                %% short for  \Theta
\newcommand{\te}{\theta}                %% short for  \theta
\newcommand{\ups}{\upsilon}             %% short for  \upsilon
\newcommand{\vf}{\varphi}               %% short for  \varphi
\newcommand{\ze}{\zeta}                 %% short for  \zeta

%Boldface letters

\newcommand{\bC}{\mathbb{C}}    %%% números complejos
\newcommand{\bN}{\mathbb{N}}    %%% números naturales
\newcommand{\bP}{\mathbb{P}}        %% números enteros positivos
\newcommand{\bQ}{\mathbb{Q}}    %%% números racionales
\newcommand{\bR}{\mathbb{R}}    %%% números reales
\newcommand{\bS}{\mathbb{S}}    %%% esfera
\newcommand{\bZ}{\mathbb{Z}}    %%% números enteros

%Script letters:

\newcommand{\cA}{\mathcal{A}}           %% formas diferenciales
\newcommand{\cB}{\mathcal{B}}           %% una base vectorial
\newcommand{\cC}{\mathcal{C}}           %% otra base vectorial
\newcommand{\cD}{\mathcal{D}}           %% funciones de prueba
\newcommand{\cE}{\mathcal{E}}           %% un modulo proyectivo
\newcommand{\cF}{\mathcal{F}}           %% espacio de Fock
\newcommand{\cG}{\mathcal{G}}           %% funtor de Gelfand
\newcommand{\cH}{\mathcal{H}}           %% espacio de Hilbert
\newcommand{\cI}{\mathcal{I}}           %% un funtor de inclusion
\newcommand{\cJ}{\mathcal{J}}           %% otro funtor
\newcommand{\cK}{\mathcal{K}}           %% otro espacio de Hilbert
\newcommand{\cL}{\mathcal{L}}           %% operadores lineales
\newcommand{\cM}{\mathcal{M}}           %% multiplicadores
\newcommand{\cN}{\mathcal{N}}           %% funciones nulas
\newcommand{\cO}{\mathcal{O}}           %% funciones de crec-to lento
\newcommand{\cP}{\mathcal{P}}           %% una particion
\newcommand{\cR}{\mathcal{R}}           %% funciones representativas
\newcommand{\cQ}{\mathcal{Q}}           %% otra particion
\newcommand{\cS}{\mathcal{S}}           %% funciones de Schwartz
\newcommand{\cT}{\mathcal{T}}           %% una topologia
\newcommand{\cU}{\mathcal{U}}           %% cubrimiento abierto
\newcommand{\cV}{\mathcal{V}}           %% vecindarios
\newcommand{\cW}{\mathcal{W}}           %% grupo de Weyl


%Brackets

\newcommand{\bonj}[1]{\left\lbrack#1\right\rbrack}
\newcommand{\obonj}[1]{\left\rbrack#1\right\lbrack}
\newcommand{\rbonj}[1]{\left\rbrack#1\right\rbrack}
\newcommand{\lbonj}[1]{\left\lbrack#1\right\lbrack}
\newcommand{\snm}[1]{\|#1\|}           %small norma
\newcommand{\nm}[1]{\left\|#1\right\|} %norma pegadita
\newcommand{\pnm}[1]{\biggl|\biggl|#1\biggr|\biggr|}
\newcommand{\set}[1]{\{\,#1\,\}}    %% set notation
\newcommand{\floor}[1]{\lfloor#1\rfloor} %% mayor entero <= x
\newcommand{\Set}[1]{\biggl\{\,#1\,\biggr\}} %% set notation (large)


%Symbols 

\renewcommand{\geq}{\geqslant}          %% mayor o igual (variante)
\newcommand{\hookto}{\hookrightarrow}     %% inclusion arrow
\newcommand{\isom}{\simeq}              %% isomorfismo
\renewcommand{\l}{\ell}                   %% ele cursiva
\renewcommand{\leq}{\leqslant}          %% menor o igual (variante)
\newcommand{\less}{\setminus}           %% set difference
\newcommand{\To}{\Rightarrow}
\newcommand{\ov}{\overline}
\newcommand{\un}{\underline}
\newcommand{\del}{\partial}

\begin{document}

\begin{frame}
  \titlepage
\end{frame}

\begin{frame}{Outline}
  \tableofcontents
  % You might wish to add the option [pausesections]
\end{frame}


% Structuring a talk is a difficult task and the following structure
% may not be suitable. Here are some rules that apply for this
% solution: 

% - Exactly two or three sections (other than the summary).
% - At *most* three subsections per section.
% - Talk about 30s to 2min per frame. So there should be between about
%   15 and 30 frames, all told.

% - A conference audience is likely to know very little of what you
%   are going to talk about. So *simplify*!
% - In a 20min talk, getting the main ideas across is hard
%   enough. Leave out details, even if it means being less precise than
%   you think necessary.
% - If you omit details that are vital to the proof/implementation,
%   just say so once. Everybody will be happy with that.

\section{Topolog\'ia en espacios m\'etricos}

\subsection{Interiores}

\begin{frame}{El interior de un conjunto}%{Subtitles are optional.}
  % - A title should summarize the slide in an understandable fashion
  %   for anyone how does not follow everything on the slide itself.
Sea $(E,d)$ un espacio m\'etrico.
  \begin{Def}
    Dado $A\subseteq E$, decimos que $x_0\in A$ es un \alert{punto interior} de $A$ si existe $r>0$ tal que $B(x_0,r)\subseteq A$.\par
    Análogamente definimos el \alert{interior} de $A$ como el conjunto de los puntos interiores de $A$. Denotamos $A^o$ al interior.
  \end{Def}
  Sea $G\subseteq A$ con $G$ abierto. Si $x_0\in G$ existe $r>0$ tal que $$B(x_0,r)\subseteq G\subseteq A.$$ 
  Luego $x_0\in A^o$ y \alert{$G\subseteq A^o$}.
\end{frame}


\begin{frame}{De la Observación al Resultado}
    \begin{Lem}
        Si $G\subseteq A$ con $G$ abierto, entonces $G\subseteq A^o$. Es decir, \underline{$A^o$ es el abierto m\'as grande contenido en $A$.}
    \end{Lem}
    Por lo tanto, al ser $A_1^o\cap A_2^o$ abierto y $A_1^o\cap A_2^o\subseteq A_1\cap A_2$, entonces
    $$A_1^o\cap A_2^o\subseteq(A_1\cap A_2)^o.$$
    \begin{Ej}\label{ej:interiorAbierto}
        Muestre que $A^o$ es un abierto.
    \end{Ej}
\end{frame}

\begin{frame}
    Si ahora $x\in(A_1\cap A_2)^o$, entonces 
    $$B(x,r)\subseteq A_1\cap A_2, r>0.$$
    As\'i $B(x,r)\subseteq A_1,A_2$ y por tanto $x\in A_1^o$ y $x\in A_2^o$. Concluimos 
    $$(A_1\cap A_2)^o\subseteq A_1^o\cap A_2^o.$$
    \begin{Lem}
        $A_1^o\cap A_2^o=(A_1\cap A_2)^o$.
    \end{Lem}
    \begin{Ej}\label{ej:interiorRespetaSubconj}
        Si $A\subseteq B$ entonces $A^o\subseteq B^o$.
    \end{Ej}
    \alert{Use el lema $G\subseteq A^o$ para probar este lema y el ejercicio de una forma alternativa.}
\end{frame}

\begin{frame}{La Distancia entre Conjuntos}
    \begin{Def}
        Dados $A,B\subseteq E$ definimos su distancia como
        $$d(A,B) = \inf\set{d(x,y): (x,y)\in A\times B}.$$
      \end{Def}
      Si $x\in E$ y $A\subseteq E$ con $d(x,A)=0$, entonces existe $x_n\in A$ tal que $d(x,x_n)<\frac1n$ para todo $n\in\bN$.\par 
      Es decir $(x_n)_{n=1}^\infty\subseteq A$ es una sucesión tal que $x_n\to x$ cuando $n\to \infty$.
      
\end{frame}

\subsection{Clausuras}

\begin{frame}{La Clausura de un Conjunto}
    \begin{Lem}
        Dados $x\in E$ y $A\subseteq E$ son equivalentes:
        \begin{enumerate}
            \item $d(x,A)=0$.
            \item Existe $(x_n)_{n=1}^\infty\subseteq A$ tal que $x_n\to x$ cuando $n\to \infty$.
        \end{enumerate}
    \end{Lem}
    Dado $A\subseteq E$, definimos la \alert{clausura} de $A$ como 
    $$\ov A=\set{x\in E:\ d(x,A)=0}.$$
    Inmediatamente vemos que $A\subseteq \ov A$.\par 
    A los elementos de $\ov A$, les llamaremos \alert{puntos de adherencia} de $A$.
\end{frame}

\begin{frame}{Propiedades de la Clausura}
    Sea $z\in \ov A$, existe $a_r\in A$ tal que $d(a_r,z)<r$ para $r>0$ pues $d(z,A)=0$. Entonces 
    $$a_r\in B(z,r)\cap A.$$
    Si $F\subseteq E$ con $A\subseteq F$, como 
    $$\forall x\in E,\ \inf_{f\in F} d(x,f)\leq \inf_{a\in A} d(x,a),$$
    entonces $d(x,F)\leq d(x,A)$. As\'i $d(x,A)=0$ implica $d(x,F)=0$ por lo que
    $$\ov A\subseteq \ov F.$$
\end{frame}

\begin{frame}{Trabajamos con la Clausura}
    Si $A$ es cerrado y $x\not\in A$, $E\less A$ es abierto y 
    $$B(x,r)\subseteq E\less A,\ r>0\To B(x,r)\cap A=\emptyset.$$
    Luego si $a\in A$, $d(x,a)\geq r$ y as\'i $x\not\in\ov A$. Por lo tanto, si $A$ es cerrado entonces $A=\ov A$.
    \begin{center}
        \emph{Es $\ov A$ un cerrado?}
    \end{center}
\end{frame}

\begin{frame}{Respondamos la pregunta\dots}
    \begin{columns}
    \begin{column}{0.5\textwidth}
        \begin{itemize}
            \item  Si $z\in\ov A$, entonces 
            $$d(z,A)=r>0.$$ 
            \pause
            \item As\'i para $a\in A$, $d(z,a)\geq r$.
            \pause
            \item  Si $w\in B\left(z,\frac r2\right)$, entonces $d(w,A)\geq \frac r2$. 
        \end{itemize}
       
     \end{column}
     \begin{column}{0.5\textwidth}  %%dibujo
        
     \end{column}
     \end{columns}
\end{frame}

\begin{frame}
    Mostremos que  
    $$w\in B\left(z,\frac r2\right)\To d(w,A)\geq \frac r2 $$
    Sabemos que\par
    $$r\leq d(z,a)\leq d(z,w)+d(w,a)<\frac r2+d(w,a)$$
    y as\'i $\frac r2<d(w,a)$.\par 
    Por tanto 
    $$B\left(z,\frac r2\right)\subseteq E\less\ov A$$
    concluimos que $\ov A$ es cerrado.
\end{frame}

\begin{frame}{Ampliemos las propiedades}
    \begin{Lem}
        Dado $A\subseteq E$, $A$ es cerrado si y s\'olo si $A=\ov A$.
    \end{Lem}
    \begin{Ej}\label{ej:abiertoIffInterior}
        Dado $B\subseteq E$, $B$ es abierto si y s\'olo si $B=B^o$. 
    \end{Ej}
    \begin{Ej}\label{ej:propsClausura}
        Dados $A,B\subseteq E$, vale que 
        \begin{enumerate}
            \item $\ov{A\cup B}=\ov A\cup\ov B$.
            \item $\ov{A\cap B}\subseteq \ov A\cap \ov B$.
        \end{enumerate}
    \end{Ej}
\end{frame}

\subsection{Acumulaci\'on y Frontera}
\begin{frame}{Puntos de Acumulaci\'on}
    \begin{Def}
        Un punto $x_0\in E$ es un \alert{punto de acumulaci\'on} de $A$ si 
        $$\left(B(x_0,r)\setminus\set{x_0}\right)\cap A \neq \emptyset$$
         para todo $r>0$. 
    \end{Def}

    Es decir, $B(x,r)$ contiene puntos de $A$ distintos de $x$.
   \begin{Ej}\label{ej:ptosAccumComoSuccs}
        Caracterice los puntos de acumulación en términos de sucesiones.
      \end{Ej}

\end{frame}

\begin{frame}{Puntos Frontera}
    \begin{Def}
        Dado $A\subseteq E$ y $x\in E$, diremos que $x$ es un \alert{punto frontera} de $A$ si para $r>0$
        $$B(x,r)\cap A\neq\emptyset,\ B(x,r)\cap A^c\neq\emptyset.$$
        
    \end{Def}
    Es decir $d(x,A)=d(x,A^c)=0$. Denotamos
    $$\del A=\set{\text{puntos frontera}}=\set{x:\ d(x,A)=d(x,A^c)=0}.$$
\end{frame}

\begin{frame}{Ejemplos}
    \begin{enumerate}
        \item Si $A=\Set{\frac1n:\ n\geq 1}$ tenemos que $\ov A=A\cup\set0$. El punto 0 es de acumulaci\'on y es el \'unico pues $B\left(\frac 1n,\frac{1}{(n+1)^2}\right)\cap A=\set{\frac1n}$ si $n\geq 1$.
        \item Sean $A=\Set{n+\frac1n}$, $B=\bN$. Ambos son cerrados pero 
        $$d\left(n+\frac1n,n\right)=\frac{1}{n}\to 0$$
        por lo que $d(A,B)=0$.
    \end{enumerate}
    
\end{frame}

\begin{frame}
    Si $A,B$ son como antes, entonces 
    $$\bR\less B=\bigcup_{n\in\bZ}]n,n+1[$$
    es un conjunto abierto. 
    \begin{center}
        \emph{Qu\'e es $\bR\less A$?}
    \end{center}
\end{frame}

%%%Ended 1:31:11 37 faltan pgs 9 y 10

\subsection{Vecindarios}

\begin{frame}{El Vecindario de un Punto} %(no del Chavo)}
Dado $x\in E$, diremos que $V$ es un \alert{vecindario} de $x$ si existe $r>0$ tal que $B(x,r)\subseteq V$.\par 
\begin{itemize}
    \item Si $V$ es un abierto, $V$ es un vecindario de todos sus puntos.
\end{itemize}

\end{frame}

\begin{frame}{El de un conjunto}
    Sea $A\subseteq E$, definimos 
    $$V_r(A)=\set{x\in E:\ d(x,A)<r}.$$
    \begin{itemize}
        \item Si $a\in A$, $d(x,A)\leq d(x,a)$. 
        \item Y si $x\in B(a,r)$, vale
        $$d(x,A)\leq d(x,a)<r.$$
    \end{itemize}
    Por tanto $B(a,r)\subseteq V_r(A)$.
    \begin{Ej}\label{ej:bolasEnVecindarios}
        Para $x\in V_r(A)$ y $r_1<r-d(x,A)$, muestre que 
        $$B(x,r_1)\subseteq V_r(A).$$
    \end{Ej}
    
\end{frame}

\begin{frame}{Propiedades del Vecindario}
    \begin{Lem}
        Si $A\subseteq E$, $V_r(A)$ es abierto para $r>0$. 
    \end{Lem}

    ahora si $x\in \bigcap_rV_r(A)$, entonces $d(x,A)<r$ para $r>0$. Es decir, $d(x,A)=0$. Por lo tanto $\ov A=\bigcap_rV_r(A)$.\par 
    \emph{Es decir, la clausira de un conjunto se puede escribir como intersecci\'on de abiertos.}
\end{frame}

\begin{frame}{Densidad} %en este caso no es masa entre volumen
    \begin{Def}
        A un conjunto $A\subseteq E$ le llamaremos \alert{denso} en $E$ si $\ov A=E$.\par
        A un espacio m\'etrico $(E,d)$ lo llamamos \alert{separable} si pose\'e un conjunto denso y numerable.
    \end{Def}
    
    Por ejemplo, $(\bR,|\cdot|)$ es separable pues $\bQ$ es denso en $\bR$. En general $(\bR^d,\nm{\cdot})$ es separable pues $\bQ^d$ es denso en $\bR^d$.
\end{frame}
\section*{Resumen}

\subsection*{Qu\'e vimos hoy}
\begin{frame}{Resumen}

  % Keep the summary *very short*.
  \begin{itemize}
  \item Definici\'on de interior y propiedades.
  \item Definición de distancia entre conjuntos.
  \item Definici\'on de clausura y la clausura en efecto es un cerrado.
  \item Propiedades de clausura.
  \item Definición de puntos de acumulaci\'on y puntos frontera.
  \item Ejemplos de lo mencionado anteriormente.
  \end{itemize}
  
\end{frame}

\subsection*{Ejercicios a trabajar}
\begin{frame}{Ejercicios}
    
  \begin{itemize}
    \item
      Lista 3
      \begin{itemize}
      \item El interior en efecto es un abierto. \ref{ej:interiorAbierto} 
      \item Interior respeta subconjuntos. \ref{ej:interiorRespetaSubconj}
      \item Pruebas alternativas usando maximalidad del interior. $(\ast)$
      \item Caracterizaci\'on de abiertos por interiores. \ref{ej:abiertoIffInterior}
      \item Uniones e intersecciones de clausuras. \ref{ej:propsClausura}
      \item Puntos de acumulación como sucesiones. \ref{ej:ptosAccumComoSuccs}
      \item Las bolas est\'an dentro de los vecindarios. \ref{ej:bolasEnVecindarios}
      \end{itemize}
    \end{itemize}
  
\end{frame}


% All of the following is optional and typically not needed. 
\appendix
\section<presentation>*{\appendixname}
\subsection<presentation>*{Lectura adicional}

\begin{frame}[allowframebreaks]
  \frametitle<presentation>{Lecturas adicionales}
    
  \begin{thebibliography}{10}
    
  \beamertemplatebookbibitems
  % Start with overview books.

  \bibitem{CambroNotas}
    S.Cambronero.
    \newblock {\em Notas MA0505}.
    \newblock 20XX.

    \bibitem{NachoNotas}
    I.Rojas
    \newblock {\em Notas MA0505}.
    \newblock 2018.
 
  \end{thebibliography}
  
\end{frame}

\end{document}


