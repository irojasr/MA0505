\documentclass[utf8]{beamer}

\mode<presentation>
{
  \usetheme{Warsaw}
  \setbeamercovered{transparent}
}


\usepackage{amsfonts,mathtools,amssymb}
\usepackage[spanish]{babel}
\usepackage{times}
\usepackage[T1]{fontenc}
\usepackage[shortlabels]{enumitem}
\usepackage{tikz}
\usepackage{physics}

\title[MA0505]{MA0505 - An\'alisis I}
\subtitle{Lecci\'on XI: La Integral de Riemann-Stieltjes}

\author{Pedro M\'endez\inst{1}}

\institute[Universidad de Costa Rica] % (optional, but mostly needed)
{
  \inst{1}%
  Departmento de Matem\'atica Pura y Ciencias Actuariales\\
  Universidad de Costa Rica
  }

\date[I-2021] {Semestre I, 2021}

%%%%%%%%% === Theorems and suchlike === %%%%%%%%%%%%%%

\theoremstyle{plain}
\newtheorem{Th}{Teorema}               %%% Theorem 1.1.1
\newtheorem{Tmon}{Teoremón}
\newtheorem{Prop}{Proposición}         %%% Proposition 1.1.2
\newtheorem{Lem}{Lema}                 %%% Lemma 3
\newtheorem{Cor}{Corolario}            %%% Corollary 4

\theoremstyle{definition}
\newtheorem{Def}{Definición}           %%% Definition 5
\newtheorem{Ex}{Ejemplo}               %%% Example 6
\newtheorem{Ej}{Ejercicio}             %%% Ejercicio 7
\newtheorem{Hec}[Th]{Hecho}            %%% Hecho 1.1.8

\theoremstyle{remark}
\newtheorem{Rmk}[Th]{Observación}      %%%Remark 1.1.9
\newtheorem*{nonum-Rmk}{Observación}         %%% No number Fact
\newtheorem*{Notn}{Notaci\'on}        %% Notaciones
\newtheorem*{Warn}{Advertencia}       %% Advertencias

\numberwithin{equation}{section}

%% Accomodations 

\makeatletter
\def\moverlay{\mathpalette\mov@rlay}
\def\mov@rlay#1#2{\leavevmode\vtop{%
   \baselineskip\z@skip \lineskiplimit-\maxdimen
   \ialign{\hfil$\m@th#1##$\hfil\cr#2\crcr}}}
\newcommand{\charfusion}[3][\mathord]{
    #1{\ifx#1\mathop\vphantom{#2}\fi
        \mathpalette\mov@rlay{#2\cr#3}
      }
    \ifx#1\mathop\expandafter\displaylimits\fi}
\makeatother

% Greek letters:

\newcommand{\al}{\alpha}                %% short for  \alpha
\newcommand{\bt}{\beta}                 %% short for  \beta
\newcommand{\Dl}{\Delta}                %% short for  \Delta
\newcommand{\dl}{\delta}                %% short for  \delta
\newcommand{\eps}{\varepsilon}          %% short for  \varepsilon
\newcommand{\Ga}{\Gamma}                %% short for  \Gamma
\newcommand{\ga}{\gamma}                %% short for  \gamma
\newcommand{\La}{\Lambda}               %% short for  \Lambda
\newcommand{\la}{\lambda}               %% short for  \lambda
\newcommand{\Om}{\Omega}                %% short for  \Omega
\newcommand{\om}{\omega}                %% short for  \omega
\newcommand{\Sg}{\Sigma}                %% short for  \Sigma
\newcommand{\sg}{\sigma}                %% short for  \sigma
\newcommand{\te}{\theta}                %% short for  \theta
\newcommand{\vf}{\varphi}               %% short for  \varphi
\newcommand{\ze}{\zeta}                 %% short for  \zeta

%Boldface letters

\newcommand{\bC}{\mathbb{C}}    %%% números complejos
\newcommand{\bN}{\mathbb{N}}    %%% números naturales
\newcommand{\bP}{\mathbb{P}}        %% números enteros positivos
\newcommand{\bQ}{\mathbb{Q}}    %%% números racionales
\newcommand{\bR}{\mathbb{R}}    %%% números reales
\newcommand{\bS}{\mathbb{S}}    %%% esfera
\newcommand{\bZ}{\mathbb{Z}}    %%% números enteros

%Script letters:

\newcommand{\cA}{\mathcal{A}}           %% formas diferenciales
\newcommand{\cB}{\mathcal{B}}           %% una base vectorial
\newcommand{\cC}{\mathcal{C}}           %% otra base vectorial
\newcommand{\cF}{\mathcal{F}}           %% espacio de Fock
\newcommand{\cL}{\mathcal{L}}           %% operadores lineales
\newcommand{\cM}{\mathcal{M}}           %% multiplicadores
\newcommand{\cN}{\mathcal{N}}           %% funciones nulas
\newcommand{\cP}{\mathcal{P}}           %% una particion
\newcommand{\cR}{\mathcal{R}}           %% funciones representativas
\newcommand{\cS}{\mathcal{S}}           %% funciones de Schwartz


%Brackets

\newcommand{\bonj}[1]{\left\lbrack#1\right\rbrack}
\newcommand{\obonj}[1]{\left\rbrack#1\right\lbrack}
\newcommand{\rbonj}[1]{\left\rbrack#1\right\rbrack}
\newcommand{\lbonj}[1]{\left\lbrack#1\right\lbrack}
\newcommand{\snm}[1]{\|#1\|}           %small norma
\newcommand{\nm}[1]{\left\|#1\right\|} %norma pegadita
\newcommand{\pnm}[1]{\biggl|\biggl|#1\biggr|\biggr|}
\newcommand{\set}[1]{\{\,#1\,\}}    %% set notation
\newcommand{\floor}[1]{\lfloor#1\rfloor} %% mayor entero <= x
\newcommand{\Set}[1]{\biggl\{\,#1\,\biggr\}} %% set notation (large)
\newcommand\quot[2]{
        \mathchoice
            {% \displaystyle
                \text{\raise1ex\hbox{$#1$}\Big/\lower1ex\hbox{$#2$}}%
            }
            {% \textstyle
                {^{ #1}/_{ #2}}
            }
            {% \scriptstyle
                {^{ #1}/_{ #2}}
            }
            {% \scriptscriptstyle
                {^{ #1}/_{ #2}}
            }
    }
\newcommand*\squot[2]{{^{ #1}/_{ #2}}}%%%small quotient

%Symbols 

\newcommand{\x}{\times}
\renewcommand{\geq}{\geqslant}          %% mayor o igual (variante)
\newcommand{\hookto}{\hookrightarrow}     %% inclusion arrow
\newcommand{\isom}{\simeq}              %% isomorfismo
\renewcommand{\l}{\ell}                   %% ele cursiva
\renewcommand{\leq}{\leqslant}          %% menor o igual (variante)
\newcommand{\less}{\setminus}           %% set difference
\newcommand{\To}{\Rightarrow}
\newcommand{\ov}{\overline}
\newcommand{\un}{\underline}
\newcommand{\del}{\partial}

%%% Small fractions in displays:

\newcommand{\half}{{\mathchoice{\nhalf}{\thalf}{\shalf}{\shalf}}} %%display text script script^2
\newcommand{\happi}{{\tfrac{\pi}{2}}} %% small fraction  \pi/2
\newcommand{\quarter}{\tfrac{1}{4}} %% small fraction  1/4
\newcommand{\nhalf}{\frac{1}{2}}
\newcommand{\shalf}{{\scriptstyle\frac{1}{2}}} %% tiny fraction 1/2
\newcommand{\thalf}{{\tfrac{1}{2}}} %% small fraction  1/2
\newcommand{\third}{\tfrac{1}{3}}   %% small fraction  1/3 %Hay que renew porque mathabx toma second y third como x'' y x''' por ejemplo

\newcommand{\ihalf}{{\tfrac{i}{2}}} %% small fraction  i/2

\newcommand{\suci}{_{i=1}^\infty} %% diminutivo
\newcommand{\suck}{_{k=1}^\infty} %% diminutivo
\newcommand{\sucm}{_{m=1}^\infty} %% diminutivo
\newcommand{\sucn}{_{n=1}^\infty} %% diminutivo

\newcommand*{\Cdot}{{\raisebox{-0.25ex}{\scalebox{1.5}{$\cdot$}}}}      %% cdot más grande
\renewcommand{\.}{\Cdot}                %% producto escalar
\newcommand{\cupdot}{\charfusion[\mathbin]{\cup}{\.}}
\newcommand{\bigcupdot}{\charfusion[\mathbin]{\bigcup}{\.}}
\DeclareMathOperator{\Var}{Var}     %%%variance

\begin{document}

\begin{frame}
  \titlepage
\end{frame}

\begin{frame}{Agenda}
  \tableofcontents
  % You might wish to add the option [pausesections]
\end{frame}


% Structuring a talk is a difficult task and the following structure
% may not be suitable. Here are some rules that apply for this
% solution: 

% - Exactly two or three sections (other than the summary).
% - At *most* three subsections per section.
% - Talk about 30s to 2min per frame. So there should be between about
%   15 and 30 frames, all told.

% - A conference audience is likely to know very little of what you
%   are going to talk about. So *simplify*!
% - In a 20min talk, getting the main ideas across is hard
%   enough. Leave out details, even if it means being less precise than
%   you think necessary.
% - If you omit details that are vital to the proof/implementation,
%   just say so once. Everybody will be happy with that.

\section{La Nueva Integral}

\subsection{Sumas de Riemann y Stieltjes}

\begin{frame}{Las Sumas de Riemann y Stieltjes}
  Sea $\phi:[a,b]\to\bR$. Dados $f:[a,b]\to\bR$ acotada, una partición $\Ga=\set{x_0=a<x_1<\dots<x_n=b}$ de $[a,b]$ y $x_{i-1}\leq\xi_i\leq x_i$ para $1\leq i\leq n$, definimos la suma de Riemann-Stieltjes como 
  $$R(f,\Ga,\phi)=\sum_{i=1}^nf(\xi_i)[\phi(x_i)-\phi(x_{i-1})].$$
\end{frame}

\begin{frame}{La Definición de la Integral}
  \begin{Def}\label{def:rsIntegrable}
    Decimos que $f$ es Riemann-Stieltjes integrable con respecto a $\phi$ en $[a,b]$ si existe $I\in\bR$ que satisface
    $$\forall\eps>0\ \exists\dl>0\ (|\Ga|<\dl\To|R(f,\Ga,\phi)-I|<\eps)$$
    para cualquier escogencia de puntos $\set{\xi_1,\dots,\xi_n}$.\par Denotamos $I=\int\limits_a^bf\dd\phi$.
  \end{Def}
\end{frame}

\subsection{Sumas Superiores e Inferiores}

\begin{frame}
  Sea $f:[a,b]\to\bR$ acotada y $\Ga=\set{x_0=a<x_1<\dots<x_n=b}$, definimos 
  $$m_i=\inf_{x_{i-1}\leq\xi_i\leq x_i} f(\xi),\quad M_i=\sup_{x_{i-1}\leq\xi_i\leq x_i} f(\xi)$$
  para $1\leq i\leq n$. De forma similar a la integral de Riemann, definimos
  \begin{itemize}
    \item $L(f,\Ga,\phi)=\sum_{i=1}^nm_i[\phi(x_i)-\phi(x_{i-1})].$
    \item $U(f,\Ga,\phi)=\sum_{i=1}^nM_i[\phi(x_i)-\phi(x_{i-1})].$
  \end{itemize}
  Claramente si $f$ es creciente entonces
  $$L(f,\Ga,\phi)\leq R(f,\Ga,\phi)\leq U(f,\Ga,\phi)$$
  y caso que $\phi(x)=x$ para $x\in [a,b]$ se obtiene la integral de Riemann ordinaria.
\end{frame}

\begin{frame}{El Criterio de Cauchy}
  \begin{Ej}\label{ej:CauchyRS}
    La función $f$ es Riemann-Stieltjes integrable respecto a $\phi$ si y sólo si dado $\eps>0$, existe un $\dl>0$ tal que si $|\Ga|<\dl$ y $|\Ga'|<\dl$, vale entonces que
    $$|R(f,\Ga,\phi)-(f,\Ga',\phi)|<\eps.$$
  \end{Ej}
\end{frame}

\begin{frame}
  \begin{Lem}\label{lem:NotRSIntCuando}
Asuma que existe $z_0\in [a,b]$ tal que $f$ y $\phi$ son discontinuas en $z_0$. Entonces $f$ no es Riemann-Stieltjes integrable respecto a $\phi$.
    \end{Lem}\begin{itemize}
      \item  Supongamos que 
    $$\phi(z_0)\neq\lim_{x\to z_0^+}\phi(x)=\lim_{x\to z_0^-}\phi(x).$$
    \item Entonces existe un $\eps>0$ tal que para $\dl>0$, existen $\ov x_\dl,\ov y_\dl$ que satisfacen 
    \begin{itemize}
      \item $\ov x_\dl<z_0<\ov y_\dl$.
      \item $|\phi(z_0)-\phi(\ov x_\dl)|\geq\sqrt{\eps}$.
      \item $|\phi(z_0)-\phi(\ov y_\dl)|\geq\sqrt{\eps}$.
    \end{itemize}
    \end{itemize}
\end{frame}

\begin{frame}{Continuamos la Prueba}
  Además, existe un $\xi_\dl$ tal que 
  \begin{itemize}
    \item $|\xi_\dl-z_0|<\frac{\dl_1}{2}$.
    \item $|f(\xi_\dl)-f(z_0)|\geq\sqrt{\eps}$.
  \end{itemize}
  Donde 
  $$\dl_1=\min\Set{\half|z_0-\ov x_\dl|,\half|z_0-\ov y_\dl|}.$$
  Supogamos que $\xi_\dl>z_0$ y tomemos $\Ga$ una partición tal que 
  $$z_0=x_{i_0}<\xi_\dl<x_{i_0+1}=\ov y_\dl$$
  con $|\Ga|<\dl$. ($\Ga=\set{x_0=a<x_1<\dots<x_n=b}$).
\end{frame}

\begin{frame}{Continuamos la Prueba}
Considere ahora dos suma de R-S con $\Ga$ de partición y los mismos $\set{\xi_i}$ para $i=i_0$.\par 
En el intervalo $[x_{i_0},x_{i_0+1}]$ vale que
 \begin{itemize}
  \item $R(f,\Ga,\phi)$ es la suma de R-S con punto $\xi_{i_0}=\xi_\dl$.
  \item $R'(f,\Ga,\phi)$ es la suma de R-S con punto $\xi_{i_0}=z_0$.
\end{itemize}
Así 
$$|R(f,\Ga,\phi)-(f,\Ga',\phi)|=|f(\xi_\dl)-f(z_0)||\phi(x_{i_0})-\phi(x_{i_0+1})|\geq\eps.$$
En el caso que $z_0>\xi_\dl$, la prueba es similar. Y el caso en el que $\lim_{x\to z_0^+}\phi(x)\neq\lim_{x\to z_0^-}\phi(x)$ será un \alert{ejercicio}
 \end{frame}

 \begin{frame}{Como las Sumas de Darboux}
   Analizamos $U$ y $L$. El resultado a continuación tiene una análogo en las sumas de Darboux.
   \begin{Lem}\label{lem:PropsSumasInfSup}
     Sea $f:[a,b]\to\bR$ acotada y $\phi:[a,b]\to\bR$ creciente.
     \begin{enumerate}
       \item Si $\Ga_1\subseteq\Ga_2$, entonces 
       $$L(f,\Ga_1,\phi)\leq L(f,\Ga_2,\phi),\quad U(f,\Ga_1,\phi)\leq U(f,\Ga_2,\phi).$$
       \item Si $\Ga_1,\Ga_2$ son dos particiones cualesquiera 
       $$L(f,\Ga_1,\phi)\leq U(f,\Ga_2,\phi).$$
     \end{enumerate}
   \end{Lem}
 \end{frame}

 \begin{frame}{Primer Inciso}
   Sean 
   \begin{gather*}
    \Ga_1=\set{x_0=a<x_1<\dots<x_n=b},\\
    \Ga_2=\set{y_0=a<y_1<\dots<y_m=b}
   \end{gather*}
   con $n\leq m$. Dado $1\leq i\leq n$ asuma que existe $y_i$ tal que $x_i<y_i<x_{i+1}$. Entonces
   $$\sup_{[x_i,y_i]}f,\sup_{[y_i,x_{i+1}]}f\leq \sup_{[x_i,x_{i+1}}f$$
   y por lo tanto
   \begin{align*}
    &\sup_{[x_i,y_i]}f(\phi(y_i)-\phi(x_i))+\sup_{[y_i,x_{i+1}]}f(\phi(x_{i+1})-\phi(y_i))\\
    \leq&\hspace{-6pt}\sup_{[x_i,x_{i+1}]}\hspace{-6pt}f(\phi(y_i)-\phi(x_i)+\phi(x_{i+1})-\phi(y_i))
    =\hspace{-6pt}\sup_{[x_i,x_{i+1}]}\hspace{-6pt}f(\phi(x_{i+1})-\phi(x_i)).
   \end{align*}
 \end{frame}

 \begin{frame}{Terminamos el Inciso}
   Usando un argumento similar se prueba que si 
   $$y_{j-1}=x_{i-1}<y_j<y_{j+1}<\dots<x_i=y_{j+m},$$
   entonces
   $$\sum_{j=1}^m\sup_{[y_{j-1},y_j]}f(\phi(y_j)-\phi(y_{j-1}))\leq\sup_{[x_{i-1},x_i]}f(\phi(x_{i})-\phi(x_{i-1})).$$
   Por lo tanto 
   $$U(f,\Ga_1,\phi)\geq U(f,\Ga_2,\phi).$$
   De forma similar se prueba la desigualdad para $L$.
 \end{frame}

 \begin{frame}{El Segundo Inciso}
Si asumimos el primer inciso y tomamos $\Ga_1,\Ga_2$ particiones con $\Ga=\Ga_1\cup \Ga_2$, entonces
$$L(f,\Ga_1,\phi)\leq L(f,\Ga,\phi)\leq U(f,\Ga,\phi)\leq U(f,\Ga_2,\phi).$$ 
 \end{frame}

 \section{Propiedades de la Integral}

 \subsection{La Integral es Acotada}
 \begin{frame}
   Utilizando la definición es fácil probar que si $\int\limits_a^bf\dd\phi_1$ y $\int\limits_a^bf\dd\phi_2$ existen, y vale $\phi=\phi_1-\phi_2$, entonces $\int\limits_a^bf\dd\phi$ existe y además 
   $$\int\limits_a^bf\dd\phi=\int\limits_a^bf\dd\phi_1-\int\limits_a^bf\dd\phi_2.$$
   En el caso que $\phi$ sea de variación acotada, podemos reducir el problema de que $f$ sea R-S integrable respecto a $\phi$ al caso que $\phi$ sea creciente y positiva.
 \end{frame}

 \begin{frame}
   \begin{Th}
Sea $f:[a,b]\to\bR$ continua y $\phi:[a,b]\to\bR$ de variación acotada en $[a,b]$. Entonces $\int\limits_a^bf\dd\phi$ existe y Además
$$\left|\int\limits_a^bf\dd\phi\right|\leq\sup_{[a,b]}|f|\Var(\phi,[a,b]).$$
   \end{Th}
   Basta probar el resultado en el caso que $\phi$ es creciente. Entonces, dado $\eps>0$ existe $\dl_1>0$ tal que
   $$|x-y|<\dl_1\To |f(x)-f(y)|<\frac{\eps}{2(\phi(b)-\phi(a))}.$$
 \end{frame}

 \begin{frame}{Primer Paso}
   Si $|\Ga|<\dl_1$, vamos a probar que 
   $$U(f,\Ga,\phi)-L(f,\Ga,\phi)<\frac\eps 2.$$
   Sea $\Ga=\set{x_0=a<x_1<\dots<x_n=b}$, como $f$ es continua, existen $\xi_i,\eta_i$ que satisfacen 
   \begin{itemize}
    \item $f(\xi_i)=\sup_{[x_i,x_{i+1}]}f=M_i$ para $0\leq i\leq n-1$.
    \item $f(\eta_i)=\inf_{[x_i,x_{i+1}]}f=m_i$ para $0\leq i\leq n-1$.
   \end{itemize}
 \end{frame}

 \begin{frame}{Primer Paso}
   Tenemos que 
   \begin{gather*}
     U(f,\Ga,\phi)-L(f,\Ga,\phi)\\
     =\sum_{i=0}^{n-1}(M_i-m_i)(\phi(x_{i+1})-\phi(x_i))\\
     =\sum_{i=0}^{n-1}(f(\xi_i)-f(\eta_i))(\phi(x_{i+1})-\phi(x_i))\\
     \leq \frac{\eps}{2}\frac{1}{\phi(b)-\phi(a)}(\phi(b)-\phi(a))
   \end{gather*}
   puesto que 
   $$|\xi_i-\eta_i|\leq |x_{i+1}-x_i|\leq |\Ga|\leq \dl_1$$
 \end{frame}

 \begin{frame}{Segundo Paso}
  Vamos a mostrar que existe $I$ tal que para $\eta>0$, existe un $\dl>0$ que satisface
  $$|\Ga|<\dl\To |U(f,\Ga,\phi)-I|<\frac\eps 2.$$
  Para ese efecto, tomemos $\set{\Ga_k}\suck$ una sucesión de particiones tal que $\lim_{k\to\infty}|\Ga_k|=0$.\par 
  Considere $\Ga_k'=\bigcup_{j=1}^k\Ga_j$, entonces vale
  \begin{itemize}
    \item $\Ga_k'\subseteq\Ga_{k+1}'$, y
    \item $\Ga_k\subseteq\Ga_k'$.
  \end{itemize}  
   \end{frame}

\begin{frame}{Segundo Paso}
    Tomemos $U=\inf_{1\leq k}U(f,\Ga_k,\phi)$. Dado que $|\Ga_k|<\dl_1$ para $k\geq k_1$, 
    $$U(f,\Ga_k,\phi)\leq L(f,\Ga_k,\phi)+\frac\eps2\leq U(f,\Ga_k',\phi)+\frac{\eps}{2}.$$
  Sea $k_0\geq k$, tal que 
  $$0\leq U(f,\Ga_k',\phi)-U<\frac\eps2$$
  cuando $k\geq k_0$. Entonces cuando $k\geq k_0$ vale
  \begin{gather*}
    |U(f,\Ga_k,\phi)-U|\\
    \leq |U(f,\Ga_k',\phi)-U|+|U(f,\Ga_k,\phi)-U(f,\Ga_k',\phi)|\\
    \leq \frac{\eps}{2}+\frac{\eps}{2}.
  \end{gather*}
     \end{frame}

\begin{frame}{Segundo Paso}
  Si $\set{\tilde\Ga_k}\suck$ es otra sucesión de particiones que satisface
  $$\lim_{k\to\infty}|\tilde{\Ga}_k|=0.$$
  Tome $\hat\Ga_k=\tilde\Ga_k\cup\Ga_k$ y $k_0$ tal que 
  $$k\geq k_0\To |\tilde\Ga_k|<\dl_1,\ |\Ga_k|<\dl_1.$$
  Entonces 
  \begin{itemize}
    \item $U(f,\Ga_k,\phi)\leq U(f,\hat\Ga_k,\phi)+\frac\eps2$.
    \item $U(f,\hat\Ga_k,\phi)\leq U(f,\Ga_k,\phi) +\frac\eps2$.
  \end{itemize}
\end{frame}

\begin{frame}{Terminamos}
  Luego 
  \begin{align*}
    -\eps&<U(f,\Ga_k,\phi)-U\leq U(f,\hat\Ga_k,\phi)-U+\frac\eps2\\
    &\leq U(f,\Ga_k,\phi)-U+\eps<2\eps.
  \end{align*}
  Finalmente 
  \begin{enumerate}[a)]
    \item $R(f,\Ga,\phi)-U\leq U(f,\Ga,\phi)-U\leq \frac\eps2$.
    \item $R(f,\Ga,\phi)-U\geq L(f,\Ga,\phi)-U\geq U(f,\Ga,\phi)-U-\frac\eps2>\eps$.
  \end{enumerate}
\end{frame}

\subsection{Linealidad}

\begin{frame}
  Las propiedades básicas de la integral de Riemann se mantienen. (Probar este teorema es un \alert{ejercicio}).
  \begin{Th}\label{thm:linealidadRS}
    Asuma que para $1\leq i,j\leq 2$, que $f_i:[a,b]\to\bR$ es R-S integrable respecto a $\phi_j:[a,b]\to\bR$. Si $c\in\bR$, tenemos que 
    \begin{enumerate}[(a)]
      \item $\int\limits_a^b(cf_1+f_2)\dd\phi_1=c\int\limits_a^bf_1\dd\phi_1+\int\limits_a^bf_2\dd\phi_1$.
      \item $\int\limits_a^bf_1\dd(c\phi_1)=c\int\limits_a^bf_1\dd\phi_1$.
      \item $\int\limits_a^bf_1\dd(\phi_1+\phi_2)=\int\limits_a^bf_1\dd\phi_1+\int\limits_a^bf_1\dd\phi_2$.
      \item $\int\limits_a^bf_1\dd\phi_1=\int\limits_a^cf_1\dd\phi_1+\int\limits_b^cf_1\dd\phi_1$.
    \end{enumerate}
  \end{Th}
  
\end{frame}

\subsection{Valor Medio, Partes y Convergencia Uniforme}
\begin{frame}{Valor Medio}
  Si $\phi:[a,b]\to\bR$ es creciente, tenemos que
  \begin{enumerate}[a)]
    \item $L(f,\Ga,\phi)\leq R(f,\Ga,\phi)\leq U(f,\Ga,\phi)$.
    \item $U(f,\Ga,\phi)\leq\sup_{[a,b]}f(\phi(b)-\phi(a))$.
    \item $L(f,\Ga,\phi)\geq\inf_{[a,b]}f(\phi(b)-\phi(a))$.
  \end{enumerate}
  Entonces vale 
  $$\inf_{[a,b]}f(\phi(b)-\phi(a))\leq\int\limits_a^bf\dd\phi\leq\sup_{[a,b]}f(\phi(b)-\phi(a)).$$
  
\end{frame}

\begin{frame}{Valor Medio}
  \begin{Lem}\label{lem:valorMedio}
    Si $f$ es continua y R-S integrable respecto a $\phi$, creciente, entonces
    $$\exists\xi\in[a,b]\left(\int\limits_a^bf\dd\phi=f(\xi)(\phi(b)-\phi(a))\right).$$
  \end{Lem}
\end{frame}

\begin{frame}{Integración por Partes}
  \begin{Th}\label{thm:partes}
    Si $\int_a^bf\dd\phi$ existe, entonces $\int_a^b\phi\dd f $ existe y vale 
    $$\int_a^bf\dd\phi+\int_a^b\phi\dd f=f(b)\phi(b)-f(a)\phi(a).$$
  \end{Th}
  Comenzamos por tomar $\Ga=\set{x_0=a<x_1<\dots<x_n=b}$ y $x_{i-1}\leq \xi_i\leq x_i$ para $1\leq i\leq n$. Definimos $\Ga'=\set{a,b}\cup\set{\xi_1,\dots,\xi_n}$ y $\xi_0=a$, $\xi_{n+1}=b$.
\end{frame}

\begin{frame}
  Así tenemos que $R(f,\Ga,\phi)$ es
  \small{
  \begin{align*}
    =&\sum\sucn f(xi_i)(\phi(x_i)-\phi(x_{i-1}))=\sum\sucn f(xi_i)\phi(x_i)-\sum_{i=0}^{n-1}f(\xi_{i+1})\phi(x_{i})\\
    =&-\sum_{i=1}^{n-1}\phi(x_i)(f(\xi_{i+1})-f(\xi_i))+f(\xi_n)\phi(b)-\phi(a)f(\xi_1)\\
    =&-\sum_{i=1}^{n-1}\phi(x_i)(f(\xi_{i+1})-\phi(a)(f(\xi_1)-f(\xi_0))-\phi(b)(f(\xi_{n+1})-f(\xi_n))\\
    &-\phi(a)f(a)+\phi(b)f(b)\\
    =&-\sum_{i=0}^{n}\phi(x_i)(f(\xi_{i+1})+\phi(b)f(b)-\phi(a)f(a)
  \end{align*}}
  Y por tanto $R(f,\Ga,\phi)=R(\phi,\Ga',f)+f(b)\phi(b)-f(a)\phi(a)$. El resultado se concluye tomando límites.
\end{frame}

\begin{frame}{Convergencia Uniforme}\label{ej:integralNoDefn}
  Considere $\set{r_n}\sucn=\bQ\cap[0,1]$. Defina 
  $$f_n:[0,1]\to\bR,\ x\mapsto\begin{cases}
    1,\ x=r_1,\dots,r_n.\\
    0,\ \text{si no}.
  \end{cases}$$
  Note que $f_n\leq f_{n+1}$ y que 
  $$\lim_{n\to\infty}f_n=\mapsto\begin{cases}
    1,\ x\in\bQ\cap[0,1].\\
    0,\ \text{si no}.
  \end{cases}$$
  Es un \alert{ejercicio} probar que
  \begin{itemize}
    \item $\int\limits_0^1f_n(x)\dd x=0$.
    \item $f$ no es Riemann integrable.
  \end{itemize}
  Luego $\int_0^1f(x)\dd x$ no se puede  definir.
\end{frame}

\begin{frame}{Sucesiones Uniformemente Convergentes}
  \begin{Lem}\label{lem:ConvUnifRS}
    Sea $\phi$ de variación acotada y $\set{f_n}\sucn$ una sucesión de funciones R-S integrables respecto a $\phi$.\par 
    Si $f_n\xrightarrow[n\to\infty]{}f$ uniformemente y $f$ es R-S integrable respecto a $\phi$, entonces
    $$\lim_{n\to\infty}\int\limits_a^bf_n\dd\phi=\int\limits_a^bf\dd\phi.$$
  \end{Lem}
\end{frame}
\section*{Resumen}

\subsection*{Qu\'e vimos hoy}

\begin{frame}{Resumen}

  % Keep the summary *very short*.
  \begin{itemize}
  \item La definición \ref{def:rsIntegrable} de funciones Riemann-Stieltjes integrables.
  \item El criterio \ref{lem:NotRSIntCuando} que nos dice cuando una función NO es R-S integrable.
  \item El lema \ref{lem:PropsSumasInfSup} que resume algunas propiedades de las sumas superiores e inferiores.
  \item El teorema \ref{thm:linealidadRS} sobre la linealidad de la integral.
  \item El lema \ref{lem:valorMedio} sobre la propiedad del valor medio.
  \item El teorema \ref{thm:partes} de integración por partes.
  \item El lema \ref{lem:ConvUnifRS} de convergencia uniforme de sucesiones de funciones.
  \end{itemize}
  
\end{frame}

\subsection*{Ejercicios a trabajar}
\begin{frame}{Ejercicios}
    
  \begin{itemize}
    \item
      Lista 11
      \begin{itemize}
      \item El ejercicio \ref{ej:CauchyRS} sobre el criterio de Cauchy para integrales de Riemann y Stieltjes.
      \item Terminar la prueba del lema \ref{lem:NotRSIntCuando} es un ejercicio.
      \item Realizar la prueba del teorema \ref{thm:linealidadRS} es un ejercicio.
      \item El ejercicio \ref{ej:integralNoDefn} sobre la función indicadora de los primeros $n$ racionales en $[0,1]$.
      \end{itemize}
    \end{itemize}
  
\end{frame}


% All of the following is optional and typically not needed. 
\appendix
\section<presentation>*{\appendixname}
\subsection<presentation>*{Lectura adicional}

\begin{frame}[allowframebreaks]
  \frametitle<presentation>{Lecturas adicionales}
    
  \begin{thebibliography}{10}
    
  \beamertemplatebookbibitems
  % Start with overview books.

  \bibitem{CambroNotas}
    S.Cambronero.
    \newblock {\em Notas MA0505}.
    \newblock 20XX.

    \bibitem{NachoNotas}
    I.Rojas
    \newblock {\em Notas MA0505}.
    \newblock 2018.
 
  \end{thebibliography}
  
\end{frame}
%% 6 - 2:10:52 48
%% 7 - 1:17:44 44
%% 8 - 27:37 69
%% 9 - 1:07:47 86 + 59:38 40 approx 2 h c 7 min
%% 10 - 2:22:39 63
\end{document}


