\documentclass[utf8]{beamer}

\mode<presentation>
{
  \usetheme{Warsaw}
  \setbeamercovered{transparent}
}


\usepackage[spanish]{babel}
\usepackage{times}
\usepackage[T1]{fontenc}

\title[MA0505]{MA0505 - An\'alisis I}
\subtitle{Lecci\'on I: Repaso}

\author{Pedro M\'endez\inst{1}}

\institute[Universidad de Costa Rica] % (optional, but mostly needed)
{
  \inst{1}%
  Departmento de Matem\'atica Pura y Ciencias Actuariales\\
  Universidad de Costa Rica
  }

\date[I-2021] {Semestre I, 2021}

%%%%%%%%% === Theorems and suchlike === %%%%%%%%%%%%%%

\theoremstyle{plain}
\newtheorem{Th}{Teorema}               %%% Theorem 1.1.1
\newtheorem{Tmon}{Teoremón}
\newtheorem{Prop}{Proposición}         %%% Proposition 1.1.2
\newtheorem{Lem}{Lema}                 %%% Lemma 3
\newtheorem{Cor}{Corolario}            %%% Corollary 4

\theoremstyle{definition}
\newtheorem{Def}{Definición}           %%% Definition 5
\newtheorem{Ex}{Ejemplo}               %%% Example 6
\newtheorem{Ej}{Ejercicio}             %%% Ejercicio 7
\newtheorem{Hec}[Th]{Hecho}            %%% Hecho 1.1.8

\theoremstyle{remark}
\newtheorem{Rmk}[Th]{Observación}      %%%Remark 1.1.9
\newtheorem*{nonum-Rmk}{Observación}         %%% No number Fact
\newtheorem*{Notn}{Notaci\'on}        %% Notaciones
\newtheorem*{Warn}{Advertencia}       %% Advertencias

\numberwithin{equation}{section}

% Greek letters:

\newcommand{\al}{\alpha}                %% short for  \alpha
\newcommand{\bt}{\beta}                 %% short for  \beta
\newcommand{\Dl}{\Delta}                %% short for  \Delta
\newcommand{\dl}{\delta}                %% short for  \delta
\newcommand{\eps}{\varepsilon}          %% short for  \varepsilon
\newcommand{\Ga}{\Gamma}                %% short for  \Gamma
\newcommand{\ga}{\gamma}                %% short for  \gamma
\newcommand{\kp}{\kappa}                %% short for  \kappa
\newcommand{\La}{\Lambda}               %% short for  \Lambda
\newcommand{\la}{\lambda}               %% short for  \lambda
\newcommand{\Om}{\Omega}                %% short for  \Omega
\newcommand{\om}{\omega}                %% short for  \omega
\newcommand{\Sg}{\Sigma}                %% short for  \Sigma
\newcommand{\sg}{\sigma}                %% short for  \sigma
\newcommand{\Te}{\Theta}                %% short for  \Theta
\newcommand{\te}{\theta}                %% short for  \theta
\newcommand{\ups}{\upsilon}             %% short for  \upsilon
\newcommand{\vf}{\varphi}               %% short for  \varphi
\newcommand{\ze}{\zeta}                 %% short for  \zeta

%Boldface letters

\newcommand{\bC}{\mathbb{C}}    %%% números complejos
\newcommand{\bN}{\mathbb{N}}    %%% números naturales
\newcommand{\bP}{\mathbb{P}}        %% números enteros positivos
\newcommand{\bQ}{\mathbb{Q}}    %%% números racionales
\newcommand{\bR}{\mathbb{R}}    %%% números reales
\newcommand{\bS}{\mathbb{S}}    %%% esfera
\newcommand{\bZ}{\mathbb{Z}}    %%% números enteros

%Script letters:

\newcommand{\cA}{\mathcal{A}}           %% formas diferenciales
\newcommand{\cB}{\mathcal{B}}           %% una base vectorial
\newcommand{\cC}{\mathcal{C}}           %% otra base vectorial
\newcommand{\cD}{\mathcal{D}}           %% funciones de prueba
\newcommand{\cE}{\mathcal{E}}           %% un modulo proyectivo
\newcommand{\cF}{\mathcal{F}}           %% espacio de Fock
\newcommand{\cG}{\mathcal{G}}           %% funtor de Gelfand
\newcommand{\cH}{\mathcal{H}}           %% espacio de Hilbert
\newcommand{\cI}{\mathcal{I}}           %% un funtor de inclusion
\newcommand{\cJ}{\mathcal{J}}           %% otro funtor
\newcommand{\cK}{\mathcal{K}}           %% otro espacio de Hilbert
\newcommand{\cL}{\mathcal{L}}           %% operadores lineales
\newcommand{\cM}{\mathcal{M}}           %% multiplicadores
\newcommand{\cN}{\mathcal{N}}           %% funciones nulas
\newcommand{\cO}{\mathcal{O}}           %% funciones de crec-to lento
\newcommand{\cP}{\mathcal{P}}           %% una particion
\newcommand{\cR}{\mathcal{R}}           %% funciones representativas
\newcommand{\cQ}{\mathcal{Q}}           %% otra particion
\newcommand{\cS}{\mathcal{S}}           %% funciones de Schwartz
\newcommand{\cT}{\mathcal{T}}           %% una topologia
\newcommand{\cU}{\mathcal{U}}           %% cubrimiento abierto
\newcommand{\cV}{\mathcal{V}}           %% vecindarioas
\newcommand{\cW}{\mathcal{W}}           %% grupo de Weyl


%Brackets

\newcommand{\bonj}[1]{\left\lbrack#1\right\rbrack}
\newcommand{\obonj}[1]{\left\rbrack#1\right\lbrack}
\newcommand{\rbonj}[1]{\left\rbrack#1\right\rbrack}
\newcommand{\lbonj}[1]{\left\lbrack#1\right\lbrack}
\newcommand{\snm}[1]{\|#1\|}           %small norma
\newcommand{\nm}[1]{\left\|#1\right\|} %norma pegadita
\newcommand{\pnm}[1]{\biggl|\biggl|#1\biggr|\biggr|}
\begin{document}

\begin{frame}
  \titlepage
\end{frame}

\begin{frame}{Outline}
  \tableofcontents
  % You might wish to add the option [pausesections]
\end{frame}


% Structuring a talk is a difficult task and the following structure
% may not be suitable. Here are some rules that apply for this
% solution: 

% - Exactly two or three sections (other than the summary).
% - At *most* three subsections per section.
% - Talk about 30s to 2min per frame. So there should be between about
%   15 and 30 frames, all told.

% - A conference audience is likely to know very little of what you
%   are going to talk about. So *simplify*!
% - In a 20min talk, getting the main ideas across is hard
%   enough. Leave out details, even if it means being less precise than
%   you think necessary.
% - If you omit details that are vital to the proof/implementation,
%   just say so once. Everybody will be happy with that.

\section{Espacios m\'etricos}

\subsection{Definiciones b\'asicas}

\begin{frame}{Definci\'on de Espacios M\'etricos}%{Subtitles are optional.}
  % - A title should summarize the slide in an understandable fashion
  %   for anyone how does not follow everything on the slide itself.

  Sea $E$ un conjunto, una \alert{m\'etrica} es una funci\'on 
  $$d: E\times E\to[0,+\infty[$$
  que satisface
  \begin{enumerate}
  \item $d(x,y)=d(y,x)$.
  \item $d(x,y)=0$ si y s\'olo si $x=y$.
  \item  $d(x,y)\leq d(x,z)+d(z,y)$. 
  \end{enumerate}
\end{frame}

\begin{frame}{La m\'etrica en $\mathbb R^d$.}
  Observemos que la tercera condici\'on es la desigualdad triangular en $\mathbb R^d$ pues si
  $$d(x,y)=\left\| x-y\right\|$$
  entonces
  \begin{align*}
      d(x,y)&=\left\| x-y\right\|=\left\| x-z+z-y\right\|\\
      &\leq \left\| x-z\right\|+\left\| z-y\right\|\\
      &=d(x,y)+d(z,y).
  \end{align*}
\end{frame}

\section{Topolog\'ia de los Espacios M\'etricos}

\subsection{Definiciones}

\begin{frame}{Bolas, Unidades de la Topolog\'ia M\'etrica.}
    Recordemos que 
    $$B(x_0,r)=\left\lbrace y\in E:\ d(x_0,y)\leq r\right\rbrace.$$
    Diremos que $D\subseteq E$ es un \alert{conjunto abierto} si para $x_0\in D$, existe un $r>0$ tal que 
    $$B(x_0,r)\subseteq D.$$
    De aqu\'i vemos que $\emptyset$ y $E$ son abiertos.
\end{frame}

\begin{frame}{Las Bolas Abiertas son Abiertos.}
    \begin{Lem}
        $B(x_0,r)$ es un abierto para todo $x_0\in E$ y $r>0$.
    \end{Lem}
        Sea $x_1\in B(x_0,r)$, debemos encontrar $r_1>0$ tal que $B(x_1,r_1)\subseteq B(x_0,r)$. Para ese efecto veremos que si 
        $$r_1< r-d(x_0,x_1),$$
        (es decir, $r_1+d(x_0,x_1)<r$), entonces $B(x_1,r_1)\subseteq B(x_0,r)$.\par 
        Sean $r_1$ como pedimos y $y\in B(x_1,r_1)$, vale que
        \begin{align*}
          d(x_0,y)&\leq d(x_0,x_1)+d(x_1,y)\\
          &< d(x_0,x_1)+r_1\\
          &< d(x_0,x_1)+r-d(x_0,x_1)=r.
        \end{align*}
        Por lo tanto $y\in B(x_0,r)$.
\end{frame}

\subsection{Propiedades B\'asicas}

\begin{frame}{Intersecciones.}
  Supongamos que $G_1,G_2$ on abiertos. Si $x_0\in G_1\cap G_2$, entonces existen $r_1,r_2$ tales que 
  $$B(x_0,r_1)\subseteq G_1,\ B(x_0,r_2)\subseteq G_2.$$
  Si $r=\min(r_1,r_2)$, entonces 
  $$B(x_0,r)\subseteq B(x_0,r_1)\cap B(x_0,r_2)\subseteq G_1\cap G_2.$$
  Hemos probado as\'i que la intersecci\'on de abiertos es un abierto. M\'as generalmente, si $G_1,\dots,G_m$ es una colecci\'on de abiertos, entonces $\bigcap_{i=1}^mG_i$ es un abierto. (\alert{Ejercicio})
\end{frame}

\begin{frame}{Y ahora Uniones.}
  Ahora si $(G_i)_{i=1}^\infty$ es una colecci\'on de abiertos, tome $x_0\in \bigcup_{i=1}^\infty$. As\'i $x_0\in G_{i_0}$ para alg\'un $i_0$. Como $G_{i_0}$ es abierto, existe $r>0$ tal que 
  $$B(x_0,r)\subseteq G_{i_0}\bigcup_{i=1}^\infty G_i.$$
  El resultado que hemos probado es
  \begin{Lem}
    Dados $G_\lambda$, $\lambda\in\Lambda$, abiertos vale que $\bigcup_{\lambda\in\Lambda}G_\lambda$ es abierto.
  \end{Lem}
\end{frame}

\begin{frame}{Cerrados.}
  Recuerde que $F$ es \alert{cerrado} si $E\setminus F$ es abierto.
  \begin{enumerate}
    \item Si $(F_\lambda)$ es una colecci\'on de cerrados, entonces 
     $$E\setminus \bigcap_{\lambda\in\Lambda}F_\lambda=\bigcup_{\lambda\in\Lambda} E\setminus F_\lambda$$
     es una uni\'on de abiertos. Es decir $\bigcap_{\lambda\in\Lambda}F_\lambda$ es cerrado.
     \item De igual forma podemos probar que si $F_1,\dots, F_m$ son cerrados, entonces $\bigcup_{i=1}^mF_i$ es un cerrado.
  \end{enumerate}
\end{frame}

\begin{frame}{Un par de ejemplos.}
  
  \begin{itemize}
    \item Podemos ver que
    $$\bigcap_{n=1}^\infty \left]a-\frac1n,a+\frac1n\right[=\{ a\}$$
    no es un abierto. Es decir, no vale que la intersecci\'on contable de abiertos sea un abierto.
    \item Adem\'as 
    $$]a,b[=\bigcup_{m=1}^\infty \left[a+\frac1n,b-\frac1n\right]$$
    no es un cerrado. As\'i la uni\'on contable de cerrados no es un cerrado.
  \end{itemize}
\end{frame}

\section*{Resumen}

\begin{frame}{Resumen}

  % Keep the summary *very short*.
  \begin{itemize}
  \item Definici\'on de m\'etrica y de abiertos.
  \item Las bolas abiertas son abiertos.
  \item Definici\'on de cerrado.
  \item Propiedades de abiertos y cerrados.
  \end{itemize}
  
  \vskip0pt plus.5fill
  \begin{itemize}
  \item
    Ejercicios
    \begin{itemize}
    \item La intersección finita de abiertos es un abierto.
    \end{itemize}
  \end{itemize}
\end{frame}



% All of the following is optional and typically not needed. 
\appendix
\section<presentation>*{\appendixname}
\subsection<presentation>*{Lectura adicional}

\begin{frame}[allowframebreaks]
  \frametitle<presentation>{Lecturas adicionales}
    
  \begin{thebibliography}{10}
    
  \beamertemplatebookbibitems
  % Start with overview books.

  \bibitem{CambroNotas}
    S.Cambronero.
    \newblock {\em Notas MA0505}.
    \newblock 20XX.

    \bibitem{NachoNotas}
    I.Rojas
    \newblock {\em Notas MA0505}.
    \newblock 2018.
 
  \end{thebibliography}
\end{frame}

\end{document}


