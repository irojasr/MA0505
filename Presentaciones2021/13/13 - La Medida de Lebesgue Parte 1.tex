\documentclass[utf8]{beamer}

\mode<presentation>
{
  \usetheme{Warsaw}
  \setbeamercovered{transparent}
}


\usepackage{amsfonts,mathtools,amssymb}
\usepackage[spanish]{babel}
\usepackage{times}
\usepackage[T1]{fontenc}
\usepackage[shortlabels]{enumitem}
\usepackage{tikz}
\usepackage{physics}

\title[MA0505]{MA0505 - An\'alisis I}
\subtitle{Lecci\'on XIII: La Medida de Lebesgue I}

\author{Pedro M\'endez\inst{1}}

\institute[Universidad de Costa Rica] % (optional, but mostly needed)
{
  \inst{1}%
  Departmento de Matem\'atica Pura y Ciencias Actuariales\\
  Universidad de Costa Rica
  }

\date[I-2021] {Semestre I, 2021}

%%%%%%%%% === Theorems and suchlike === %%%%%%%%%%%%%%

\theoremstyle{plain}
\newtheorem{Th}{Teorema}               %%% Theorem 1.1.1
\newtheorem{Tmon}{Teoremón}
\newtheorem{Prop}{Proposición}         %%% Proposition 1.1.2
\newtheorem{Lem}{Lema}                 %%% Lemma 3
\newtheorem{Cor}{Corolario}            %%% Corollary 4

\theoremstyle{definition}
\newtheorem{Def}{Definición}           %%% Definition 5
\newtheorem{Ex}{Ejemplo}               %%% Example 6
\newtheorem{Ej}{Ejercicio}             %%% Ejercicio 7
\newtheorem{Hec}[Th]{Hecho}            %%% Hecho 1.1.8

\theoremstyle{remark}
\newtheorem{Rmk}[Th]{Observación}      %%%Remark 1.1.9
\newtheorem*{nonum-Rmk}{Observación}         %%% No number Fact
\newtheorem*{Notn}{Notaci\'on}        %% Notaciones
\newtheorem*{Warn}{Advertencia}       %% Advertencias

\numberwithin{equation}{section}

%% Accomodations 

\makeatletter
\def\moverlay{\mathpalette\mov@rlay}
\def\mov@rlay#1#2{\leavevmode\vtop{%
   \baselineskip\z@skip \lineskiplimit-\maxdimen
   \ialign{\hfil$\m@th#1##$\hfil\cr#2\crcr}}}
\newcommand{\charfusion}[3][\mathord]{
    #1{\ifx#1\mathop\vphantom{#2}\fi
        \mathpalette\mov@rlay{#2\cr#3}
      }
    \ifx#1\mathop\expandafter\displaylimits\fi}
\makeatother

% Greek letters:

\newcommand{\al}{\alpha}                %% short for  \alpha
\newcommand{\bt}{\beta}                 %% short for  \beta
\newcommand{\Dl}{\Delta}                %% short for  \Delta
\newcommand{\dl}{\delta}                %% short for  \delta
\newcommand{\eps}{\varepsilon}          %% short for  \varepsilon
\newcommand{\Ga}{\Gamma}                %% short for  \Gamma
\newcommand{\ga}{\gamma}                %% short for  \gamma
\newcommand{\La}{\Lambda}               %% short for  \Lambda
\newcommand{\la}{\lambda}               %% short for  \lambda
\newcommand{\Om}{\Omega}                %% short for  \Omega
\newcommand{\om}{\omega}                %% short for  \omega
\newcommand{\Sg}{\Sigma}                %% short for  \Sigma
\newcommand{\sg}{\sigma}                %% short for  \sigma
\newcommand{\te}{\theta}                %% short for  \theta
\newcommand{\vf}{\varphi}               %% short for  \varphi
\newcommand{\ze}{\zeta}                 %% short for  \zeta

%Boldface letters

\newcommand{\bC}{\mathbb{C}}    %%% números complejos
\newcommand{\bN}{\mathbb{N}}    %%% números naturales
\newcommand{\bP}{\mathbb{P}}        %% números enteros positivos
\newcommand{\bQ}{\mathbb{Q}}    %%% números racionales
\newcommand{\bR}{\mathbb{R}}    %%% números reales
\newcommand{\bS}{\mathbb{S}}    %%% esfera
\newcommand{\bZ}{\mathbb{Z}}    %%% números enteros

%Script letters:

\newcommand{\cA}{\mathcal{A}}           %% formas diferenciales
\newcommand{\cB}{\mathcal{B}}           %% una base vectorial
\newcommand{\cC}{\mathcal{C}}           %% otra base vectorial
\newcommand{\cF}{\mathcal{F}}           %% espacio de Fock
\newcommand{\cL}{\mathcal{L}}           %% operadores lineales
\newcommand{\cM}{\mathcal{M}}           %% multiplicadores
\newcommand{\cN}{\mathcal{N}}           %% funciones nulas
\newcommand{\cP}{\mathcal{P}}           %% una particion
\newcommand{\cR}{\mathcal{R}}           %% funciones representativas
\newcommand{\cS}{\mathcal{S}}           %% funciones de Schwartz


%Brackets

\newcommand{\bonj}[1]{\left\lbrack#1\right\rbrack}
\newcommand{\obonj}[1]{\left\rbrack#1\right\lbrack}
\newcommand{\rbonj}[1]{\left\rbrack#1\right\rbrack}
\newcommand{\lbonj}[1]{\left\lbrack#1\right\lbrack}
\newcommand{\snm}[1]{\|#1\|}           %small norma
\newcommand{\nm}[1]{\left\|#1\right\|} %norma pegadita
\newcommand{\pnm}[1]{\biggl|\biggl|#1\biggr|\biggr|}
\newcommand{\set}[1]{\{\,#1\,\}}    %% set notation
\newcommand{\floor}[1]{\lfloor#1\rfloor} %% mayor entero <= x
\newcommand{\Set}[1]{\biggl\{\,#1\,\biggr\}} %% set notation (large)
\newcommand\quot[2]{
        \mathchoice
            {% \displaystyle
                \text{\raise1ex\hbox{$#1$}\Big/\lower1ex\hbox{$#2$}}%
            }
            {% \textstyle
                {^{ #1}/_{ #2}}
            }
            {% \scriptstyle
                {^{ #1}/_{ #2}}
            }
            {% \scriptscriptstyle
                {^{ #1}/_{ #2}}
            }
    }
\newcommand*\squot[2]{{^{ #1}/_{ #2}}}%%%small quotient

%Symbols 

\newcommand{\x}{\times}
\renewcommand{\geq}{\geqslant}          %% mayor o igual (variante)
\newcommand{\hookto}{\hookrightarrow}     %% inclusion arrow
\newcommand{\isom}{\simeq}              %% isomorfismo
\renewcommand{\l}{\ell}                   %% ele cursiva
\renewcommand{\leq}{\leqslant}          %% menor o igual (variante)
\newcommand{\less}{\setminus}           %% set difference
\newcommand{\To}{\Rightarrow}
\newcommand{\ov}{\overline}
\newcommand{\un}{\underline}
\newcommand{\del}{\partial}
\newcommand{\ind}{\mathbf{1}}       %%%indicator function

%%% Small fractions in displays:

\newcommand{\half}{{\mathchoice{\nhalf}{\thalf}{\shalf}{\shalf}}} %%display text script script^2
\newcommand{\happi}{{\tfrac{\pi}{2}}} %% small fraction  \pi/2
\newcommand{\quarter}{\tfrac{1}{4}} %% small fraction  1/4
\newcommand{\nhalf}{\frac{1}{2}}
\newcommand{\shalf}{{\scriptstyle\frac{1}{2}}} %% tiny fraction 1/2
\newcommand{\thalf}{{\tfrac{1}{2}}} %% small fraction  1/2
\newcommand{\third}{\tfrac{1}{3}}   %% small fraction  1/3 %Hay que renew porque mathabx toma second y third como x'' y x''' por ejemplo

\newcommand{\ihalf}{{\tfrac{i}{2}}} %% small fraction  i/2

\newcommand{\suci}{_{i=1}^\infty} %% diminutivo
\newcommand{\suck}{_{k=1}^\infty} %% diminutivo
\newcommand{\sucl}{_{\l=1}^\infty} %% diminutivo
\newcommand{\sucm}{_{m=1}^\infty} %% diminutivo
\newcommand{\sucn}{_{n=1}^\infty} %% diminutivo

\newcommand*{\Cdot}{{\raisebox{-0.25ex}{\scalebox{1.5}{$\cdot$}}}}      %% cdot más grande
\renewcommand{\.}{\Cdot}                %% producto escalar
\newcommand{\cupdot}{\charfusion[\mathbin]{\cup}{\.}}
\newcommand{\bigcupdot}{\charfusion[\mathbin]{\bigcup}{\.}}
\DeclareMathOperator{\Var}{Var}     %%%variance

\begin{document}

\begin{frame}
  \titlepage
\end{frame}

\begin{frame}{Agenda}
  \tableofcontents
  % You might wish to add the option [pausesections]
\end{frame}


% Structuring a talk is a difficult task and the following structure
% may not be suitable. Here are some rules that apply for this
% solution: 

% - Exactly two or three sections (other than the summary).
% - At *most* three subsections per section.
% - Talk about 30s to 2min per frame. So there should be between about
%   15 and 30 frames, all told.

% - A conference audience is likely to know very little of what you
%   are going to talk about. So *simplify*!
% - In a 20min talk, getting the main ideas across is hard
%   enough. Leave out details, even if it means being less precise than
%   you think necessary.
% - If you omit details that are vital to the proof/implementation,
%   just say so once. Everybody will be happy with that.

\section{Conjuntos Medibles}

\begin{frame}{Cerrados}
  \begin{Lem}\label{ej:parcial1I2018Ej7}
    Sea $G\subseteq\bR^d$ abierto, entonces $G=\bigcup_{k=1}^\infty I_k$ con $I_k=\bonj{a_1^k,b_1^k}\x\dots\x\bonj{a_d^k,b_d^k}$ y $I_k^o\cap I_\ell^o=\emptyset$ para $k\neq\ell$.
  \end{Lem}
  La prueba de este lema es un \alert{ejercicio}.
  \begin{Th}\label{th:closedMeasurable}
    Sea $F\subseteq\bR^d$ un cerrado, entonces $F$ es medible.
  \end{Th}
\end{frame}

\begin{frame}{Prueba del Teorema}
  Si asumimos adicionalmente que $F$ es compacto, entonces dado $\eps>0$ existe un $G$ abierto que satisface:
  $$m_e(G)\leq m_e(F)+\frac{\eps}{2}.$$
  Note que $G\less F$ es abierto. Luego existen $\set{I_k}\suck$ cubos cerrados tales que 
  $$G\less F=\bigcup\suck I_k.$$
  Como $F$ y $\bigcup_{k=1}^\l I_k$ son cerrados disjuntos y $F$ es compacto, entonces $d\left(F,\bigcup_{k=1}^\l I_k\right)>0$ y así
  $$m_e(G)\geq m_e\left(F\cup\bigcup_{k=1}^\l I_k\right)=m_e\left(F\right)+m_e\left(\bigcup_{k=1}^\l I_k\right).$$
\end{frame}

\begin{frame}{Terminamos la Prueba}
  De esta manera 
  \begin{align*}
    &m_e\left(\bigcup_{k=1}^\l I_k\right)\leq m_e(G)-m_e(F)\leq \frac{\eps}{2}\\
    \To&\forall \l\geq 1\left(\sum_{k=1}^\l m_e(I_k)\leq \frac{\eps}{2}\right)\\
    \To&\sum\suck m_e(I_k)\leq\frac\eps2\To m_e(G\less F)<\eps.
  \end{align*}
  Y finalmente si $F$ no es compacto, podemos tomar $F=\bigcup F_k$ con $F_k=F\cap\ov{B(0,k)}$ que sí es compacto. De esta manera, $F$ es medible.
\end{frame}

\begin{frame}{Complementos}
  \begin{Th}\label{th:complementMeasurable}
    El complemento de un conjunto medible es medible.
  \end{Th}
  Sea $E$ medible, dado $k\geq 1$, existe $G_k$ un abierto tal que 
  $$E\subseteq G_k,\quad m_e(G_k\less E)<\frac1k.$$
  Note que $E\subseteq\bigcap\suck G_k$ y que $G_k^c$ es cerrado para $k\geq 1$. Así $H=\bigcup\suck G_k^c$ es medible con 
  $$H=\left(\bigcap\suck G_k\right)^c=\bigcup\suck G_k^c\subseteq E^c.$$
\end{frame}

\begin{frame}{Terminamos la Prueba}
  Ahora vale 
  $$E^c\less H\subseteq E^c\less G_k^c=G_k\less E.$$
  Es decir, para $k\geq 1$ tenemos $m_e(E^c\less H)\less \frac1k$. Concluimos que $Z=E^c\less H$ tiene medida exterior cero y entonces podemos expresar 
  $$E^c=Z\cup H$$
  de manera que $E^c$ es medible.
\end{frame}

\begin{frame}{Extraemos un Resultado}
  Si $\set{E_k}\suck$ es una colección de medibles, entonces 
  \begin{enumerate}[(i)]
    \item $\bigcup\suck E_k$ es medible.
    \item $\bigcup\suck E_k^c$ es medible.
  \end{enumerate}
  Entonces tenemos que 
  $$\bigcap\suck E_k=\left(\bigcup\suck E_k^c\right)^c$$
  es medible también.\par 
  Además $E_1\less E_2=E_1\cap E_2^c$ es medible.
\end{frame}

\section{Sigma Álgebras}

\begin{frame}
  \begin{Def}\label{def:sgAlgebra}
  Sea $\Om$ un conjunto, una \alert{$\sg$-álgebra} $\cF$ es un subconjunto de $2^\Om$ que satisface
  \begin{enumerate}[(i)]
    \item Si $E\in\cF$, entonces $E^c\in\cF$.
    \item $\Om\in\cF$.
    \item Si $\set{F_k}\suck\subseteq F$ entonces $\bigcup\suck F_k\in\cF$.
  \end{enumerate}
  \end{Def}

  \begin{Lem}\label{lem:whoisSigmaAlg}
    Sea $\cF$ una $\sg$-álgebra y $\set{E_k}\suck\subseteq\cF$. Entonces
    \begin{enumerate}[(i)]
      \item $\bigcap\suck E_k\in\cF$.
      \item $E_1\less E_2\in\cF$.
      \item $\emptyset\in\cF$.
    \end{enumerate}
  \end{Lem}
  La prueba de este lema es un \alert{ejercicio}.
\end{frame}

\begin{frame}{El Conjunto de los Medibles}
  Sea $\cM=\set{E\subseteq\bR^d:\ E\ \text{medible}}$. Los resultados anteriores se resumen en el siguiente lema.
  \begin{Lem}\label{lem:MesSgAlg}
    $\cM$ es una $\sg$-álgebra.
  \end{Lem}
  Sea $\cS\subseteq 2^\Om$, la \alert{$\sg$-álgebra generada por $\cS$} es 
  $$\sg(\cS)=\bigcap_{\substack{\cF\ \sg-\text{álg.}\\ \cS\subseteq \cF}}\cF$$
  y de esto se desprende que $\sg(\cS)$ es la $\sg$-álgebra más pequeña que contiene a $\cS$.
\end{frame}

\begin{frame}{Borelianos}
  \begin{Def}\label{def:borelSets}
    $\cB$ la $\sg$-álgebra \alert{Boreliana}, es la $\sg$-álgebra generada por los abiertos de $\bR^d$.
  \end{Def}
  Al ser $\cB$ una $\sg$-álgebra, $\cB$ contiene a los cerrados, $G_\dl$, $F_\sg$ y además $\cB\subseteq\cM$.
\end{frame}

\section{Medida Nuevamente}

\begin{frame}
  Los medibles se pueden definir en términos de cerrados.
  \begin{Lem}\label{lem:equivMedible}
    Sea $E\subseteq\bR^d$, entonces $E$ es medible si y sólo si dado $\eps>0$, existe un $F\subseteq E$ cerrado que satisface 
    $$m_e(E\less F)<\eps.$$
  \end{Lem}
  Sabemos que $E$ es medible si y sólo si $E^c$ es medible. Dado $\eps>0$, existe $G$ abierto tal que 
  $$E^c\subseteq G,\quad m_e(G\less E^c)<\eps.$$
  Entonces $G^c=F\subseteq E$ y 
  $$m_e(E\less F)=m_e(E\cap G)=m_e(G\less E^c)<\eps.$$
  Es un \alert{ejercicio} terminar la prueba.
\end{frame}

\begin{frame}
  Con el resultado anterior en mano procedemos a probar este:
  \begin{Th}\label{th:sumaMedidaDisjuntos}
    Sea $\set{E_k}\suck$ una colección contable de conjuntos disjuntos y medibles. Entonces 
    $$m\left(\bigcup\suck E_k\right)=\sum\suck m(E_k).$$
  \end{Th}
\end{frame}

\begin{frame}{Prueba del Teorema}
  Comenzamos por asumir $E_k$ es acotado.
  \begin{itemize}
    \item Sabemos que existe $F_k\subseteq E_k$ cerrado tal que $m(E_k\less F_k)<\frac{\eps}{2^k}$.
    \item Dado que $F_k$ es compacto, para $k\geq 1$ y $F_k\cap F_\l=\emptyset$ se tiene que 
    $$m\left(\bigcup_{i=1}^m F_i\right)=\sum_{i=1}^m m(F_i).$$
    \item Entonces 
    $$\sum_{i=1}^m m(F_i)\leq m\left(\bigcup\suci E_i\right)\To\sum\suci m(F_i)\leq m\left(\bigcup\suci E_i\right).$$
  \end{itemize}
\end{frame}

\begin{frame}{Terminamos la Prueba}
  Ahora 
  $$m(E_k)=m(F_k\cup E_k\less F_k)\leq m(F_k)+m(E_k\less F_k)\leq m(F_k)+\frac{\eps}{2^k}.$$
  Luego 
  $$\sum\suci m(E_i)-\frac{\eps}{2^i}\leq m\left(\bigcup\suci E_i\right)$$
  y por tanto 
  $$\sum\suci m(E_i)\leq m\left(\bigcup\suci E_i\right)+\eps.$$
  El resto de la prueba es un \alert{ejercicio}
\end{frame}

\begin{frame}
  Sea $I^k=I_1^k\x\dots\x I_d^k$ donde $I^k_i$ es un intervalo. Si vale que $(I^k)^o\cap(I^l)^o=\emptyset$, tenemos que:
  $$m\left(\bigcup\suck(I^k)^o\right)=\sum\suck m((I_k)^o)=\sum\suck m(I^k).$$
  Así tenemos que 
  $$\sum\suck m(I^k)\leq m\left(\bigcup\suck I^k\right)\To \sum\suck m(I^k)= m\left(\bigcup\suck I^k\right).$$
\end{frame}

\begin{frame}
  Sean $E_1,E_2$ medibles con $E_1\subseteq E_2$. Tenemos que 
  $$m(E_2)=m(E_1\cup E_2\less E_1)=m(E_1)+m(E_2\less E_1).$$
  Si ocurre que $m(E_1)<\infty$, entonces 
  $$m(E_2\less E_1)=m(E_2)-m(E_1).$$
  Por otro lado si $\set{E_i}\suci$ es una sucesión de medibles tales que $E_i\subseteq E_{i+1}$, tenemos que 
  $$\bigcup\suci E_i=\bigcupdot\suci E_i\less E_{i-1}$$
  donde asumimos $E_0=\emptyset$.
\end{frame}

\begin{frame}
  De esto tenemos que
  \begin{align*}
    &m\left(\bigcup\suci E_i\right)=\sum\suci m(E_i\less E_{i-1})\\
    =&\lim_{n\to\infty}\sum_{k=1}^n m(E_i)-m(E_{i-1})=\lim_{n\to\infty}m(E_n)
  \end{align*}
  siempre que $m(E_n)<\infty$ para $k\geq 1$. Note que la identidad es cierta si $m(E_k)=\infty$ para algún $k$. 
  
\end{frame}

\begin{frame}
  Si asumimos ahora que $E_i\supseteq E_{i+1}$ para $i\geq 1$ y $m(E_1)<\infty$, entonces
  $$E_1=\bigcup\suci E_i=\left(\bigcup\suci E_i\less E_{i+1}\right)\cup E$$
  con $E=\bigcap\suci E_i$. Entonces 
  $$m(E_1)=\sum\suci m(E_i\less E_{i+1})+m(E).$$ 
\end{frame}

\begin{frame}
  Como $\sum\suci m(E_i)-m(E_{i+1})=\lim_{n\to\infty}m(E_1)-m(E_n)$, concluimos que 
  \begin{align*}
    &m(E_1)=\lim_{n\to\infty}(m(E_1)-m(E_n))+m\left(\bigcap\suci E_i\right)\\
\To &m\left(\bigcap\suci E_i\right)=\lim_{n\to\infty}m(E_n).
  \end{align*}
  En este caso \alert{sí es necesario} asumir que $m(E_1)<\infty$.
\end{frame}

\begin{frame}
  Considere ahora $E_k=\bonj{-k,k}^c$. Entonces 
  $$E_{k+1}\subseteq E_k,\ m(E_k)=\infty,\ \bigcap\suck E_k=\emptyset.$$
  \begin{Th}
    Sea $\set{E_k}\suck$ una sucesión de conjuntos medibles:
    \begin{enumerate}[(i)]
      \item Si $E_i\subseteq E_{i+1}$, $1\leq i$, entonces 
      $$m\left(\bigcup\suci E_i\right)=\lim_{k\to\infty}m(E_k).$$
      \item Si $E_{i+1}\subseteq E_i$ y existe $i_0\geq 1$ tal que $m(E_{i_0})<\infty$, entonces 
      $$m\left(\bigcap\suci E_i\right)=\lim_{i\to\infty}m(E_i).$$
    \end{enumerate}
  \end{Th}

\end{frame}

\begin{frame}{Una Generalización}
  El resultado previo se puede generalizar. Dados $E_k\subseteq\bR^d$, no necesariamente medibles, que satisface $E_k\subseteq E_{k+1}$ para $k\geq 1$, tomemos $G_k$ $G_\dl$'s
 tales que
$$E_k\subseteq G_k,\quad m_e(E_k)=m(G_k).$$
No podemos deducir que $G_k\subseteq G_{k+1}$, entonces tome $V_k=\bigcap_{j=k}^\infty G_j$ y así $V_k\subseteq V_{k+1}$.\par 
Como $E_k\subseteq E_{k+\l}\subseteq G_{k+\l}$, tenemos que $E_k\subseteq V_k$. Y además 
$$m_e(E_k)\leq m_e(V_k)=m(V_k)$$
y $m(V_k)\leq m(G_k)=m_e(E_k)$. Es decir $m_e(E_k)=m(V_k)$.
\end{frame}

\begin{frame}
  Finalmente 
  \begin{align*}
    m_e(E_k)&\leq m_e\left(\bigcup\suck E_k\right)\leq m_e\left(\bigcup\suck V_k\right)\\
    &\leq \lim_{k\to\infty}m(V_k)=\lim_{k\to\infty}m_e(E_k).
  \end{align*}
  Con esto tenemos el teorema 
  \begin{Th}\label{th:sucDeConjsMedida}
    Si $E_k\subseteq E_{k+1}\subseteq\bR^d$, entonces 
    $$m_e\left(\bigcup\suck E_k\right)=\lim_{k\to\infty}m_e(E_k).$$
  \end{Th}
\end{frame}
\section*{Resumen}

\subsection*{Qu\'e vimos hoy}

\begin{frame}{Resumen}

  % Keep the summary *very short*.
  \begin{itemize}
  \item El lema \ref{ej:parcial1I2018Ej7} sobre la caracterización de abiertos.
  \item Los teoremas \ref{th:closedMeasurable} y \ref{th:complementMeasurable} sobre cerrados y complementos medibles.
  \item La definición \ref{def:sgAlgebra} y el lema \ref{lem:whoisSigmaAlg} sobre $\sg$-álgebras.
  \item Por el lema \ref{lem:MesSgAlg}, el conjunto de medibles es una $\sg$-álgebra.
  \item La definición \ref{def:borelSets} de la $\sg$-álgebra Boreliana.
  \item La caracterización \ref{lem:equivMedible} de medibles por cerrados.
  \item El teorema \ref{th:sumaMedidaDisjuntos} sobre medidas de conjuntos disjuntos.
  \item El teorema \ref{th:sucDeConjsMedida}.
  \end{itemize}
  
\end{frame}

\subsection*{Ejercicios a trabajar}
\begin{frame}{Ejercicios}
    
  \begin{itemize}
    \item
      Lista 13
      \begin{itemize}
      \item El lema \ref{ej:parcial1I2018Ej7} sobre la caracterización de abiertos.
      \item El lema \ref{lem:whoisSigmaAlg} sobre quienes están dentro de una $\sg$-álgebra.
      \item Terminar la prueba del lema \ref{lem:equivMedible}.
      \item Terminar la prueba del teorema \ref{th:sumaMedidaDisjuntos}.
      \end{itemize}
    \end{itemize}
  
\end{frame}


% All of the following is optional and typically not needed. 
\appendix
\section<presentation>*{\appendixname}
\subsection<presentation>*{Lectura adicional}

\begin{frame}[allowframebreaks]
  \frametitle<presentation>{Lecturas adicionales}
    
  \begin{thebibliography}{10}
    
  \beamertemplatebookbibitems
  % Start with overview books.

  \bibitem{CambroNotas}
    S.Cambronero.
    \newblock {\em Notas MA0505}.
    \newblock 20XX.

    \bibitem{NachoNotas}
    I.Rojas
    \newblock {\em Notas MA0505}.
    \newblock 2018.
 
  \end{thebibliography}
  
\end{frame}
%% 6 - 2:10:52 48
%% 7 - 1:17:44 44
%% 8 - 27:37 69
%% 9 - 1:07:47 86 + 59:38 40 approx 2 h c 7 min
%% 10 - 2:22:39 63
%% 11 - Approx 2h c 2 min
%% 12 - Motiv y figs 1:00:17.53, en tot: 2:22:23.43
%% 13 - 1:18:03.51
\end{document}


