\documentclass[utf8]{beamer}

\mode<presentation>
{
  \usetheme{Warsaw}
  \setbeamercovered{transparent}
}


\usepackage{amsfonts,mathtools,amssymb}
\usepackage[spanish]{babel}
\usepackage{times}
\usepackage[T1]{fontenc}
\usepackage[shortlabels]{enumitem}
\usepackage{tikz}
\usepackage{physics}

\title[MA0505]{MA0505 - An\'alisis I}
\subtitle{Lecci\'on XIX: La Integral de Lebesgue II}

\author{Pedro M\'endez\inst{1}}

\institute[Universidad de Costa Rica] % (optional, but mostly needed)
{
  \inst{1}%
  Departmento de Matem\'atica Pura y Ciencias Actuariales\\
  Universidad de Costa Rica
  }

\date[I-2021] {Semestre I, 2021}

%%%%%%%%% === Theorems and suchlike === %%%%%%%%%%%%%%

\theoremstyle{plain}
\newtheorem{Th}{Teorema}               %%% Theorem 1.1.1
\newtheorem{Tmon}{Teoremón}
\newtheorem{Prop}{Proposición}         %%% Proposition 1.1.2
\newtheorem{Lem}{Lema}                 %%% Lemma 3
\newtheorem{Cor}{Corolario}            %%% Corollary 4

\theoremstyle{definition}
\newtheorem{Def}{Definición}           %%% Definition 5
\newtheorem{Ex}{Ejemplo}               %%% Example 6
\newtheorem{Ej}{Ejercicio}             %%% Ejercicio 7
\newtheorem{Hec}[Th]{Hecho}            %%% Hecho 1.1.8

\theoremstyle{remark}
\newtheorem{Rmk}[Th]{Observación}      %%%Remark 1.1.9
\newtheorem*{nonum-Rmk}{Observación}         %%% No number Fact
\newtheorem*{Notn}{Notaci\'on}        %% Notaciones
\newtheorem*{Warn}{Advertencia}       %% Advertencias

\numberwithin{equation}{section}

%% Accomodations 

\makeatletter
\def\moverlay{\mathpalette\mov@rlay}
\def\mov@rlay#1#2{\leavevmode\vtop{%
   \baselineskip\z@skip \lineskiplimit-\maxdimen
   \ialign{\hfil$\m@th#1##$\hfil\cr#2\crcr}}}
\newcommand{\charfusion}[3][\mathord]{
    #1{\ifx#1\mathop\vphantom{#2}\fi
        \mathpalette\mov@rlay{#2\cr#3}
      }
    \ifx#1\mathop\expandafter\displaylimits\fi}
\makeatother

% Greek letters:

\newcommand{\al}{\alpha}                %% short for  \alpha
\newcommand{\bt}{\beta}                 %% short for  \beta
\newcommand{\Dl}{\Delta}                %% short for  \Delta
\newcommand{\dl}{\delta}                %% short for  \delta
\newcommand{\eps}{\varepsilon}          %% short for  \varepsilon
\newcommand{\Ga}{\Gamma}                %% short for  \Gamma
\newcommand{\ga}{\gamma}                %% short for  \gamma
\newcommand{\La}{\Lambda}               %% short for  \Lambda
\newcommand{\la}{\lambda}               %% short for  \lambda
\newcommand{\kp}{\kappa}                %% short for  \kappa
\newcommand{\Om}{\Omega}                %% short for  \Omega
\newcommand{\om}{\omega}                %% short for  \omega
\newcommand{\Sg}{\Sigma}                %% short for  \Sigma
\newcommand{\sg}{\sigma}                %% short for  \sigma
\newcommand{\te}{\theta}                %% short for  \theta
\newcommand{\vf}{\varphi}               %% short for  \varphi
\newcommand{\ze}{\zeta}                 %% short for  \zeta

%Boldface letters

\newcommand{\bC}{\mathbb{C}}    %%% números complejos
\newcommand{\bN}{\mathbb{N}}    %%% números naturales
\newcommand{\bP}{\mathbb{P}}        %% números enteros positivos
\newcommand{\bQ}{\mathbb{Q}}    %%% números racionales
\newcommand{\bR}{\mathbb{R}}    %%% números reales
\newcommand{\bS}{\mathbb{S}}    %%% esfera
\newcommand{\bZ}{\mathbb{Z}}    %%% números enteros

%Script letters:

\newcommand{\cA}{\mathcal{A}}           %% formas diferenciales
\newcommand{\cB}{\mathcal{B}}           %% una base vectorial
\newcommand{\cC}{\mathcal{C}}           %% otra base vectorial
\newcommand{\cF}{\mathcal{F}}           %% espacio de Fock
\newcommand{\cL}{\mathcal{L}}           %% operadores lineales
\newcommand{\cM}{\mathcal{M}}           %% multiplicadores
\newcommand{\cN}{\mathcal{N}}           %% funciones nulas
\newcommand{\cP}{\mathcal{P}}           %% una particion
\newcommand{\cR}{\mathcal{R}}           %% funciones representativas
\newcommand{\cS}{\mathcal{S}}           %% funciones de Schwartz


%Brackets

\newcommand{\bonj}[1]{\left\lbrack#1\right\rbrack}
\newcommand{\obonj}[1]{\left\rbrack#1\right\lbrack}
\newcommand{\rbonj}[1]{\left\rbrack#1\right\rbrack}
\newcommand{\lbonj}[1]{\left\lbrack#1\right\lbrack}
\newcommand{\snm}[1]{\|#1\|}           %small norma
\newcommand{\nm}[1]{\left\|#1\right\|} %norma pegadita
\newcommand{\pnm}[1]{\biggl|\biggl|#1\biggr|\biggr|}
\newcommand{\set}[1]{\{\,#1\,\}}    %% set notation
\newcommand{\floor}[1]{\lfloor#1\rfloor} %% mayor entero <= x
\newcommand{\Set}[1]{\biggl\{\,#1\,\biggr\}} %% set notation (large)
\newcommand\quot[2]{
        \mathchoice
            {% \displaystyle
                \text{\raise1ex\hbox{$#1$}\Big/\lower1ex\hbox{$#2$}}%
            }
            {% \textstyle
                {^{ #1}/_{ #2}}
            }
            {% \scriptstyle
                {^{ #1}/_{ #2}}
            }
            {% \scriptscriptstyle
                {^{ #1}/_{ #2}}
            }
    }
\newcommand*\squot[2]{{^{ #1}/_{ #2}}}%%%small quotient

%Symbols 

\newcommand{\x}{\times}
\renewcommand{\geq}{\geqslant}          %% mayor o igual (variante)
\newcommand{\hookto}{\hookrightarrow}     %% inclusion arrow
\newcommand{\isom}{\simeq}              %% isomorfismo
\renewcommand{\l}{\ell}                   %% ele cursiva
\renewcommand{\leq}{\leqslant}          %% menor o igual (variante)
\newcommand{\less}{\setminus}           %% set difference
\newcommand{\To}{\Rightarrow}
\newcommand{\ov}{\overline}
\newcommand{\un}{\underline}
\newcommand{\del}{\partial}
\newcommand{\ind}{\mathbf{1}}       %%%indicator function

%%% Small fractions in displays:

\newcommand{\half}{{\mathchoice{\nhalf}{\thalf}{\shalf}{\shalf}}} %%display text script script^2
\newcommand{\happi}{{\tfrac{\pi}{2}}} %% small fraction  \pi/2
\newcommand{\quarter}{\tfrac{1}{4}} %% small fraction  1/4
\newcommand{\nhalf}{\frac{1}{2}}
\newcommand{\shalf}{{\scriptstyle\frac{1}{2}}} %% tiny fraction 1/2
\newcommand{\thalf}{{\tfrac{1}{2}}} %% small fraction  1/2
\newcommand{\third}{\tfrac{1}{3}}   %% small fraction  1/3 %Hay que renew porque mathabx toma second y third como x'' y x''' por ejemplo

\newcommand{\ihalf}{{\tfrac{i}{2}}} %% small fraction  i/2

\newcommand{\suci}{_{i=1}^\infty} %% diminutivo
\newcommand{\sucj}{_{j=1}^\infty} %% diminutivo
\newcommand{\suck}{_{k=1}^\infty} %% diminutivo
\newcommand{\sucl}{_{\l=1}^\infty} %% diminutivo
\newcommand{\sucm}{_{m=1}^\infty} %% diminutivo
\newcommand{\sucn}{_{n=1}^\infty} %% diminutivo

\newcommand*{\Cdot}{{\raisebox{-0.25ex}{\scalebox{1.5}{$\cdot$}}}}      %% cdot más grande
\renewcommand{\.}{\Cdot}                %% producto escalar
\newcommand{\cupdot}{\charfusion[\mathbin]{\cup}{\.}}
\newcommand{\bigcupdot}{\charfusion[\mathbin]{\bigcup}{\.}}
\DeclareMathOperator{\Var}{Var}     %%%variance

\begin{document}

\begin{frame}
  \titlepage
\end{frame}

\begin{frame}{Agenda}
  \tableofcontents
  % You might wish to add the option [pausesections]
\end{frame}


% Structuring a talk is a difficult task and the following structure
% may not be suitable. Here are some rules that apply for this
% solution: 

% - Exactly two or three sections (other than the summary).
% - At *most* three subsections per section.
% - Talk about 30s to 2min per frame. So there should be between about
%   15 and 30 frames, all told.

% - A conference audience is likely to know very little of what you
%   are going to talk about. So *simplify*!
% - In a 20min talk, getting the main ideas across is hard
%   enough. Leave out details, even if it means being less precise than
%   you think necessary.
% - If you omit details that are vital to the proof/implementation,
%   just say so once. Everybody will be happy with that.

\section{La Integral de Lebesgue}

\subsection{Teoremas de Convergencia}
\begin{frame}{Convergencia Monótona}
\begin{Th}\label{th:TCMLebesgue}
    Sea $\set{f_k}\suck$ una sucesión de funciones medibles tales que 
    $$f_k(x)\leq f_{k+1}(x), x\in E.$$
Si vale que $\lim_{k\to\infty}f_k(x)=f(x)$ casi por doquier en $E$, entonces
$$\lim_{k\to\infty}\int\limits_Ef_k(x)\dd x=\int_Ef(x)\dd x.$$
\end{Th}
\end{frame}

\begin{frame}{Prueba del Teorema}
Recordemos que 
$$R(f,E)=\Ga(f,E)\cup\bigcup\suck R(f_k,E).$$
Como $R(f_k,E)\subseteq R(f_{k+1},E)$ tenemos que 
$$m(R(f,E))=\lim_{k\to\infty}m(R(f_k,E)).$$
\end{frame}

\begin{frame}
  En el caso que $E=\bigcup\suci E_i$ con $E_i\cap E_j=\emptyset$ si $i\neq j$, tenemos que
  $$R(f,E)=\bigcupdot\suck R(f,E_k).$$
  De esta manera
  $$\int\limits_Ef(x)\dd x=m(R(f,E))=\sum\suck m(R(f,E_k))=\sum\suck\int\limits_{E_k}f(x)\dd x.$$
  Esta fórmula es válida también para uniones finitas pues $R(f,\emptyset)=\emptyset$.
\end{frame}

\begin{frame}{Una Fórmula Útil}
  Dada $f:E\to\lbonj{0,\infty}$ medible, definimos 
  $$\sum(f,\set{E_i}_{i=1}^N)=\sum_{i=1}^N(\inf_{E_i}f)\ind_{E_i}.$$
  \begin{Th}\label{th:formulaInfSup}
    Sea $f:E\to\bR$ medible, no negativa. Entonces
    $$\int\limits_E f(x)\dd x=\sup_{E=\bigcupdot_{i=1}^NE_i}\sum(f,\set{E_i}_{i=1}^N).$$
  \end{Th}
\end{frame}

\begin{frame}{Prueba del Teorema}
  Dados $E_1,\dots,E_N$ tales que $E=\bigcupdot_{i=1}^NE_i$, tenemos que 
  \begin{gather*}  
    \sum(f,\set{E_i}_{i=1}^N)\leq f,\ \text{i.e.}\\
    \sum_{i=1}^N(\inf_{E_i}f)m(E_i)\leq\int\limits_E f(x)\dd x.
  \end{gather*}
  Ahora dado $k\geq 1$ y $1\leq i\leq k2^k$ diremos que
  $$E_k^0=\set{f\geq k},\ E_k^i=\Set{\frac{i-1}{2^k}\leq f(x)\leq \frac{i}{2^k}}.$$
  Tenemos entonces $E=\bigcup_{i=0}^{k2^k}E_k^i$.
\end{frame}

\begin{frame}{Continuamos la Prueba}
  Si $f_k(x)=k\ind_{E_k^o}+\sum_{i=1}^{k2^k}\frac{i-1}{2^k}\ind_{E^i_k}$, entonces
  \begin{itemize}
    \item $f_k\leq f_{k+1}$.
    \item $\lim_{k\to\infty}f_k=f$.
  \end{itemize}
  Por el TCM, $\int_Ef_k\to\int_E f$ y como
  $$f_k\leq \sum(f,\set{E_k^i}_{i=1}^{k2^k})$$
  entonces tenemos que
  $$\lim_{k\to\infty}\sum_{i=0}^{k2^k}(\inf_{E_i}f)\ind_{E_k^i}=\int\limits_E f(x)\dd x.$$
\end{frame}

\begin{frame}{Un Corolario}
  Note que este resultado implica lo siguiente:
  \begin{Cor}\label{cor:ALaFormInfSup}
Sea $f:E\to\bR$ medible. Entonces 
$$\int\limits_E f(x)\dd x=\sup\Set{\int\limits_E\phi(x)\dd x:\ \phi\ \text{simple},\ \phi\leq f\ \text{en }E}.$$
  \end{Cor}
\end{frame}

\subsection{Propiedades de la Integral}

\begin{frame}{Medida Cero}
  \begin{Lem}\label{lem:intSobreMedCero}
  Sea $f:E\to\lbonj{0,\infty}$. Si $m(E)=0$, entonces $\int\limits_Ef(x)\dd x=0$.
  \end{Lem}
Sean $f_k=\sum_{i=1}^ma_i\ind_{A_i}$ funciones simples tales que 
$$\int\limits_Ef_k\dd x\xrightarrow[k\to\infty]{}\int\limits_Ef(x)\dd x.$$
Note que 
$$\int\limits_Ef_k(x)\dd x)\sum_{i=1}^ma_im(A_i).$$
Pero $A_i\subseteq E$ nos dice que $m(A_i)=0$ para $1\leq i\leq m$.
\end{frame}

\begin{frame}{Desigualdades}
  \begin{Th}\label{th:intPreservaOrden}
    Sean $f,g:E\to\lbonj{0,\infty}$ tales que $g(x)\leq f(x)$ casi por doquier en $E$. Entonces vale que 
    $$\int\limits_Eg(x)\dd x\leq\int\limits_Ef(x)\dd x.$$
    En particular si $f=g$ c.p.d. en $E$, entonces
    $$\int\limits_Eg(x)\dd x=\int\limits_Ef(x)\dd x.$$
  \end{Th}
\end{frame}

\begin{frame}{Prueba del Teorema}
  Sea $A=\set{x\in E:\ g(x)\leq f(x)}$. Entonces $E=A\cup Z$ con $m(Z)=0$. Así tenemos
  \begin{align*}
    \int\limits_Ef(x)\dd x&=\int\limits_Af(x)\dd x+\int\limits_Zf(x)\dd x\\
    &\geq \int\limits_Ag(x)\dd x\\
    &= \int\limits_Ag(x)\dd x+\int\limits_ZSg(x)\dd x+\\
&=\int\limits_Eg(x)\dd x.
  \end{align*}
\end{frame}

\begin{frame}{Una Desigualdad Familiar\dots}\label{Markov}
  Sea $\al> 0$, entonces
  $$\al\ind_{\set{f\geq \al}}\leq f\ind_{\set{f\geq \al}}\leq f$$
  cuando $f:E\to\lbonj{0,\infty}$. Entonces
  $$\al m(\set{f\geq \al})\leq\int\limits_{\set{f\geq \al}}f(x)\dd x\leq \int\limits_Ef(x)\dd x.$$
  A la desigualdad $ m(\set{f\geq \al})\leq\frac{1}{\al}\int\limits_Ef(x)\dd x$ se le conoce como \alert{desigualdad de Markov}.
\end{frame}

\begin{frame}
  Ahora si $f:E\to\lbonj{0,\infty}$ es medible y $\int\limits_Ef(x)\dd x=0$ tenemos que 
  $$m(\set{f\geq \al})=0$$
  para $\al\geq 0$. Y luego 
  $$\set{f\geq 0}=\bigcup\suck\set{f\geq\frac{1}{k}}$$
  es un conjunto de medida cero. Es decir, $f=0$ c.p.d.
  \begin{Cor}\label{cor:aMarkov}
    Sea $f:E\to\lbonj{0,\infty}$ tal que $\int\limits_E f(x)\dd x=0$. Entonces $f=0$ casi por doquier.
  \end{Cor}
\end{frame}

\begin{frame}{Linealidad}
  Por otro lado si $c>0$ entonces veremos que 
  $$\int\limits_E(cf(x)+g(x))\dd x=c\int\limits_Ef(x)\dd x+\int\limits_Eg(x)\dd x.$$
  Sea $f_k$ una sucesión de funciones simples tales que 
  \begin{itemize}
    \item $0\leq f_k\leq f_{k+1}$.
    \item $\lim_{k\to\infty}f_k=f$.
  \end{itemize}
  Si $f_k=\sum_{i=1}^{m_k}a_i\ind_{A_i}$, tenemos que 
  $$cf_k=\sum_{i=1}^{m_k}ca_i\ind_{A_i}\xrightarrow[k\to\infty]{}cf(x)$$
  y $cf_k\leq cf_{k+1}$.
\end{frame}

\begin{frame}{Linealidad}
  Entonces tenemos 
  \begin{align*}
    \int\limits_Ecf(x)\dd x&=\lim_{k\to\infty}\int\limits_Ecf_k(x)\dd x\\
    &=\lim_{k\to\infty}\sum_{i=1}^{m_k}ca_im(A_i)\\
    &=c\lim_{k\to\infty}\sum_{i=1}^{m_k}a_im(A_i)\\
    &=c\int\limits_Ef(x)\dd x.
  \end{align*}

\end{frame}

\begin{frame}{Linealidad}
  \begin{Th}\label{th:intLineal}
    Sean $f,g:E\to\lbonj{0,\infty}$ medibles y $c\geq 0$. Entonces
    $$\int\limits_E(cf(x)+g(x))\dd x=c\int\limits_Ef(x)\dd x+\int\limits_Eg(x)\dd x.$$
  \end{Th}
  Falta mostrar la aditividad. Para el efecto tomemos $g_k$ una sucesión de funciones simples tales que 
  $$g_k\leq g_{k+1},\quad \lim_{k\to\infty}g_k=g.$$
  Entonces se sigue que 
  $$g_k+f_k\leq g_{k+1}+f_{k+1},\quad \lim_{k\to\infty}g_k+f_k=g+f.$$
\end{frame}

\begin{frame}{Linealidad}
  Debemos mostrar que 
  $$\int\limits_E(g_k(x)+f_k(x))\dd x=\int\limits_Eg_k(x)\dd x+\int\limits_Ef_k(x)\dd x.$$
  Si $g_k=\sum_{i=1}^{\l_k}b_i\ind_{B_i}$, con $E=\bigcupdot_{i=1}^{\l_k}B_i=\bigcupdot_{j=1}^{m_k}A_j$, entonces 
  $$g_k(x)+f_k(x)=\sum_{i=1}^{\l_k}\sum_{j=1}^{m_k}(a_j+b_i)\ind_{A_j\cap B_i}.$$
\end{frame}

\begin{frame}{Linealidad}
Ahora
\begin{align*}
  \sum_{i=1}^{\l_k}\sum_{i=1}^{m_k}m(A_j\cap B_i)&=\sum_{i=1}^{\l_k}m\left(B_i\cap\bigcup_{j=1}^{m_k}A_j
  \right)\\
  &=\sum_{i=1}^{\l_k}m\left(B_i\cap E\right) =\sum_{i=1}^{\l_k}m\left(B_i\right).
\end{align*}
De igual forma tenemos 
$$\sum_{i=1}^{m_k}\sum_{i=1}^{\l_k}m(A_j\cap B_i)=\sum_{i=1}^{m_k}m(A_i).$$
Fácilmente deducimos que 
$$\int\limits_E(g_k(x)+f_k(x))\dd x=\int\limits_Eg_k(x)\dd x+\int\limits_Ef_k(x)\dd x.$$
\end{frame}

\section*{Resumen}

\subsection*{Qu\'e vimos hoy}

\begin{frame}{Resumen}

  % Keep the summary *very short*.
  \begin{itemize}
  \item El teorema \ref{th:TCMLebesgue} de convergencia monótona.
  \item La fórmula \ref{th:formulaInfSup} para la integral y su corolario \ref{cor:ALaFormInfSup}.
  \item Entre la propiedades de la integral tenemos:
  \begin{itemize}
    \item La de medida cero: \ref{lem:intSobreMedCero}.
    \item Que preserva el orden: \ref{th:intPreservaOrden}.
    \item La desigualdad de Markov \ref{Markov} y su corolario \ref{cor:aMarkov}.
  \end{itemize}
  \item La integral es un operador lineal \ref{th:intLineal}.
  \end{itemize}
  
\end{frame}

\subsection*{Ejercicios a trabajar}
\begin{frame}{Ejercicios}
    
  \begin{itemize}
    \item
      Lista 19
      \begin{itemize}
      \item No hay ejercicios en esta presentación.
      \end{itemize}
    \end{itemize}
  
\end{frame}


% All of the following is optional and typically not needed. 
\appendix
\section<presentation>*{\appendixname}
\subsection<presentation>*{Lectura adicional}

\begin{frame}[allowframebreaks]
  \frametitle<presentation>{Lecturas adicionales}
    
  \begin{thebibliography}{10}
    
  \beamertemplatebookbibitems
  % Start with overview books.

  \bibitem{CambroNotas}
    S.Cambronero.
    \newblock {\em Notas MA0505}.
    \newblock 20XX.

    \bibitem{NachoNotas}
    I.Rojas
    \newblock {\em Notas MA0505}.
    \newblock 2018.
 
  \end{thebibliography}
  
\end{frame}
%% 6 - 2:10:52 48
%% 7 - 1:17:44 44
%% 8 - 27:37 69
%% 9 - 1:07:47 86 + 59:38 40 approx 2 h c 7 min
%% 10 - 2:22:39 63
%% 11 - Approx 2h c 2 min
%% 12 - Motiv y figs 1:00:17.53, en tot: 2:22:23.43
%% 13 - 1:18:03.51
%% 14 - 2:05:13.72
%% 15 - 1:13:41.49
%% 16 - 46:11.67
%% 17 - 44:18.85
%% 18 - 1:03:50.64
%% 19 - 51:57.37
\end{document}


