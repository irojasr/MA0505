\documentclass[utf8]{beamer}

\mode<presentation>
{
  \usetheme{Warsaw}
  \setbeamercovered{transparent}
}


\usepackage{amsfonts,mathtools,amssymb}
\usepackage[spanish]{babel}
\usepackage{times}
\usepackage[T1]{fontenc}

\title[MA0505]{MA0505 - An\'alisis I}
\subtitle{Lecci\'on IV: Continuidad}

\author{Pedro M\'endez\inst{1}}

\institute[Universidad de Costa Rica] % (optional, but mostly needed)
{
  \inst{1}%
  Departmento de Matem\'atica Pura y Ciencias Actuariales\\
  Universidad de Costa Rica
  }

\date[I-2021] {Semestre I, 2021}

%%%%%%%%% === Theorems and suchlike === %%%%%%%%%%%%%%

\theoremstyle{plain}
\newtheorem{Th}{Teorema}               %%% Theorem 1.1.1
\newtheorem{Tmon}{Teoremón}
\newtheorem{Prop}{Proposición}         %%% Proposition 1.1.2
\newtheorem{Lem}{Lema}                 %%% Lemma 3
\newtheorem{Cor}{Corolario}            %%% Corollary 4

\theoremstyle{definition}
\newtheorem{Def}{Definición}           %%% Definition 5
\newtheorem{Ex}{Ejemplo}               %%% Example 6
\newtheorem{Ej}{Ejercicio}             %%% Ejercicio 7
\newtheorem{Hec}[Th]{Hecho}            %%% Hecho 1.1.8

\theoremstyle{remark}
\newtheorem{Rmk}[Th]{Observación}      %%%Remark 1.1.9
\newtheorem*{nonum-Rmk}{Observación}         %%% No number Fact
\newtheorem*{Notn}{Notaci\'on}        %% Notaciones
\newtheorem*{Warn}{Advertencia}       %% Advertencias

\numberwithin{equation}{section}

% Greek letters:

\newcommand{\al}{\alpha}                %% short for  \alpha
\newcommand{\bt}{\beta}                 %% short for  \beta
\newcommand{\Dl}{\Delta}                %% short for  \Delta
\newcommand{\dl}{\delta}                %% short for  \delta
\newcommand{\eps}{\varepsilon}          %% short for  \varepsilon
\newcommand{\Ga}{\Gamma}                %% short for  \Gamma
\newcommand{\ga}{\gamma}                %% short for  \gamma
\newcommand{\kp}{\kappa}                %% short for  \kappa
\newcommand{\La}{\Lambda}               %% short for  \Lambda
\newcommand{\la}{\lambda}               %% short for  \lambda
\newcommand{\Om}{\Omega}                %% short for  \Omega
\newcommand{\om}{\omega}                %% short for  \omega
\newcommand{\Sg}{\Sigma}                %% short for  \Sigma
\newcommand{\sg}{\sigma}                %% short for  \sigma
\newcommand{\Te}{\Theta}                %% short for  \Theta
\newcommand{\te}{\theta}                %% short for  \theta
\newcommand{\ups}{\upsilon}             %% short for  \upsilon
\newcommand{\vf}{\varphi}               %% short for  \varphi
\newcommand{\ze}{\zeta}                 %% short for  \zeta

%Boldface letters

\newcommand{\bC}{\mathbb{C}}    %%% números complejos
\newcommand{\bN}{\mathbb{N}}    %%% números naturales
\newcommand{\bP}{\mathbb{P}}        %% números enteros positivos
\newcommand{\bQ}{\mathbb{Q}}    %%% números racionales
\newcommand{\bR}{\mathbb{R}}    %%% números reales
\newcommand{\bS}{\mathbb{S}}    %%% esfera
\newcommand{\bZ}{\mathbb{Z}}    %%% números enteros

%Script letters:

\newcommand{\cA}{\mathcal{A}}           %% formas diferenciales
\newcommand{\cB}{\mathcal{B}}           %% una base vectorial
\newcommand{\cC}{\mathcal{C}}           %% otra base vectorial
\newcommand{\cD}{\mathcal{D}}           %% funciones de prueba
\newcommand{\cE}{\mathcal{E}}           %% un modulo proyectivo
\newcommand{\cF}{\mathcal{F}}           %% espacio de Fock
\newcommand{\cG}{\mathcal{G}}           %% funtor de Gelfand
\newcommand{\cH}{\mathcal{H}}           %% espacio de Hilbert
\newcommand{\cI}{\mathcal{I}}           %% un funtor de inclusion
\newcommand{\cJ}{\mathcal{J}}           %% otro funtor
\newcommand{\cK}{\mathcal{K}}           %% otro espacio de Hilbert
\newcommand{\cL}{\mathcal{L}}           %% operadores lineales
\newcommand{\cM}{\mathcal{M}}           %% multiplicadores
\newcommand{\cN}{\mathcal{N}}           %% funciones nulas
\newcommand{\cO}{\mathcal{O}}           %% funciones de crec-to lento
\newcommand{\cP}{\mathcal{P}}           %% una particion
\newcommand{\cR}{\mathcal{R}}           %% funciones representativas
\newcommand{\cQ}{\mathcal{Q}}           %% otra particion
\newcommand{\cS}{\mathcal{S}}           %% funciones de Schwartz
\newcommand{\cT}{\mathcal{T}}           %% una topologia
\newcommand{\cU}{\mathcal{U}}           %% cubrimiento abierto
\newcommand{\cV}{\mathcal{V}}           %% vecindarios
\newcommand{\cW}{\mathcal{W}}           %% grupo de Weyl


%Brackets

\newcommand{\bonj}[1]{\left\lbrack#1\right\rbrack}
\newcommand{\obonj}[1]{\left\rbrack#1\right\lbrack}
\newcommand{\rbonj}[1]{\left\rbrack#1\right\rbrack}
\newcommand{\lbonj}[1]{\left\lbrack#1\right\lbrack}
\newcommand{\snm}[1]{\|#1\|}           %small norma
\newcommand{\nm}[1]{\left\|#1\right\|} %norma pegadita
\newcommand{\pnm}[1]{\biggl|\biggl|#1\biggr|\biggr|}
\newcommand{\set}[1]{\{\,#1\,\}}    %% set notation
\newcommand{\floor}[1]{\lfloor#1\rfloor} %% mayor entero <= x
\newcommand{\Set}[1]{\biggl\{\,#1\,\biggr\}} %% set notation (large)


%Symbols 

\renewcommand{\geq}{\geqslant}          %% mayor o igual (variante)
\newcommand{\hookto}{\hookrightarrow}     %% inclusion arrow
\newcommand{\isom}{\simeq}              %% isomorfismo
\renewcommand{\l}{\ell}                   %% ele cursiva
\renewcommand{\leq}{\leqslant}          %% menor o igual (variante)
\newcommand{\less}{\setminus}           %% set difference
\newcommand{\To}{\Rightarrow}
\newcommand{\ov}{\overline}
\newcommand{\un}{\underline}
\newcommand{\del}{\partial}

\begin{document}

\begin{frame}
  \titlepage
\end{frame}

\begin{frame}{Agenda}
  \tableofcontents
  % You might wish to add the option [pausesections]
\end{frame}


% Structuring a talk is a difficult task and the following structure
% may not be suitable. Here are some rules that apply for this
% solution: 

% - Exactly two or three sections (other than the summary).
% - At *most* three subsections per section.
% - Talk about 30s to 2min per frame. So there should be between about
%   15 and 30 frames, all told.

% - A conference audience is likely to know very little of what you
%   are going to talk about. So *simplify*!
% - In a 20min talk, getting the main ideas across is hard
%   enough. Leave out details, even if it means being less precise than
%   you think necessary.
% - If you omit details that are vital to the proof/implementation,
%   just say so once. Everybody will be happy with that.

\section{Continuidad}

\subsection{Continuidad en n\'umeros reales}

\begin{frame}{Recordemos\dots}%{Subtitles are optional.}
  % - A title should summarize the slide in an understandable fashion
  %   for anyone how does not follow everything on the slide itself.
Si $f:\ I\to\bR$ decimos que $\lim_{x\to a}f(x)=\l$ si para $\eps>0$, existe $\dl>0$ tal que 

$$|x-a|<\dl\To |f(x)-\l|<\eps.$$
Note que $d(x,a)=|x-a|$ en el espacio de partida $X$. Adem\'as $|f(x)-\l|=\rho(f(x),\l)$ en el espacio de llegada $Y$. Es decir
$$d(x,a)<\dl\To\rho(f(x),\l)<\eps.$$

\end{frame}

\subsection{Continuidad en Espacios M\'etricos}

\begin{frame}{La definición\dots}

    Sean $(X,d),(Y,\rho)$ dos espacios métricos y $f\colon X\to Y$. Decimos 
    que $\lim_{z\to a}f(z)=\l$ si para $\eps>0$, existe $\dl>0$ tal que 
    $$d(z,a)<\dl\To\rho(f(z),\l)<\eps.$$
    Note que para definir el l\'imite en de $f$ en $a$ no es necesario que $f$ est\'e definida en $a$.
    \begin{Def}\label{def:continudadEspMet}
    La funci\'on $f:\ X\to Y$ es \alert{continua} en $a$ si $\lim_{z\to a}f(z)=f(a)$. En general $f$ es continua si es es continua en todo punto de $X$. En este caso
    $$d(z,a)<\dl\To\rho(f(z),\l)<\eps.$$
    \end{Def}
    
\end{frame}

\begin{frame}
    \begin{columns}
        \begin{column}{0.5\textwidth}
            Si comenzamos con 
            $$\set{z:\ d(z,a)<\dl}\subseteq X$$
            y aplicamos $f$, llegamos a 
            $$\set{y:\ \rho(f(a),y)<\eps}\subseteq Y.$$
            $\therefore f(B_x(a,\dl))\subseteq B_y(f(a),\eps).$
        \end{column}
        
        \begin{column}{0.5\textwidth}
        \end{column}
    \end{columns}
\end{frame}

\begin{frame}{Ahora un ejemplo}
    Si $(X,d)$ es un espacio m\'etrico y $a\in X$, tomemos $f(x)=d(x,a)$. Sabemos que 
    $$|d(x,a)-d(y,a)|\leq d(x,y)$$
    y de esta desigualdad extraemos que $f$ es continua. Esta funci\'on de hecho es un ejemplo de una funci\'on 1-\alert{Lipschitz}.
    \begin{Def}\label{def:Lipschitz}
        Una funci\'on $f: X\to Y$ es $\la$-Lipschitz si para $x,y\in X$ vale
        $$\rho(f(x),f(y))\leq \la d(x,y).$$
    \end{Def}
    \begin{Ej}\label{ej:distConjEsLip}
        Para $A\subseteq X$, verifique que $f(x)=d(x,A)$ es 1-Lipschitz.
    \end{Ej}
\end{frame}

\begin{frame}{No nos olvidemos de las bolas}
    
    \begin{columns}
        \begin{column}{0.5\textwidth}
            \begin{itemize}
                \item Sea $G\subseteq Y$ abierto y $y_0=f(x_0)\in G$.\pause
                \item Entonces existe $\eps>0$ tal que $B(y_0,\eps)\subseteq G$.\pause
                \item Si $f$ es continua, existe $\dl>0$ tal que 
                $$f(B_x(x_0,\dl))\subseteq B_y(y_0,\eps)\subseteq G.$$
            \end{itemize}
            

                \end{column} 
                \begin{column}{0.5\textwidth}%dibujo
                    
                        \end{column}
                    \end{columns}
\end{frame}

\begin{frame}{Esclareciendo la Relaci\'on}
    Si $y_0\in G$, existe $\dl>0$ tal que si $z\in B_x(x_0,\dl)$ se tiene que $f(z)\in G$. Es decir,
    $$B_X(x_0,\dl)\subseteq \set{z\in X:\ f(z)\in G}=f^{-1}(G).$$
    Por tanto si $x_0\in f^{-1}(G)$, existe $\dl>0$ tal que 
    $$B_X(x_0,\dl)\subseteq f^{-1}(G).$$
    \begin{Lem}\label{lem:equiv1Cont}
        Si $f:\ X\to Y$ es continua y $G\subseteq Y$ es abierto, entonces $f^{-1}(G)$ es abierto.
    \end{Lem}

\end{frame}

\begin{frame}{No Olvidemos a los Cerrados}
    Por otro lado si $F\subseteq Y$ es un cerrado, entonces 
    $$f^{-1}(Y\less F)=X\less f^{-1}(F)$$
    es un abierto. Entonces $f^{-1}(F)$ es un cerrado.\par 
    \emph{En general recuerde que $f^{-1}(A\less B)=f^{-1}(A)\less f^{-1}(B)$.}
\end{frame}

\begin{frame}{El Teorema Resumen}
    \begin{Th}\label{thm:teoremaResumenContinuidad}
        Sea $f\colon X\to Y$, son equivalentes:
        \begin{enumerate}
          \item $f$ es continua
          \item $f^{-1}(G)$ para todo $G\subseteq Y$ abierto.
          \item $f^{-1}(F)$ es cerrado para todo $F\subseteq Y$ cerrado.
          \item $f(\overline{B})\subseteq \overline{f(B)}$ para todo $B\subseteq X$. Donde la primera cerradura es respecto a $X$ y la segunda respecto a $Y$.
        \end{enumerate}
      \end{Th}
\end{frame}

\begin{frame}{Probando el resultado}
    \begin{itemize}
        \item La primera implicaci\'on de 1 a 2 est\'a lista. \pause
        \item $(\mathit{2}\Rightarrow\mathit{1})$ Dado $\varepsilon>0$ y $a\in X$, $B(f(a),\varepsilon))$ abierto implica que $f^{-1}(B(f(a),\varepsilon))$ es abierto. Así $\exists\delta>0$ tal que $B(a,\delta)\subseteq f^{-1}( B(f(a),\varepsilon))\Rightarrow f(B(x_0,\delta))\subseteq B(f(x_0),\varepsilon)$.\pause
        \item \alert{Las dem\'as equivalencias son ejercicios}
    \end{itemize}
\end{frame}

\begin{frame}{Caracterización por sucesiones}
    \begin{Th}\label{thm:equivSuccCont}
        Sea $f\colon X\to Y$, $f$ es continua en $a$ si y sólo si para cualquier sucesión $(x_n)_{n\in\bN}\subseteq X$ tal que $x_n\to a$ se tiene que $f(x_n)\to f(a)$.
      \end{Th}
      Si $f$ no fuese continua, existe $\eps>0$ tal que para todo $\dl$, existe $x_\dl\in X$ que satisface
      $$d(x_\dl,a)<\dl\ \land\ \rho(f(x_\dl),f(a))>\eps.$$
      En particular si $n\in\bN$, existe $x_n$ tal que 
      $$d(x_n,a)<\frac{1}{n},\ \rho(f(x_n),f(a))\geq \eps.$$
      Es decir, encontramos $(x_n)$ tal que $x_n\to a$ pero $f(x_n)\not\to a$.
\end{frame}

\begin{frame}{Homeomorfismos}
    \begin{Def}\label{def:homeomorfismo}
        Llamamos a $f:\ X\to Y$ un \alert{homeomorfismo} si es continua y biyectiva. Adem\'as debe cumplir que $f^{-1}$ es continua.\par 
        En este caso diremos que $X$ y $Y$ son \alert{homeomorfos}.
    \end{Def}
   Si $A\subseteq X$ es abierto, entonces $f(A)=(f^{-1})^{-1}(A)$ es un abierto. Es decir $f$ env\'ia abiertos en abiertos.
   \begin{Ej}\label{ej:homeomorfismoDeInterior}
       En este caso muestre que $f(A^o)=(f(A))^o$.
   \end{Ej}
\end{frame}

\section*{Resumen}

\subsection*{Qu\'e vimos hoy}
\begin{frame}{Resumen}

  % Keep the summary *very short*.
  \begin{itemize}
  \item Funciones continuas en espacios m\'etricos. \ref{def:continudadEspMet}
  \item Funciones Lipschitz continuas. \ref{def:Lipschitz}
  \item El teorema resumen sobre continuidad \ref{thm:teoremaResumenContinuidad}
  \item Equivalencia en continuidad secuencial y m\'etrica. \ref{thm:equivSuccCont}
  \item Definición de homeomorfismo. \ref{def:homeomorfismo}
  \end{itemize}
  
\end{frame}

\subsection*{Ejercicios a trabajar}
\begin{frame}{Ejercicios}
    
  \begin{itemize}
    \item
      Lista 4
      \begin{itemize}
      \item La distancia a conjuntos es Lipschitz. \ref{ej:distConjEsLip}
      \item Las dem\'as equivalencias del teorema resumen. \ref{thm:teoremaResumenContinuidad}
      \item Interiores y homeomorfismos \ref{ej:homeomorfismoDeInterior}
      \end{itemize}
    \end{itemize}
  
\end{frame}


% All of the following is optional and typically not needed. 
\appendix
\section<presentation>*{\appendixname}
\subsection<presentation>*{Lectura adicional}

\begin{frame}[allowframebreaks]
  \frametitle<presentation>{Lecturas adicionales}
    
  \begin{thebibliography}{10}
    
  \beamertemplatebookbibitems
  % Start with overview books.

  \bibitem{CambroNotas}
    S.Cambronero.
    \newblock {\em Notas MA0505}.
    \newblock 20XX.

    \bibitem{NachoNotas}
    I.Rojas
    \newblock {\em Notas MA0505}.
    \newblock 2018.
 
  \end{thebibliography}
  
\end{frame}
%% 1:01:10
\end{document}


