\documentclass[utf8]{beamer}

\mode<presentation>
{
  \usetheme{Warsaw}
  \setbeamercovered{transparent}
}


\usepackage{amsfonts,mathtools,amssymb}
\usepackage[spanish]{babel}
\usepackage{times}
\usepackage[T1]{fontenc}

\title[MA0505]{MA0505 - An\'alisis I}
\subtitle{Lecci\'on II: Repaso}

\author{Pedro M\'endez\inst{1}}

\institute[Universidad de Costa Rica] % (optional, but mostly needed)
{
  \inst{1}%
  Departmento de Matem\'atica Pura y Ciencias Actuariales\\
  Universidad de Costa Rica
  }

\date[I-2021] {Semestre I, 2021}

%%%%%%%%% === Theorems and suchlike === %%%%%%%%%%%%%%

\theoremstyle{plain}
\newtheorem{Th}{Teorema}               %%% Theorem 1.1.1
\newtheorem{Tmon}{Teoremón}
\newtheorem{Prop}{Proposición}         %%% Proposition 1.1.2
\newtheorem{Lem}{Lema}                 %%% Lemma 3
\newtheorem{Cor}{Corolario}            %%% Corollary 4

\theoremstyle{definition}
\newtheorem{Def}{Definición}           %%% Definition 5
\newtheorem{Ex}{Ejemplo}               %%% Example 6
\newtheorem{Ej}{Ejercicio}             %%% Ejercicio 7
\newtheorem{Hec}[Th]{Hecho}            %%% Hecho 1.1.8

\theoremstyle{remark}
\newtheorem{Rmk}[Th]{Observación}      %%%Remark 1.1.9
\newtheorem*{nonum-Rmk}{Observación}         %%% No number Fact
\newtheorem*{Notn}{Notaci\'on}        %% Notaciones
\newtheorem*{Warn}{Advertencia}       %% Advertencias

\numberwithin{equation}{section}

% Greek letters:

\newcommand{\al}{\alpha}                %% short for  \alpha
\newcommand{\bt}{\beta}                 %% short for  \beta
\newcommand{\Dl}{\Delta}                %% short for  \Delta
\newcommand{\dl}{\delta}                %% short for  \delta
\newcommand{\eps}{\varepsilon}          %% short for  \varepsilon
\newcommand{\Ga}{\Gamma}                %% short for  \Gamma
\newcommand{\ga}{\gamma}                %% short for  \gamma
\newcommand{\kp}{\kappa}                %% short for  \kappa
\newcommand{\La}{\Lambda}               %% short for  \Lambda
\newcommand{\la}{\lambda}               %% short for  \lambda
\newcommand{\Om}{\Omega}                %% short for  \Omega
\newcommand{\om}{\omega}                %% short for  \omega
\newcommand{\Sg}{\Sigma}                %% short for  \Sigma
\newcommand{\sg}{\sigma}                %% short for  \sigma
\newcommand{\Te}{\Theta}                %% short for  \Theta
\newcommand{\te}{\theta}                %% short for  \theta
\newcommand{\ups}{\upsilon}             %% short for  \upsilon
\newcommand{\vf}{\varphi}               %% short for  \varphi
\newcommand{\ze}{\zeta}                 %% short for  \zeta

%Boldface letters

\newcommand{\bC}{\mathbb{C}}    %%% números complejos
\newcommand{\bN}{\mathbb{N}}    %%% números naturales
\newcommand{\bP}{\mathbb{P}}        %% números enteros positivos
\newcommand{\bQ}{\mathbb{Q}}    %%% números racionales
\newcommand{\bR}{\mathbb{R}}    %%% números reales
\newcommand{\bS}{\mathbb{S}}    %%% esfera
\newcommand{\bZ}{\mathbb{Z}}    %%% números enteros

%Script letters:

\newcommand{\cA}{\mathcal{A}}           %% formas diferenciales
\newcommand{\cB}{\mathcal{B}}           %% una base vectorial
\newcommand{\cC}{\mathcal{C}}           %% otra base vectorial
\newcommand{\cD}{\mathcal{D}}           %% funciones de prueba
\newcommand{\cE}{\mathcal{E}}           %% un modulo proyectivo
\newcommand{\cF}{\mathcal{F}}           %% espacio de Fock
\newcommand{\cG}{\mathcal{G}}           %% funtor de Gelfand
\newcommand{\cH}{\mathcal{H}}           %% espacio de Hilbert
\newcommand{\cI}{\mathcal{I}}           %% un funtor de inclusion
\newcommand{\cJ}{\mathcal{J}}           %% otro funtor
\newcommand{\cK}{\mathcal{K}}           %% otro espacio de Hilbert
\newcommand{\cL}{\mathcal{L}}           %% operadores lineales
\newcommand{\cM}{\mathcal{M}}           %% multiplicadores
\newcommand{\cN}{\mathcal{N}}           %% funciones nulas
\newcommand{\cO}{\mathcal{O}}           %% funciones de crec-to lento
\newcommand{\cP}{\mathcal{P}}           %% una particion
\newcommand{\cR}{\mathcal{R}}           %% funciones representativas
\newcommand{\cQ}{\mathcal{Q}}           %% otra particion
\newcommand{\cS}{\mathcal{S}}           %% funciones de Schwartz
\newcommand{\cT}{\mathcal{T}}           %% una topologia
\newcommand{\cU}{\mathcal{U}}           %% cubrimiento abierto
\newcommand{\cV}{\mathcal{V}}           %% vecindarioas
\newcommand{\cW}{\mathcal{W}}           %% grupo de Weyl


%Brackets

\newcommand{\bonj}[1]{\left\lbrack#1\right\rbrack}
\newcommand{\obonj}[1]{\left\rbrack#1\right\lbrack}
\newcommand{\rbonj}[1]{\left\rbrack#1\right\rbrack}
\newcommand{\lbonj}[1]{\left\lbrack#1\right\lbrack}
\newcommand{\snm}[1]{\|#1\|}           %small norma
\newcommand{\nm}[1]{\left\|#1\right\|} %norma pegadita
\newcommand{\pnm}[1]{\biggl|\biggl|#1\biggr|\biggr|}
\newcommand{\set}[1]{\{\,#1\,\}}    %% set notation
\newcommand{\floor}[1]{\lfloor#1\rfloor} %% mayor entero <= x
\newcommand{\Set}[1]{\biggl\{\,#1\,\biggr\}} %% set notation (large)


%Symbols 

\renewcommand{\geq}{\geqslant}          %% mayor o igual (variante)
\newcommand{\hookto}{\hookrightarrow}     %% inclusion arrow
\newcommand{\isom}{\simeq}              %% isomorfismo
\renewcommand{\l}{\ell}                   %% ele cursiva
\renewcommand{\leq}{\leqslant}          %% menor o igual (variante)
\newcommand{\less}{\setminus}           %% set difference
\newcommand{\To}{\Rightarrow}

\begin{document}

\begin{frame}
  \titlepage
\end{frame}

\begin{frame}{Outline}
  \tableofcontents
  % You might wish to add the option [pausesections]
\end{frame}


% Structuring a talk is a difficult task and the following structure
% may not be suitable. Here are some rules that apply for this
% solution: 

% - Exactly two or three sections (other than the summary).
% - At *most* three subsections per section.
% - Talk about 30s to 2min per frame. So there should be between about
%   15 and 30 frames, all told.

% - A conference audience is likely to know very little of what you
%   are going to talk about. So *simplify*!
% - In a 20min talk, getting the main ideas across is hard
%   enough. Leave out details, even if it means being less precise than
%   you think necessary.
% - If you omit details that are vital to the proof/implementation,
%   just say so once. Everybody will be happy with that.

\section{Espacios normados}

\subsection{Normas en $\bR^d$}

\begin{frame}{Las normas en $\bR^d$}%{Subtitles are optional.}
  % - A title should summarize the slide in an understandable fashion
  %   for anyone how does not follow everything on the slide itself.

  En $\bR^d$ tenemos las normas:
  \begin{enumerate}
    \item Eucl\'idea: $\nm{(x_1,\dots,x_d)}=(x_1^2+\dots+x_d^2)^{\frac12}$.
    \item $\nm{(x_1,\dots,x_d)}_\infty=\max\set{|x_i|:\ 1\leq i\leq d}$.
    \item $\nm{(x_1,\dots,x_d)}_p=(x_1^p+\dots+x_d^p)^{\frac1p}$.\par 
     Note que el caso de $p=2$ coincide con la norma Eucl\'idea.
  \end{enumerate}

  En base a estas normas definimos:
  $$B_p(x_0,r)=\set{x\in\bR^d:\ \nm{x-x_0}_p<r}.$$
  \emph{Si $G$ es un abierto respecto a la norma $\nm{\cdot}$, es abierto respecto a la norma $\nm{\cdot}_p$?}
\end{frame}

\begin{frame}{Estudiamos la pregunta}
    Si vale que para $x\in G$, existe $r>0$ tal que $B_2(x,r)\subseteq G$, entonces es cierto que tambi\'en existe $r_p>0$ tal que $B_p(x,r_p)\subseteq G$?\par 
    Basta probar que dado $r>0$, existe $r_p>0$ tal que 
    \begin{gather*}
        B_p(x,r_p)\subseteq B_2(x,r)\\
        \iff \set{y\in\bR^d: \nm{x-y}_p<r_p}\subseteq \set{y\in\bR^d: \nm{x-y}<r}.
    \end{gather*}
    \begin{center}
        \emph{Hay una relaci\'on entre $\nm{\cdot}_p$ y $\nm\cdot$?}
    \end{center}
    
\end{frame}

\begin{frame}{Relaci\'on entre normas}
    Consideremos $p=\infty$, entonces
    \begin{align*}
        \nm{(x_1,\dots,x_d)}&=\sqrt{x_1^2+\dots+x_d^2}\\
        &\leq (d\max_{1\leq i\leq d}|x_i|^2)^{\frac12}\\
        &=\sqrt{d}\nm{(x_1,\dots,x_d)}_\infty.
    \end{align*}
    Note que $|x_i|\leq \nm{(x_1,\dots,x_d)}$ para $1\leq i\leq d$. Es decir
    $$\nm{(x_1,\dots,x_d)}_\infty\leq \nm{(x_1,\dots,x_d)}.$$
\end{frame}

\begin{frame}{Ahora con $p$ general}
    De igual forma 
    $$\nm{(x_1,\dots,x_d)}_p\leq d^{\frac1p}\nm{(x_1,\dots,x_d)}_\infty$$
    y 
    $$\nm{(x_1,\dots,x_d)}_\infty\leq \nm{(x_1,\dots,x_d)}_p.$$
    Por lo que
    $$\nm{(x_1,\dots,x_d)}\leq \sqrt{d}\nm{(x_1,\dots,x_d)}_p.$$
\end{frame}

\begin{frame}{Ya podemos responder la pregunta!}
    Si ocurre que $\nm{x-y}_p<r_p$ entonces
    $$\nm{x-y}\leq \sqrt{d}\nm{x-y}_p<\sqrt{d}r_p.$$
    De manera que si tomamos $r_p=\frac{r}{\sqrt{d}}$, tenemos
    $$B_p(x,r_p)=B_p\left(x,\frac{r}{\sqrt{d}}\right)\subseteq B(x,r).$$
    De igual forma 
    $$B\left(x,\frac{r}{d^{\frac{1}{p}}}\right)\subseteq B_p(x,r).$$
    \begin{Lem}
        Las normas $\nm{\cdot}_p$ definen los mismos abiertos en $\bR^d$.
    \end{Lem}
\end{frame}

\section{Conexidad}

\subsection{Arcoconexidad}

\begin{frame}{Curvas y conjuntos arcoconexos}
    Una \alert{curva} es un funci\'on $\ga:\ [a,b]\to\bR^d$ continua.
    \begin{Def}
        Decimos que $E\subseteq\bR^d$ es arcoconexo si dados $x_0, x_1$ en $E$, existe $\ga$ una curva tal que $\ga(a)=x_0,\ \ga(b)=x$
        y $\ga(t)\in E$ para $t\in[a,b]$.
    \end{Def}        
    
    Si $\ga:[a,b]\to\bR^d$ es continua, entonces $\ga_1:[0,1]\to\bR^d$ dada por $\ga_1(s)=\ga((b-a)s+a)$ es continua. Es decir, podemos tomar $a=0$ y $b=1$ en la definición.
\end{frame}

\begin{frame}{Podemos \emph{desconectar} un arcoconexo?}
    Si $E$ es arcoconexo, pueden existir $G_0,G_1$ abiertos no vac\'ios tales que 
    $$E=G_0\cup G_1\ \text{y}\ G_0\cap G_1=\emptyset?$$\par 
    Tome $x_0\in G_0\ x_1\in G_1$. Entonces existe $\ga: [0,1]\to E$ tal que $\ga(0)=x_0$ y $\ga(1)=x_1$. \par Como $G_0$ es abierto, existe $r>0$ tal que $B(x_0,r)\subseteq G_0$. Al ser $\ga$ continua, existe $\dl>0$ tal que $\nm{\ga(0)-\ga(t)}<r$ cuando $|t|<\dl$.\par  Es decir $\nm{x_0-\ga(t)}<r$ y por tanto $\ga(t)\in B(x_0,r)\subseteq G_0$ para $0<t<\dl$.
\end{frame}

\begin{frame}{Un dibujo\dots}
    
\end{frame}

\begin{frame}
    Considere 
    $$t_0=\sup\set{t>0: \ga(s)\in G_0,\ 0\leq s\leq t}.$$
    Noe que $\frac\dl 2\leq t_0$. si $\ga(t_0)\in G_0$, existe $r_1$ tal que $B(\ga(t_0),r_1)\subseteq G_0$. Por continuidad, existe $\dl_1>0$ tal que 
    $$|t_0-s|<\dl_1\To \nm{\ga(t_0)-\ga(s)}<r_1.$$
    Luego 
    $$t_0-\dl_1<s<t_0+\dl_1\To \ga(s)\in B(\ga(t_0),r_1)\subseteq G_0.$$
    \begin{Ej}
        Si $s_1<t_0$, entonces $\ga(s)\in G_0$ para $0<s<s_1$.
    \end{Ej}
\end{frame}

\begin{frame}
    De lo anterior vale que 
    $$0<s\leq t_0-\frac{\dl_1}{2},\ t_0+\frac{\dl_1}{2}\To \ga(s)\in G_0.$$
    Esto contradice que $t_0$ sea el supremo por lo que $\ga(t_0)\in E\less G_0=G_1$.\par 
    Con un argumento similar mostramos que si $\ga(t_0)\in G_1$, existen $r_2, \dl_2$ tales que 
    $$t_0-\dl_2<s<t_0+\dl_2\To \ga(s)\in B(\ga(t_0),r_2)\subseteq G_1.$$
    En particular $t_0-\dl_2<s<t_0\To \ga(s)\in G_1\less G_0$ lo que es una contradicción.
\end{frame}

\begin{frame}{El resultado en cuestión}
    \begin{Lem}
        Sea $G$ arcoconexo y abierto. Entonces no existen $G_0, G_1$ abiertos no vac\'ios tales que $G=G_0\cup G_1$ y $G_0\cap G_1=\emptyset$.
    \end{Lem}

    \begin{center}
        \emph{Hace falta que $G$ sea abierto?}
    \end{center}
    No, pero hay que tomar $G=G\cap (G_0\cup G_1)$.
\end{frame}

\subsection{Conexidad}

\begin{frame}{Conjuntos conexos
}
    \begin{Def}
        $G$ es \alert{disconexo} si existen $G_0,G_1$ tales que 
        $$G_0\cap G, G_1\cap G\neq\emptyset = G_0\cap G_1$$
        y $G\subseteq G_0\cup G_1$. Un conjunto se dice \alert{conexo} si no es disconexo.
    \end{Def}
    \begin{Th}
        \begin{enumerate}
            \item Si $G$ es arcoconexo, es conexo.
            \item Si $G$ es abierto y conexo, es arcoconexo.
        \end{enumerate}
    \end{Th}
\end{frame}
\section*{Resumen}

\begin{frame}{Resumen}

  % Keep the summary *very short*.
  \begin{itemize}
  \item Definici\'on de normas en $\bR^d$.
  \item Equivalencia de normas y su consecuencia.
  \item Definici\'on de camino y arcoconexo.
  \item Un conjunto arcoconexo no es disconexo.
  \item Definci\'on de conexidad y resultados.
  \end{itemize}
  
  \vskip0pt plus.5fill
  \begin{itemize}
  \item
    Ejercicios
    \begin{itemize}
    \item Detalle en prueba sobre que los arcoconexos no son disconexos.
    \end{itemize}
  \end{itemize}
\end{frame}



% All of the following is optional and typically not needed. 
\appendix
\section<presentation>*{\appendixname}
\subsection<presentation>*{Lectura adicional}

\begin{frame}[allowframebreaks]
  \frametitle<presentation>{Lecturas adicionales}
    
  \begin{thebibliography}{10}
    
  \beamertemplatebookbibitems
  % Start with overview books.

  \bibitem{CambroNotas}
    S.Cambronero.
    \newblock {\em Notas MA0505}.
    \newblock 20XX.

    \bibitem{NachoNotas}
    I.Rojas
    \newblock {\em Notas MA0505}.
    \newblock 2018.
 
  \end{thebibliography}
  \alert{Existen m\'as pruebas que muestran que un arcoconexo es conexo. Investigue!}
\end{frame}

\end{document}


