\documentclass[utf8]{beamer}

\mode<presentation>
{
  \usetheme{Warsaw}
  \setbeamercovered{transparent}
}


\usepackage{amsfonts,mathtools,amssymb}
\usepackage[spanish]{babel}
\usepackage{times}
\usepackage[T1]{fontenc}

\title[MA0505]{MA0505 - An\'alisis I}
\subtitle{Lecci\'on VI: Completitud}

\author{Pedro M\'endez\inst{1}}

\institute[Universidad de Costa Rica] % (optional, but mostly needed)
{
  \inst{1}%
  Departmento de Matem\'atica Pura y Ciencias Actuariales\\
  Universidad de Costa Rica
  }

\date[I-2021] {Semestre I, 2021}

%%%%%%%%% === Theorems and suchlike === %%%%%%%%%%%%%%

\theoremstyle{plain}
\newtheorem{Th}{Teorema}               %%% Theorem 1.1.1
\newtheorem{Tmon}{Teoremón}
\newtheorem{Prop}{Proposición}         %%% Proposition 1.1.2
\newtheorem{Lem}{Lema}                 %%% Lemma 3
\newtheorem{Cor}{Corolario}            %%% Corollary 4

\theoremstyle{definition}
\newtheorem{Def}{Definición}           %%% Definition 5
\newtheorem{Ex}{Ejemplo}               %%% Example 6
\newtheorem{Ej}{Ejercicio}             %%% Ejercicio 7
\newtheorem{Hec}[Th]{Hecho}            %%% Hecho 1.1.8

\theoremstyle{remark}
\newtheorem{Rmk}[Th]{Observación}      %%%Remark 1.1.9
\newtheorem*{nonum-Rmk}{Observación}         %%% No number Fact
\newtheorem*{Notn}{Notaci\'on}        %% Notaciones
\newtheorem*{Warn}{Advertencia}       %% Advertencias

\numberwithin{equation}{section}

% Greek letters:

\newcommand{\al}{\alpha}                %% short for  \alpha
\newcommand{\bt}{\beta}                 %% short for  \beta
\newcommand{\Dl}{\Delta}                %% short for  \Delta
\newcommand{\dl}{\delta}                %% short for  \delta
\newcommand{\eps}{\varepsilon}          %% short for  \varepsilon
\newcommand{\Ga}{\Gamma}                %% short for  \Gamma
\newcommand{\ga}{\gamma}                %% short for  \gamma
\newcommand{\kp}{\kappa}                %% short for  \kappa
\newcommand{\La}{\Lambda}               %% short for  \Lambda
\newcommand{\la}{\lambda}               %% short for  \lambda
\newcommand{\Om}{\Omega}                %% short for  \Omega
\newcommand{\om}{\omega}                %% short for  \omega
\newcommand{\Sg}{\Sigma}                %% short for  \Sigma
\newcommand{\sg}{\sigma}                %% short for  \sigma
\newcommand{\Te}{\Theta}                %% short for  \Theta
\newcommand{\te}{\theta}                %% short for  \theta
\newcommand{\ups}{\upsilon}             %% short for  \upsilon
\newcommand{\vf}{\varphi}               %% short for  \varphi
\newcommand{\ze}{\zeta}                 %% short for  \zeta

%Boldface letters

\newcommand{\bC}{\mathbb{C}}    %%% números complejos
\newcommand{\bN}{\mathbb{N}}    %%% números naturales
\newcommand{\bP}{\mathbb{P}}        %% números enteros positivos
\newcommand{\bQ}{\mathbb{Q}}    %%% números racionales
\newcommand{\bR}{\mathbb{R}}    %%% números reales
\newcommand{\bS}{\mathbb{S}}    %%% esfera
\newcommand{\bZ}{\mathbb{Z}}    %%% números enteros

%Script letters:

\newcommand{\cA}{\mathcal{A}}           %% formas diferenciales
\newcommand{\cB}{\mathcal{B}}           %% una base vectorial
\newcommand{\cC}{\mathcal{C}}           %% otra base vectorial
\newcommand{\cD}{\mathcal{D}}           %% funciones de prueba
\newcommand{\cE}{\mathcal{E}}           %% un modulo proyectivo
\newcommand{\cF}{\mathcal{F}}           %% espacio de Fock
\newcommand{\cG}{\mathcal{G}}           %% funtor de Gelfand
\newcommand{\cH}{\mathcal{H}}           %% espacio de Hilbert
\newcommand{\cI}{\mathcal{I}}           %% un funtor de inclusion
\newcommand{\cJ}{\mathcal{J}}           %% otro funtor
\newcommand{\cK}{\mathcal{K}}           %% otro espacio de Hilbert
\newcommand{\cL}{\mathcal{L}}           %% operadores lineales
\newcommand{\cM}{\mathcal{M}}           %% multiplicadores
\newcommand{\cN}{\mathcal{N}}           %% funciones nulas
\newcommand{\cO}{\mathcal{O}}           %% funciones de crec-to lento
\newcommand{\cP}{\mathcal{P}}           %% una particion
\newcommand{\cR}{\mathcal{R}}           %% funciones representativas
\newcommand{\cQ}{\mathcal{Q}}           %% otra particion
\newcommand{\cS}{\mathcal{S}}           %% funciones de Schwartz
\newcommand{\cT}{\mathcal{T}}           %% una topologia
\newcommand{\cU}{\mathcal{U}}           %% cubrimiento abierto
\newcommand{\cV}{\mathcal{V}}           %% vecindarios
\newcommand{\cW}{\mathcal{W}}           %% grupo de Weyl


%Brackets

\newcommand{\bonj}[1]{\left\lbrack#1\right\rbrack}
\newcommand{\obonj}[1]{\left\rbrack#1\right\lbrack}
\newcommand{\rbonj}[1]{\left\rbrack#1\right\rbrack}
\newcommand{\lbonj}[1]{\left\lbrack#1\right\lbrack}
\newcommand{\snm}[1]{\|#1\|}           %small norma
\newcommand{\nm}[1]{\left\|#1\right\|} %norma pegadita
\newcommand{\pnm}[1]{\biggl|\biggl|#1\biggr|\biggr|}
\newcommand{\set}[1]{\{\,#1\,\}}    %% set notation
\newcommand{\floor}[1]{\lfloor#1\rfloor} %% mayor entero <= x
\newcommand{\Set}[1]{\biggl\{\,#1\,\biggr\}} %% set notation (large)
\newcommand\quot[2]{
        \mathchoice
            {% \displaystyle
                \text{\raise1ex\hbox{$#1$}\Big/\lower1ex\hbox{$#2$}}%
            }
            {% \textstyle
                {^{ #1}/_{ #2}}
            }
            {% \scriptstyle
                {^{ #1}/_{ #2}}
            }
            {% \scriptscriptstyle
                {^{ #1}/_{ #2}}
            }
    }
\newcommand*\squot[2]{{^{ #1}/_{ #2}}}%%%small quotient

%Symbols 

\renewcommand{\geq}{\geqslant}          %% mayor o igual (variante)
\newcommand{\hookto}{\hookrightarrow}     %% inclusion arrow
\newcommand{\isom}{\simeq}              %% isomorfismo
\renewcommand{\l}{\ell}                   %% ele cursiva
\renewcommand{\leq}{\leqslant}          %% menor o igual (variante)
\newcommand{\less}{\setminus}           %% set difference
\newcommand{\To}{\Rightarrow}
\newcommand{\ov}{\overline}
\newcommand{\un}{\underline}
\newcommand{\del}{\partial}

%%% Small fractions in displays:

\newcommand{\half}{{\mathchoice{\nhalf}{\thalf}{\shalf}{\shalf}}} %%display text script script^2
\newcommand{\happi}{{\tfrac{\pi}{2}}} %% small fraction  \pi/2
\newcommand{\quarter}{\tfrac{1}{4}} %% small fraction  1/4
\newcommand{\nhalf}{\frac{1}{2}}
\newcommand{\shalf}{{\scriptstyle\frac{1}{2}}} %% tiny fraction 1/2
\newcommand{\thalf}{{\tfrac{1}{2}}} %% small fraction  1/2
\newcommand{\third}{\tfrac{1}{3}}   %% small fraction  1/3 %Hay que renew porque mathabx toma second y third como x'' y x''' por ejemplo

\newcommand{\ihalf}{{\tfrac{i}{2}}} %% small fraction  i/2

\newcommand{\sucm}{_{m=1}^\infty} %% diminutivo
\newcommand{\suck}{_{k=1}^\infty} %% diminutivo
\newcommand{\sucn}{_{n=1}^\infty} %% diminutivo

\begin{document}

\begin{frame}
  \titlepage
\end{frame}

\begin{frame}{Agenda}
  \tableofcontents
  % You might wish to add the option [pausesections]
\end{frame}


% Structuring a talk is a difficult task and the following structure
% may not be suitable. Here are some rules that apply for this
% solution: 

% - Exactly two or three sections (other than the summary).
% - At *most* three subsections per section.
% - Talk about 30s to 2min per frame. So there should be between about
%   15 and 30 frames, all told.

% - A conference audience is likely to know very little of what you
%   are going to talk about. So *simplify*!
% - In a 20min talk, getting the main ideas across is hard
%   enough. Leave out details, even if it means being less precise than
%   you think necessary.
% - If you omit details that are vital to the proof/implementation,
%   just say so once. Everybody will be happy with that.

\section{Definición de Completitud}

\begin{frame}{Un Recordatorio\dots}%{Subtitles are optional.}
  % - A title should summarize the slide in an understandable fashion
  %   for anyone how does not follow everything on the slide itself.
  A diferencia de \emph{compacidad}, este concepto es intr\'inseco a los espacios m\'etricos.\par 
  Recordemos que una sucesi\'on $(x_n)_{n=1}^\infty$ es de \alert{Cauchy} si para $\eps>0$, existe un $n_0\in\bN$ tal que 
  $$n,m\geq n_0\To d(x_n,x_m)<\eps.$$
  \begin{Ej}\label{ej:RecordatorioCauchy}
    \begin{itemize}
\item Toda sucesi\'on convergente es de Cauchy.
\item Toda sucesi\'on de Cauchy es acotada.
    \end{itemize}   
  \end{Ej}
\end{frame}

\begin{frame}{La Definición}
    \begin{Def}\label{def:EspacioCompleto}
    A un espacio $(X,d)$ le llamamos \alert{completo} si toda sucesi\'on de Cauchy en $X$ es convergente.
    \end{Def}
Como ejemplo consideremos el espacio $\set{f:\ [0,1]\to\bR,\ f\ \text{continua}}$. La distancia es
$$d_\infty(f,g)=\sup_{x\in[0,1]}|f(x)-g(x)|.$$
Si $x\in[0,1]$ y $(f_n)_{n=1}^\infty$ es de Cauchy, entonces existe $n_0\in\bN$ tal que $d_\infty(f_n,f_m)<\frac{\eps}{2}$ cuando $n,m\geq n_0$.
\end{frame}

\begin{frame}{Continuamos el ejemplo}
  Entonces vale que 
  $$|f_n(x)-f_m(x)|\leq d_\infty(f_n,f_m)<\eps$$
  por lo que $(f_n(x))_{n=1}^\infty$ es de Cauchy en $\bR$. Al ser $\bR$ completo, llamemos
  $$f(x)=\lim_{n\to\infty}f_n(x)$$
  Vamos a mostrar que $f\in X$, es decir que $f$ es continua. Para ello, si $x\in[0,1]$, existe $m_1\geq n_0$ tal que 
  $$m\geq m_1\To |f(x)-f_m(x)|<\frac{\eps}{2}.$$
\end{frame}

\begin{frame}{¡Aquí concluimos!}
  Entonces 
  \begin{align*}
    |f(x)-f_n(x)|&\leq |f(x)-f_{m_1}(x)|+|f_{m_1}(x)-f_n(x)|\\
    &<\half\eps+\half\eps=\eps
  \end{align*}
  cuando $n\geq n_0$.\par 
  Por lo tanto $f_n\xrightarrow[n\to\infty]{} f$ uniformemente y concluimos que $f$ es continua.
  \begin{Rmk}
    Como es usual para probar que una sucesión de Cauchy converge, empezamos encontrando un candidato para el límite.
  \end{Rmk}
\end{frame}

\section{Topología y Completitud}

\subsection{Cerrados}

\begin{frame}{Espacios Completos y Conjuntos Cerrados}
  Tomemos $C\subseteq X$ cerrado y $X$ completo. Sea $(x_n)\sucn\subseteq C\subseteq X$ de Cauchy.\par 
  Como $X$ es completo, existe $x_0\in X$:
  $$x_n\xrightarrow[n\to\infty]{}x_0.$$
  Como $C$ es cerrado, $x_0\in C$ y así $(C,d)$ es un espacio completo.\par 
  
\end{frame}

\begin{frame}{El Otro Lado}
  Por otro lado si $(C,d)$ es completo con $C\subseteq X$, tomemos $(x_n)\sucn\subseteq C$ y $x_0\in\ov C$ tal que en $X$ vale:
  $$x_n\xrightarrow[n\to\infty]{}x_0.$$
  Dado que toda sucesión convergente es de Cauchy, tenemos que para $\eps>0$, existe $n_0$ tal que 
  $$n,m\geq n_0\To d(x_n,x_m)<\eps.$$
  Entonces $(x_n)\sucn$ es de Cauchy en $(C,d)$ y por tanto 
  $$x_n\xrightarrow[n\to\infty]{}x_0'\in C\To \ov C\subseteq C.$$
  %Es decir, $x_0=x_0'\in C$ lo que nos dice que $\ov C\subseteq C$.
\end{frame}

\begin{frame}{El Resultado}
  \begin{Lem}\label{lem:completoYCerrado}
    Sea $(X,d)$ un espacio métrico, entonces 
    \begin{enumerate}
      \item Si $(C,d)$ es completo, entonces $C$ es cerrado en $(X,d)$.
      \item Si $C\subseteq X$ es cerrado, entonces $(C,d)$ es completo.
    \end{enumerate}
  \end{Lem}
\end{frame}

\subsection{Compactos}

\begin{frame}{Analicemos los Compactos}
  Por otro lado si $(X,d)$ es compacto y $(x_n)\sucn$ es de Cauchy, existe $(x_{n_k})\suck\subseteq(x_n)\sucn$ que converge a $x_0\in X$.\par 
  Existe $n_0$ tal que 
  $$n,m\geq n_0\To d(x_n,x_m)<\half\eps.$$
  Sea $k_0\geq n_0$ tal que 
  $$k\geq k_0\To d(x_0,x_{n_k})<\half\eps.$$
\end{frame}

\begin{frame}{Juntamos lo Anterior}
  Vale que 
  \begin{align*}
    d(x_n,x_0)&\leq  d(x_n,x_{n_k})+d(x_{n_k},x_0)\\
    &<\half\eps+\half\eps=\eps
  \end{align*}
  cuando $n\geq n_0$. 
  \begin{Lem}\label{lem:compactoToCompleto}
    Sea $(X,d)$ un espacio métrico compacto. Entonces $(X,d)$ es completo.
  \end{Lem}

  \begin{Rmk}
    Compacto $\To$ completo. Pero completo $\not\To$ compacto. Vea por ejemplo que $\bR$ es completo pero no compacto. 
  \end{Rmk}
\end{frame}

\begin{frame}{¿Completo + (Algo) $\To$ Compacto?}
  La siguiente propiedad junto con completitud implica compacidad.
  \begin{Def}\label{def:paracompacidad}
    Llamamos a un espacio métrico \alert{totalmente acotado} (o \emph{paracompacto}) si para $\eps>0$ existen $x_1,\dots,x_m$ tales que
    $$X\subseteq \bigcup_{i=1}^mB(x_i,\eps).$$
  \end{Def}
  Note que si
  $$X\subseteq \bigcup_{x\in X}B(x,\eps),$$
  entonces todo espacio compacto es totalmente acotado.
\end{frame}

\begin{frame}{¿Acotado y Totalmente Acotado?}
  \begin{Ex}
    Consideremos la métrica definida en $\bN$: 
    $$\rho(n,m)=\left\lbrace\begin{aligned}
      1,\ n\neq m,\\
      0,\ n=m.
    \end{aligned}\right.$$
    Podemos ver que $\bN= B(n,2)$ para cualquier $n$. Sin embargo $B\left(n,\half\right)=\set{n}$ y por lo tanto no podemos cubrir $\bN$ por un número finito de bolas de radio $\half$.\par 
    Se sigue que $(\bN,\rho)$ es acotado pero no totalmente acotado.
  \end{Ex}
  \end{frame}

\begin{frame}{Vizlumbrando la relación}
  Analicemos la relación entre compacidad, completitud y ser totalmente acotado.\par 
  Tomemos $(x_n)\sucn$ en $(X,d)$ totalmente acotado.
  \begin{Ej}\label{ej:sucesiones}
    Si el conjunto $\set{x_n:\ n\in \bN}$ es finito, entonces existe $(x_{n_k})\suck\subseteq(x_n)\sucn$ que converge a un punto de la sucesión.
  \end{Ej}
  Así, asumimos que $(x_n)\sucn$ tiene una cantidad infinita de puntos distintos. Vamos a ver que existe una subsucesión de Cauchy.
\end{frame}

\begin{frame}
  Como $X$ es totalmente acotado, existen $y_1,\dots,y_m$ tales que 
  $$X\subseteq \bigcup_{i=1}^m B(y_i,1).$$
  Luego existe $y_{i_1}$ tal que $B(y_{i_1},1)$ contiene una cantidad infinita de puntos de $(x_n)\sucn$. Sea $(x_{n_{k,1}})\suck$ la subsucesión de todos los puntos dentro de $B(y_{i_1},1)$.
  \begin{Rmk}
    Recordemos que $n_{k,1}\leq n_{k+1,1}$ se cumple puesto que estamos en $\bN$.
  \end{Rmk}
\end{frame}

\begin{frame}
  De igual forma existe $\tilde{y}_2\in X$ con $B\left(\tilde{y}_2,\half\right)$ que contiene infinitos puntos de $(x_{n_{k,1}})\suck$.\par 
  Llamemos $(x_{n_{k,2}})\suck$ a la subsucesión de estos puntos. Iterando el proceso, existe
  $$(x_{n_{k,\l}})\suck\subseteq B\left(\tilde{y}_\l,\frac1\l\right)$$
   y un punto $\tilde{y}_{\l+1}$ tal que $B\left(\tilde{y}_{\l+1},\frac{1}{\l+1}\right)$ contiene una cantidad infinita de puntos de la subsucesión $(x_{n_{k,\l}})\suck$.
\end{frame}

\begin{frame}
  Sea $(x_{n_{k,\l+1}})\suck$ la subsucesión de estos puntos. Note que 
  $$d(x_{n_{k,\l+1}},x_{n_{s,\l+1}})<\frac{2}{\l+1}.$$
  Consideremos la subsucesión de los $x_{n_{\l,\l}}$. Si $\l\leq s$, entonces 
  $$d(x_{n_{\l,\l}},x_{n_{s,s}})\leq \frac{2}{\l+1}$$
  lo que nos permite concluir que $(x_{n_{\l,\l}})_{\l=1}^\infty$ es de Cauchy.
\end{frame}

\begin{frame}{El Resultado}
  \begin{Lem}\label{lem:caractParacompacidadCauchy}
  Un espacio $(X,d)$ es totalmente acotado si y sólo si toda sucesión $(x_n)\sucn\subseteq X$ poseé una subsucesión de Cauchy.
  \end{Lem}
  Con lo anterior hemos probado una dirección, resta por ver la otra.
\end{frame}

\begin{frame}{La Otra Dirección}
  En la otra dirección, supongamos que $(X,d)$ no es totalmente acotado. Entonces existe $\eps>0$ tal que
    $$X\not\subseteq \bigcup_{i=1}^m B(x_i,\eps)$$
    para cualquier conjunto $\set{x_1,\dots,x_m}$. Dado $x_0\in X$, existe $x_1\in X\less B(x_0,\eps)$.\par 
  Iteramos y obtenemos que existe $x_n$:
    $$x_n\in X\less \bigcup_{i=1}^{n-1}B(x_i,\eps).$$
    Así obtenemos una sucesión tal que $d(x_i,x_j)\geq \eps$ cuando $i\neq j$. 
\end{frame}

\begin{frame}{Podemos probar que\dots}
  \begin{Th}\label{thm:compactoIffCompletoYParacompacto}
    Un espacio métrico es compacto si y sólo si es completo y totalmente acotado.
  \end{Th}
  Ya sabemos que todo espacio compacto es totalmente acotado y completo.\par 
  Tomemos entonces $(x_n)\sucn$, poseé $(x_{n_k})\suck\subseteq(x_n)\sucn$ una subsucesión de Cauchy. Esta converge al ser el espacio completo. 
\end{frame}

\section{Compleción}

\begin{frame}{La Idea}
  Vamos a finalizar probando que dado un espacio métrico $(X,d)$, existe un espacio métrico completo $(X^\sharp,d^\sharp)$ y una función inyectiva que preserva distancia $i:\ X\to X^\sharp$.\par 
  Note que si $X\subseteq X^\sharp$ y $(x_n)\sucn\subseteq X$ es de Cauchy, entonces existe $y\in X^\sharp$ tal que $x_n\xrightarrow[n\to\infty]{}y$. Esto significa que $X^\sharp$ contiene todos los posibles límites. Vamos a definir una clase de equivalencia entre sucesiones de Cauchy. Los límites serán las distintas clases de equivalencia.
\end{frame}

\begin{frame}{El Primer Paso}
  Definimos una métrica entre sucesiones de Cauchy. Sean $(x_n)\sucn,(y_n)\sucn$ dos sucesiones de Cauchy. Note que 
  $$d(x_m,y_m)\leq d(x_m,x_n)+d(x_n,y_n)+d(y_n,y_m).$$
  Luego 
  $$d(x_m,y_m)-d(x_n,y_n)\leq d(x_m,x_n)+d(y_m,y_n)$$
  y la simetría del argumento nos muestra que 
  $$|d(x_m,y_m)-d(x_n,y_n)|\leq d(x_m,x_n)+d(y_m,y_n).$$
  
  \end{frame}

  \begin{frame}{Una Nueva Métrica}
    Basado en lo anterior, existe $n_0$ tal que 
    $$n,m\geq n_0\To |d(x_m,y_m)-d(x_n,y_n)|<\eps$$
    y por tanto $(d(x_m,y_m))_{m=1}^\infty$ es una sucesión de Cauchy en $\bR$.\par 
    Definimos 
    $$\tilde{d}((x_n)\sucn,(y_n)\sucn)=\lim_{m\to\infty}d(x_m,y_m).$$
    
  \end{frame}

  \begin{frame}{En efecto, sí es métrica*}
    Si $x_m\xrightarrow[m\to\infty]{}z$ y $y_m\xrightarrow[m\to\infty]{}z$, entonces
    $$d(x_m,y_m)\leq d(x_m,z)+d(z,y_m)$$ converge a cero cuando $m\to\infty$. Es decir $\tilde{d}((x_n)\sucn,(y_n)\sucn)=0$. Además como $d(x_m,y_m)=d(y_m,x_m)$, entonces 
    $$\tilde{d}((x_n)\sucn,(y_n)\sucn)=\tilde{d}((y_n)\sucn,(x_n)\sucn).$$
    Y dada otra sucesión de Cauchy $(z_n)\sucn$, vale 
    \begin{gather*}
    d(x_m,y_m)\leq d(x_m,z_m)+d(z_m,y_m)\\
    \To \tilde{d}((x_n)\sucn,(y_n)\sucn)\leq \tilde{d}((x_n)\sucn,(z_n)\sucn)+\tilde{d}((z_n)\sucn,(y_n)\sucn)
    \end{gather*} 
    
  \end{frame}

  \begin{frame}{Una\dots semimétrica}
    La función $\tilde{d}$ satisface la definición de ser métrica excepto por la condición 
    $$\tilde{d}((x_n)\sucn,(y_n)\sucn)=0\iff (x_n)\sucn=(y_n)\sucn.$$
     A estas funciones les llamamos \alert{semimétricas}. Para obtener una métrica, diremos que $(x_n)\sucn\sim(y_n)\sucn$ cuando 
     $$\tilde{d}((x_n)\sucn,(y_n)\sucn)=0.$$
     
  \end{frame}
\begin{frame}{El siguiente paso}
  Vamos a probar que si
     $$X^\sharp=\quot{\set{(x_n)\sucn\subseteq X\ \text{de Cauchy}}}{\sim}$$
     es el conjunto de clases de equivalencia y 
     $$d^\sharp(\un x,\un y)=\tilde{d}((x_n)\sucn,(y_n)\sucn),$$
     con $\un x=[(x_n)\sucn]$ y $\un y=[(y_n)\sucn]$, entonces $(X^\sharp,d^\sharp)$ es el espacio buscado.
\end{frame}

\begin{frame}{El Argumento Diagonal}
  %%Do Dibujo
\end{frame}

%Llevamos 1:33:29:57 y estamos por empezar pag 14

\begin{frame}{$d^\sharp$ está bien definida}
Tomemos $(x_n)\sucn,(x_n')\sucn$ tales que $(x_n)\sucn\sim(x_n')\sucn$. Entonces vale que $\lim_{m\to\infty}d(x_m,x_m')=0$. Dada $(y_n)\sucn$ sucesión de Cauchy vale que
$$d(x_m,y_m)\leq d(x_m,x_m')+d(x_m',y_m)$$
y por tanto $\lim_{m\to\infty}d(x_m,y_m)\leq\lim_{m\to\infty}d(x_m',y_m)$. De igual forma probamos 
$$\lim_{m\to\infty}d(x_m',y_m)\leq\lim_{m\to\infty}d(x_m,y_m)$$
y así $\lim_{m\to\infty}d(x_m',y_m)=\lim_{m\to\infty}d(x_m,y_m)$.
\end{frame}

\begin{frame}{Paso 2: Densidad}
  Tomemos $x\in X$ y $x_n=x$ para $n\in\bN$. Definimos $i(x)=[(x_n)\sucn]$ y vemos que 
  $$d^\sharp(i(x),i(y))=\lim_{m\to\infty}d(x_m,y_m)=d(x,y)$$
  por lo que vemos que $i$ preserva distancias. Si $(x_n)\sucn$ es de Cauchy y $\eps>0$, existe un $n_0$ tal que 
  $$n\geq n_0\To d(x_n,x_{n_0})<\eps.$$
  Así $\lim_{m\to\infty}d(x_m,d_{n_0})\leq\eps$ por lo tanto 
  $$d^\sharp([(x_n)\sucn],i(x_{n_0}))\leq\eps$$
  y así $i(X)$ es denso en $X^\sharp$.
\end{frame}

\begin{frame}{Tercero: $X^\sharp$ es completo}
Tomemos $(\un x^m)\sucm$ una sucesión de Cauchy de clases de equivalencia y $y_n\in X$ tal que $d^\sharp(\un x^n,i(y_n))<\frac1n$. Notemos que
\begin{align*}
  &d(y_m,y_n)=d^\sharp(i(y_m),i(y_n))\\
  \leq&d^\sharp(i(y_m),\un x^m)+d^\sharp(\un x^m,\un x^n)+d^\sharp(\un x^n,i(y_n))\\
  \leq&\frac 1n+\frac1m+d^\sharp(\un x^m,\un x^n)
\end{align*}  
y con esto vemos que $(y_m)\sucm$ también es de Cauchy.
\end{frame}

\begin{frame}{Continuamos con el Tercer Paso}
Finalmente
\begin{align*}
  &d^\sharp(\un x^m,[(y_m)\sucm])\\
\leq& d^\sharp(\un x^m,i(y_m))+d^\sharp(i(y_m),[(y_m)\sucm])\\
\leq & \frac{1}{m}+d^\sharp(i(y_m),[(y_m)\sucm]).
\end{align*}  
Como $i(y_\l)\xrightarrow[\l\to\infty]{}[(y_m)\sucm]$ entonces podemos concluir que 
$$\un x^n\xrightarrow[n\to\infty]{}[(y_m)\sucm].$$
\end{frame}

\begin{frame}{El Resultado}
  \begin{Th}\label{thm:ExistenciaDeComplec}
    Sea $(X,d)$ un espacio métrico. Entonces 
    \begin{enumerate}
      \item Existe un espacio métrico $(X^\sharp,d^\sharp)$ completo y una función inyectiva $i:X\to X^\sharp$ que preserva distancias.
      \item Si $(X,d)$ es completo, entonces $(X^\sharp,d^\sharp)$ es isométrico a $(X,d)$.
    \end{enumerate}
  \end{Th}
 
\end{frame}

\begin{frame}{¡Pero falta una parte!}
  Falta que veamos la segunda parte: Tomemos $(x_n)\sucn$ de Cauchy. Si $(X,d)$ es completo, existe $y\in X$ tal que $x_n\xrightarrow[n\to\infty]{}y$.\par 
  Dado $\eps>0$, 
  $$d^\sharp([(x_n)\sucn],i(y))=\lim_{m\to\infty}d(x_m,y)=0.$$
  Por lo tanto $\un i(y)=[(x_n)\sucn]$ y así $i$ es sobreyectiva.

  \begin{Lem}\label{lem:propUnivEspacioCompleto}
    Sea $(Y,\rho)$ un espacio métrico completo y $j:X\to Y$ que preserva distancias con $j(X)\subseteq Y$ denso en $Y$. Entonces existe $\te: X^\sharp\to Y$, una isometría tal que $\te\circ i=j$.
  \end{Lem}
  La prueba de este lema es un \alert{ejercicio}.
\end{frame}
\section*{Resumen}

\subsection*{Qu\'e vimos hoy}
\begin{frame}{Resumen}

  % Keep the summary *very short*.
  \begin{itemize}
  \item Hemos recordado la definición de sucesiones de Cauchy.
  \item Qué era un espacio en el que todas las sucesiones Cauchy eran convergentes. \ref{def:EspacioCompleto}
  \item El comportamiento de conjuntos cerrados respecto a completitud. \ref{lem:completoYCerrado}
  \item Que los compactos son completos. \ref{lem:compactoToCompleto}
  \item Qué es que un conjunto sea totalmente acotado. \ref{def:paracompacidad}
  \item Una caracterización de la propiedad anterior. \ref{lem:caractParacompacidadCauchy}
  \item Una equivalencia de compacidad. \ref{thm:compactoIffCompletoYParacompacto}
  \item La existencia de un espacio completo que contiene a los espacios métricos. \ref{thm:ExistenciaDeComplec}
  \item La propiedad universal de la compleción de un espacio. \ref{lem:propUnivEspacioCompleto}
  \end{itemize}
  
\end{frame}

\subsection*{Ejercicios a trabajar}
\begin{frame}{Ejercicios}
    
  \begin{itemize}
    \item
      Lista 6
      \begin{itemize}
      \item Un recordatorio sobre sucesiones de Cauchy. \ref{ej:RecordatorioCauchy}
      \item Un detalle sobre convergencia en sucesiones que alcanzan finitos valores. \ref{ej:sucesiones}
      \item La prueba de la propiedad universal. \ref{lem:propUnivEspacioCompleto}
      \end{itemize}
    \end{itemize}
  
\end{frame}


% All of the following is optional and typically not needed. 
\appendix
\section<presentation>*{\appendixname}
\subsection<presentation>*{Lectura adicional}

\begin{frame}[allowframebreaks]
  \frametitle<presentation>{Lecturas adicionales}
    
  \begin{thebibliography}{10}
    
  \beamertemplatebookbibitems
  % Start with overview books.

  \bibitem{CambroNotas}
    S.Cambronero.
    \newblock {\em Notas MA0505}.
    \newblock 20XX.

    \bibitem{NachoNotas}
    I.Rojas
    \newblock {\em Notas MA0505}.
    \newblock 2018.
 
  \end{thebibliography}
  
\end{frame}
%% 2:10:52 48
\end{document}


