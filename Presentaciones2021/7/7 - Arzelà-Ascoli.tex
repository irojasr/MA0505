\documentclass[utf8]{beamer}

\mode<presentation>
{
  \usetheme{Warsaw}
  \setbeamercovered{transparent}
}


\usepackage{amsfonts,mathtools,amssymb}
\usepackage[spanish]{babel}
\usepackage{times}
\usepackage[T1]{fontenc}

\title[MA0505]{MA0505 - An\'alisis I}
\subtitle{Lecci\'on VII: Arzelà-Ascoli}

\author{Pedro M\'endez\inst{1}}

\institute[Universidad de Costa Rica] % (optional, but mostly needed)
{
  \inst{1}%
  Departmento de Matem\'atica Pura y Ciencias Actuariales\\
  Universidad de Costa Rica
  }

\date[I-2021] {Semestre I, 2021}

%%%%%%%%% === Theorems and suchlike === %%%%%%%%%%%%%%

\theoremstyle{plain}
\newtheorem{Th}{Teorema}               %%% Theorem 1.1.1
\newtheorem{Tmon}{Teoremón}
\newtheorem{Prop}{Proposición}         %%% Proposition 1.1.2
\newtheorem{Lem}{Lema}                 %%% Lemma 3
\newtheorem{Cor}{Corolario}            %%% Corollary 4

\theoremstyle{definition}
\newtheorem{Def}{Definición}           %%% Definition 5
\newtheorem{Ex}{Ejemplo}               %%% Example 6
\newtheorem{Ej}{Ejercicio}             %%% Ejercicio 7
\newtheorem{Hec}[Th]{Hecho}            %%% Hecho 1.1.8

\theoremstyle{remark}
\newtheorem{Rmk}[Th]{Observación}      %%%Remark 1.1.9
\newtheorem*{nonum-Rmk}{Observación}         %%% No number Fact
\newtheorem*{Notn}{Notaci\'on}        %% Notaciones
\newtheorem*{Warn}{Advertencia}       %% Advertencias

\numberwithin{equation}{section}

% Greek letters:

\newcommand{\al}{\alpha}                %% short for  \alpha
\newcommand{\bt}{\beta}                 %% short for  \beta
\newcommand{\Dl}{\Delta}                %% short for  \Delta
\newcommand{\dl}{\delta}                %% short for  \delta
\newcommand{\eps}{\varepsilon}          %% short for  \varepsilon
\newcommand{\Ga}{\Gamma}                %% short for  \Gamma
\newcommand{\ga}{\gamma}                %% short for  \gamma
\newcommand{\kp}{\kappa}                %% short for  \kappa
\newcommand{\La}{\Lambda}               %% short for  \Lambda
\newcommand{\la}{\lambda}               %% short for  \lambda
\newcommand{\Om}{\Omega}                %% short for  \Omega
\newcommand{\om}{\omega}                %% short for  \omega
\newcommand{\Sg}{\Sigma}                %% short for  \Sigma
\newcommand{\sg}{\sigma}                %% short for  \sigma
\newcommand{\Te}{\Theta}                %% short for  \Theta
\newcommand{\te}{\theta}                %% short for  \theta
\newcommand{\ups}{\upsilon}             %% short for  \upsilon
\newcommand{\vf}{\varphi}               %% short for  \varphi
\newcommand{\ze}{\zeta}                 %% short for  \zeta

%Boldface letters

\newcommand{\bC}{\mathbb{C}}    %%% números complejos
\newcommand{\bN}{\mathbb{N}}    %%% números naturales
\newcommand{\bP}{\mathbb{P}}        %% números enteros positivos
\newcommand{\bQ}{\mathbb{Q}}    %%% números racionales
\newcommand{\bR}{\mathbb{R}}    %%% números reales
\newcommand{\bS}{\mathbb{S}}    %%% esfera
\newcommand{\bZ}{\mathbb{Z}}    %%% números enteros

%Script letters:

\newcommand{\cA}{\mathcal{A}}           %% formas diferenciales
\newcommand{\cB}{\mathcal{B}}           %% una base vectorial
\newcommand{\cC}{\mathcal{C}}           %% otra base vectorial
\newcommand{\cD}{\mathcal{D}}           %% funciones de prueba
\newcommand{\cE}{\mathcal{E}}           %% un modulo proyectivo
\newcommand{\cF}{\mathcal{F}}           %% espacio de Fock
\newcommand{\cG}{\mathcal{G}}           %% funtor de Gelfand
\newcommand{\cH}{\mathcal{H}}           %% espacio de Hilbert
\newcommand{\cI}{\mathcal{I}}           %% un funtor de inclusion
\newcommand{\cJ}{\mathcal{J}}           %% otro funtor
\newcommand{\cK}{\mathcal{K}}           %% otro espacio de Hilbert
\newcommand{\cL}{\mathcal{L}}           %% operadores lineales
\newcommand{\cM}{\mathcal{M}}           %% multiplicadores
\newcommand{\cN}{\mathcal{N}}           %% funciones nulas
\newcommand{\cO}{\mathcal{O}}           %% funciones de crec-to lento
\newcommand{\cP}{\mathcal{P}}           %% una particion
\newcommand{\cR}{\mathcal{R}}           %% funciones representativas
\newcommand{\cQ}{\mathcal{Q}}           %% otra particion
\newcommand{\cS}{\mathcal{S}}           %% funciones de Schwartz
\newcommand{\cT}{\mathcal{T}}           %% una topologia
\newcommand{\cU}{\mathcal{U}}           %% cubrimiento abierto
\newcommand{\cV}{\mathcal{V}}           %% vecindarios
\newcommand{\cW}{\mathcal{W}}           %% grupo de Weyl


%Brackets

\newcommand{\bonj}[1]{\left\lbrack#1\right\rbrack}
\newcommand{\obonj}[1]{\left\rbrack#1\right\lbrack}
\newcommand{\rbonj}[1]{\left\rbrack#1\right\rbrack}
\newcommand{\lbonj}[1]{\left\lbrack#1\right\lbrack}
\newcommand{\snm}[1]{\|#1\|}           %small norma
\newcommand{\nm}[1]{\left\|#1\right\|} %norma pegadita
\newcommand{\pnm}[1]{\biggl|\biggl|#1\biggr|\biggr|}
\newcommand{\set}[1]{\{\,#1\,\}}    %% set notation
\newcommand{\floor}[1]{\lfloor#1\rfloor} %% mayor entero <= x
\newcommand{\Set}[1]{\biggl\{\,#1\,\biggr\}} %% set notation (large)
\newcommand\quot[2]{
        \mathchoice
            {% \displaystyle
                \text{\raise1ex\hbox{$#1$}\Big/\lower1ex\hbox{$#2$}}%
            }
            {% \textstyle
                {^{ #1}/_{ #2}}
            }
            {% \scriptstyle
                {^{ #1}/_{ #2}}
            }
            {% \scriptscriptstyle
                {^{ #1}/_{ #2}}
            }
    }
\newcommand*\squot[2]{{^{ #1}/_{ #2}}}%%%small quotient

%Symbols 

\renewcommand{\geq}{\geqslant}          %% mayor o igual (variante)
\newcommand{\hookto}{\hookrightarrow}     %% inclusion arrow
\newcommand{\isom}{\simeq}              %% isomorfismo
\renewcommand{\l}{\ell}                   %% ele cursiva
\renewcommand{\leq}{\leqslant}          %% menor o igual (variante)
\newcommand{\less}{\setminus}           %% set difference
\newcommand{\To}{\Rightarrow}
\newcommand{\ov}{\overline}
\newcommand{\un}{\underline}
\newcommand{\del}{\partial}

%%% Small fractions in displays:

\newcommand{\half}{{\mathchoice{\nhalf}{\thalf}{\shalf}{\shalf}}} %%display text script script^2
\newcommand{\happi}{{\tfrac{\pi}{2}}} %% small fraction  \pi/2
\newcommand{\quarter}{\tfrac{1}{4}} %% small fraction  1/4
\newcommand{\nhalf}{\frac{1}{2}}
\newcommand{\shalf}{{\scriptstyle\frac{1}{2}}} %% tiny fraction 1/2
\newcommand{\thalf}{{\tfrac{1}{2}}} %% small fraction  1/2
\newcommand{\third}{\tfrac{1}{3}}   %% small fraction  1/3 %Hay que renew porque mathabx toma second y third como x'' y x''' por ejemplo

\newcommand{\ihalf}{{\tfrac{i}{2}}} %% small fraction  i/2

\newcommand{\sucm}{_{m=1}^\infty} %% diminutivo
\newcommand{\suck}{_{k=1}^\infty} %% diminutivo
\newcommand{\sucn}{_{n=1}^\infty} %% diminutivo

\begin{document}

\begin{frame}
  \titlepage
\end{frame}

\begin{frame}{Agenda}
  \tableofcontents
  % You might wish to add the option [pausesections]
\end{frame}


% Structuring a talk is a difficult task and the following structure
% may not be suitable. Here are some rules that apply for this
% solution: 

% - Exactly two or three sections (other than the summary).
% - At *most* three subsections per section.
% - Talk about 30s to 2min per frame. So there should be between about
%   15 and 30 frames, all told.

% - A conference audience is likely to know very little of what you
%   are going to talk about. So *simplify*!
% - In a 20min talk, getting the main ideas across is hard
%   enough. Leave out details, even if it means being less precise than
%   you think necessary.
% - If you omit details that are vital to the proof/implementation,
%   just say so once. Everybody will be happy with that.

\section{El Teorema de Arzelà-Ascoli}

\begin{frame}{El Espacio de Funciones Continuas}
  \begin{itemize}
    \item Sea $(X,d)$ un espacio métrico y $K\subseteq X$ un compacto. Definimos 
    $$\cC(K,\bR)=\set{f:\ K\to\bR,\ f\ \text{continua}}$$
    y $d_\infty(f,g)=\sup_{x\in K} \set{|f(x)-g(x)|}$ para $f,g\in\cC(K,\bR)$.\par 
  \item  Sabemos que $(\cC(K,\bR),d_\infty)$ es un espacio métrico. La convergencia en este espacio, es la convergencia uniforme de funciones continuas.   
  \end{itemize}
 
\end{frame}

\begin{frame}{Compacidad en este Espacio}
  \begin{itemize}
    \item Vamos a analizar la compacidad en estos espacios. Tomemos $C\subseteq\cC(K,\bR)$ un compacto.
    \item Si $\set{f_n}\sucn\subseteq C$ vamos a probar que dados $\eps>0$ y $x_0\in K$, existe un $\dl>0$ tal que 
    $$d(x_0,y)<\dl\To|f_n(y)-f_n(x_0)|<\eps$$
    para $n\geq 1$.
  \end{itemize}
\end{frame}


\begin{frame}{Probamos esto\dots}
  \begin{itemize}
    \item Asumamos por el contrario que no, entonces existe un $\eps>0$ y $x_0\in K$ tales que para $n\in\bN$ existen $y_n,k_n$ que satisfacen
    $$d(x_0,y_n)<\frac1n\ \land\ |f_{k_n}(x_0)-f_{k_n}(y_n)|\geq\eps.$$
    \item Como $C$ es compacto, existe $\set{f_{m_n}}\sucn$ subsucesión de $\set{f_{k_n}}\sucn$ que converge uniformemente a $f: K\to\bR$ continua.
    \item Al ser $f$ continua y $y_n\xrightarrow[n\to\infty]{}x_0$ existe un $n_0$ tal que 
    $$n\geq n_0\To |f(x_n)-f(x_0)|<\frac{\eps}{3}.$$
  \end{itemize}
\end{frame}

\begin{frame}{Terminamos la Prueba\dots}
  De esta manera
  \begin{align*}
    &|f_{m_n}(x_n)-f_{m_n}(x_0)|\\
    \leq&|f_{m_n}(x_n)-f(x_n)|+|f(x_n)-f(x_0)|+|f_{m_n}(x_0)-f(x_0)|\\
    \leq&\frac{\eps}{3}+2d_\infty(f_{m_n},f)
  \end{align*}
  y esto resulta ser una contradicción.%Donde? No la veo :s un toque no afilado aquí
\end{frame}

\begin{frame}{Equicontinuidad}
  \begin{Def}\label{def:equicontinuidad}
    Sea $\set{f_\al}_{\al\in\Om}$ una familia de funciones $f_\al:X\to Y$ con $(X,d),(Y,\rho)$ espacios métricos.\par 
Diremos que $\set{f_\al}_{\al\in\Om}$ es \alert{equicontinua} en $x_0$ si para $\eps>0$, existe $\dl>0$ que satisface
$$d(x_0,y)<\dl\To\rho(f_\al(y),f_\al(x_0))<\eps$$
si $\al\in\Om$.\par
La familia entera es equicontinua si lo es en todos los puntos.
  \end{Def}
\end{frame}

\begin{frame}{Un Lema Útil}
  \begin{Lem}\label{lem:equicontToUnifConv}
    Sea $\set{f_n}\sucn$, $f_n: X\to Y$ una familia equicontinua. Sea $K\subseteq X$ un compacto tal que $f_n(x)\xrightarrow[n\to\infty]{}f_0(x)$ con $f_0$ continua en $K$. Entonces $\set{f_n}\sucn$ converge uniformemente a $f_0$ en $K$.
  \end{Lem}
\end{frame}

\begin{frame}{Probamos el Lema}
\begin{itemize}
  \item Dado $x\in K$ y $\eps>0$, existe $\dl_x>0$ tal que 
  $$y\in B(x,\dl_x)\To \rho(f_n(x),f_n(y))<\frac{\eps}{3}$$
  para $n\geq 0$.
  \item Al ser $K$ compacto, existen $x_1,\dots,x_m$ tales que 
  $$K\subseteq\bigcup\sucm B(x_i,\dl_{x_i}).$$
  \item Tome $n_0\in\bN$ tal que para $1\leq i\leq m$ valga
  $$\rho(f_n(x_i),f_0(x_i))<\frac{\eps}{3}.$$
\end{itemize}
\end{frame}

\begin{frame}{Terminamos la Prueba}
  Sea $y\in K$, entonces existe $1\leq i\leq m$ tal que $y\in B(x_i,\dl_{x_i})$ y así
  \begin{align*}
    &\rho(f_n(y),f_0(y))\\
    \leq &\rho(f_n(y),f_n(x_i))+\rho(f_n(x_i),f_0(x_i))+\rho(f_0(x_i),f_0(y))\\
    <&\frac{\eps}{3}+\frac{\eps}{3}+\frac{\eps}{3}.
  \end{align*}
En esencia, este lema nos dice que la equicontinuidad nos permite pasar de convergencia puntual a uniforme.\par 
\emph{¿Qué hacemos si no tenemos de antemano que $f$ es continua?}
\end{frame}

\begin{frame}{Solventando el Problema}
  \begin{Lem}\label{lem:casoNoContEquicont}
    Sea $f_n:X\to Y$ una sucesión equicontinua con $Y$ completo. Suponga que además $\set{f_n}\sucn$ converge para $x\in D\subseteq X$ con $D$ denso en $X$.\par 
    Entonces $f_n$ converge en $X$ y su límite es continuo.
  \end{Lem}
\end{frame}

\begin{frame}{Probamos el Lema}
Sea $x\in X$ vamos a probar que $\set{f_n(x)}\sucn$ es de Cauchy.
\begin{itemize}
  \item Por equicontinuidad existe $\dl>0$ tal que para $n\geq 1$ vale
  $$d(x,y)<\dl\To\rho(f_n(x),f_n(y))<\frac{\eps}{3}.$$
  \item Sea $y\in B(x,\dl)\cap D$, así existe $n_0$ tal que 
  $$n,m\geq n_0\To \rho(f_n(y),f_m(y))<\frac{\eps}{3}.$$
  
\end{itemize}  
\end{frame}

\begin{frame}{Terminamos la Prueba}
  Así, cuando $n,m\geq 0$ vale que
  \begin{align*}
    &\rho(f_n(x),f_m(x))\\
    \leq&\rho(f_n(x),f_n(y))+\rho(f_n(y),f_m(y))+\rho(f_m(y),f_m(x))<\eps.
  \end{align*}
  Sea $f(x)=\lim_{n\to\infty}f_n(x)$, note que 
  $$\rho(f(x),f(y))\leq\lim_{n\to\infty}\rho(f_n(x),f_n(y))\leq\frac{\eps}{3}$$
  cuando $d(x,y)<\dl$.
  \begin{Rmk}
    El resultado es válido siempre que $\ov{\set{f_n(x)}\sucn}$ sea completo. Por ejemplo, cuando el conjunto es compacto.
  \end{Rmk}
\end{frame}

\begin{frame}{Otro Lema, pero de Ejercicio}
\begin{Lem}\label{lem:equicontConvToUnif}
  Sea $f_n: X\to Y$ una familia equicontinua y $K\subseteq X$ compacto. Asuma que $\set{f_n(x)}\sucn$ converge para $x\in K$ a $f_0: X\to Y$. Entonces $f_n$ converge uniformemente en $K$.
\end{Lem}
\begin{Ej}
  ¡Pruebe el lema!
\end{Ej}
\end{frame}

\begin{frame}{El Teorema de Arzelà-Ascoli}
  \begin{Th}\label{thm:ArzelaAscoli}
    Sea $\set{f_n}\sucn$ una familia equicontinua con $f_n: X\to Y$ y $X$ separable. Supongamos que $\ov{\set{f_n(x)}\sucn}$ es compacto. Entonces existe una subsucesión de $\set{f_n}\sucn$ que converge puntualmente a una función continua $f: X\to Y$ y la convergencia es uniforme para cada compacto $K\subseteq X$. 
  \end{Th}
\end{frame}

\begin{frame}{La Prueba del Teorema}
  \begin{itemize}
    \item Sea $\set{x_n}\sucn=D\subseteq X$ denso, que existe por separabilidad. 
    \item Sabemos que $\ov{\set{f_n(x_1):n\geq 1}}$ es compacto. 
    \item Entonces existe una subsucesión $f_{n_k}^1$ tal que $\set{f_{n_k}^1(x_1)}\suck$ converge. 
    \item De igual forma $\ov{\set{f_{n_k}^1(x_2):n\geq 1}}$ es compacto. \alert{¿Por qué?}
    \item Entonces existe una subsucesión $f_{n_k}^2$ de $f_{n_k}^1$ que satisface que $\set{f_{n_k}^2(x_2)}\suck$ converge.
  \end{itemize}
\end{frame}

\begin{frame}{Continuamos la Prueba}
  \begin{itemize}
    \item Iterando, dado $f_{n_k}^m$ tal que $\set{f_{n_k}^m(x_m)}\suck$ converge, existe una subsucesión $f_{n_k}^{m+1}$ que satisface que $\set{f_{n_k}^{m+1}(x_{m+1})}\suck$ converge.
    \item Por construcción $\set{f_{n_k}^m(x_\l)}\suck$ convergencia para $1\leq \l\leq m$.
    \item Dado que para $\l\leq m$, $\set{f_{n_m}^m}_{m=\l}^\infty$ es una subsucesión de $f_{n_k}^\l$, tenemos que $\set{f_{n_m}^m(x_i)}_{m=1}^\infty$ converge para $1\leq i\leq \l$.
  \end{itemize}
\end{frame}

\begin{frame}{Terminamos la Prueba}
  Por lo tanto $\set{f_{n_m}^m}\sucm$ converge para todos los puntos de $D$.\par 
  Por los resultados anteriores, $\set{f_{n_m}^m}\sucm$ converge en $X$ a una función continua y la convergencia es uniforme en compactos.
\end{frame}

%Vamos por pg 8.2, llevamos 1:03:13 85.

\section*{Resumen}

\subsection*{Qu\'e vimos hoy}
\begin{frame}{Resumen}

  % Keep the summary *very short*.
  \begin{itemize}
  \item La definición \ref{def:equicontinuidad} de equicontinuidad.
  \item El primer lema \ref{lem:equicontToUnifConv} sobre convergencia uniforme.
  \item Qué pasaba en el caso no continuo. \ref{lem:casoNoContEquicont}
  \item El teorema \ref{thm:ArzelaAscoli} de Arzelà-Ascoli.
  \end{itemize}
  
\end{frame}

\subsection*{Ejercicios a trabajar}
\begin{frame}{Ejercicios}
    
  \begin{itemize}
    \item
      Lista 7
      \begin{itemize}
      \item A probar el lema \ref{lem:equicontConvToUnif}
      \end{itemize}
    \end{itemize}
  
\end{frame}


% All of the following is optional and typically not needed. 
\appendix
\section<presentation>*{\appendixname}
\subsection<presentation>*{Lectura adicional}

\begin{frame}[allowframebreaks]
  \frametitle<presentation>{Lecturas adicionales}
    
  \begin{thebibliography}{10}
    
  \beamertemplatebookbibitems
  % Start with overview books.

  \bibitem{CambroNotas}
    S.Cambronero.
    \newblock {\em Notas MA0505}.
    \newblock 20XX.

    \bibitem{NachoNotas}
    I.Rojas
    \newblock {\em Notas MA0505}.
    \newblock 2018.
 
  \end{thebibliography}
  
\end{frame}
%% 6 - 2:10:52 48
%% 7 - 
\end{document}


