\documentclass[utf8]{beamer}

\mode<presentation>
{
  \usetheme{Warsaw}
  \setbeamercovered{transparent}
}


\usepackage{amsfonts,mathtools,amssymb}
\usepackage[spanish]{babel}
\usepackage{times}
\usepackage[T1]{fontenc}

\title[MA0505]{MA0505 - An\'alisis I}
\subtitle{Lecci\'on V: Compacidad}

\author{Pedro M\'endez\inst{1}}

\institute[Universidad de Costa Rica] % (optional, but mostly needed)
{
  \inst{1}%
  Departmento de Matem\'atica Pura y Ciencias Actuariales\\
  Universidad de Costa Rica
  }

\date[I-2021] {Semestre I, 2021}

%%%%%%%%% === Theorems and suchlike === %%%%%%%%%%%%%%

\theoremstyle{plain}
\newtheorem{Th}{Teorema}               %%% Theorem 1.1.1
\newtheorem{Tmon}{Teoremón}
\newtheorem{Prop}{Proposición}         %%% Proposition 1.1.2
\newtheorem{Lem}{Lema}                 %%% Lemma 3
\newtheorem{Cor}{Corolario}            %%% Corollary 4

\theoremstyle{definition}
\newtheorem{Def}{Definición}           %%% Definition 5
\newtheorem{Ex}{Ejemplo}               %%% Example 6
\newtheorem{Ej}{Ejercicio}             %%% Ejercicio 7
\newtheorem{Hec}[Th]{Hecho}            %%% Hecho 1.1.8

\theoremstyle{remark}
\newtheorem{Rmk}[Th]{Observación}      %%%Remark 1.1.9
\newtheorem*{nonum-Rmk}{Observación}         %%% No number Fact
\newtheorem*{Notn}{Notaci\'on}        %% Notaciones
\newtheorem*{Warn}{Advertencia}       %% Advertencias

\numberwithin{equation}{section}

% Greek letters:

\newcommand{\al}{\alpha}                %% short for  \alpha
\newcommand{\bt}{\beta}                 %% short for  \beta
\newcommand{\Dl}{\Delta}                %% short for  \Delta
\newcommand{\dl}{\delta}                %% short for  \delta
\newcommand{\eps}{\varepsilon}          %% short for  \varepsilon
\newcommand{\Ga}{\Gamma}                %% short for  \Gamma
\newcommand{\ga}{\gamma}                %% short for  \gamma
\newcommand{\kp}{\kappa}                %% short for  \kappa
\newcommand{\La}{\Lambda}               %% short for  \Lambda
\newcommand{\la}{\lambda}               %% short for  \lambda
\newcommand{\Om}{\Omega}                %% short for  \Omega
\newcommand{\om}{\omega}                %% short for  \omega
\newcommand{\Sg}{\Sigma}                %% short for  \Sigma
\newcommand{\sg}{\sigma}                %% short for  \sigma
\newcommand{\Te}{\Theta}                %% short for  \Theta
\newcommand{\te}{\theta}                %% short for  \theta
\newcommand{\ups}{\upsilon}             %% short for  \upsilon
\newcommand{\vf}{\varphi}               %% short for  \varphi
\newcommand{\ze}{\zeta}                 %% short for  \zeta

%Boldface letters

\newcommand{\bC}{\mathbb{C}}    %%% números complejos
\newcommand{\bN}{\mathbb{N}}    %%% números naturales
\newcommand{\bP}{\mathbb{P}}        %% números enteros positivos
\newcommand{\bQ}{\mathbb{Q}}    %%% números racionales
\newcommand{\bR}{\mathbb{R}}    %%% números reales
\newcommand{\bS}{\mathbb{S}}    %%% esfera
\newcommand{\bZ}{\mathbb{Z}}    %%% números enteros

%Script letters:

\newcommand{\cA}{\mathcal{A}}           %% formas diferenciales
\newcommand{\cB}{\mathcal{B}}           %% una base vectorial
\newcommand{\cC}{\mathcal{C}}           %% otra base vectorial
\newcommand{\cD}{\mathcal{D}}           %% funciones de prueba
\newcommand{\cE}{\mathcal{E}}           %% un modulo proyectivo
\newcommand{\cF}{\mathcal{F}}           %% espacio de Fock
\newcommand{\cG}{\mathcal{G}}           %% funtor de Gelfand
\newcommand{\cH}{\mathcal{H}}           %% espacio de Hilbert
\newcommand{\cI}{\mathcal{I}}           %% un funtor de inclusion
\newcommand{\cJ}{\mathcal{J}}           %% otro funtor
\newcommand{\cK}{\mathcal{K}}           %% otro espacio de Hilbert
\newcommand{\cL}{\mathcal{L}}           %% operadores lineales
\newcommand{\cM}{\mathcal{M}}           %% multiplicadores
\newcommand{\cN}{\mathcal{N}}           %% funciones nulas
\newcommand{\cO}{\mathcal{O}}           %% funciones de crec-to lento
\newcommand{\cP}{\mathcal{P}}           %% una particion
\newcommand{\cR}{\mathcal{R}}           %% funciones representativas
\newcommand{\cQ}{\mathcal{Q}}           %% otra particion
\newcommand{\cS}{\mathcal{S}}           %% funciones de Schwartz
\newcommand{\cT}{\mathcal{T}}           %% una topologia
\newcommand{\cU}{\mathcal{U}}           %% cubrimiento abierto
\newcommand{\cV}{\mathcal{V}}           %% vecindarios
\newcommand{\cW}{\mathcal{W}}           %% grupo de Weyl


%Brackets

\newcommand{\bonj}[1]{\left\lbrack#1\right\rbrack}
\newcommand{\obonj}[1]{\left\rbrack#1\right\lbrack}
\newcommand{\rbonj}[1]{\left\rbrack#1\right\rbrack}
\newcommand{\lbonj}[1]{\left\lbrack#1\right\lbrack}
\newcommand{\snm}[1]{\|#1\|}           %small norma
\newcommand{\nm}[1]{\left\|#1\right\|} %norma pegadita
\newcommand{\pnm}[1]{\biggl|\biggl|#1\biggr|\biggr|}
\newcommand{\set}[1]{\{\,#1\,\}}    %% set notation
\newcommand{\floor}[1]{\lfloor#1\rfloor} %% mayor entero <= x
\newcommand{\Set}[1]{\biggl\{\,#1\,\biggr\}} %% set notation (large)


%Symbols 

\renewcommand{\geq}{\geqslant}          %% mayor o igual (variante)
\newcommand{\hookto}{\hookrightarrow}     %% inclusion arrow
\newcommand{\isom}{\simeq}              %% isomorfismo
\renewcommand{\l}{\ell}                   %% ele cursiva
\renewcommand{\leq}{\leqslant}          %% menor o igual (variante)
\newcommand{\less}{\setminus}           %% set difference
\newcommand{\To}{\Rightarrow}
\newcommand{\ov}{\overline}
\newcommand{\un}{\underline}
\newcommand{\del}{\partial}
\newcommand{\x}{\times}
\newcommand{\xyx}{\times\dots\times}

%%% Small fractions in displays:

\newcommand{\half}{{\mathchoice{\nhalf}{\thalf}{\shalf}{\shalf}}} %%display text script script^2
\newcommand{\happi}{{\tfrac{\pi}{2}}} %% small fraction  \pi/2
\newcommand{\quarter}{\tfrac{1}{4}} %% small fraction  1/4
\newcommand{\nhalf}{\frac{1}{2}}
\newcommand{\shalf}{{\scriptstyle\frac{1}{2}}} %% tiny fraction 1/2
\newcommand{\thalf}{{\tfrac{1}{2}}} %% small fraction  1/2
\newcommand{\third}{\tfrac{1}{3}}   %% small fraction  1/3 %Hay que renew porque mathabx toma second y third como x'' y x''' por ejemplo

\newcommand{\ihalf}{{\tfrac{i}{2}}} %% small fraction  i/2
\newcommand{\sucm}{_{m=1}^\infty} %% diminutivo
\newcommand{\suck}{_{k=1}^\infty} %% diminutivo
\newcommand{\sucn}{_{n=1}^\infty} %% diminutivo

\begin{document}

\begin{frame}
  \titlepage
\end{frame}

\begin{frame}{Agenda}
  \tableofcontents
  % You might wish to add the option [pausesections]
\end{frame}


% Structuring a talk is a difficult task and the following structure
% may not be suitable. Here are some rules that apply for this
% solution: 

% - Exactly two or three sections (other than the summary).
% - At *most* three subsections per section.
% - Talk about 30s to 2min per frame. So there should be between about
%   15 and 30 frames, all told.

% - A conference audience is likely to know very little of what you
%   are going to talk about. So *simplify*!
% - In a 20min talk, getting the main ideas across is hard
%   enough. Leave out details, even if it means being less precise than
%   you think necessary.
% - If you omit details that are vital to the proof/implementation,
%   just say so once. Everybody will be happy with that.

\section{Compacidad Secuencial}

\begin{frame}{La Definición de Compacidad Secuencial}%{Subtitles are optional.}
  % - A title should summarize the slide in an understandable fashion
  %   for anyone how does not follow everything on the slide itself.
   Dado $C\subseteq X$, diremos que $C$ es \alert{secuencialmente compacto}\label{def:compacidadSecuencial} si cualquier 
   sucesi\'on $\{x_n\}_{n=1}^\infty\subseteq C$ pose\'e una subsucesi\'on convergente.\vspace{1cm}
   
  
  Ahora si $\{x_{n_k}\}_{k=1}^\infty$ converge a $x_0$, tenemos que dado $\eps>0$, existe $k_0$ tal que 
   $$k\geq k_0\To d(x_{n_k},x_0)<\eps.$$
  
\end{frame}

\begin{frame}{Analizando la definici\'on}

  De lo anterior, 
  $$x_{n_k}\in B(x_0,\eps),$$ 
  cuando $k\geq k_0$. Dado que $n_k \geq k$ 
  $$B(x_0,\eps)\cap\set{x_m:\ m\geq k}\neq \emptyset,$$
  para $k\geq 1$.
  Por lo tanto, para todo  $k\geq 1$
  $$x_0\in \ov{\set{x_m:\ m\geq k}},$$   y en consecuencia 
  $$x_0\in \bigcap_{k=1}^\infty\ov{\set{x_m:\ m\geq k}}.$$

\end{frame}

\begin{frame}{Analizando la definici\'on}
  Por otro lado, si  $x_0 \in \bigcap_{k=1}^\infty\ov{\set{x_m:\ m\geq k}}$ podemos tomar iterativamente:
  
  \begin{itemize}
    \item $x_{k_1}\in B(x_0,1)\cap\set{x_m:\ m\geq 1}$.
    \item $x_{k_2}\in B\left(x_0,\half\right)\cap\set{x_m:\ m\geq k_1+1}$.
  \end{itemize}
  \hspace{1cm}$\vdots$
  \begin{itemize}
    \item $x_{k_{n+1}}\in B\left(x_0,\frac1n\right)\cap\set{x_m:\ m\geq k_{n}+1}$.
  \end{itemize}
  As\'i $$x_{k_n}\to x_0,$$ 
  y 
  \[\{x_{n_k}\}_{k=1}^{\infty} \text{ es una subsucesi\'on de } \{x_n\}_{n=1}^\infty.\]
\end{frame}

\begin{frame}{Condensamos lo anterior}
  \begin{Lem}\label{lem:equivSecCompYEncajados}
    Si $(X,d)$ es un espacio m\'etrico y $C\subseteq X$, son equivalentes
    \begin{enumerate}
      \item $C$ es secuencialmente compacto.
      \item $C\cap \bigcap_{k=1}^\infty\ov{\set{x_m:\ m\geq k}}\neq \emptyset$ para toda $\{x_n\}_{n=1}^\infty\subseteq C$.
    \end{enumerate}
  \end{Lem}
\end{frame}

\begin{frame}{Propiedades}
  Sea $x\in\ov C$, entonces existe $\{x_n\}_{n=1}^{\infty}\subseteq C$ tal que $x_n\to x$. Si $C$ es \emph{secuencialmente 
  compacto}, existe una subsucesi\'on $\{x_{n_k}\}_{k=1}^\infty$ que converge a un punto de $C$. 
  Como el l\'{i}mite es \'{u}nico, tenemos que 
  $$z\in C.$$
   
  \begin{Lem}\label{lem:equiv1Compacidad}
Sea $C$ secuencialmente compacto, entonces $C$ es cerrado y acotado.
  \end{Lem}

  \begin{Ej}
    Complete la prueba del lema anterior.
  \end{Ej}
\end{frame}

\section{Compacidad}

\begin{frame}{Cubrimientos por abiertos}
  \begin{Def}\label{def:recubrimiento}
  Una colecci\'on $\cU=\set{U_\al:\ \al\in\La}$ de abiertos es llamado un \alert{recubrimiento} de un conjunto $A$ si 
  $$A\subseteq \bigcup_{\al\in\La}U_\al.$$
  \end{Def}
  \begin{Lem}\label{lem:lemaTecnico}
    Sea $C$  secuencialmente compacto y $\cU$  un recubrimiento de $C$. Entonces existe $\eps>0$ tal que para todo  $x\in C$, existe $U\in\cU$ 
    que satisface
    $$B(x,\eps) \subseteq U.$$
  \end{Lem}
\end{frame}

\begin{frame}{Prueba del lema}
  Supongamos que para todo $\eps>0$ existe $x_\eps\in C$ tal que 
  $$B(x_\eps,\eps)\subsetneq U_\al,$$
  para todo $\ \al\in \La.$ En particular, para todo $n \in\bN$  existe 
  $$\{x_n\}_{n=1}^\infty\subseteq C$$
  tal que 
  $$B\left(x_n,\frac1n\right)\subsetneq U_\al,$$
  para $\al\in \La$.
  
    
  
  

\end{frame}

\begin{frame}{Prueba del lema}
  Al ser $C$ secuencialmente compacto, existe $\{x_{n_k}\}_{k=1}^\infty\subseteq C$ y $x_0\in C$ tal que $x_{n_k}\to x_0$.
  Como $\cU$ es recubrimiento, existe $U_{\al_0} \in \cU$ tal que 
  $$x_0\in U_{\al_0}.$$ 
  Sea $\eps>0$ tal que 
  $$B(x_0,\eps)\subseteq U_{\al_0}$$ 
  y tomemos $k_0$ tal que 
  $$k\geq k_0\To d(x_{n_k},x_0)<\frac{\eps}{2}.$$
  
  
  \end{frame}

\begin{frame}{Prueba del lema}
  
  Con argumentos usuales podemos probar que 
  $$B\left(x_{n_k},\frac{1}{n_k}\right)\subseteq B(x_0,\eps)$$
  cuando 
  $\frac{1}{n_k}<\frac{\eps}{2}$. 
  Es decir
  
  \begin{align*}
    B\left(x_{n_k},\frac{1}{n_k}\right)&\subseteq B(x_0,\eps)\\
    &\subseteq U_{\al_0}.
  \end{align*}
    
      
   
\end{frame}

\begin{frame}{Compacidad}
  
  \begin{Def}\label{def:compacidad}
  Un conjunto $C \subseteq X$ es compacto si dado un recubrimeinto $\cU=\set{U_\al:\ \al\in\La}$ de $C$, existen 
  $U_{\alpha_1}, \ldots, U_{\alpha_m}$ tales que   
  $$C\subseteq \bigcup_{k=1}^{m} U_{\al_k}.$$
  \end{Def}
  
Es decir, dado un recubrimeinto  $\cU$ por abiertos de $C$, existe una colecci\'on finita de  $\cU$ que  recubre a $C$.   
  
\end{frame}

\begin{frame}{Equivalencia entre ambas definiciones.}
 
  \begin{Lem}\label{lem:equivalencia}
    Dado  $C \subseteq X$ son equivalentes:
    \begin{enumerate}       
    \item $C$ es secuencialmente compacto.
    \item $C$ es compacto.
    \end{enumerate}
  \end{Lem}
\end{frame}

\begin{frame}{Secuencial implica compacidad}
Asuma que $C$ es secuencialmente compacto. Sea $x_1 \in C$,  sabemos que existe $\epsilon > 0$  y $\al_1$ tal que 
\[ B(x_1,\eps) \subseteq U_{\al_1}. \]
Ahora, si 
$$ C   \subseteq B(x_1,\eps) \subseteq U_{\al_1}$$
entonces tenemos un recubrimiento finito. De lo contrario, existe 
\[ x_2 \in C \setminus  B(x_1,\eps).\]
 
\end{frame}

\begin{frame}{Secuencial implica compacidad}
Entonces existe  $\al_2$ tal que 
\[ B(x_2,\eps) \subseteq U_{\al_2}. \]
En general si 
$$ C   \subsetneq \bigcup_{i=1}^{m} B(x_i,\eps),$$
entonces existe 
\[ x_{m+1} \in C \setminus   \bigcup_{i=1}^{m} B(x_i,\eps).\]
Note que por construcci\'on, si $ i \not= j$ 
$$ d( x_i, x_j) \geq \eps.$$
 
\end{frame}

\begin{frame}{Secuencial implica compacidad}
Entonces existe  $\al_2$ tal que 
\[ B(x_2,\eps) \subseteq U_{\al_2}. \]
En general si 
$$ C   \subsetneq \bigcup_{i=1}^{m} B(x_i,\eps),$$
entonces existe 
\[ x_{m+1} \in C \setminus   \bigcup_{i=1}^{m} B(x_i,\eps).\]
Note que por construcci\'on, si $ i \not= j$ 
$$ d( x_i, x_j) \geq \eps.$$
 
\end{frame}
\begin{frame}{Secuencial implica compacidad}
Concluimos que existe $m$ tal que 
$$ C   \subseteq \bigcup_{i=1}^{m} B(x_i,\eps) \subseteq \bigcup_{i=1}^{m} U_{\al_i},$$
o existe una sucesi\'on $\{x_k\}_{k=1}^{\infty}$ tal que, si $ i \not= j$, 
$$ d( x_i, x_j) \geq \eps.$$
Esta sucesi\'on no tiene subsucesiones de Cauchy, i.e. no tiene subseciones convergentes. 
 

\end{frame}
\begin{frame}{Compacidad implica secuencial}
Sea  $\{x_m\}_{m=1}^{\infty}$ una sucesi\'on en $C$ y considere 
$$U_n=\left( \ov{\set{x_m:\ m\geq n}} \right)^{c}.$$
Entonces 
$$ U_n \subset U_{n+1},$$
$$ X \setminus \bigcup_{n=1}^{\infty} U_n= \bigcap_{n=1}^{\infty} \ov{\set{x_m:\ m\geq n}}.$$
Ahora, si $\{x_m\}_{m=1}^{\infty}$ no tiene una subsucesi\'on convergente en $C$ tenemos que 
$$C\cap \bigcap_{k=1}^\infty\ov{\set{x_m:\ m\geq k}} = \emptyset.$$  
\end{frame}

\begin{frame}{Compacidad implica secuencial}
Luego 
$$C \subset  \bigcup_{n=1}^\infty \left( \ov{\set{x_m:\ m\geq n}} \right)^{c} =\bigcup_{n=1}^\infty U_n.$$
Por compacidad, existe $\al_1, \ldots, \al_m$ tal que 
$$ C \subset  \bigcup_{n=1}^m U_{\al_m}.$$

\end{frame}

\begin{frame}{Compacidad implica secuencial}
En particular 
$$ C \subset  \bigcup_{n=1}^{m_0} U_{n}=U_{m_0},$$
donde $m_0=\max\{\al_1,\ldots,\al_m\}.$\\
Pero
$$ \set{x_k:\ k\geq m_0} \subset C \subset \left( \ov{\set{x_k:\ k\geq m_0}} \right)^{c}.$$

\end{frame}

\begin{frame}{Compacidad y funciones continuas}
Sea $f:[a,b] \to \bR$ una funci\'on continua, entonces existen $x_1, x_2 \in [a,b]$ tal que 
\[f(x_1) \leq f(x) \leq f(x_2),\]
para todo $ x \in [a,b]$. Por el Teorema del valor intermedio
\[f\left( [a,b]\right) \leq [f(x_1),f(x_2)].\]
Luego el compacto $ [a,b]$ es mapeado en el compacto $ [f(x_1),f(x_2)]$.\\

\end{frame}


\begin{frame}{Compacidad y funciones continuas}
Sean $f:X \to Y$ una funci\'on continua y $ K \subset X$ compacto. Tome $K_1=f(K)$ y  
considere 
$$\cU=\set{U_\al:\ \al\in\La_1}$$
un cubrimiento de $K_1$. Entonces 
$$ f(K) \subset \bigcup_{\al \in \La_1}U_{\al}.$$
Luego 
$$ K_1 \subset f^{-1}\left\{ \bigcup_{\al \in \La_1}U_{\al}\right\}.$$
\end{frame}

\begin{frame}{Compacidad y funciones continuas}
Como $K$ es compacto, existen $\al_1,\ldots,\al_m$ tal que 
$$ K \subset f^{-1}\left\{ \bigcup_{i=1}^{m}U_{\al_i}\right\}.$$
Es decir, si $x \in K$, entonces 
$$ f(x) \in  \bigcup_{i=1}^{m}U_{\al_i},$$
i.e. 
$$ f(K) \subset \bigcup_{i=1}^{m}U_{\al_i}.$$
\end{frame}

\subsection{Algunas Propiedades}

\begin{frame}{Compacidad y funciones continuas}
  \begin{Lem}\label{lem:CompactosEnCompactos}
    Sean $f:X \to Y$ una funci\'on continua y $ K \subset X$ compacto. Entonces $f(K)$ es compacto.
  \end{Lem}
\end{frame}

\begin{frame}{Compacidad }
Sea 
$$\cF=\set{F_\al:\ \al\in\La}$$
una colecci\'on de conjuntos cerrados tales que 
$$ \bigcap_{\al\in A} F_\al\not= \emptyset.$$
si $A \subset \La$ es finito. Asuma que 
$$ \bigcap_{\al\in \La} F_\al = \emptyset.$$
Entonces 
$$ X = \bigcup_{\al\in \La} X \setminus F_{\al}.$$
\end{frame}

\begin{frame}{Compacidad }
Ahora si $X$ es compacto, entonces existen $\al_1,\ldots,\al_m$ tal que 
$$ X = \bigcup_{i=1}^{m} X \setminus F_{\al_i},$$
i.e. 
$$ \emptyset = \bigcap_{i=1}^{m}  F_{\al_i}.$$
\end{frame}

\begin{frame}{Compacidad }
Ahora si $X$ es compacto, entonces existen $\al_1,\ldots,\al_m$ tal que 
$$ X = \bigcup_{i=1}^{m} X \setminus F_{\al_i},$$
i.e. 
$$ \emptyset = \bigcap_{i=1}^{m}  F_{\al_i}.$$
\end{frame}

\begin{frame}{Compacidad }
\begin{Th}\label{thm:equivCompacidadConIntersecc}
Sea $(X,d)$ un espacio m\'etrico, son equivalentes 
\begin{enumerate}
\item $X$ es compacto.
\item Si $\cF=\set{F_\al:\ \al\in\La}$ es 
una colecci\'on de conjuntos cerrados tales que 
$$ \bigcap_{\al\in A} F_\al\not= \emptyset.$$
si $A \subset \La$ es finito. Entonces 
$$ \bigcap_{\al\in \La} F_\al \not= \emptyset.$$
\end{enumerate}
\end{Th}

\end{frame}

\begin{frame}{La Prueba es Ejercicio}
  \begin{itemize}
  \item La prueba del teorema se deja como \alert{ejercicio}.
  \item A la segunda propiedad del teorema \ref{thm:equivCompacidadConIntersecc} le llamamos la \alert{propiedad de intersecciones finitas}.
  \end{itemize}
\end{frame}

\subsection{El caso de $\bR^d$}

\begin{frame}{Compacidad en $\bR^d$}
  Asuma que 
  $$C=[a_1,b_1]\x\dots\x[a_d,b_d]$$
  es compacto si $a_i\leq b_i$ para $1\leq i\leq d$. Sabemos que en este caso todo compacto es cerrado y acotado.
\end{frame}

\begin{frame}{¿Cerrado + Acotado = Compacto?}
  La respuesta general es no, ¡pero en $\bR^d$ sí!
  \begin{itemize}
    \item Sea $F$ cerrado y acotado en $\bR^d$. Existe $C_n=[-n,n]\x\dots\x[-n,n]$ tal que $F\subseteq C_n$.
    \item Sea $\{x_n\}\sucn\subseteq F$. Como $C_n$ es compacto, existe $\{x_{n_k}\}\suck$ subsucesión de $\{x_n\}\sucn$ tal que 
    $$x_{n_k}\xrightarrow[n_k\to\infty]{}x_0\in C_n$$
  \end{itemize}
  Como $\{x_{n_k}\}\suck\subseteq F$ y $F$ es cerrado, entonces $x_0\in F$. Es decir, $F$ es \un{secuencialmente compacto}.
\end{frame}

\begin{frame}{El Teorema de Heine y Borel}
  \begin{Th}{Heine-Borel}\label{thm:HeineBorel}
    Sea $C\subseteq\bR^d$. Entonces $C$ es compacto respecto a la norma euclídea si y sólo si es cerrado y acotado.
  \end{Th}
\end{frame}

\begin{frame}{Cajas}
\begin{Lem}\label{lem:cajasCompactas}
  Dados $a_i\leq b_i$, $1\leq i\leq d$, el conjunto
  $$C=[a_1,b_1]\x\dots\x[a_d,b_d]$$
  es compacto en $(\bR^d,\nm{\cdot})$.
\end{Lem}  
Dividimos $C$ en $2^d$ rectángulos de la forma 
$$[c_1^i,e_1^i]\xyx[c_d^i,e_d^i]=C_i$$
donde $c_j^i=a_j$ ó $\frac{a_j+b_j}{2}$ y $e_j^i=b_j$ ó $\frac{a_j+b_j}{2}$.
\end{frame}

\begin{frame}{Prueba del Lema}
  \begin{itemize}
    \item Tomemos $\cU=\set{U_\al:\ \al\in\Om}$ es un recubrimiento de $C$ tal que $C\subsetneq\bigcup\sucm U_{\al_i}$ para ${\al_1,\dots,\al_m}\subseteq \La$.
    \item Si dado $1\leq i\leq 2^d$ existen $\al^i_1,\dots,\al_{m_i}^i$ tal que $C_i\subseteq \bigcup_{j=1}^{m_i}U_{\al_j^i}$, entonces 
    $$C\subseteq \bigcup_{i=1}^{2^d}\bigcup_{j=1}^{m_i}U_{\al_j^i}.$$
  \end{itemize}
  
\end{frame}{Prueba del Lema}
\begin{itemize}
  \item Es decir, existe $i_0$ tal que 
  $$C_{i_0}\subsetneq\bigcup\sucm U_{\al_i}$$
  para $\set{\al_1,\dots,\al_m}\subseteq\La$
  \item Dividiendo $C_{i_0}$ en $2^d$ rectángulos obtenemos 
  $$C_{i_1}\subseteq C_{i_0},\ \text{ y }\ C_{i_1}\subsetneq \bigcup\sucm U_{\al_i}$$
  para cualquier $\set{\al_1,\dots,\al_m}\subseteq\La$
  
\end{itemize}

\begin{frame}{Prueba del Lema - Iterando}
  \begin{itemize}
    \item Además 
    \begin{align*}
      &[c_1^{i_0},e_1^{i_0}]\xyx[c_d^{i_0},e_d^{i_0}]=C_{i_0}\\
      &[c_1^{i_1},e_1^{i_1}]\xyx[c_d^{i_1},e_d^{i_1}]=C_{i_1},
    \end{align*}
    con $[c_j^{i_1},e_j^{i_1}]\subseteq[c_j^{i_0},e_j^{i_0}]$ y 
    \begin{align*}
      |e_j^{i_1}-c_j^{i_1}|&=\half|e_j^{i_0}-c_j^{i_0}|\\
      &=\frac14|b_j-a_j|.
    \end{align*}
  \end{itemize}
\end{frame}

\begin{frame}{Prueba del Lema - El caso general}
  Iterando el proceso obtenemos 
  $$[c_1^{i_j},e_1^{i_j}]\xyx[c_d^{i_j},e_d^{i_j}]=C_{i_j}$$
  tal que 
  $$[c_\l^{i_j},e_\l^{i_j}]\subseteq[c_\l^{i_j-1},e_\l^{i_j-1}]$$
  y 
  $$|c_\l^{i_j}-e_\l^{i_j}|=\frac{1}{2^j}|b_\l-a_\l|.$$
\end{frame}

\begin{frame}{Prueba del Lema - Intervalos Encajados}
  Por el teorema de los Intervalos Encajados, existe $z_\l$ tal que $\bigcap_{j=1}^\infty[c_\l^{i_j},e_\l^{i_j}]=\set{z_\l}$. Es decir,
  $$\bigcap_{j=1}^\infty C_{i_j}=\set{z},\ z=(z_1,\dots,z_d).$$
  Como $z\in C$, existe $\al_0$ tal que $U_{\al_0}\in U$ y $z\in U_{\al_0}$. Sea $\eps> 0$ tal que $B(z,\eps)\subseteq U_{\al_0}$.
\begin{Ej}\label{ej:ultimoDetallePrueba}
  Pruebe que existe $j_0$ tal que $C_{i_j}\subseteq B(z,\eps)$ para $j\geq j_0$.
\end{Ej}
\end{frame}

\begin{frame}{Valores Extremos}
  Si $K\subseteq X$ y $f:\ X\to\bR$ es continua, entonces $f(K)$ es compacto y en particular, acotado.\par 
  Sean $a=\inf_{x\in K}f(x)$ y $b=\sup_{x\in K}$, nos preguntamos:\par 
  \begin{center}
    \emph{¿Se alcanzan ``$a$'' y ``$b$''?}
  \end{center}
\end{frame}
\begin{frame}{Damos Respuesta}
  \begin{itemize}
    \item Dado $n\in\bN$, existe $x_n\in K$ tal que 
    \begin{equation}\label{eqn:TmaValExt}
      a\leq f(x_n)\leq a+\frac1n  
    \end{equation}
    \item Como $\set{x_n}\sucn\subseteq K$, existe $\set{x_{n_k}}\suck$ que satisface
    \item $$x_{n_k}\xrightarrow[n_k\to\infty]{}y_0\in K.$$
    \item Entonces $f(x_{n_k})\to f(y_0)$ cuando $n_k\to\infty$.
    \item De \ref{eqn:TmaValExt} se tiene que $f(x_{n_k})\to a$ cuando $n_k\to\infty$. Por lo tanto $a=f(y_0)$.
  \end{itemize}
  
\end{frame}
\section*{Resumen}

\subsection*{Qu\'e vimos hoy}
\begin{frame}{Resumen}

  % Keep the summary *very short*.
  \begin{itemize}
  \item Definición de compacidad secuencial. \ref{def:compacidadSecuencial}
  \item Equivalencia entre compacidad secuencial y la propiedad de intersección. \ref{lem:equivSecCompYEncajados}.
  \item Compacidad secuencial implica compacidad usual. \ref{lem:equiv1Compacidad}
  \item Definición de recubrimiento por abiertos. \ref{def:recubrimiento}.
  \item Equivalencia entre compacidad y compacidad secuencial. \ref{lem:equivalencia}
  \item Que las funciones continuas mandan compactos en compactos. \ref{lem:CompactosEnCompactos}
  \item El Teorema de las Intersecciones Finitas. \ref{thm:equivCompacidadConIntersecc}
  \item El Teorema Heine-Borel sobre compactos en $\bR^d$. \ref{thm:HeineBorel}
  \item Las cajas son compactos de $\bR^d$. \ref{lem:cajasCompactas}
  \end{itemize}
  
\end{frame}

\subsection*{Ejercicios a trabajar}
\begin{frame}{Ejercicios}
    
  \begin{itemize}
    \item
      Lista 5
      \begin{itemize}
      \item Terminar el detalle en la prueba del lema de compacidad secuencial a la usual. \ref{lem:equiv1Compacidad}.
      \item El último detalle de la prueba sobre las cajas compactas. \ref{ej:ultimoDetallePrueba}
      \end{itemize}
    \end{itemize}
  
\end{frame}


% All of the following is optional and typically not needed. 
\appendix
\section<presentation>*{\appendixname}
\subsection<presentation>*{Lectura adicional}

\begin{frame}[allowframebreaks]
  \frametitle<presentation>{Lecturas adicionales}
    
  \begin{thebibliography}{10}
    
  \beamertemplatebookbibitems
  % Start with overview books.

  \bibitem{CambroNotas}
    S.Cambronero.
    \newblock {\em Notas MA0505}.
    \newblock 20XX.

    \bibitem{NachoNotas}
    I.Rojas
    \newblock {\em Notas MA0505}.
    \newblock 2018.
 
  \end{thebibliography}
  
\end{frame}
%% 1:01:10
%% 5 - 49:48 54
\end{document}


