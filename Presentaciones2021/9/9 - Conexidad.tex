\documentclass[utf8]{beamer}

\mode<presentation>
{
  \usetheme{Warsaw}
  \setbeamercovered{transparent}
}


\usepackage{amsfonts,mathtools,amssymb}
\usepackage[spanish]{babel}
\usepackage{times}
\usepackage[T1]{fontenc}
\usepackage[shortlabels]{enumitem}


\title[MA0505]{MA0505 - An\'alisis I}
\subtitle{Lecci\'on IX: Conexidad}

\author{Pedro M\'endez\inst{1}}

\institute[Universidad de Costa Rica] % (optional, but mostly needed)
{
  \inst{1}%
  Departmento de Matem\'atica Pura y Ciencias Actuariales\\
  Universidad de Costa Rica
  }

\date[I-2021] {Semestre I, 2021}

%%%%%%%%% === Theorems and suchlike === %%%%%%%%%%%%%%

\theoremstyle{plain}
\newtheorem{Th}{Teorema}               %%% Theorem 1.1.1
\newtheorem{Tmon}{Teoremón}
\newtheorem{Prop}{Proposición}         %%% Proposition 1.1.2
\newtheorem{Lem}{Lema}                 %%% Lemma 3
\newtheorem{Cor}{Corolario}            %%% Corollary 4

\theoremstyle{definition}
\newtheorem{Def}{Definición}           %%% Definition 5
\newtheorem{Ex}{Ejemplo}               %%% Example 6
\newtheorem{Ej}{Ejercicio}             %%% Ejercicio 7
\newtheorem{Hec}[Th]{Hecho}            %%% Hecho 1.1.8

\theoremstyle{remark}
\newtheorem{Rmk}[Th]{Observación}      %%%Remark 1.1.9
\newtheorem*{nonum-Rmk}{Observación}         %%% No number Fact
\newtheorem*{Notn}{Notaci\'on}        %% Notaciones
\newtheorem*{Warn}{Advertencia}       %% Advertencias

\numberwithin{equation}{section}

%% Accomodations 

\makeatletter
\def\moverlay{\mathpalette\mov@rlay}
\def\mov@rlay#1#2{\leavevmode\vtop{%
   \baselineskip\z@skip \lineskiplimit-\maxdimen
   \ialign{\hfil$\m@th#1##$\hfil\cr#2\crcr}}}
\newcommand{\charfusion}[3][\mathord]{
    #1{\ifx#1\mathop\vphantom{#2}\fi
        \mathpalette\mov@rlay{#2\cr#3}
      }
    \ifx#1\mathop\expandafter\displaylimits\fi}
\makeatother

% Greek letters:

\newcommand{\al}{\alpha}                %% short for  \alpha
\newcommand{\bt}{\beta}                 %% short for  \beta
\newcommand{\Dl}{\Delta}                %% short for  \Delta
\newcommand{\dl}{\delta}                %% short for  \delta
\newcommand{\eps}{\varepsilon}          %% short for  \varepsilon
\newcommand{\Ga}{\Gamma}                %% short for  \Gamma
\newcommand{\ga}{\gamma}                %% short for  \gamma
\newcommand{\kp}{\kappa}                %% short for  \kappa
\newcommand{\La}{\Lambda}               %% short for  \Lambda
\newcommand{\la}{\lambda}               %% short for  \lambda
\newcommand{\Om}{\Omega}                %% short for  \Omega
\newcommand{\om}{\omega}                %% short for  \omega
\newcommand{\Sg}{\Sigma}                %% short for  \Sigma
\newcommand{\sg}{\sigma}                %% short for  \sigma
\newcommand{\Te}{\Theta}                %% short for  \Theta
\newcommand{\te}{\theta}                %% short for  \theta
\newcommand{\ups}{\upsilon}             %% short for  \upsilon
\newcommand{\vf}{\varphi}               %% short for  \varphi
\newcommand{\ze}{\zeta}                 %% short for  \zeta

%Boldface letters

\newcommand{\bC}{\mathbb{C}}    %%% números complejos
\newcommand{\bN}{\mathbb{N}}    %%% números naturales
\newcommand{\bP}{\mathbb{P}}        %% números enteros positivos
\newcommand{\bQ}{\mathbb{Q}}    %%% números racionales
\newcommand{\bR}{\mathbb{R}}    %%% números reales
\newcommand{\bS}{\mathbb{S}}    %%% esfera
\newcommand{\bZ}{\mathbb{Z}}    %%% números enteros

%Script letters:

\newcommand{\cA}{\mathcal{A}}           %% formas diferenciales
\newcommand{\cB}{\mathcal{B}}           %% una base vectorial
\newcommand{\cC}{\mathcal{C}}           %% otra base vectorial
\newcommand{\cD}{\mathcal{D}}           %% funciones de prueba
\newcommand{\cE}{\mathcal{E}}           %% un modulo proyectivo
\newcommand{\cF}{\mathcal{F}}           %% espacio de Fock
\newcommand{\cG}{\mathcal{G}}           %% funtor de Gelfand
\newcommand{\cH}{\mathcal{H}}           %% espacio de Hilbert
\newcommand{\cI}{\mathcal{I}}           %% un funtor de inclusion
\newcommand{\cJ}{\mathcal{J}}           %% otro funtor
\newcommand{\cK}{\mathcal{K}}           %% otro espacio de Hilbert
\newcommand{\cL}{\mathcal{L}}           %% operadores lineales
\newcommand{\cM}{\mathcal{M}}           %% multiplicadores
\newcommand{\cN}{\mathcal{N}}           %% funciones nulas
\newcommand{\cO}{\mathcal{O}}           %% funciones de crec-to lento
\newcommand{\cP}{\mathcal{P}}           %% una particion
\newcommand{\cR}{\mathcal{R}}           %% funciones representativas
\newcommand{\cQ}{\mathcal{Q}}           %% otra particion
\newcommand{\cS}{\mathcal{S}}           %% funciones de Schwartz
\newcommand{\cT}{\mathcal{T}}           %% una topologia
\newcommand{\cU}{\mathcal{U}}           %% cubrimiento abierto
\newcommand{\cV}{\mathcal{V}}           %% vecindarios
\newcommand{\cW}{\mathcal{W}}           %% grupo de Weyl


%Brackets

\newcommand{\bonj}[1]{\left\lbrack#1\right\rbrack}
\newcommand{\obonj}[1]{\left\rbrack#1\right\lbrack}
\newcommand{\rbonj}[1]{\left\rbrack#1\right\rbrack}
\newcommand{\lbonj}[1]{\left\lbrack#1\right\lbrack}
\newcommand{\snm}[1]{\|#1\|}           %small norma
\newcommand{\nm}[1]{\left\|#1\right\|} %norma pegadita
\newcommand{\pnm}[1]{\biggl|\biggl|#1\biggr|\biggr|}
\newcommand{\set}[1]{\{\,#1\,\}}    %% set notation
\newcommand{\floor}[1]{\lfloor#1\rfloor} %% mayor entero <= x
\newcommand{\Set}[1]{\biggl\{\,#1\,\biggr\}} %% set notation (large)
\newcommand\quot[2]{
        \mathchoice
            {% \displaystyle
                \text{\raise1ex\hbox{$#1$}\Big/\lower1ex\hbox{$#2$}}%
            }
            {% \textstyle
                {^{ #1}/_{ #2}}
            }
            {% \scriptstyle
                {^{ #1}/_{ #2}}
            }
            {% \scriptscriptstyle
                {^{ #1}/_{ #2}}
            }
    }
\newcommand*\squot[2]{{^{ #1}/_{ #2}}}%%%small quotient

%Symbols 

\newcommand{\x}{\times}
\renewcommand{\geq}{\geqslant}          %% mayor o igual (variante)
\newcommand{\hookto}{\hookrightarrow}     %% inclusion arrow
\newcommand{\isom}{\simeq}              %% isomorfismo
\renewcommand{\l}{\ell}                   %% ele cursiva
\renewcommand{\leq}{\leqslant}          %% menor o igual (variante)
\newcommand{\less}{\setminus}           %% set difference
\newcommand{\To}{\Rightarrow}
\newcommand{\ov}{\overline}
\newcommand{\un}{\underline}
\newcommand{\del}{\partial}

%%% Small fractions in displays:

\newcommand{\half}{{\mathchoice{\nhalf}{\thalf}{\shalf}{\shalf}}} %%display text script script^2
\newcommand{\happi}{{\tfrac{\pi}{2}}} %% small fraction  \pi/2
\newcommand{\quarter}{\tfrac{1}{4}} %% small fraction  1/4
\newcommand{\nhalf}{\frac{1}{2}}
\newcommand{\shalf}{{\scriptstyle\frac{1}{2}}} %% tiny fraction 1/2
\newcommand{\thalf}{{\tfrac{1}{2}}} %% small fraction  1/2
\newcommand{\third}{\tfrac{1}{3}}   %% small fraction  1/3 %Hay que renew porque mathabx toma second y third como x'' y x''' por ejemplo

\newcommand{\ihalf}{{\tfrac{i}{2}}} %% small fraction  i/2

\newcommand{\suci}{_{i=1}^\infty} %% diminutivo
\newcommand{\suck}{_{k=1}^\infty} %% diminutivo
\newcommand{\sucm}{_{m=1}^\infty} %% diminutivo
\newcommand{\sucn}{_{n=1}^\infty} %% diminutivo

\newcommand*{\Cdot}{{\raisebox{-0.25ex}{\scalebox{1.5}{$\cdot$}}}}      %% cdot más grande
\renewcommand{\.}{\Cdot}                %% producto escalar
\newcommand{\cupdot}{\charfusion[\mathbin]{\cup}{\.}}
\newcommand{\bigcupdot}{\charfusion[\mathbin]{\bigcup}{\.}}

\begin{document}

\begin{frame}
  \titlepage
\end{frame}

\begin{frame}{Agenda}
  \tableofcontents
  % You might wish to add the option [pausesections]
\end{frame}


% Structuring a talk is a difficult task and the following structure
% may not be suitable. Here are some rules that apply for this
% solution: 

% - Exactly two or three sections (other than the summary).
% - At *most* three subsections per section.
% - Talk about 30s to 2min per frame. So there should be between about
%   15 and 30 frames, all told.

% - A conference audience is likely to know very little of what you
%   are going to talk about. So *simplify*!
% - In a 20min talk, getting the main ideas across is hard
%   enough. Leave out details, even if it means being less precise than
%   you think necessary.
% - If you omit details that are vital to the proof/implementation,
%   just say so once. Everybody will be happy with that.

\section{La Definición de Conexidad}

\begin{frame}{Conjuntos Disconexos}
  \begin{Def}\label{def:disconexo}
    Un espacio $(X,d)$ es \alert{disconexo} si existen $A,B$ abiertos no vacíos tales que 
    $$X=A\cupdot B,\quad (A\cap B =\emptyset).$$
    Diremos que un espacio es \alert{conexo} si no es disconexo.
  \end{Def}

  Si $X$ es conexo y $X=A\cupdot B$, con $A,B$ abiertos, es necesario que $A=X$ ó $B=X$.
\end{frame}

\begin{frame}{Subespacios y Conexidad}
  Dado $E\subseteq X$ podemos definir 
  $$d_E:E\x E\to\bR,\ (x,y)\mapsto d(x,y).$$
  \begin{Ej}\label{ej:abiertosEnSubespacios}
    $(E,d_E)$ es un espacio métrico. Pruebe que $D\subseteq E$ es abierto en $(E,d_E)$ si y sólo si existe $O\subseteq X$ abierto en $(X,d)$ tal que $D=E\cap O$.
  \end{Ej}
  \begin{Def}\label{def:subconjsDisconexos}
    $E\subseteq X$ es disconexo si existen $A,B$ abiertos en $(X,d)$ tales que
    $$E=(A\cap E)\cupdot (B\cap E),\ A\cap E\neq \emptyset \neq B\cap E.$$
  \end{Def}
\end{frame}

\subsection{Conexidad en $\bR$}

\begin{frame}{Un Ejemplo}
  Sea $I=]a,b[\subseteq\bR$. Vamos a probar que $I$ es conexo.
  \begin{itemize}
    \item Asumamos que $I$ es disconexo, entonces existen $A,B$ abiertos tales que 
     $$I=(I\cap A)\cupdot(I\cap B),\ I\cap A,\ I\cap B\neq \emptyset.$$
     \item Sean $s\in I\cap A$, $t\in I\cap B$ con $s<t$. Así $[s,t]\subseteq I$.
  \end{itemize}
\end{frame}

\begin{frame}{Continuamos con el Ejemplo}
  \begin{itemize}
    \item Como $s\in A$, entonces existe $\dl_1$ tal que 
    $$]s-\dl_1,s+\dl_1[\subseteq A.$$
    \item Como $[s,s+\dl_1]\subseteq [s,t]$ entonces
    $$[s,s+\dl_1]\subseteq [s,t]\cap A.$$
    \item Llamemos $u=\sup\set{x\in [s,t]\cap A}$.
    \item De esta manera $s<u\leq t$.
  \end{itemize}
\end{frame}

\begin{frame}{Continuamos con el Ejemplo}
  \begin{itemize}
    \item Si fuese que $u\in B$, entonces existe $\dl_2>0$ tal que 
     $$]u-\dl_2,u+\dl_2[\subseteq B.$$
     \item De manera análoga $]u-\dl_2,u[\subseteq [s,t]$ y por tanto
     $$]u-\dl_2,u[\subseteq B\cap [s,t].$$
     \item Sin embargo, por propiedades del sup, existe $w\in [s,t]\cap A$ tal que 
     $$u-\dl_2<w\leq u.$$
     Esto es una contradicción.
  \end{itemize}
\end{frame}

\begin{frame}{Terminamos el Ejemplo}
  De forma similar se prueba que si $u\in A$, entonces existe un $\dl_3$ tal que 
  $$[u,u+\dl_3]\subseteq[s,t]\cap A.$$
\end{frame}

\section{Propiedades de los Conexos}
\subsection{Conexidad y Funciones Continuas}
%% pag 3+eps 32:28
\begin{frame}{Funciones Continuas Preservan Conexidad}
  \begin{Lem}\label{lem:ContPreservaConexo}
    Sea $f:X\to Y$ una función continua. Si $E\subseteq X$ es conexo, entonces $f(E)$ es conexo.
  \end{Lem}
  Asumamos que existen $B,C\subseteq Y$ abiertos tales que 
  $$f(E)=(f(E)\cap C)\cupdot(f(E)\cap B),\ f(E)\cap C\neq\emptyset\neq f(E)\cap B).$$
  
\end{frame}

\begin{frame}{Terminamos la Prueba}
  Notemos que 
  \begin{align*}
    f^{-1}(f(E)\cap C)\supseteq E\cap f^{-1}(C)\neq\emptyset,\\
    f^{-1}(f(E)\cap B)\supseteq E\cap f^{-1}(B)\neq\emptyset.
  \end{align*}
  Además
  $$E\subseteq (E\cap f^{-1}(C))\cupdot (E\cap f^{-1}(B))$$
  pues $f^{-1}(C)$ y $f^{-1}(B)$ son abiertos.
\end{frame}

\begin{frame}{Los Reales son un Conjunto Conexo}
\begin{itemize}
  \item Supongamos que $\bR^d=A\cupdot B$ con $A,B$ abiertos no vacíos.
  \item Tomemos $x\in A, y\in B$ y consideremos 
  $$f:[0,1]\to\bR^d,\ t\mapsto (1-t)x+ty.$$
  \item Como $f$ es continua, $[x,y]=f([0,1])$. El \emph{segmento} entre $x$ y $y$ es conexo.
  \item Pero $[x,y]=([x,y]\cap A)\cupdot([x,y]\cap B)$ con $[x,y]\cap A$ y $[x,y]\cap B$ no vacíos.
\end{itemize}
Esto es una contradicción.
\end{frame}

\subsection{Uniones de Conexos}
\begin{frame}{Uniones}
  \begin{Lem}\label{lem:unionConexos}
    Sea $\set{E_\al:\ \al\in A}$ una familia de conexos tal que $\bigcap_{\al\in A}E_\al$ es no vacío. Entonces $E=\bigcup_{\al\in A} E_\al$ es conexo.
  \end{Lem}
  Tomemos $x$ en la intersección. Si fuese que $E$ es disconexo, existen $A,B$ tales que 
  $$E=(A\cap E)\cupdot(B\cap E),\ A\cap E,\ B\cap E\neq\emptyset.$$
\end{frame}

\begin{frame}{Terminamos la Prueba}
\begin{itemize}
  \item Asumamos que $x\in B\cap E$.
  \item Como $A\cap E\neq\emptyset$, existe $\al_0$ tal que $A\cap E_{\al_0}\neq \emptyset$.
  \item Además, como $x\in B$ y $x\in \bigcap_{\al\in A}E_\al$ tenemos que $x\in B\cap E_{\al_0}$.
  \item Como $E_{\al_0}\cap E=E_{\al_0}$, entonces 
   $$E_{\al_0}=(E_{\al_0}\cap A)\cupdot(E_{\al_0}\cap B).$$
\end{itemize}
Esto es imposible pues $E_{\al_0}$ es conexo.
\end{frame}

\section{Componentes Conexos}

\begin{frame}{Definición de Componente}
  \begin{Def}\label{def:compConexa}
    Sea $(X,d)$ un espacio métrico. Dado $x\in X$, definimos la \alert{componente conexa} de $x$ como la unión de todos lo conexos que lo contienen. Es decir, si
    $$\La =\set{E\subseteq X:\ E\ \text{conexo},\ x\in E},$$ 
    entonces la componente conexa de $x$ es 
    $$C(x)=\bigcup_{E\in\La}E.$$
  \end{Def}
  Por el lema \ref{lem:unionConexos} anterior, $C(x)$ es conexo para todo $x\in X$.
\end{frame}

\begin{frame}{Una Relación Útil}
  Podemos definir la relación de equivalencia $\cR$ en $X$ por medio de 
  $$x\cR y\iff \exists C,\ \text{conexo}(x,y\in C).$$
  \begin{Ej}\label{ej:componentesSonClases}
    Si $[x]$ es la clase de equivalencia de $x$, entonces $[x]=C(x)$.
  \end{Ej}
\end{frame}

\begin{frame}{Segmentos en Conexos}
  \begin{Ex}
    Sea $E\subseteq\bR$ conexo con $a,b\in E$. Vamos a mostrar que $[a,b]\subseteq E$.
  \end{Ex}
  \begin{itemize}
    \item Asuma que $x\in[a,b]\less E$.
    \item Entonces $E=(E\cap]-\infty,x[)\cupdot(]x,\infty[\cap E)$. Tome $a=\inf E$ y $b=$
    %% estamos pegados aquí 59:38 40
    %%Finish
  \end{itemize}
  
\end{frame}

\begin{frame}
  \begin{Ej}\label{ej:caracAbiertosEnR}
    Sea $G\subseteq\bR$ un abierto, entonces existen $\set{]a_i,b_i[}\suci$ tales que $G=\bigcupdot\suci]a_i,b_i[$.
  \end{Ej}
\end{frame}

\section{Arcoconexidad}

\begin{frame}{Recordamos la Definición}
  

\begin{Def}
  Dado $E\subseteq \bR^d$, diremos que $E$ es \alert{arcoconexo} si dados $x_0,x_1\in E$, existe una curva $\ga:[0,1]\to\bR^d$ tal que 
  \begin{enumerate}
    \item $\ga(0)=x_0$ y $\ga(1)=x_1$.
    \item $\ga(t)\in E$ para $t\in [0,1]$.
  \end{enumerate}
\end{Def}

\begin{center}
  ¿Es la arcoconexidad equivalente a la conexidad?
\end{center}
\end{frame}

\begin{frame}{Respondamos la Pregunta}
  Considere 
  $$E=\set{0}\x[-1,1]\cup\set{(x,y):0<x\leq 1,\ y=\sin\left(\frac1x\right)}.$$
  \begin{itemize}
    \item Sean $A,B$ abiertos tales que $A\cap E, B\cap E$ y $A\cap B$ son todos no vacíos y 
    $$E\subseteq (A\cap E)\cupdot(B\cap E).$$
    \item Supongamos además que $(0,0)\in A$.
  \end{itemize}
\end{frame}

\begin{frame}{Primeros Pasos}
  \newcounter{saveenumi}
  \begin{enumerate}
    \item Usando el hecho de que $\set{0}\x[-1,1]$ es conexo, muestre que está contenido en $A$.
    \item La sucesión $\set{x_n}\sucn$ definida por 
    $$x_n=\left(\frac{1}{\pi n},\sin(\pi n)\right)=\left(\frac{1}{\pi n},0\right)$$
    converge a $(0,0)$. 
    \item Luego existe un $n_0$ tal que $\left(\frac{1}{\pi n},\sin(\pi n)\right)\in A$ para todo $n\geq n_0$. ¿Por qué?
    \setcounter{saveenumi}{\value{enumi}}
  \end{enumerate}
\end{frame}

\begin{frame}{Los Pasos que Siguen}
  \begin{enumerate}
    \setcounter{enumi}{\value{saveenumi}}
    \item Considere el conjunto 
     $$E_n=\Set{(x,y):\frac{1}{n\pi}\leq x\leq 1,\ y=\sin\left(\frac{1}{x}\right)}.$$
    \item Como 
    $$\ga:\bonj{\frac{1}{\pi n},1}\to E_n,\ x\mapsto \left(x,\sin\left(\frac{1}{x}\right)\right),$$
    tenemos que $E_n$ es conexo. Luego $E_n\cap A$ ó $E_n\cap B$ es no vacío.
    \setcounter{saveenumi}{\value{enumi}}
    
  \end{enumerate}
  
\end{frame}

\begin{frame}{Continuamos}
  \begin{enumerate}
    \setcounter{enumi}{\value{saveenumi}}
    \item Sin perdida de generalidad $E_n\cap B =\emptyset$, entonces $E_n\subseteq A$. Entonces $\left(x,\sin\left(\frac{1}{x}\right)\right)\in A$ para $0<x\leq 1$.
    \item De aquí se concluye que $E\subseteq A$ y eso es imposible.
    \item Asumimos pues que existe 
     $$\tilde{\ga}:[0,1]\to E$$
     continua tal que 
     $$\tilde{\ga}(0)=(0,0),\ \tilde{\ga}(1)=\left(\frac{1}{n_0\pi},\sin(n_0\pi)\right).$$
     \setcounter{saveenumi}{\value{enumi}}
  \end{enumerate}
  
\end{frame}

\begin{frame}{Continuamos con la Respuesta}
  
  \begin{enumerate}
    \setcounter{enumi}{\value{saveenumi}}
    \item Si $t_n\downarrow 0$ cuando $n\to\infty$, entonces
     $$\tilde{\ga}(t_n)=(\tilde{\ga}_1(t_n),\tilde{\ga}_2(t_n))\xrightarrow[n\to\infty]{}(0,0).$$
     \item En particular $\tilde{\ga}_1(t_n)\xrightarrow[n\to\infty]{}0$ y 
     $$\tilde{\ga}_2(t_n)=\sin\left(\frac{1}{\tilde{\ga}_1(t_n)}\right).$$
     Es decir, existe $x_n\xrightarrow[n\to\infty]{}0$ tal que 
     $$\left(x_n,\sin\left(\frac{1}{x_n}\right)\right)\in\tilde{\ga}([0,1])$$
  \end{enumerate}
\end{frame}

\begin{frame}{Un Ejercicio}
  \begin{Ej}\label{ej:detalleArcoconexidad}
    Basado en lo anterior, pruebe que si $x_n<x_{n+1}$ entonces para $x_n\leq x\leq x_{n+1}$ vale $\left(x,\sin\left(\frac{1}{x}\right)\right)\in\tilde{\ga}([0,1])$.\par 
    Concluya que 
    \begin{enumerate}[a)]
      \item $B=\Set{\left(x,\sin\left(\frac{1}{x}\right)\right):\ 0<x<\frac{1}{n_0\pi}}\subseteq \tilde{\ga}([0,1])$.
      \item Sea $t_0=\sup_{t>0}\set{\tilde{\ga}(t)=(0,0)}$. Entonces 
      $$\tilde{\ga}([t_0,1])\less B=\set{(0,0)}.$$
      \item $\tilde{\ga}$ no es continua en $t_0$.
    \end{enumerate}
  \end{Ej}
\end{frame}

\begin{frame}{Nociones Equivalentes}
  Si asumimos que nuestro conjunto es abierto en $\bR^d$ entonces las nociones son equivalentes.
  \begin{Lem}\label{lem:EnRdConexoIffArcoconexo}
    Sea $E\subseteq\bR^d$ abierto. Entonces $E$ es conexo si y sólo si $E$ es arcoconexo.
  \end{Lem}

  La prueba de este hecho es un \alert{ejercicio}.
\end{frame}
\section*{Resumen}

\subsection*{Qu\'e vimos hoy}

\begin{frame}{Resumen}

  % Keep the summary *very short*.
  \begin{itemize}
  \item La definción \ref{def:disconexo} de espacios conexos.
  \item La definción \ref{def:subconjsDisconexos} de subconjuntos disconexos.
  \item Las funciones continuas preservan conexidad: \ref{lem:ContPreservaConexo}.
  \item Las uniones de conexos a veces son conexas: \ref{lem:unionConexos}.
  \item La definción \ref{def:compConexa} de componente conexo.
  \item El lema \ref{lem:EnRdConexoIffArcoconexo} que garantiza equivalencia entre conexidad y arcoconexidad.
  \end{itemize}
  
\end{frame}

\subsection*{Ejercicios a trabajar}
\begin{frame}{Ejercicios}
    
  \begin{itemize}
    \item
      Lista 9
      \begin{itemize}
      \item El ejercicio \ref{ej:abiertosEnSubespacios} sobre abiertos dentro de subespacios.
      \item El ejercicio \ref{ej:componentesSonClases} que nos dice cuales son las clases de equivalencia en conexidad.
      \item El ejercicio \ref{ej:caracAbiertosEnR} que da una caracterización de los abiertos en $\bR$.
      \item El ejercicio \ref{ej:detalleArcoconexidad} para terminar los detalles de la curva.
      \end{itemize}
    \end{itemize}
  
\end{frame}


% All of the following is optional and typically not needed. 
\appendix
\section<presentation>*{\appendixname}
\subsection<presentation>*{Lectura adicional}

\begin{frame}[allowframebreaks]
  \frametitle<presentation>{Lecturas adicionales}
    
  \begin{thebibliography}{10}
    
  \beamertemplatebookbibitems
  % Start with overview books.

  \bibitem{CambroNotas}
    S.Cambronero.
    \newblock {\em Notas MA0505}.
    \newblock 20XX.

    \bibitem{NachoNotas}
    I.Rojas
    \newblock {\em Notas MA0505}.
    \newblock 2018.
 
  \end{thebibliography}
  
\end{frame}
%% 6 - 2:10:52 48
%% 7 - 1:17:44 44
%% 8 - 27:37 69
%% 9 - 1:07:47 86 + 59:38 40 approx 2 h c 7 min
\end{document}


