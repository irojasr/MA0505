\documentclass[utf8]{beamer}

\mode<presentation>
{
  \usetheme{Warsaw}
  \setbeamercovered{transparent}
}


\usepackage{amsfonts,mathtools,amssymb}
\usepackage[spanish]{babel}
\usepackage{times}
\usepackage[T1]{fontenc}
\usepackage[shortlabels]{enumitem}
\usepackage{tikz}
\usepackage{physics}

\title[MA0505]{MA0505 - An\'alisis I}
\subtitle{Lecci\'on XV: Funciones Medibles}

\author{Pedro M\'endez\inst{1}}

\institute[Universidad de Costa Rica] % (optional, but mostly needed)
{
  \inst{1}%
  Departmento de Matem\'atica Pura y Ciencias Actuariales\\
  Universidad de Costa Rica
  }

\date[I-2021] {Semestre I, 2021}

%%%%%%%%% === Theorems and suchlike === %%%%%%%%%%%%%%

\theoremstyle{plain}
\newtheorem{Th}{Teorema}               %%% Theorem 1.1.1
\newtheorem{Tmon}{Teoremón}
\newtheorem{Prop}{Proposición}         %%% Proposition 1.1.2
\newtheorem{Lem}{Lema}                 %%% Lemma 3
\newtheorem{Cor}{Corolario}            %%% Corollary 4

\theoremstyle{definition}
\newtheorem{Def}{Definición}           %%% Definition 5
\newtheorem{Ex}{Ejemplo}               %%% Example 6
\newtheorem{Ej}{Ejercicio}             %%% Ejercicio 7
\newtheorem{Hec}[Th]{Hecho}            %%% Hecho 1.1.8

\theoremstyle{remark}
\newtheorem{Rmk}[Th]{Observación}      %%%Remark 1.1.9
\newtheorem*{nonum-Rmk}{Observación}         %%% No number Fact
\newtheorem*{Notn}{Notaci\'on}        %% Notaciones
\newtheorem*{Warn}{Advertencia}       %% Advertencias

\numberwithin{equation}{section}

%% Accomodations 

\makeatletter
\def\moverlay{\mathpalette\mov@rlay}
\def\mov@rlay#1#2{\leavevmode\vtop{%
   \baselineskip\z@skip \lineskiplimit-\maxdimen
   \ialign{\hfil$\m@th#1##$\hfil\cr#2\crcr}}}
\newcommand{\charfusion}[3][\mathord]{
    #1{\ifx#1\mathop\vphantom{#2}\fi
        \mathpalette\mov@rlay{#2\cr#3}
      }
    \ifx#1\mathop\expandafter\displaylimits\fi}
\makeatother

% Greek letters:

\newcommand{\al}{\alpha}                %% short for  \alpha
\newcommand{\bt}{\beta}                 %% short for  \beta
\newcommand{\Dl}{\Delta}                %% short for  \Delta
\newcommand{\dl}{\delta}                %% short for  \delta
\newcommand{\eps}{\varepsilon}          %% short for  \varepsilon
\newcommand{\Ga}{\Gamma}                %% short for  \Gamma
\newcommand{\ga}{\gamma}                %% short for  \gamma
\newcommand{\La}{\Lambda}               %% short for  \Lambda
\newcommand{\la}{\lambda}               %% short for  \lambda
\newcommand{\Om}{\Omega}                %% short for  \Omega
\newcommand{\om}{\omega}                %% short for  \omega
\newcommand{\Sg}{\Sigma}                %% short for  \Sigma
\newcommand{\sg}{\sigma}                %% short for  \sigma
\newcommand{\te}{\theta}                %% short for  \theta
\newcommand{\vf}{\varphi}               %% short for  \varphi
\newcommand{\ze}{\zeta}                 %% short for  \zeta

%Boldface letters

\newcommand{\bC}{\mathbb{C}}    %%% números complejos
\newcommand{\bN}{\mathbb{N}}    %%% números naturales
\newcommand{\bP}{\mathbb{P}}        %% números enteros positivos
\newcommand{\bQ}{\mathbb{Q}}    %%% números racionales
\newcommand{\bR}{\mathbb{R}}    %%% números reales
\newcommand{\bS}{\mathbb{S}}    %%% esfera
\newcommand{\bZ}{\mathbb{Z}}    %%% números enteros

%Script letters:

\newcommand{\cA}{\mathcal{A}}           %% formas diferenciales
\newcommand{\cB}{\mathcal{B}}           %% una base vectorial
\newcommand{\cC}{\mathcal{C}}           %% otra base vectorial
\newcommand{\cF}{\mathcal{F}}           %% espacio de Fock
\newcommand{\cL}{\mathcal{L}}           %% operadores lineales
\newcommand{\cM}{\mathcal{M}}           %% multiplicadores
\newcommand{\cN}{\mathcal{N}}           %% funciones nulas
\newcommand{\cP}{\mathcal{P}}           %% una particion
\newcommand{\cR}{\mathcal{R}}           %% funciones representativas
\newcommand{\cS}{\mathcal{S}}           %% funciones de Schwartz


%Brackets

\newcommand{\bonj}[1]{\left\lbrack#1\right\rbrack}
\newcommand{\obonj}[1]{\left\rbrack#1\right\lbrack}
\newcommand{\rbonj}[1]{\left\rbrack#1\right\rbrack}
\newcommand{\lbonj}[1]{\left\lbrack#1\right\lbrack}
\newcommand{\snm}[1]{\|#1\|}           %small norma
\newcommand{\nm}[1]{\left\|#1\right\|} %norma pegadita
\newcommand{\pnm}[1]{\biggl|\biggl|#1\biggr|\biggr|}
\newcommand{\set}[1]{\{\,#1\,\}}    %% set notation
\newcommand{\floor}[1]{\lfloor#1\rfloor} %% mayor entero <= x
\newcommand{\Set}[1]{\biggl\{\,#1\,\biggr\}} %% set notation (large)
\newcommand\quot[2]{
        \mathchoice
            {% \displaystyle
                \text{\raise1ex\hbox{$#1$}\Big/\lower1ex\hbox{$#2$}}%
            }
            {% \textstyle
                {^{ #1}/_{ #2}}
            }
            {% \scriptstyle
                {^{ #1}/_{ #2}}
            }
            {% \scriptscriptstyle
                {^{ #1}/_{ #2}}
            }
    }
\newcommand*\squot[2]{{^{ #1}/_{ #2}}}%%%small quotient

%Symbols 

\newcommand{\x}{\times}
\renewcommand{\geq}{\geqslant}          %% mayor o igual (variante)
\newcommand{\hookto}{\hookrightarrow}     %% inclusion arrow
\newcommand{\isom}{\simeq}              %% isomorfismo
\renewcommand{\l}{\ell}                   %% ele cursiva
\renewcommand{\leq}{\leqslant}          %% menor o igual (variante)
\newcommand{\less}{\setminus}           %% set difference
\newcommand{\To}{\Rightarrow}
\newcommand{\ov}{\overline}
\newcommand{\un}{\underline}
\newcommand{\del}{\partial}
\newcommand{\ind}{\mathbf{1}}       %%%indicator function

%%% Small fractions in displays:

\newcommand{\half}{{\mathchoice{\nhalf}{\thalf}{\shalf}{\shalf}}} %%display text script script^2
\newcommand{\happi}{{\tfrac{\pi}{2}}} %% small fraction  \pi/2
\newcommand{\quarter}{\tfrac{1}{4}} %% small fraction  1/4
\newcommand{\nhalf}{\frac{1}{2}}
\newcommand{\shalf}{{\scriptstyle\frac{1}{2}}} %% tiny fraction 1/2
\newcommand{\thalf}{{\tfrac{1}{2}}} %% small fraction  1/2
\newcommand{\third}{\tfrac{1}{3}}   %% small fraction  1/3 %Hay que renew porque mathabx toma second y third como x'' y x''' por ejemplo

\newcommand{\ihalf}{{\tfrac{i}{2}}} %% small fraction  i/2

\newcommand{\suci}{_{i=1}^\infty} %% diminutivo
\newcommand{\suck}{_{k=1}^\infty} %% diminutivo
\newcommand{\sucl}{_{\l=1}^\infty} %% diminutivo
\newcommand{\sucm}{_{m=1}^\infty} %% diminutivo
\newcommand{\sucn}{_{n=1}^\infty} %% diminutivo

\newcommand*{\Cdot}{{\raisebox{-0.25ex}{\scalebox{1.5}{$\cdot$}}}}      %% cdot más grande
\renewcommand{\.}{\Cdot}                %% producto escalar
\newcommand{\cupdot}{\charfusion[\mathbin]{\cup}{\.}}
\newcommand{\bigcupdot}{\charfusion[\mathbin]{\bigcup}{\.}}
\DeclareMathOperator{\Var}{Var}     %%%variance

\begin{document}

\begin{frame}
  \titlepage
\end{frame}

\begin{frame}{Agenda}
  \tableofcontents
  % You might wish to add the option [pausesections]
\end{frame}


% Structuring a talk is a difficult task and the following structure
% may not be suitable. Here are some rules that apply for this
% solution: 

% - Exactly two or three sections (other than the summary).
% - At *most* three subsections per section.
% - Talk about 30s to 2min per frame. So there should be between about
%   15 and 30 frames, all told.

% - A conference audience is likely to know very little of what you
%   are going to talk about. So *simplify*!
% - In a 20min talk, getting the main ideas across is hard
%   enough. Leave out details, even if it means being less precise than
%   you think necessary.
% - If you omit details that are vital to the proof/implementation,
%   just say so once. Everybody will be happy with that.

\section{Funciones Lebesgue Medibles}

\subsection{Definición y Propiedades}

\begin{frame}{Definición}
\begin{Def}\label{def:funcMedible}
Sea $f:E\to\ov\bR=\bR\cup\set{\pm\infty}$. Decimos que $f$ es \alert{medible} si para todo $a\in\bR$ vale que 
$$\set{x\in E:\ f(x)>a}\in\cM.$$
\end{Def}
De ahora en adelante
$$\set{f>a}=\set{x\in E:\ f(x)>a}.$$
Note que 
$$E=\bigcup\suck\set{f>-k}\cup\set{f=-\infty}.$$
Entonces $E$ es medible si y s\'olo si $\set{f=-\infty}$ es medible. Por el resto de esta secci\'on asumimos que $E$ es medible.
\end{frame}

\begin{frame}{Ejemplos}
  \begin{enumerate}[(i)]
    \item Sea $f:\bR^d\to\bR$ continua. Entonces $\set{f>a}$ es abierto.
    \item Si $f=\ind_A$, entonces 
    $$\set{f>a}=\begin{cases}
      E&\text{si}\ a<0\\
      A&\text{si}\ 0\leq a<1\\
      \emptyset&\text{si}\ a\geq 1
    \end{cases}$$
  \end{enumerate}
\end{frame}

\begin{frame}{Equivalencias}
  \begin{Th}\label{th:equivsMedib}
    Sea $f:E\to\bR$ con $E$ medible. Entonces $f$ es medible si se cumple cualquiera de los siguientes postulados para todo $a\in\bR$.
    \begin{enumerate}[(i)]
      \item $\set{f>a}$ es medible.
      \item $\set{f<a}$ es medible.
      \item $\set{f\leq a}$ es medible.
      \item $\set{f\geq a}$ es medible.
    \end{enumerate}
  \end{Th}
\end{frame}

\begin{frame}{Prueba del Teorema}
  Note que
  \begin{align*}
    \set{f>a}&=\bigcup\sucn \Set{f\geq a+\frac1n}=\set{f\leq a}^c\\
    \set{f\geq a}&=\bigcap\sucn \Set{f> a-\frac1n}=\set{f< a}^c
  \end{align*}
\end{frame}

\begin{frame}
  Si $f: E\to\bR$ es medible, entonces los conjuntos
  \begin{gather*}
    \set{f>-\infty}=\bigcup\suck\set{f>-k},\ \set{f<\infty}=\bigcup\suck\set{f\leq k},\\
    \set{f=\infty},\ \set{a\leq f\leq b},\ \set{a\leq f<b}
  \end{gather*}
  son medibles.
\end{frame}

\begin{frame}{Funciones Borel Medibles}
  \begin{Def}\label{def:funcBorelMedible}
    Diremos que $f:E\to\bR$ es \alert{Borel medible} si 
    \begin{enumerate}
      \item $E\in\cB$.
      \item $\set{f>a}\in\cB$ para $a\in\bR$. 
    \end{enumerate}
  \end{Def}
  Esta definición nos será útil para realizar los ejercicios.
  \begin{Th}\label{th:caracMedibAbiertos}
    Sea $f:E\to\bR$. Entonces $f$ es medible si y sólo si $f^{-1}(G)$ es medible para todo abierto $G$.
  \end{Th}
\end{frame}

\begin{frame}{Prueba del Teorema}
  \begin{itemize}
    \item Supongamos que la imagen inversa de abiertos es medible. Entonces si $G=\obonj{a,\infty}$, tenemos que 
    $$f^{-1}(G)=\set{x\in E:\ f(x)>a}.$$
    \item Por otro lado, si $G$ es un abierto en $\bR$, entonces $G=\bigcup\suck\obonj{a_k,b_k}$. De esta manera 
    $$f^{-1}(G)=\bigcup\suck f^{-1}(\obonj{a_k,b_k})=\bigcup\suck\set{a_k<f<b_k}.$$
  \end{itemize}
\end{frame}

\begin{frame}{Composición}
  \begin{Lem}\label{lem:compContConMed}
    Sea $f:E\to\bR$ medible y $\phi:\bR\to\bR$ continua. Entonces $\phi\circ f$ es medible.
  \end{Lem}
  Si $G$ es abierto, entonces 
  $$(\phi\circ f)^{-1}(G)=f^{-1}(\phi^{-1}(G)).$$
  Como $\phi$ es continua, entonces $\phi^{-1}(G)$ es abierto y $f^{-1}(\phi^{-1}(G))$ es medible.\par 
  En particular si $f$ es medible, $|f|$, $|f|^p$, $e^{cf}$, $f^{+}=\max\set{f,0}$ y $f^{-}=\min\set{f,0}$ son medibles.
\end{frame}

\begin{frame}{Casi Por Doquier}
  \begin{Def}\label{def:cpd}
    Diremos que una propiedad se cumple \alert{casi por doquier} si se cumple excepto en un conjunto de medida cero.
  \end{Def}
  \begin{Lem}\label{lem:igualdadCPDimplicaMedible}
    Sean $f:E\to\bR$, $g:E\to\bR$. Si $f$ es medible y $g=f$ casi por doquier, entonces $g$ es medible.
  \end{Lem}
\end{frame}

\begin{frame}{Prueba del Lema}
  Sea $a\in\bR$, entonces
  $$\set{g>a}=(\set{g>a}\cap \set{f=g})\cup(\set{g>a}\cap\set{f\neq g}).$$
  El conjunto $\set{g>a}\cap \set{f=g}=\set{f>a}\cap \set{f=g}$ es medible y $\set{f\neq g}$ tiene medida cero de manera tal que $\set{g>a}$ es la unión de medibles.
\end{frame}

\begin{frame}{Una Variante}
  Tenemos una variante del lema \ref{lem:compContConMed} para funciones con valores infinitos. 
  \begin{Lem}\label{lem:VarCompContMed}
    Sea $f:E\to\ov\bR$ medible con 
    $$m(f=\infty)=m(f=-\infty)=0.$$
    Entonces $\phi\circ f$ es medible para $\phi$ continua.
  \end{Lem}
  Si llamamos $F=\set{x\in E:\ f\in\bR}$ y consideramos $f_1=f$ cuando $x\in F$ y cero si no, entonces $\phi\circ f_1=\phi\circ f$ c.p.d. Como $\phi\circ f_1$ es medible, entonces $\phi\circ f$ también lo es.
\end{frame}

\subsection{Álgebra de Funciones Medibles}

\begin{frame}{Espacio Vectorial de Funciones Medibles}
  \begin{Lem}\label{lem:conjMedibEntreFuncs}
    Sean $f:E\to\bR$, $g:E\to\bR$ medibles. Entonces $\set{f>g}$ es medible.
  \end{Lem}
  Sea $\set{q_n}\sucn$ una enumeración de $\bQ$. Entonces 
  $$\set{f>g}=\bigcup\sucn\set{f>q_n>g}=\bigcup\sucn\set{f>q_n}\cap\set{q_n>g}$$
  es una unión de medibles.
\end{frame}

\section*{Resumen}

\subsection*{Qu\'e vimos hoy}

\begin{frame}{Resumen}

  % Keep the summary *very short*.
  \begin{itemize}
  \item La definición \ref{def:funcMedible} de función medible y el teorema \ref{th:equivsMedib} que nos da condiciones equivalentes.
  \item La definición \ref{def:funcBorelMedible} de función Borel medible.
  \item La caracterización \ref{th:caracMedibAbiertos} de funciones medibles con abiertos.
  \item El lema \ref{lem:compContConMed} sobre la composición de funciones medibles con continuas.
  \item La definición \ref{def:cpd} de propiedades que se dan casi por doquier junto con el lema \ref{lem:igualdadCPDimplicaMedible}.
  \item El lema \ref{lem:VarCompContMed} que es levemente diferente del lema \ref{lem:compContConMed}.
  \item El lema \ref{lem:conjMedibEntreFuncs} técnico necesario para hablar del álgebra de medibles.
  \end{itemize}
  
\end{frame}

\subsection*{Ejercicios a trabajar}
\begin{frame}{Ejercicios}
    
  \begin{itemize}
    \item
      Lista 15
      \begin{itemize}
      \item 
      \end{itemize}
    \end{itemize}
  
\end{frame}


% All of the following is optional and typically not needed. 
\appendix
\section<presentation>*{\appendixname}
\subsection<presentation>*{Lectura adicional}

\begin{frame}[allowframebreaks]
  \frametitle<presentation>{Lecturas adicionales}
    
  \begin{thebibliography}{10}
    
  \beamertemplatebookbibitems
  % Start with overview books.

  \bibitem{CambroNotas}
    S.Cambronero.
    \newblock {\em Notas MA0505}.
    \newblock 20XX.

    \bibitem{NachoNotas}
    I.Rojas
    \newblock {\em Notas MA0505}.
    \newblock 2018.
 
  \end{thebibliography}
  
\end{frame}
%% 6 - 2:10:52 48
%% 7 - 1:17:44 44
%% 8 - 27:37 69
%% 9 - 1:07:47 86 + 59:38 40 approx 2 h c 7 min
%% 10 - 2:22:39 63
%% 11 - Approx 2h c 2 min
%% 12 - Motiv y figs 1:00:17.53, en tot: 2:22:23.43
%% 13 - 1:18:03.51
%% 14 - 2:05:13.72
\end{document}


