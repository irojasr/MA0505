\documentclass[utf8]{beamer}

\mode<presentation>
{
  \usetheme{Warsaw}
  \setbeamercovered{transparent}
}


\usepackage{amsfonts,mathtools,amssymb}
\usepackage[spanish]{babel}
\usepackage{times}
\usepackage[T1]{fontenc}
\usepackage[shortlabels]{enumitem}
\usepackage{tikz}
\usepackage{physics}

\title[MA0505]{MA0505 - An\'alisis I}
\subtitle{Lecci\'on XVIII: La Integral de Lebesgue I}

\author{Pedro M\'endez\inst{1}}

\institute[Universidad de Costa Rica] % (optional, but mostly needed)
{
  \inst{1}%
  Departmento de Matem\'atica Pura y Ciencias Actuariales\\
  Universidad de Costa Rica
  }

\date[I-2021] {Semestre I, 2021}

%%%%%%%%% === Theorems and suchlike === %%%%%%%%%%%%%%

\theoremstyle{plain}
\newtheorem{Th}{Teorema}               %%% Theorem 1.1.1
\newtheorem{Tmon}{Teoremón}
\newtheorem{Prop}{Proposición}         %%% Proposition 1.1.2
\newtheorem{Lem}{Lema}                 %%% Lemma 3
\newtheorem{Cor}{Corolario}            %%% Corollary 4

\theoremstyle{definition}
\newtheorem{Def}{Definición}           %%% Definition 5
\newtheorem{Ex}{Ejemplo}               %%% Example 6
\newtheorem{Ej}{Ejercicio}             %%% Ejercicio 7
\newtheorem{Hec}[Th]{Hecho}            %%% Hecho 1.1.8

\theoremstyle{remark}
\newtheorem{Rmk}[Th]{Observación}      %%%Remark 1.1.9
\newtheorem*{nonum-Rmk}{Observación}         %%% No number Fact
\newtheorem*{Notn}{Notaci\'on}        %% Notaciones
\newtheorem*{Warn}{Advertencia}       %% Advertencias

\numberwithin{equation}{section}

%% Accomodations 

\makeatletter
\def\moverlay{\mathpalette\mov@rlay}
\def\mov@rlay#1#2{\leavevmode\vtop{%
   \baselineskip\z@skip \lineskiplimit-\maxdimen
   \ialign{\hfil$\m@th#1##$\hfil\cr#2\crcr}}}
\newcommand{\charfusion}[3][\mathord]{
    #1{\ifx#1\mathop\vphantom{#2}\fi
        \mathpalette\mov@rlay{#2\cr#3}
      }
    \ifx#1\mathop\expandafter\displaylimits\fi}
\makeatother

% Greek letters:

\newcommand{\al}{\alpha}                %% short for  \alpha
\newcommand{\bt}{\beta}                 %% short for  \beta
\newcommand{\Dl}{\Delta}                %% short for  \Delta
\newcommand{\dl}{\delta}                %% short for  \delta
\newcommand{\eps}{\varepsilon}          %% short for  \varepsilon
\newcommand{\Ga}{\Gamma}                %% short for  \Gamma
\newcommand{\ga}{\gamma}                %% short for  \gamma
\newcommand{\La}{\Lambda}               %% short for  \Lambda
\newcommand{\la}{\lambda}               %% short for  \lambda
\newcommand{\kp}{\kappa}                %% short for  \kappa
\newcommand{\Om}{\Omega}                %% short for  \Omega
\newcommand{\om}{\omega}                %% short for  \omega
\newcommand{\Sg}{\Sigma}                %% short for  \Sigma
\newcommand{\sg}{\sigma}                %% short for  \sigma
\newcommand{\te}{\theta}                %% short for  \theta
\newcommand{\vf}{\varphi}               %% short for  \varphi
\newcommand{\ze}{\zeta}                 %% short for  \zeta

%Boldface letters

\newcommand{\bC}{\mathbb{C}}    %%% números complejos
\newcommand{\bN}{\mathbb{N}}    %%% números naturales
\newcommand{\bP}{\mathbb{P}}        %% números enteros positivos
\newcommand{\bQ}{\mathbb{Q}}    %%% números racionales
\newcommand{\bR}{\mathbb{R}}    %%% números reales
\newcommand{\bS}{\mathbb{S}}    %%% esfera
\newcommand{\bZ}{\mathbb{Z}}    %%% números enteros

%Script letters:

\newcommand{\cA}{\mathcal{A}}           %% formas diferenciales
\newcommand{\cB}{\mathcal{B}}           %% una base vectorial
\newcommand{\cC}{\mathcal{C}}           %% otra base vectorial
\newcommand{\cF}{\mathcal{F}}           %% espacio de Fock
\newcommand{\cL}{\mathcal{L}}           %% operadores lineales
\newcommand{\cM}{\mathcal{M}}           %% multiplicadores
\newcommand{\cN}{\mathcal{N}}           %% funciones nulas
\newcommand{\cP}{\mathcal{P}}           %% una particion
\newcommand{\cR}{\mathcal{R}}           %% funciones representativas
\newcommand{\cS}{\mathcal{S}}           %% funciones de Schwartz


%Brackets

\newcommand{\bonj}[1]{\left\lbrack#1\right\rbrack}
\newcommand{\obonj}[1]{\left\rbrack#1\right\lbrack}
\newcommand{\rbonj}[1]{\left\rbrack#1\right\rbrack}
\newcommand{\lbonj}[1]{\left\lbrack#1\right\lbrack}
\newcommand{\snm}[1]{\|#1\|}           %small norma
\newcommand{\nm}[1]{\left\|#1\right\|} %norma pegadita
\newcommand{\pnm}[1]{\biggl|\biggl|#1\biggr|\biggr|}
\newcommand{\set}[1]{\{\,#1\,\}}    %% set notation
\newcommand{\floor}[1]{\lfloor#1\rfloor} %% mayor entero <= x
\newcommand{\Set}[1]{\biggl\{\,#1\,\biggr\}} %% set notation (large)
\newcommand\quot[2]{
        \mathchoice
            {% \displaystyle
                \text{\raise1ex\hbox{$#1$}\Big/\lower1ex\hbox{$#2$}}%
            }
            {% \textstyle
                {^{ #1}/_{ #2}}
            }
            {% \scriptstyle
                {^{ #1}/_{ #2}}
            }
            {% \scriptscriptstyle
                {^{ #1}/_{ #2}}
            }
    }
\newcommand*\squot[2]{{^{ #1}/_{ #2}}}%%%small quotient

%Symbols 

\newcommand{\x}{\times}
\renewcommand{\geq}{\geqslant}          %% mayor o igual (variante)
\newcommand{\hookto}{\hookrightarrow}     %% inclusion arrow
\newcommand{\isom}{\simeq}              %% isomorfismo
\renewcommand{\l}{\ell}                   %% ele cursiva
\renewcommand{\leq}{\leqslant}          %% menor o igual (variante)
\newcommand{\less}{\setminus}           %% set difference
\newcommand{\To}{\Rightarrow}
\newcommand{\ov}{\overline}
\newcommand{\un}{\underline}
\newcommand{\del}{\partial}
\newcommand{\ind}{\mathbf{1}}       %%%indicator function

%%% Small fractions in displays:

\newcommand{\half}{{\mathchoice{\nhalf}{\thalf}{\shalf}{\shalf}}} %%display text script script^2
\newcommand{\happi}{{\tfrac{\pi}{2}}} %% small fraction  \pi/2
\newcommand{\quarter}{\tfrac{1}{4}} %% small fraction  1/4
\newcommand{\nhalf}{\frac{1}{2}}
\newcommand{\shalf}{{\scriptstyle\frac{1}{2}}} %% tiny fraction 1/2
\newcommand{\thalf}{{\tfrac{1}{2}}} %% small fraction  1/2
\newcommand{\third}{\tfrac{1}{3}}   %% small fraction  1/3 %Hay que renew porque mathabx toma second y third como x'' y x''' por ejemplo

\newcommand{\ihalf}{{\tfrac{i}{2}}} %% small fraction  i/2

\newcommand{\suci}{_{i=1}^\infty} %% diminutivo
\newcommand{\sucj}{_{j=1}^\infty} %% diminutivo
\newcommand{\suck}{_{k=1}^\infty} %% diminutivo
\newcommand{\sucl}{_{\l=1}^\infty} %% diminutivo
\newcommand{\sucm}{_{m=1}^\infty} %% diminutivo
\newcommand{\sucn}{_{n=1}^\infty} %% diminutivo

\newcommand*{\Cdot}{{\raisebox{-0.25ex}{\scalebox{1.5}{$\cdot$}}}}      %% cdot más grande
\renewcommand{\.}{\Cdot}                %% producto escalar
\newcommand{\cupdot}{\charfusion[\mathbin]{\cup}{\.}}
\newcommand{\bigcupdot}{\charfusion[\mathbin]{\bigcup}{\.}}
\DeclareMathOperator{\Var}{Var}     %%%variance

\begin{document}

\begin{frame}
  \titlepage
\end{frame}

\begin{frame}{Agenda}
  \tableofcontents
  % You might wish to add the option [pausesections]
\end{frame}


% Structuring a talk is a difficult task and the following structure
% may not be suitable. Here are some rules that apply for this
% solution: 

% - Exactly two or three sections (other than the summary).
% - At *most* three subsections per section.
% - Talk about 30s to 2min per frame. So there should be between about
%   15 and 30 frames, all told.

% - A conference audience is likely to know very little of what you
%   are going to talk about. So *simplify*!
% - In a 20min talk, getting the main ideas across is hard
%   enough. Leave out details, even if it means being less precise than
%   you think necessary.
% - If you omit details that are vital to the proof/implementation,
%   just say so once. Everybody will be happy with that.

\section{La Integral de Lebesgue}

\subsection{Áreas y Gráficos}
\begin{frame}{El Área Bajo la Curva}
  Sea $f:E\to\ov\bR$ medible con $E$ medible y $f\geq 0$. Definimos 
  $$R(f,E)=\set{(x,y)\in\bR^{d+1}:\ x\in E,\ 0\leq y\leq f(x)}.$$
  Surge la pregunta, ¿es $R(f,E)$ medible? Analicemos el caso $f(x)=a$ dentro de $A\subseteq E$ y $a>0$.s
\end{frame}

\begin{frame}
  Entonces ocurre que 
  \begin{gather*}
    R(f,E)=\set{(x,y)\in\bR^{d+1}:\ x\in E,\ 0\leq y\leq f(x)}\\
    =\set{(x,y)\in\bR^{d+1}:\ x\in E,\ 0\leq y\leq a}\\
    \cup\set{(x,y)\in\bR^{d+1}:\ x\in E\less A,\ y=0}\\
    =A\x[0,a]\cup E\less A\x \set{0}.
  \end{gather*}
  \begin{Lem}\label{lem:casoConstArea}
    Sea $A\subseteq E$ medible y $a\geq 0$. Entonces $A\x [0,a]$ es medible y su medida es $am(A)$.
  \end{Lem}
\end{frame}

\begin{frame}{Prueba del Lema}
  \begin{enumerate}
    \item Sea $A=[a_1,b_1]\x\dots\x[a_d,b_d]$. Entonces es claro que se cumple el lema.
    \item Sea $A$ un abierto, entonces existen $I_k$'s, cajas en $d$ dimensiones, tales que $A=\bigcup\suck I_k$ con $I_k^o\cap I_k^o=\emptyset$ si $j\neq k$. Entonces $A\x[0,a]=\bigcup\suck I_k\x[0,a]$ es un conjunto medible. Y como 
    $$(I_k\x [0,a])^o\cap (I_j\x [0,a])^o=\emptyset,\ k\neq j,$$
    entonces 
    $$m(A\x[0,a])=\sum\suci(I_i\x[0,a])=a\sum\suci m(I_i)=am(A).$$
  \end{enumerate}
\end{frame}

\begin{frame}{Prueba del Lema}
  \newcounter{saveenumi}
  \begin{enumerate}
    \item Sea $A=[a_1,b_1]\x\dots\x[a_d,b_d]$. Entonces es claro que se cumple el lema.
    \item Sea $A$ un abierto, entonces existen $I_k$'s, cajas en $d$ dimensiones, tales que $A=\bigcup\suck I_k$ con $I_k^o\cap I_k^o=\emptyset$ si $j\neq k$. Entonces $A\x[0,a]=\bigcup\suck I_k\x[0,a]$ es un conjunto medible. Y como 
    $$(I_k\x [0,a])^o\cap (I_j\x [0,a])^o=\emptyset,\ k\neq j,$$
    entonces 
    $$m(A\x[0,a])=\sum\suci(I_i\x[0,a])=a\sum\suci m(I_i)=am(A).$$
    \setcounter{saveenumi}{\value{enumi}}
  \end{enumerate}
\end{frame}

\begin{frame}{Continuamos}\label{ej:medCeroTimesInter}
  \begin{enumerate}
    \setcounter{enumi}{\value{saveenumi}}
    \item Sea $A=\bigcap\sucj G_j$ un $G_\dl$ con $G_i$ abierto para $i\geq 1$ y $G_{i+1}\subseteq G_i$. \alert{¿Por qué podemos asumir esto?} Luego $A\x [0,a]=\bigcap\sucj G_j\x[0,a]$ con $G_{i+1}\x[0,a]\subseteq G_i\x[0,a]$ medibles con medida finita. Entonces 
    $$m(A\x [0,a])=\lim_{i\to\infty}m(G_i\x[0,a])=\lim_{i\to\infty}am(G_i)=am(A).$$
    \item Si $A=H\less Z$ con $H$ $G_\dl$ y $Z$ de medida cero, entonces por el paso anterior $H\x[0,a]$ es medible y $m(H\x[0,a])=a m(H)$. \alert{(Ej:$m_e(Z\x[0,a])=0$)} Entonces 
    $$A\x[0,a]=(H\x[0,a])\less (Z\x[0,a])$$
    y $m(A\x[0,a])=m(H\x[0,a])=a m(H)=am(A)$.
    \setcounter{saveenumi}{\value{enumi}}
  \end{enumerate}
\end{frame}

\begin{frame}{Terminamos}
  \begin{enumerate}
    \setcounter{enumi}{\value{saveenumi}}
    \item Finalmente si $m(A)=\infty$, entonces llamemos $A_k=A\cap B(0,k)$. De esta manera $A=\bigcup\suck A_k$, con $A_k\subseteq A_{k+1}$ medibles y acotados. Por los argumentos anteriores 
    $$A_k\x[0,a]\subseteq A_{k+1}\x[0,a]$$
    son conjuntos medibles tales que $m(A_k\x[0,a])=am(A_k)$. Entonces
    \begin{gather*}
      m(A\x[0,a])=m\left(\bigcup\suck A_k\x [0,a]\right)=\lim_{k\to\infty}m(A_k\x[0,a])\\
      =\lim_{k\to\infty} m(A_k\x[0,a])=\lim_{k\to\infty} am(A_k)=am(A).
    \end{gather*}
    
  \end{enumerate}
\end{frame}

\begin{frame}{Gráficos}
  Sea $f:E\to\bR$ medible con $f\geq 0$. Entonces existe una sucesión creciente de funciones simple $\set{\phi_k}\suck$ que satisfacen
  $$\phi_k(x)\xrightarrow[k\to\infty]{}f(x)\ \text{c.p.d.},\ x\in E.$$
  Luego si $0\leq y\leq f(x)$, existe $\phi_k$ simple tal que $0\leq y\leq \phi_k(x)$. Entonces
  $$R(f,E)=\bigcup\suck R(\phi_k,E)\cup\set{(x,y)\in\bR^{d+1}:\ x\in E,\ f(x)=y}.$$
  Concluimos que $R(f,E)$ es medible si 
  \begin{enumerate}[(a)]
    \item $R(\phi_k,E)$ es medible.
    \item $\Ga(f,E)=\set{(x,f(x)):\ x\in E}$ es medible.
  \end{enumerate}
\end{frame}

\begin{frame}{El Gráfico tiene Medida Cero}
  \begin{Lem}\label{lem:GrafMedidaCero}
    Sea $f:E\to\bR$ medible, entonces $m_e(\Ga(f,E))=0$.
  \end{Lem}
  Asumamos primero que $E$ tiene medida finita. Sea $\eps>0$ y 
  $$E_k=\set{x\in E:\ k\eps\leq f(x)<(k+1)\eps},$$
  entonces $E=\bigcup\suck E_k$. Note que 
  $$\Ga(f,E_k)\subseteq (E_k\x[0,(k+1)\eps]\less(E_k\x[0,k\eps])).$$
\end{frame}

\begin{frame}{El Gráfico tiene Medida Cero}
  De lo anterior tenemos 
  $$m_e(\Ga(f,E_k))\leq m(E_k\x[0,(k+1)\eps])-m(E_k\x[0,k\eps])\leq\eps m(E).$$
  Por lo tanto vale que 
  $$m_e(\Ga(f,E_k))=0$$
  y así 
  $$m_E(\bigcup\suck\Ga(f,E_k))=m_e(\Ga(f,E))=0.$$
  \begin{Ej}\label{ej:termPruebaGrafMedCero}
    Terminar la prueba de este lema es un ejercicio.
  \end{Ej}
\end{frame}

\begin{frame}{La Concluisión}
  Finalmente si 
  $$\phi_k(x)=\sum_{i=1}^ma_i\ind_{A_i},$$
  con $a_i\neq a_j$, $A_i\cap A_j=\emptyset$ para $i\neq j$ y $E=\bigcup_{i=1}^m A_i$, tenemos que 
  \begin{gather*}
    \set{(x,y)\in\bR^d:\ x\in E,\ 0\leq y\leq\phi_k(x)}\\
    =\bigcup_{k=1}^m\set{(x,y)\in\bR^d:\ x\in A_k,\ 0\leq y\leq a_k}
  \end{gather*}
  es un conjunto medible.
\begin{Th}\label{th:AreaBajoCurvaMedible}
Sea $f:E\to\bR$ medible, con $E$ medible. Entonces $R(f,E)\subseteq\bR^{d+1}$ es un conjunto medible.
\end{Th}
\end{frame}

\subsection{La Integral y sus Propiedades}

\begin{frame}{La Definición}
  \begin{Def}\label{def:intLebesgue}
  Dada $f:E\to\bR$ medible tal que $f\geq 0$, definimos su \alert{integral de Lebesgue} como 
  $$\int\limits_E f\dd x=m(R(f,E)).$$
  \end{Def}
  
\end{frame}

\begin{frame}{Una Observación}
  Note que si $\phi(x)=\sum_{i=1}^ma_i\ind_{A_i}$ con $A_i\cap A_j=\emptyset$ para $i\neq j$. Entonces, si $a_i\neq 0$,
  $$\int\limits_E f(x)\dd x=\sum_{i=1}^ma_im(A_i)$$
  pues 
  \begin{align*}
    &m(\set{(x,y):x\in E,\ 0\leq y\leq f(x)})\\
    =&m(\set{f=0}\x\set{0})+m\left(\bigcup_{i=1}^m\set{(x,y):\ x\in A_j,\ 0\leq y\leq a_i}\right)\\
    =&\sum_{i=1}^ma_im(A_i).
  \end{align*}
\end{frame}

\begin{frame}{Propiedades}
  \begin{Th}\label{th:PropsIntLeb}
    Sean $f,g:E\to\lbonj{0,\infty}$ medibles. 
    \begin{enumerate}[(i)]
      \item Si $0\leq g\leq f$, entonces 
      $$\int\limits_E g(x)\dd x\leq\int\limits_E f(x)\dd x.$$
      \item Si $\int\limits_E f(x)\dd x<\infty$, entonces $f\neq\infty$ c.p.d.
      \item Si $E_1\subseteq E_2\subseteq E$, entonces 
      $$\int\limits_{E_1}f(x)\dd x\leq \int\limits_{E_2}f(x)\dd x.$$
    \end{enumerate}
  \end{Th}
\end{frame}

\begin{frame}{Prueba del Teorema}
  \begin{itemize}
    \item Dado que $0\leq g\leq f$, tenemos que \begin{gather*}
      \set{(x,y):\ x\in E,\ 0\leq y\leq g(x)}\\
      \subseteq\set{(x,y):\ x\in E,\ 0\leq y\leq f(x)}.
    \end{gather*}
    Luego $R(g,E)\subseteq R(f,E)$.
    \item Por otro lado si $E_1\subseteq E_2$, entonces $\ind_{E_1}f\leq \ind_{E_2}f$. Como 
    \begin{gather*}
      \set{(x,y):\ x\in E,\ 0\leq y\leq \ind_{E_i}f}\\
      =(E\less E_i\x\set{0})\cup\set{(x,y):\ x\in E_1,\ 0\leq y\leq f}
    \end{gather*}
    Entonces $R(\ind_{E_i}f,E)=R(f,E_i)$. Es decir $\int\limits_{E_i}f(x)\dd x\leq \int\limits_{E_2}f(x)\dd x$.
  \end{itemize}
\end{frame}

\begin{frame}{Terminamos la Prueba}
  \begin{itemize}
    \item Por otro lado si $E_1=\set{f=\infty}$, tomemos $f(x)=n\ind_{E_i}(x)$. Entonces $g(x)\leq f(x)$ para $x\in E$. Luego 
    $$nm(E_1)\leq\int\limits_Ef(x)\dd x,$$
    y por lo tanto $m(E_1)=0$
  \end{itemize}
\end{frame}

%%Llevamos 1:03:50.64
\section*{Resumen}

\subsection*{Qu\'e vimos hoy}

\begin{frame}{Resumen}

  % Keep the summary *very short*.
  \begin{itemize}
  \item El lema \ref{lem:casoConstArea} sobre el área bajo la curva de una función constante.
  \item El lema \ref{lem:GrafMedidaCero} sobre medida cero del gráfico.
  \item El teorema \ref{th:AreaBajoCurvaMedible} sobre el area bajo la curva.
  \item La definición \ref{def:intLebesgue} de la integral de Lebesgue.
  \item El teorema \ref{th:PropsIntLeb} sobre las propiedades de la integral de Lebesgue.
  \end{itemize}
  
\end{frame}

\subsection*{Ejercicios a trabajar}
\begin{frame}{Ejercicios}
    
  \begin{itemize}
    \item
      Lista 18
      \begin{itemize}
      \item El ejercicio \ref{ej:medCeroTimesInter} sobre los conjuntos de medida cero.
      \item Terminar la prueba del lema \ref{lem:GrafMedidaCero} es el ejercicio \ref{ej:termPruebaGrafMedCero}.
      \end{itemize}
    \end{itemize}
  
\end{frame}


% All of the following is optional and typically not needed. 
\appendix
\section<presentation>*{\appendixname}
\subsection<presentation>*{Lectura adicional}

\begin{frame}[allowframebreaks]
  \frametitle<presentation>{Lecturas adicionales}
    
  \begin{thebibliography}{10}
    
  \beamertemplatebookbibitems
  % Start with overview books.

  \bibitem{CambroNotas}
    S.Cambronero.
    \newblock {\em Notas MA0505}.
    \newblock 20XX.

    \bibitem{NachoNotas}
    I.Rojas
    \newblock {\em Notas MA0505}.
    \newblock 2018.
 
  \end{thebibliography}
  
\end{frame}
%% 6 - 2:10:52 48
%% 7 - 1:17:44 44
%% 8 - 27:37 69
%% 9 - 1:07:47 86 + 59:38 40 approx 2 h c 7 min
%% 10 - 2:22:39 63
%% 11 - Approx 2h c 2 min
%% 12 - Motiv y figs 1:00:17.53, en tot: 2:22:23.43
%% 13 - 1:18:03.51
%% 14 - 2:05:13.72
%% 15 - 1:13:41.49
%% 16 - 46:11.67
%% 17 - 44:18.85
\end{document}


