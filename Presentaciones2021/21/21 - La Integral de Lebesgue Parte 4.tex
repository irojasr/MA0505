\documentclass[utf8]{beamer}

\mode<presentation>
{
  \usetheme{Warsaw}
  \setbeamercovered{transparent}
}


\usepackage{amsfonts,mathtools,amssymb}
\usepackage[spanish]{babel}
\usepackage{times}
\usepackage[T1]{fontenc}
\usepackage[shortlabels]{enumitem}
\usepackage{tikz}
\usepackage{physics}

\title[MA0505]{MA0505 - An\'alisis I}
\subtitle{Lecci\'on XI: La Integral de Lebesgue IV}

\author{Pedro M\'endez\inst{1}}

\institute[Universidad de Costa Rica] % (optional, but mostly needed)
{
  \inst{1}%
  Departmento de Matem\'atica Pura y Ciencias Actuariales\\
  Universidad de Costa Rica
  }

\date[I-2021] {Semestre I, 2021}

%%%%%%%%% === Theorems and suchlike === %%%%%%%%%%%%%%

\theoremstyle{plain}
\newtheorem{Th}{Teorema}               %%% Theorem 1.1.1
\newtheorem{Tmon}{Teoremón}
\newtheorem{Prop}{Proposición}         %%% Proposition 1.1.2
\newtheorem{Lem}{Lema}                 %%% Lemma 3
\newtheorem{Cor}{Corolario}            %%% Corollary 4

\theoremstyle{definition}
\newtheorem{Def}{Definición}           %%% Definition 5
\newtheorem{Ex}{Ejemplo}               %%% Example 6
\newtheorem{Ej}{Ejercicio}             %%% Ejercicio 7
\newtheorem{Hec}[Th]{Hecho}            %%% Hecho 1.1.8

\theoremstyle{remark}
\newtheorem{Rmk}[Th]{Observación}      %%%Remark 1.1.9
\newtheorem*{nonum-Rmk}{Observación}         %%% No number Fact
\newtheorem*{Notn}{Notaci\'on}        %% Notaciones
\newtheorem*{Warn}{Advertencia}       %% Advertencias

\numberwithin{equation}{section}

%% Accomodations 

\makeatletter
\def\moverlay{\mathpalette\mov@rlay}
\def\mov@rlay#1#2{\leavevmode\vtop{%
   \baselineskip\z@skip \lineskiplimit-\maxdimen
   \ialign{\hfil$\m@th#1##$\hfil\cr#2\crcr}}}
\newcommand{\charfusion}[3][\mathord]{
    #1{\ifx#1\mathop\vphantom{#2}\fi
        \mathpalette\mov@rlay{#2\cr#3}
      }
    \ifx#1\mathop\expandafter\displaylimits\fi}
\makeatother

% Greek letters:

\newcommand{\al}{\alpha}                %% short for  \alpha
\newcommand{\bt}{\beta}                 %% short for  \beta
\newcommand{\Dl}{\Delta}                %% short for  \Delta
\newcommand{\dl}{\delta}                %% short for  \delta
\newcommand{\eps}{\varepsilon}          %% short for  \varepsilon
\newcommand{\Ga}{\Gamma}                %% short for  \Gamma
\newcommand{\ga}{\gamma}                %% short for  \gamma
\newcommand{\La}{\Lambda}               %% short for  \Lambda
\newcommand{\la}{\lambda}               %% short for  \lambda
\newcommand{\kp}{\kappa}                %% short for  \kappa
\newcommand{\Om}{\Omega}                %% short for  \Omega
\newcommand{\om}{\omega}                %% short for  \omega
\newcommand{\Sg}{\Sigma}                %% short for  \Sigma
\newcommand{\sg}{\sigma}                %% short for  \sigma
\newcommand{\te}{\theta}                %% short for  \theta
\newcommand{\vf}{\varphi}               %% short for  \varphi
\newcommand{\ze}{\zeta}                 %% short for  \zeta

%Boldface letters

\newcommand{\bC}{\mathbb{C}}    %%% números complejos
\newcommand{\bN}{\mathbb{N}}    %%% números naturales
\newcommand{\bP}{\mathbb{P}}        %% números enteros positivos
\newcommand{\bQ}{\mathbb{Q}}    %%% números racionales
\newcommand{\bR}{\mathbb{R}}    %%% números reales
\newcommand{\bS}{\mathbb{S}}    %%% esfera
\newcommand{\bZ}{\mathbb{Z}}    %%% números enteros

%Script letters:

\newcommand{\cA}{\mathcal{A}}           %% formas diferenciales
\newcommand{\cB}{\mathcal{B}}           %% una base vectorial
\newcommand{\cC}{\mathcal{C}}           %% otra base vectorial
\newcommand{\cF}{\mathcal{F}}           %% espacio de Fock
\newcommand{\cL}{\mathcal{L}}           %% operadores lineales
\newcommand{\cM}{\mathcal{M}}           %% multiplicadores
\newcommand{\cN}{\mathcal{N}}           %% funciones nulas
\newcommand{\cP}{\mathcal{P}}           %% una particion
\newcommand{\cR}{\mathcal{R}}           %% funciones representativas
\newcommand{\cS}{\mathcal{S}}           %% funciones de Schwartz


%Brackets

\newcommand{\bonj}[1]{\left\lbrack#1\right\rbrack}
\newcommand{\obonj}[1]{\left\rbrack#1\right\lbrack}
\newcommand{\rbonj}[1]{\left\rbrack#1\right\rbrack}
\newcommand{\lbonj}[1]{\left\lbrack#1\right\lbrack}
\newcommand{\snm}[1]{\|#1\|}           %small norma
\newcommand{\nm}[1]{\left\|#1\right\|} %norma pegadita
\newcommand{\pnm}[1]{\biggl|\biggl|#1\biggr|\biggr|}
\newcommand{\set}[1]{\{\,#1\,\}}    %% set notation
\newcommand{\floor}[1]{\lfloor#1\rfloor} %% mayor entero <= x
\newcommand{\Set}[1]{\biggl\{\,#1\,\biggr\}} %% set notation (large)
\newcommand\quot[2]{
        \mathchoice
            {% \displaystyle
                \text{\raise1ex\hbox{$#1$}\Big/\lower1ex\hbox{$#2$}}%
            }
            {% \textstyle
                {^{ #1}/_{ #2}}
            }
            {% \scriptstyle
                {^{ #1}/_{ #2}}
            }
            {% \scriptscriptstyle
                {^{ #1}/_{ #2}}
            }
    }
\newcommand*\squot[2]{{^{ #1}/_{ #2}}}%%%small quotient

%Symbols 

\newcommand{\x}{\times}
\renewcommand{\geq}{\geqslant}          %% mayor o igual (variante)
\newcommand{\hookto}{\hookrightarrow}     %% inclusion arrow
\newcommand{\isom}{\simeq}              %% isomorfismo
\renewcommand{\l}{\ell}                   %% ele cursiva
\renewcommand{\leq}{\leqslant}          %% menor o igual (variante)
\newcommand{\less}{\setminus}           %% set difference
\newcommand{\To}{\Rightarrow}
\newcommand{\ov}{\overline}
\newcommand{\un}{\underline}
\newcommand{\del}{\partial}
\newcommand{\ind}{\mathbf{1}}       %%%indicator function

%%% Small fractions in displays:

\newcommand{\half}{{\mathchoice{\nhalf}{\thalf}{\shalf}{\shalf}}} %%display text script script^2
\newcommand{\happi}{{\tfrac{\pi}{2}}} %% small fraction  \pi/2
\newcommand{\quarter}{\tfrac{1}{4}} %% small fraction  1/4
\newcommand{\nhalf}{\frac{1}{2}}
\newcommand{\shalf}{{\scriptstyle\frac{1}{2}}} %% tiny fraction 1/2
\newcommand{\thalf}{{\tfrac{1}{2}}} %% small fraction  1/2
\newcommand{\third}{\tfrac{1}{3}}   %% small fraction  1/3 %Hay que renew porque mathabx toma second y third como x'' y x''' por ejemplo

\newcommand{\ihalf}{{\tfrac{i}{2}}} %% small fraction  i/2

\newcommand{\suci}{_{i=1}^\infty} %% diminutivo
\newcommand{\sucj}{_{j=1}^\infty} %% diminutivo
\newcommand{\suck}{_{k=1}^\infty} %% diminutivo
\newcommand{\sucl}{_{\l=1}^\infty} %% diminutivo
\newcommand{\sucm}{_{m=1}^\infty} %% diminutivo
\newcommand{\sucn}{_{n=1}^\infty} %% diminutivo

\newcommand*{\Cdot}{{\raisebox{-0.25ex}{\scalebox{1.5}{$\cdot$}}}}      %% cdot más grande
\renewcommand{\.}{\Cdot}                %% producto escalar
\newcommand{\cupdot}{\charfusion[\mathbin]{\cup}{\.}}
\newcommand{\bigcupdot}{\charfusion[\mathbin]{\bigcup}{\.}}
\DeclareMathOperator{\Var}{Var}     %%%variance

\begin{document}

\begin{frame}
  \titlepage
\end{frame}

\begin{frame}{Agenda}
  \tableofcontents
  % You might wish to add the option [pausesections]
\end{frame}


% Structuring a talk is a difficult task and the following structure
% may not be suitable. Here are some rules that apply for this
% solution: 

% - Exactly two or three sections (other than the summary).
% - At *most* three subsections per section.
% - Talk about 30s to 2min per frame. So there should be between about
%   15 and 30 frames, all told.

% - A conference audience is likely to know very little of what you
%   are going to talk about. So *simplify*!
% - In a 20min talk, getting the main ideas across is hard
%   enough. Leave out details, even if it means being less precise than
%   you think necessary.
% - If you omit details that are vital to the proof/implementation,
%   just say so once. Everybody will be happy with that.

\section{La Integral de Lebesgue para Funciones Medibles}

\subsection{Parte Positiva y Parte Negativa}
\begin{frame}
Sea $f:E\to\ov\bR$ una función medible. Entonces $f=f^+-f^-$ con $f^+=\max\set{0,f}$ y $f^-=\max\set{0,-f}$. Sabemos que $f^+$ y $f^-$ son medibles y no negativas. Entonces 
$$\int\limits_E f^+\dd x,\quad\int\limits_E f^-\dd x$$
están bien definidas. Y por tanto podemos definir
$$\int\limits_Ef\dd x=\int\limits_Ef^+\dd x-\int\limits_Ef^-\dd x.$$
\end{frame}

\begin{frame}{Las Funciones Integrables}
  Decimos que $f$ es \alert{integrable} o bien que $f\in L(E)$ si 
  $$\int\limits_Ef^+\dd x,\int\limits_Ef^-\dd x<\infty.$$
  Note que 
  $$\left|\int\limits_Ef\dd x\right|\leq \int\limits_Ef^+\dd x+\int\limits_Ef^-\dd x=\int\limits_E|f|\dd x.$$
  Si $\int\limits_E|f|\dd x<\infty$ entonces $f\in\bR$ c.p.d.
\end{frame}

\subsection{Propiedades de la Integral}

\begin{frame}
  \begin{Th}\label{th:propsInt}
Sea $f:E\to\ov{\bR}$ y $g:E\to\ov\bR$ medibles tales que $\int\limits_E f\dd x$ y $\int\limits_E g\dd x$ existen. 
\begin{enumerate}[(i)]
  \item Sea $f\leq g$ c.p.d. en $E$. Entonces $\int\limits_Ef\dd x\leq \int\limits_E g\dd x$. En particular si $f=g$ c.p.d. en $E$ entonces $\int\limits_Ef \dd x=\int\limits_E g\dd x$. 
  \item Si $E_1\subseteq E$, entonces $\int\limits_{E_1} f\dd x$ existe.
  \item Si $E=\bigcupdot\suck E_k$, entonces $\int\limits_E f\dd x=\sum\suck\int\limits_{E_k}f\dd x$.
  \item Si $E_1\subseteq E$ con $m(E_1)=0$ entonces $\int\limits_{E_1}f \dd x=0$.
\end{enumerate} 
  \end{Th}
\end{frame}

\begin{frame}{Prueba del Teorema}
  Note que si $f\leq g$ c.p.d. entonces 
  \begin{gather*}
    f^+=\max\set{0,f}\leq\max\set{0,g}=g^+,\\
    g^-=\max\set{0,-g}\leq\max\set{0,-f}=f^-.
  \end{gather*}
  Luego 
  $$\int\limits_Ef^+\dd x\leq \int\limits_Eg^+\dd x,\ \int\limits_Eg^-\dd x\leq\int\limits_Ef^-\dd x,$$
  y entonces
  $$\int\limits_Ef^+\dd x-\int\limits_Ef^-\dd x<\int\limits_Eg^+\dd x-\int\limits_Eg^-\dd x.$$
\end{frame}

\begin{frame}{Continuamos la Prueba}
  Note que 
  $$0\leq \int\limits_{E_1}f^+\dd x\leq\int\limits_{E}f^+\dd x,\ 0\leq \int\limits_{E_1}f^-\dd x\leq\int\limits_{E}f^-\dd x.$$
  Finalmente si 
  \begin{align*}
    \int\limits_{E}f^+\dd x&=\sum\suck\int\limits_{E_k}f^+\dd x<\infty,\ \text{ó}\\
    \int\limits_{E}f^-\dd x&=\sum\suck\int\limits_{E_k}f^-\dd x<\infty.
  \end{align*}
  Entonces 
  $$\int\limits_{E}f\dd x=\sum\suck\int\limits_{E_k}f\dd x.$$
\end{frame}

\section{Linealidad de la Integral}

\begin{frame}
  \begin{Th}\label{th:linealidad}
  Sean $f,g:E\to\ov\bR$ medibles tales que $\int\limits_Ef\dd x$ y $\int\limits_Eg\dd x$ existen. Entonces vale que
  \begin{enumerate}
    \item $\displaystyle \int\limits_Ecf\dd x=c\int\limits_Ef\dd x$.
    \item Además, si $f,g\in L^1(E)$, entonces
    $$\int\limits_E(f+g)\dd x=\int\limits_Ef\dd x+\int\limits_Eg\dd x.$$
  \end{enumerate}
\end{Th}

\end{frame}

\begin{frame}{Prueba del Teorema}
  Primero si $c\leq 0$ entonces 
$$(cf)^+=-cf^-,\quad (cf)^-=-cf^+.$$
Entonces vale que
$$\int\limits_Ecf\dd x=-c\int\limits_Ef^-\dd x+c\int\limits_Ef^+\dd x=c\int\limits_Ef\dd x.$$
\end{frame}

\begin{frame}{Segunda Parte}
  Si $f,g$ son integrables entonces
  $$0\leq \int\limits_E|f+g|\dd x\leq \int\limits_E|f|\dd x+ \int\limits_E|g|\dd x<\infty.$$
  Note que los siguientes conjuntos cubren $E$:
\begin{align*}
  &E_1=\set{f\geq 0,\ g\geq 0},\\
  &E_2=\set{f\leq 0,\ g\leq 0},\\
  &E_3=\set{f\geq 0,\ g< 0,\ f+g\geq 0},\\
  &E_4=\set{f< 0,\ g\geq 0,\ f+g\geq 0},\\
  &E_5=\set{f\geq 0,\ g< 0,\ f+g< 0},\\
  &E_6=\set{f< 0,\ g\geq 0,\ f+g< 0}.
\end{align*}  
\end{frame}

\begin{frame}{Continuamos la Prueba}
Note que 
$$\int\limits_{E_1}(f+g)\dd x=\hspace{-2pt}\int\limits_{E_1}(f^++g^+)\dd x=\hspace{-2pt}\int\limits_{E_1}f^+\dd x+\int\limits_{E_1}g^+\dd x=\hspace{-2pt}\int\limits_{E_1}f\dd x+\int\limits_{E_1}g\dd x$$
y también
\begin{align*}
  \int\limits_{E_2}(f+g)\dd x&=\hspace{-2pt}\int\limits_{E_2}-(f^-+g^-)\dd x=\hspace{-2pt}-\int\limits_{E_2}(f^-+g^-)\dd x\\
  &=-\hspace{-2pt}\int\limits_{E_2}f^-\dd x-\int\limits_{E_2}g^-\dd x=\int\limits_{E_2}f\dd x+\int\limits_{E_2}g\dd x.
\end{align*}
\end{frame}

\begin{frame}
  De esta manera 
  $$\int\limits_{E_3}((f+g)-g)\dd x=\int\limits_{E_3}(f+g)\dd x+\int\limits_{E_3}(-g)\dd x.$$
  Entonces 
  \begin{align*}
    \int\limits_{E_3}f\dd x&=\int\limits_{E_3}(f+g)\dd x+\int\limits_{E_3}(-g)\dd x\\
    &=\int\limits_{E_3}(f+g)\dd x-\int\limits_{E_3}g\dd x.
  \end{align*}
  Por lo tanto 
  $$\int\limits_{E_3}(f+g)\dd x=\int\limits_{E_3}f\dd x+\int\limits_{E_3}g\dd x.$$
\end{frame}

\begin{frame}{Terminamos}
 Finalmente
 $$\int\limits_{E_5}(-(f+g)+f)\dd x=\int\limits_{E_5}(-g)\dd x=\int\limits_{E_5}-(f+g)\dd x+\int\limits_{E_5}f\dd x.$$
 Así tenemos que 
 $$\int\limits_{E_5}f\dd x+\int\limits_{E_5}g\dd x=\int\limits_{E_5}(f+g)\dd x.$$
Terminar el resto de la prueba es un \alert{ejercicio}.
\end{frame}

\subsection{Y las Restas}
\begin{frame}
  Luego si $f_1,\dots,f_n\in L(E)$, entonces $\sum_{k=1}^na_kf_k\in L(E)$ y vale que
  $$\int\limits_E\left(\sum_{k=1}^na_kf_k\right)\dd x=\sum_{k=1}^na_k\int\limits_Ef_k\dd x.$$
  Ojo que en general no es cierto que 
  $$\int\limits_E(f-g)\dd x=\int\limits_Ef\dd x-\int\limits_Eg\dd x.$$
  Por ejemplo considere $f=\ind_{\bonj{n,\infty}}$ y $g=\ind_{\bonj{n+1,\infty}}.$
\end{frame}

\begin{frame}
  Sin embargo si $f$ y $\phi$ son medibles con $\phi\leq f$ c.p.d. y $\phi\in L(E)$, entonces $f^-\leq \phi^-$. Es decir
  $$\int\limits_Ef^-\dd x\leq\int\limits_E\phi^-\dd x<\infty$$
   y por lo tanto $\int\limits_Ef \dd x$ existe. Ahora si $\int\limits_Ef^+=\infty$ (es decir, $f\not\in L(E)$), entonces $f-\phi\not\in L(E)$ y 
   $$\int\limits_E(f-\phi)\dd x=\int\limits_E(f-\phi)^+\dd x=\infty.$$
\end{frame}

\begin{frame}{El Resultado}
  \begin{Prop}\label{prop:restasInt}
Si $\phi\leq f$ c.p.d.
 y $\phi\in L(E)$, entonces 
 $$\int\limits_E(f-\phi)\dd x=\int\limits_Ef\dd x-\int\limits_E\phi\dd x.$$
  \end{Prop}
\end{frame}

\subsection{¿Y qué hay de Productos?}

\begin{frame}{Productos}
  Las condiciones para que $fg$ sea integrable son más complejas. Por ejemplo si $f\in L(E)$ y $|g(x)|\leq M$ para $x\in E$, entonces $fg\in L(E)$.
\end{frame}

\section{Los Teoremas de Convergencia}

\begin{frame}{Convergencia Monótona}
  \begin{Th}\label{th:TCM}
    Sea $\set{f_k}\suck$ una sucesión de funciones medibles en $E$, tales que $\lim_{k\to\infty}f_k=f$ c.p.d. en $E$.
    \begin{enumerate}[(i)]
      \item Si existe $\phi\in L(E)$ tal que $\phi\leq f_k\leq f_{k+1}$ c.p.d. para $k\geq 1$, entonces $\lim_{k\to\infty}\int\limits_Ef_k\dd x=\int\limits_Ef\dd x$.
      \item Si existe $\phi\in L(E)$ tal que $f_{k+1}\leq f_k\leq \phi$ c.p.d. para $k\geq 1$, entonces $\lim_{k\to\infty}\int\limits_Ef_k\dd x=\int\limits_Ef\dd x$.
    \end{enumerate}
  \end{Th}
\end{frame}

\begin{frame}{Prueba TCM}
  Si se cumple la primera condición entonces $f_k-\phi\geq 0$ y $f_{k+1}-\phi\geq f_{k}-\phi$ para $k\geq 1$. Entonces 
  $$\lim_{k\to\infty}\int\limits_E(f_k-\phi)\dd x=\int\limits_E(f-\phi)\dd x.$$
  Es decir 
  $$\lim_{k\to\infty}\int\limits_Ef_k\dd x-\int\limits_E\phi\dd x=\int\limits_Ef\dd x-\int\limits_E\phi\dd x.$$
\end{frame}

\begin{frame}{Fatou}
  \begin{Th}\label{th:fatou}
    Sea $\set{f_k}\suck$ una sucesión de funciones medibles en $E$. Si existe $\phi\in L(E)$ tal que $f_k\geq\phi$ c.p.d. en $E$ para $k\geq 1$, entonces 
    $$\int\limits_E\left(\liminf_{k\to\infty}f_k\right)\dd x\leq \liminf_{k\to\infty}\int\limits_Ef_k\dd x.$$
  \end{Th}
\end{frame}

\begin{frame}{Convergencia Dominada}
  \begin{Th}\label{th:DCT}
    Sea $\set{f_k}\suck$ una sucesión de funciones medibles en $E$ tales que $\lim_{k\to\infty}f_k=f$ en $E$. Si existe $\phi\in L(E)$ tal que $|f_k|\leq \phi$ c.p.d. en $E$ para $k\geq 1$, entonces 
    $$\lim_{k\to\infty}\int\limits_Ef_k\dd x=\int\limits_Ef\dd x.$$
  \end{Th}
La prueba se basa en considerar $\phi+f_k$ y $\phi-f_k$.
\end{frame}

\begin{frame}{Convergencia Uniforme}
  \begin{Th}\label{th:UCT}
    Sea $f_k\in L(E)$, $k\geq 1$. Si $\lim_{k\to\infty}f_k=f$ uniformemente en $E$ con $m(E)<\infty$, entonces $f\in L(E)$ y 
    $$\lim_{k\to\infty}\int\limits_Ef_k\dd x=\int\limits_Ef\dd x.$$
  \end{Th}
  La prueba de este teorema es un \alert{ejercicio}.
  \begin{Rmk}
Note que $f_k(x)=\frac{1}{k}$ converge a cero en $\bR$ pero $\int\limits_\bR f_k\dd x=\infty$.
  \end{Rmk}
\end{frame}

\section*{Resumen}

\subsection*{Qu\'e vimos hoy}

\begin{frame}{Resumen}

  % Keep the summary *very short*.
  \begin{itemize}
  \item El teorema \ref{th:propsInt} que resumen las propiedades de las integrales en general.
  \item El teorema \ref{th:linealidad} sobre la linealidad de la integral en general.
  \item El caso de las restas \ref{prop:restasInt}.
  \item Los teoremas de convergencia: \ref{th:TCM} (Convergencia Monótona), \ref{th:fatou} (Fatou), \ref{th:DCT} (Convergencia Dominada) y \ref{th:UCT} (Convergencia Uniforme).
  \end{itemize}
  
\end{frame}

\subsection*{Ejercicios a trabajar}
\begin{frame}{Ejercicios}
    
  \begin{itemize}
    \item
      Lista 21
      \begin{itemize}
      \item Concluir la prueba del teorema \ref{th:linealidad} es un ejercicio.
      \item Probar el teorema \ref{th:UCT} de convergencia uniforme es un ejercicio.
      \end{itemize}
    \end{itemize}
  
\end{frame}


% All of the following is optional and typically not needed. 
\appendix
\section<presentation>*{\appendixname}
\subsection<presentation>*{Lectura adicional}

\begin{frame}[allowframebreaks]
  \frametitle<presentation>{Lecturas adicionales}
    
  \begin{thebibliography}{10}
    
  \beamertemplatebookbibitems
  % Start with overview books.

  \bibitem{CambroNotas}
    S.Cambronero.
    \newblock {\em Notas MA0505}.
    \newblock 20XX.

    \bibitem{NachoNotas}
    I.Rojas
    \newblock {\em Notas MA0505}.
    \newblock 2018.
 
  \end{thebibliography}
  
\end{frame}
%% 6 - 2:10:52 48
%% 7 - 1:17:44 44
%% 8 - 27:37 69
%% 9 - 1:07:47 86 + 59:38 40 approx 2 h c 7 min
%% 10 - 2:22:39 63
%% 11 - Approx 2h c 2 min
%% 12 - Motiv y figs 1:00:17.53, en tot: 2:22:23.43
%% 13 - 1:18:03.51
%% 14 - 2:05:13.72
%% 15 - 1:13:41.49
%% 16 - 46:11.67
%% 17 - 44:18.85
%% 18 - 1:03:50.64
%% 19 - 51:57.37
%% 20 - 30:52.52
%% 21 - 1:06:30.84
\end{document}


